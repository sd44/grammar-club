\chapter{状语从句}

状语从句的语义分析比较复杂,因为\textbf{同一个从属连词所引导的从句意思可能不
  同},而且这样的情形为数不少。例如, since从句可以是时间从句,也可以是原因分
句。\textbf{另外,有些从句把两层意思结合起来}。

\section{状语从句的从属连接词}


作时间状语的 ing 从句由以下从属连词引导: once, till, until, when, whenever,
while 和 whilst, as soon/long as:
\begin{itemize}
\item \unbf{Once having made a promise}, you should keep it.
\item The dog stayed at the entrance \unbf{until told to come in}.
\item Complete your work \unbf{as soon as possible}.
\end{itemize}

带有\textbf{until- 从句的主句}必须是\textbf{持续性}的,时间持续到until-从句的时间
为止。因为事件未发生的状态是持续性的,所以\textbf{否定从句总是持续性的},即使相应
的肯定从句并非如此。例如:
\begin{itemize}
\item I \unbf{didn't} start my meal \unbf{until} Adam arrived. [正确]
\item \sout{I \unbf{started} my meal \unbf{until} Adam arrived.} [错误]
\end{itemize}


地点从句主要由where和wherever引导,\textbf{where是具体的, wherever是非具体的}。
\begin{itemize}
\item \unbf{Where} the fire had been, we saw nothing but blackened ruins.

\item They went \unbf{wherever} they could find work. [to any place where]
\end{itemize}


\todo[inline]{本笔记对状语较少描述,以后可从 The Gramma Book等简明书籍中摘录。}

% \section{状语从句与并列从句的比较}

% 请看下例:

% \begin{enumerate}
% \item \unct{Because}{从属连接词} \unct{he needs the money}{状语从句}, \unct{he
%     works hard}{主要从句}.

%   因为他缺钱,所以他勤奋工作。
% \item \unct{He needs the money}{对等从句
%   }, \unct{and}{并列连接词} \unct{he
%     works hard}{并列从句}.

%   他需要钱,也勤奋工作。
% \end{enumerate}

% 例 1 是分成主、从的复句结构。其中状语从句 he needs the money 和主要从句he
% works hard 分别都是完整、独立的简单句,以一个连接词连起来。这和例 2中两个并列
% 从句的情形完全相同。唯一的差别是并列从句使用并列连接词(例 2中的and),连接起
% 来的两个从句地位相等,没有主从之分,也不需互相解释。

% 状语从句则使用从属连接词(例1 中的because),使得 because he needs the
% money成为从属地位的从句,当作副词使用,用来修饰主要从句中的动词 works(交
% 待works hard的原因)。

% \section{状语从句与名词从句的比较}

% 状语从句和名词从句就有较大的差别了。请看下例:

% \begin{enumerate}
% \item \unct{The witness}{S} \unct{said}{V} \unct{that}{连接词} \unct{he saw the
%     whole thing}{名词从句}.

%   证人说他目睹了事情发生的全过程。
% \item \unct{The witness}{S} \unct{said}{V} \unct{this}{O}, \unct{though}{连接词}
%   \unct{he didn't really see it}{状语从句}.

%   证人这样说,尽管他没有真正看到。
% \end{enumerate}

% 先来观察一下名词从句和状语从句的共同点。首先,两者原来都是完整、独立的简单句
% (例1 中的 He saw the whole thing 与例 2 中的 He didn't really see it)。然后,
% 两者都是加上从属连接词构成从句,但是由此开始有了差别。名词从句加的连接词
% 是that,表示“那件事情”,此外没有别的意义。状语从句加的连接词,如
% 例 2 的though,以及上节例子中的 because等等,都是\textbf{有意义的连接词},表
% 达两句话之间的逻辑关系:\textbf{though表示让步,because 表示原因,if表示条件。
%   使用的连接词不同,一个有意义,一个没有意义,这是状语从句和名词从句第一个重
%   要的差别。}

% 第二个差别是:名词从句属于名词类,要放在主要从句中的名词位置使用,状语从句则
% 不然。例1 中主要从句 The witness said 部分尚不完整,在及物动词 said之后还要有
% 个名词当宾语,构成 S+V+O 的句型才算完成。取一个独立的简单句 He saw the whole
% thing 外加上没有意义的连接词that,造成一个名词从句,就可以放入主要从句 The
% witness said后面的宾语位置使用,成为例 1 的形状。

% 状语从句情况不同。\textbf{它是修饰语的词类,要附在一个完整的主要从句上作修饰
%   语使用。}如例2 He didn't really see it 是完整的单句,外面加上表示让步的连接
% 词 though构成状语从句。主要从句 The witness said this已经是完整的句子
% (S+O+V),把副词从句 though he didn't really see it直接附上去,当作副词,用
% 来修饰动词said。因为两个从句都是完整的简单句,所以说其间的关系\textbf{很像并
%   列从句}的关系。这是状语从句与名词从句第二个重要的差别。

% \section{状语从句的种类}

% 状语从句因为结构十分单纯,所以学习状语从句的重点只是在认识各种连接词,以便写
% 作时可以选择贴切的连接词来表达各种逻辑关系。以下按照各种逻辑关系把状语从句的
% 连接词大略分类。

% \subsection{时间、地方}

% \begin{enumerate}
% \item He became more frugal \unct{after}{连接词} \unct{he got married}{副词从
%     句}.

%   他结婚以后变得比较节俭。
% \end{enumerate}
% 状语从句修饰动词 became 的时间。

% \begin{enumerate}[resume]
% \item I'll be waiting for you \unct{until}{连接词} \unct{you're married}{副词从
%     句}.

%   我会等你,直到你结婚为止。
% \end{enumerate}
% 状语从句修饰动词 will be waiting 的时间。

% 附带说明一下:\textbf{未来时间的状语从句,}虽然还没有到发生的时间,可是语气上必须当
% 作“到了那个时候”来说,所以\textbf{时态要用现在式来表示}(如例2 中的 are married)。
% 这是属于语气的问题,在从前介绍语气的单元中曾说明过。

% \begin{enumerate}[resume]
% \item It was all over \unct{when}{连接词} \unct{I got there}{状语从句}.

%   我赶到的时候事情都结束了。
% \end{enumerate}
% 状语从句修饰动词 was 的时间。

% when这个\textbf{连接词},也可以当做\textbf{关系副词}来使用,这点留待下一
% 章\textbf{关系副词从句}时再详细说明。

% \begin{enumerate}[resume]
% \item A small town grew \unct{where}{连接词} \unct{three roads met}{状语从句}.

%   一个小镇在三条路交会处发展起来。
% \end{enumerate}
% 状语从句修饰动词 grew 的地方。

% 同样的,where 这个连接词也可以当作关系词来解释。

% \subsection{条件}

% \begin{enumerate}
% \item \unct{If}{连接词} \unct{he calls}{状语从句}, I'll say you're
%   sleeping.

%   如果他打电话来,我就说你在睡觉。
% \end{enumerate}
% 状语从句修饰动词 will say 的条件——如果打来就会说,不打来就不说了。

% 在表示条件的状语从句中,如果时间是未来,也必须以“当作真正发生”的语气来说,
% 所以要用现在式的动词。同时请注意say 的宾语(名词从句)you're sleeping也用现在
% 式,因为这是当作已经打来了,自然要说“在睡觉”,而不是“要去睡觉”(will be
% sleeping)。只有主要从句 I'll say用将来时的动词,因为如果打来了“就会”说,这
% 表示现在还没说!

% \begin{enumerate}[resume]
% \item He won't have it his way, \unct{as long as}{连接词} \unct{I'm here}{副词从
%     句}.

%   只要我在,不会让他称心如意。
% \end{enumerate}
% 状语从句修饰动词 won't have 的条件。 as long as 也可以用比较级来诠释。

% \begin{enumerate}[resume]
% \item \unct{Suppose}{连接词} \unct{you were ill}{状语从句}, where would you go?

%   假定你生病了,你会到哪里去?
% \end{enumerate}
% 状语从句修饰动词 would go 的条件。

% \textbf{suppose 本来是动词,这个状语从句原来是 supposing that you were ill的句型,经过
%   省略后才成为只剩 suppose 一字当连接词用。}这个例子只是用以说明,用if远比用
% suppose要好。

% 同时请注意例 3中两个动词都是非事实的
% 假设语句。

% \subsection{原因、结果}

% \begin{enumerate}
% \item \unct{As}{连接词} \unct{there isn't much time left}{状语从句}, we might as
%   well call it a day.

%   既然时间所剩无几,我们不妨就此结束好了。
% \end{enumerate}
% 状语从句修饰动词 might call 的原因。

% \begin{enumerate}[resume]
% \item There's nothing to worry about, \unct{now that}{连接词} \unct{Father is
%     back}{状语从句}.

%   既然父亲回来了,就没什么好担心了。
% \end{enumerate}
% 状语从句修饰动词 is 的原因。

% 请注意:简单句前面加上一个单独的、没有意义的that,会成为名词从句(指“那件
% 事”)。\textbf{可是 that一旦配合其他字眼当作连接词、具有表达逻辑关系的功能时,
%   就成了副词从句的连接词,引导的是状语从句。now that 解释为“既然、所以”,所
%   以它后面的 Father is back就成了状语从句。}

% \begin{enumerate}[resume]
% \item He looked \unct{so}{连} sincere \unct{that}{接词} \unct{no one doubted his
%   story}{状语从句}.

% 他看起来是那么诚恳,所以没有人怀疑他说的话。
% \end{enumerate}
% 状语从句修饰形容词 sincere 造成什么结果。

% 连接词 so \ldots{} that 表示因果关系,所以引导的是状语从句。

% \begin{enumerate}[resume]
% \item The mother locked the door from the outside, \unct{so that}{连接词} \unct{the
%     kids couldn't}{副}\\
%   \unct{get out when they saw fire}{词从句}.

%   这位妈妈把门反锁,所以小孩看到火起时也跑不出去。
% \end{enumerate}

% 状语从句修饰动词 locked 造成什么结果。

% 连接词 so that亦表示因果关系,所以引导的是状语从句。请注意这个状语从句中又有一个
% 表示时间的状语从句when they saw fire。

% \subsection{目的}

% \begin{enumerate}
% \item The mother locked away the drugs \unct{so that}{连接词} \unct{the kids
%     wouldn't swallow}{副词}\\
%   \unct{any by mistake}{从句}.

%   这位妈妈把药锁好,目的是不让小孩误食。
% \end{enumerate}
% 状语从句修饰动词 locked 有什么目的。

% 同样是 so that 连接词,同样引导状语从句,但是这里用来表示目的。

% \begin{enumerate}[resume]
% \item I've typed out the main points in boldface, \unct{in order that}{连接词}
%   \unct{you won't miss them}{状语从句}.

%   我用黑体字把重点打出来,好让你们不会遗漏掉。
% \end{enumerate}
% 状语从句修饰动词 type out有何目的。同样的,这里的连接词不是单独、无意义的 that,
% 而是表示目的的 in order that,所以引导的是状语从句。

% \begin{enumerate}[resume]
% \item I've underlined the key points, \unct{lest}{连接词} \unct{you miss them}{副词
%     从句}.

%   我已把重点画了线,以免你们把它们漏掉。
% \end{enumerate}
% 状语从句修饰动词 have underlined 有何目的。

% \begin{enumerate}[resume]
% \item You'd better bring more money, \unct{in case}{连接词} \unct{you should need
%     it}{状语从句}.

%   你最好多带点钱,万一要用。
% \end{enumerate}
% 状语从句修饰动词 bring 的目的。

% \subsection{让步}

% \begin{enumerate}
% \item \unct{Although}{连接词} \unct{you may object}{状语从句}, I must give it a
%   try.

%   虽然你可能会反对,我仍然必须试试看。
% \end{enumerate}
% 状语从句修饰动词 must give。

% \begin{enumerate}[resume]
% \item \unct{While}{连接词} \unct{the disease is not fatal}{状语从句}, it can be
%   very dangerous.

%   这虽然不是要命的病,不过也很危险。
% \end{enumerate}
% 状语从句修饰动词 can be。

% \begin{enumerate}[resume]
% \item \emph{Wh-} 拼法的连接词,若解释为 No matter \ldots(不论),就表示让步语气,引导副
%   词从句。

% \begin{itemize}
% \item \unct{Whether (=No matter)}{连接词} \unct{you agree or not}{状语从句}, I want
%   to give it a try.

%   无论你是否同意,我都想试一试。
% \item \unct{Whoever (=No matter who) calls}{连接词 + 状语从句}, I won't answer.

%   不管谁来电话,我都不接。
% \item \unct{Whichever (=No matter which) way you go}{连接词 + 状语从句}, I'll
%   follow.

%   不论你走到哪里,我都跟定你了。
% \item \unct{However (=No matter how) cold it is}{连接词 + 状语从句}, he's always
%   wearing a shirt only.

%   不管多冷,他总是只穿件村衫。
% \item \unct{Wherever (=No matter where) he is}{连接词 + 状语从句}, I'll get
%   him!

%   不管他躲到哪儿,我都会抓到他!
% \item \unct{Whenever (=No matter when) you like}{连接词 + 状语从句}, you can call me.

%   你随时给我来电话都可以。
% \end{itemize}
% \end{enumerate}


% \subsection{限制}

% \begin{enumerate}
% \item \unct{As far as}{连接词} \unct{money is concerned}{状语从句}, you needn't
%   worry.

%   钱的方面你不必担心。
% \end{enumerate}
% 状语从句修饰动词 needn't worry,表示不必担心的事情是在某一方面,暗示也许是别的方
% 面才要担心。

% \begin{enumerate}[resume]
% \item Picasso was a revolutionary \unct{in that}{连接词} \unct{he broke all
%     traditions}{状语从句}.

%   毕加索是革命派,原因是他打破了一切传统。
% \end{enumerate}
% 状语从句修饰动词was,把“是革命派”的意思加以限制:在于打破传统,并非真的举枪起义。
% 连接词 in that 是由 in the sense that(从某种意义来说)省略而来。

% \subsection{方法,状态}

% \begin{enumerate}
% \item He played the piano \unct{as}{连接词} \unct{Horowitz would have}{副词从
%     句}.

%   他弹钢琴有如大师霍洛维兹。
% \end{enumerate}
% 状语从句修饰动词 played——如何弹法。

% \begin{enumerate}[resume]
% \item 他写字像左撇子
%   \begin{itemize}
%   \item He writes \unbf{as if} \unbf{he is left handed} .
%   \item He writes \unbf{as if} \unbf{he were left handed} .
%   \item He writes \unbf{as if} \unbf{he was left handed} .
%   \end{itemize}
% \end{enumerate}

% 上面三句中,用 is 表示他应该真的是左撇子,用 were表示他不是,只是冒充左撇子,
% 用 was则表示不一定——可能是,也可能不是。三句话都是用连接词 as if引导后面的副词
% 从句,修饰动词 writes——“如何写法”。

\section{非限定状语从句和无动词状语从句}

\begin{description}
\item[独立从句] 具有一个明显的\textbf{主语}但\textbf{不用从属连词}引导
  的\textbf{非限定性从句和无动词从句}。之所以成为独立从句是因为它们在句法上显
  然并\textbf{不与主句绑定在一起}。独立从句可以是\textbf{ -ing, -ed 或无动词从句}。
\end{description}

\begin{itemize}
\item \textbf{No further discussion a rising}, the meeting was brought to a close.
\item \textbf{Lunch finished}, the guests retired to the lounge.
\item \textbf{Christmas then only days away}, the family was pent up with excitement.
\end{itemize}

非限定从句和无动词从句中\textbf{没有主语时},用以识别主语的依附规则
是,\textbf{认为母句的主语就是其主语}:
\begin{itemize}
\item The oranges, \unbf{when (they are) ripe}, are picked and sorted
  mechanically.

\item \unbf{Driving home after work}, I accidentally went through a red
  light. [While I was driving home after work]

\item \unbf{To climb the rock face}, we had to take various precautions. [So that
  we could climb]

\end{itemize}

某些情况下,依附规则是不适用的,或者说至少是不严格的:
\begin{description}[style=nextline]
\item[从句是一个主语外接状语, 这时隐含的主语是说话者 I]

  \unbf{Putting it mildly}, you have caused us some inconvenience.

\item[隐含的主语是整个主句]

  I'll help you \unbf{if necessary}. [\ldots{} \textbf{if it is necessary}]

\item[隐含的主语是一个不定代词或支撑词 it]

  \unbf{When dining in the restaurant}, a jacket and tie are required. [\textbf{When one
  dines}]

  \unbf{Being Christmas}, the government offices were closed. [\textbf{Since it was}]
\end{description}

\textbf{没有从属连词引导的状语从句和无动词从句被称为增补从句},根据上下文,我
们可以用其表示\textbf{时间、条件、原因、让步或状况关系}。对读者或听者来说,这种伴随关
系的实质是\textbf{从语境中推断}的。
\begin{itemize}
\item \unct{Reaching}{When we Reached} the river, we pitched camp for the night.
\item Julia, \unct{being}{since she was} a nun, spent much of her time in
  prayer and meditation.
\item The sentence is ambiguous, (\unbf{if / when it is}) taken out of context.

\item We spoke \unbf{face to face}.
\end{itemize}

\section{Test}

\paragraph{请选出最适当的答案填入空格内,以使句子完整。}

\begin{enumerate}
\item Please come back \ttu you finish your work.
\begin{tasks}(2)
  \task as soon as
  \task as soon as possible
  \task as possibly soon
  \task as soon possible
\end{tasks}

\item Which of the following is correct?
\begin{tasks}
  \task He is very smart; moreover, he is diligent.
  \task He is very smart, moreover, he is diligent.
  \task He is very smart, Moreover, he is diligent.
  \task He is very smart; and moreover, he is diligent.
\end{tasks}

\item It is not safe to get off a car \ttu.
\begin{tasks}
  \task unless it is in motion
  \task until it has come to a stop
  \task after you have opened the window
  \task before the traffic light turns red
\end{tasks}

\item If you sell your rice now, you will be playing your hand very badly. Wait \ttu the price goes up.
\begin{tasks}(2)
  \task until
  \task still
  \task for
  \task that
\end{tasks}

\item (The rain is over. You must not stay any longer.) You must not stay any longer \ttu the rain is over.
\begin{tasks}(2)
  \task when
  \task that
  \task now that
  \task as for
\end{tasks}

\item It is such a good opportunity \ttu you should not miss it.
\begin{tasks}(2)
  \task as
  \task that
  \task which
  \task of which
\end{tasks}

\item Tom is dull. He works hard. He will surely pass the exam.
\begin{tasks}
  \task Though Tom is dull, he works so hard that he will surely pass the exam.
  \task Despite his dullness, Tom will surely pass the exam by work hard.
  \task Tom will surely pass the exam because he works hard with his dullness.
  \task Dull as Tom is, he will surely pass the exam with work hard.
\end{tasks}

\item She had worked several years \ttu she could continue her studies in France.
\begin{tasks}(2)
  \task as
  \task while
  \task before
  \task then
\end{tasks}

\item \ttu, he never begged for money.
\begin{tasks}(2)
  \task Despite he was poor
  \task Because he was poor
  \task Poor as he was
  \task In spite of he was poor
\end{tasks}

\item \ttu the typhoon warnings, several fishing boats set sail.
\begin{tasks}(2)
  \task Because
  \task According
  \task Despite
  \task Although
\end{tasks}

\item I knew I would never have what I needed \ttu it myself.
\begin{tasks}(2)
  \task even made
  \task without me making
  \task except making
  \task unless I made
\end{tasks}

\item Which of the following is correct?
\begin{tasks}
  \task I shall either go back to Taiwan or my family will come to England.
  \task I shall go back either to Taiwan or my family will come to England.
  \task Either I shall go back to Taiwan or my family will come to England.
\end{tasks}

\item \ttu unwilling to do so, he had to follow the traditional ways.
\begin{tasks}(2)
  \task After
  \task Although
  \task Since
  \task Once
\end{tasks}

\item Which of the following is correct?
\begin{tasks}
  \task Not only the money but also three paintings was stolen.
  \task Not only the money but also three paintings were stolen.
  \task Not only the money was stolen but also were the paintings.
\end{tasks}

\item No one was sure \ttu was going to happen.
\begin{tasks}(2)
  \task what
  \task who
  \task when
  \task where
\end{tasks}

\item \ttu she studied hard, but she didn't succeed.
\begin{tasks}(2)
  \task Though
  \task Although
  \task Indeed
  \task While
\end{tasks}

\item "You seem angry at Martha." "I am. \ttu I'm concerned, she can go away forever."
\begin{tasks}(2)
  \task As like as
  \task As many as
  \task As such as
  \task As far as
\end{tasks}

\item I'm going to tell you the number once more, \ttu you forget.
\begin{tasks}(2)
  \task don't
  \task that
  \task so that
  \task lest
\end{tasks}

\item The mother's warning \ttu there be no contact with boys was generally ignored.
\begin{tasks}(2)
  \task which
  \task that
  \task if
  \task wherever
\end{tasks}

\item Don't go away \ttu you have told me what actually happened.
\begin{tasks}(2)
  \task since
  \task then
  \task after
  \task until
\end{tasks}

\end{enumerate}

\section{Answer}

\begin{enumerate}

\item (A) 空格前后分别是完整的独立从句,中间只需要连接词,如 A,把后面的从句改为副词
  从句。B 的 as soon as possible 已经是一个从句(as soon as it is possible 的简
  化),不再是连接词。C 和 D 都不是完整的连接词。

\item (A) \textbf{moreover 是副词,不具连接词的语法功能,所以要用分号(;)来取代连接词。}

\item (B) 四个答案在语法上都对,句意则只有 B 合理:“除非车子停稳了,否则下车不安全。”

\item (A) wait 一词构成一个祈使句,与右边的 the price goes up 之间要有连接词,故可排除非连接词的 B。答案 D 的 that 会把从句变成名词从句,不合词类要求。C 的 for 可以当连接词,不过要解释为 because,在此不通,只有 A 这个连接词是引导时间状语从句用的,符合要求。

\item (C) now that 解释为“既然”,符合原意。

\item (B) 上文有 such,因而要有 that 来配合,表示因果关系。

\item (A) 句一和句二有相反关系,句二和句三有因果关系,因而分别用 though 和 so…that 来连接。B 中的 by work hard 错在以动词 work 直接放在介词后面。C 中的 he works hard with his dullness 句意十分牵强。D 与 B 相同,也是错在把动词(work)直接放在介词(with)后面。

\item (C) had worked 是过去完成时,could continue 是过去一般体,这是时间先后顺序,因
  而用 before。

\item (C) Though he was poor 可改写为 Poor as he was,注意连接词现在要
  用 as。A 和 D 都是\textbf{错用介词(despite 和 in spite of)来引导从句}。B
  的句型正确,但逻辑关系不通顺。

\item (C) 名词短语 the typhoon warnings 前面应有介词(只有 C 是)。

\item  (D) 空格中要表示“条件”,因而 C 不适合。A 多一个动词,文法错误。B 应该省略掉与主语重复的 me。D 是以 unless 的状语从句表示条件,符合要求。

\item  (C) either 和 or 之间的部分要和 or 之后的部分对称。符合条件的只有 C(从句对从句),其余答案在词类上都不对称。

\item (B) unwilling 和 had to 意思上相反,只有 although 可表示相反的关系。答
  案 B 是 although he was unwilling… 的简化。

\item (B) not only 和 but also 亦要求对称。A 虽然有对称,但是动词 was 和主语 three
  paintings 在单复数上有冲突,而 C 中应倒装的是 not only was the money stolen,不
  是后面的从句。


\item (A) 这个位置要用连接词,又要能当 was 的主语,所以要用关系代词类(A 或 B)。
  因为它前面没有先行词,不能用 who,只能用 what,故选 A。what was going to
  happen 亦可作疑问句类的名词从句看待。

\item  (C) 两个从句间已有连接词 but,不能再用连接词(A、B 和 D 都是),只剩下一个副词类的 C。

\item (D) 这个位置要用连接词。D 是表示限度的从属连接词,符合要求。B 的 as many as 则要配合复数名词才能使用。
\item (D) 这个位置要用连接词,故排除 A。B 会造成名词从句,不合句型要求。C 和 D 都是状语从句连接词,但只有 D 的 lest(以免……)符合逻辑关系。
\item (B) 从下文的 there be no contact… 来看,是间接祈使句语气,应为名词从句,故选择 B。

\item (D) 这个位置连接两个从句,要用连接词(A、C 或 D),从意思判断用 D 较合理。
\end{enumerate}

\chapter{关系副词从句}
\label{chap:reladv}

\section{关系副词从句与状语从句的区别}

\index{概念!关系副词从句 relative adverb clauses} 在英语语法中,关系副词从
句 (RELATIVE ADVERB CLAUSES) 和状语从句是两种不同类型的从句,它们在结构和功能
上有显著的区别。
  \begin{description}
  \item[关系副词从句] 关系副词从句是由\textbf{关系副词}引导的,从句中的成分通
    常用来\textbf{修饰}前面的\textbf{先行词},起到\textbf{定语}的作用。常见的关系副
    词包括\textbf{when(时间)、where(地点)和why(原因)}。例如:
    \begin{itemize}
    \item I remember \unbf{the day} \unbf{when we first met}.” (我记得我们第一次见面的那天。)

    \item Sam knows \unbf{the place} \unbf{where we're meeting} and
      \unbf{the time} \unbf{when we're meeting}, but he doesn't know
      \unbf{the reason} \unbf{why we're meeting}.

    \item The percentage of working adults is at its lowest level since
      1983, \unbf{when women were still entering the workforce.}

      自1983年妇女不断加入劳动大军以来,成年人就业率处于最低水平。

      \textbf{关系副词从句也可以是非限制性修饰语,与领句之间用逗号隔开。}
    \end{itemize}

  \item[状语从句] 状语从句是用来\textbf{修饰}\textbf{动词、形容词或其他副
      词}的,从句通常表示\textbf{时间、地点、原因、条件等},起到\textbf{状语}的
    作用。它可以由多种连接词引导,如because(因为)、if(如果
    )、when(当……时)等(见上一章)。例如:
    \begin{itemize}
    \item I will call you \unbf{when I arrive}. (我到达时会给你
      打电话。)
    \item \unbf{If it rains tomorrow}, we will stay indoors. (如果明天下雨,我们就待在室内。)
    \end{itemize}
  \end{description}

\section{关系副词与介词短语的替代}

\textbf{非正式文体中,关系副词可被介词+关系代词组成的介词短语替代},引导后面的句子:
\begin{description}
\item[where] prep + which + [place]
\item[when] prep + which + [time]
\item[why] prep + which + [reason]
\end{description}

例句:
\begin{itemize}
\item Sam knows \unbf{the time} \unbf{at which we're meeting}, but he doesn't know
      \unbf{the reason} \unbf{for which we're meeting}.
\end{itemize}

另外,副词how也可被类似介词短语替代:
\begin{description}
\item[how] prep + which + [manner/way]
\end{description}

\chapter{形容词性关系从句}

\section{形容词性关系从句和关系副词从句的异同}

\textbf{一般我们说关系从句时,就是指形容词性关系从句。它修饰先行词,提供关于
  先行词表示的人或实体的更多信息。}

在现代语法中,形容词性关系分句和关系副词分句都是用来\textbf{修饰名词或代词等}的从句。
按理说,关系副词从句也可归为形容词性关系从句,但为了减少语法混乱,我们将之单
独并列。两者的异同有:

\begin{description}
\item[形容词性关系分句] 指由关系代词(如who, which, that等)引导的从句,通常
  用于限制或定义先行詞,使其意义更加明确。。例如:\textbf{The book}
  \textbf{that I read} was fascinating.

\item[关系副词分句] 由关系副词(如when, where, why等)引导的从句。这类从句也
  用于修饰名词或代词,但其主要功能是提供时间、地点或原因等背景信息,不一定改
  变先行名詞的基本含义。例如:I remember \textbf{the day} \textbf{when we
    met}. 在这个例子中,“when we met”是一个关系副词分句,用于修饰“the
  day”。
\end{description}


如前言所述,国内语法多用\textbf{定语从句}这个概念,可以简单粗暴认为:

\textbf{定语从句 = 形容词性关系从句(见\cref{}) + 关系副词从句(见\cref{chap:reladv})}


\section{限制性和非限制性关系分句}

可根据关系从句与其先行词的关系,将之分为限制性和非限制性两种。

\begin{table}[htbp!]
  \centering \small
  \begin{talltblr}[ caption = {形容词性关系代词},
    label = {tab:relativePro},
    ]{
      width=\linewidth, colspec={lllll},
      rowsep=2pt, colsep=4pt,
      row{1} = {c, font=\bfseries},
      vlines,
    }
    \toprule
    格 & \SetCell[c=2]{c} 限制性 & & \SetCell[c=2]{c} 非限制性 & \\ \midrule
    & 人称 & 非人称 & 人称 & 非人称 \\ \midrule
    主格 & who/that & which/that & who & \SetCell[r=2]{c} which \\ \hline
    宾格 & whom/that/zero & which/that/zero & whom & \\ \hline
    属格 & \SetCell[c=4]{c} whose \\
    \bottomrule
  \end{talltblr}%
\end{table}

\textbf{关系代词引导形容词性关系从句,总是放在关系从句开头。有which, who,
  whom, whose, that 或零关系代词,他们都可以作限制性关系代词;但that或零不能
  做非限制性关系代词。}

\textbf{1. 限制性关系从句与它们的先行词或中心词在读音上是一气呵成的,以示对先行词所指
  对象的限制。}
\begin{itemize}
\item This is not \unbf{something} \unbf{that/which} would disturb me \`{A}NYway.

  无论如何,这都不会打扰我。(something that \ldots{} me连读)
\item I'd like to see \unbf{the car} \unbf{that/which/( )} you bought last week. (zero由( )表
  示)

  我想看看你上周买的那辆车。(the car that/which \ldots{} last week连读)
% \item The lady \unbf{whose} daughter you met is Mrs. Brown.

%   你见过她女儿的那位女士是布朗夫人。
\end{itemize}

\textbf{2. 非限制性关系从句是一种插入性说明,它通常是对先行词加以描述而不是对先行词作进一
  步的限定,可在关系代词前面(先行词或中心词之后)停顿。}
\begin{itemize}
\item  They operated like poliT\`{I}cians | \unbf{who notoriously have no sense of humour
  at \`{A}LL}.
\end{itemize}


\section{whose和of which}

与who和whom不同,whose可以指人称,也可以指非人称,但人们不太原意用whose来指
非人称的先行词,可以用of which来表示,但也常显得有些别扭。
\begin{itemize}
\item The lady \unbf{whose daughter you met} is Mrs. Brown.
\item The house \unbf{whose roof was damaged} has now been repaired.
\item The house \unbf{of which the roof was damaged} \ldots{}
\item The house \unbf{the roof of which was damaged}  \ldots{}
\end{itemize}

\section{句子关系从句}

\index{概念!{句子关系从句 (SENTENTIAL RELATIVE CLAUSE)}}
\textbf{修饰名词的关系从句}的先行词是\textbf{名词短语},而\textbf{句子关系从
  句 (SENTENTIAL RELATIVE CLAUSE)}的先行词是:
\begin{description}
\item[主句的谓语或谓体] They say he \unbf{plays truant}, \unbf{which he doesn't}.

  \unbf{walks for an hour each morning}, \unbf{which would bore me}.

\item[主句或整个句子] Things then improved, \unbf{which surprises me}.

  Colin married my sister and I married his brother, \unbf{which makes Colin and me in-law}.

\item[之前多个句子] \unbf{--- which is how the kangaroo came to have a pouch}.

  所以袋鼠才有了育儿袋。
\end{description}

\textbf{句子关系从句与名词短语中的非限制性后置修饰从句相似,因为它们也用语调
  或标点符号将其本身和先行词分隔开来。}但一个修饰句子,一个对先行词做非限制性
修饰,暂不详述。
\begin{itemize}
\item The plane may be several hours late, \unbf{in which case} there's no point in our waiting.

\item They were under water for several hours, \unbf{from which experience} they
  emerged unharmed.
\end{itemize}

which 之外的其他关系词也用来引导句子关系分句:whereup on [after which,
inconsequence of which], whence [from which, from where, in consequence of
which], since when [since which time尤其在英国英语中], until when, from
when和 by when等。

\begin{itemize}
\item He found himself in an unknown land, \unbf{from where} he eventually reached Greenland.
\end{itemize}

\section{Test}

\paragraph{请选出最适当的答案填入空格内,以使句子完整。}

\begin{enumerate}
\item Not long ago I wrote a letter to a friend, \ttu almost got us into a quarrel.
\begin{tasks}(2)
  \task whom
  \task where
  \task which
  \task what
\end{tasks}

\item England, \ttu is justly proud of her poets, is today ranked behind the
  continent in poetic achievement.
\begin{tasks}(2)
  \task which
  \task that
  \task where
  \task whom
\end{tasks}

\item You are the only friend \ttu he will listen to at all.
\begin{tasks}(2)
  \task where
  \task whom
  \task which
  \task that
\end{tasks}

\item Choose the correct sentence:
\begin{tasks}
  \task I have bought a book, the cover of which bears a picture of The Hague.
  \task I have bought a book; the cover of which bears a picture of The Hague.
  \task I have bought a book, the cover of which, bears a picture of The Hague.
  \task I have bought a book, of which bears a picture of The Hague.
\end{tasks}

\item This is the one encyclopedia upon \ttu I can depend.
\begin{tasks}(2)
  \task that
  \task which
  \task what
  \task it
\end{tasks}

\item \ttu likes good food and cheerful service would like the Regent Hotel.
\begin{tasks}(2)
  \task Who that
  \task Someone
  \task Whoever
  \task Who
\end{tasks}

\item This custom, \ttu, is slowly disappearing.
\begin{tasks}
  \task of many centuries ago origin
  \task which originated many centuries ago
  \task with many centuries origin
\end{tasks}

\item I find it very unfair when \ttu I do is considered mediocre or a
  failure. I can be depressed for days because of \ttu happens.

I.
\begin{tasks}(2)
  \task that
  \task those
  \task which
  \task what
\end{tasks}

II.
\begin{tasks}(2)
  \task who
  \task what
  \task that
  \task where
\end{tasks}

\item \ttu is elected President, corruption won't cease.
\begin{tasks}(2)
  \task Whatever
  \task Who
  \task How
  \task Whoever
\end{tasks}

\item Neither success nor money, to me at least, is the criterion \ttu we are to be judged.
\begin{tasks}(2)
  \task which
  \task under which
  \task under which
  \task since which
\end{tasks}

\item I'm afraid I'd never be able to see Jane again, \ttu very much.
\begin{tasks}(2)
  \task that I love
  \task I love
  \task I love her
  \task whom I love
\end{tasks}

\item Didn't you know that all \ttu is not gold?
\begin{tasks}(2)
  \task which glitters
  \task glitters
  \task who glitters
  \task that glitters
\end{tasks}

\item I have a present for \ttu his hand first.
\begin{tasks}(2)
  \task whoever raises
  \task whomever raises
  \task anyone raises
  \task whoever that raises
\end{tasks}

\item Boys \ttu in the dorm make a lot of friends.
\begin{tasks}(2)
  \task who live
  \task who lives
  \task live
  \task that living
\end{tasks}

\item The final decision will be up to \ttu everyone trusts.
\begin{tasks}(2)
  \task Judge Clemens, whom
  \task Judge Clemens, who
  \task Judge Clemens whom
  \task Judge Clemens who
\end{tasks}

\item \ttu he has in his pocket, it's not a gun.
\begin{tasks}(2)
  \task What
  \task Whatever
  \task When
  \task How
\end{tasks}

\item Abandoned flower pots are \ttu.
\begin{tasks}
  \task where do mosquitoes thrive
  \task mosquitoes thrive there
  \task where mosquitoes thrive
  \task what mosquitoes thrive
\end{tasks}

\item The author wrote his first novel \ttu he was working as a hotel clerk.
\begin{tasks}(2)
  \task which
  \task during
  \task what
  \task while
\end{tasks}

\item \ttu held upside down, the fire extinguisher begins to spray bubbles.
\begin{tasks}(2)
  \task When it is
  \task When they are
  \task Whenever they are
  \task During it is
\end{tasks}

\item I need to know \ttu the library is open.
\begin{tasks}(2)
  \task that
  \task when
  \task which
  \task if it
\end{tasks}

\end{enumerate}

\section{Answer}
\begin{enumerate}
\item (C) 这个位置是要能作 got 的主语,又要作\textbf{非限制性关系分句}的连接
  词,指代前面句子所说“写信这件事”或“这封信”,应选which。

\item (A) 同上。非限制性关系分句引导词不能用that。

\item (B或D) 先行词是 the only friend,有明显的指示功能,且关系词是关系从句中
  的宾语,可用who/whom/that。

\item (A) B 错在以分号隔开非限制性关系从句和其先行词,C 错在以逗号分隔开了关系从
  句,D 错在用介词短语 of which 作主语。

  另外,这个题目其实出的不是很合适:不加逗号:可能更好。加了逗号,是非限制性
  关系从句,辅助说明;不加逗号的话,是限制性关系从句,限制说明。

\item (B) 先行词 the one encyclopedia,接关系从句,但that没有宾格,不能紧接在
  介词(upon)后面,所以选 B。

\item (C) wh- 名词性关系从句。用whoever比who合适,喜欢服务的任何人,非特指。

\item (B) A 和 C 都在名词 origin 前面加上了短语(many centuries ago 和 many
  centuries)来修饰,可是名词前面只能用单词的形容词来修饰,所以错误。B 是正确
  的关系从句。

\item \Ronum{1} (D) \Ronum{2} (B) 两句都是wh- 名词性关系从句,而非形容词关系
  从句。

\item (D) 不管谁当选,wh- 名词性关系从句(名词从句)。


\item (C) 可还原为 We are to be judged under the criterion of
  \ldots(我们应以此标准来被衡量。),因而改成名词性关系从句,要用 under which。


\item (D) 关系从句 whom I love very much。因为关系从句前面有逗号,所以 whom
  不能省略。

  再给大家出个纠错题,以下语句有个小语法错误:

  I'm afraid I'd never be able to see Jane again, I love her very much.

  答案是:\textbf{两个独立的从句应该用正确的标点符号连接。}这里可以用分号或者
  句号来分隔两个句子,或者使用连词来使句子更流畅。
  \begin{description}
  \item[分号] I'm afraid I'd never be able to see Jane again; I love
    her very much. 分号比句号关系更紧密。
  \item[句号] I'm afraid I'd never be able to see Jane again. I love her
    very much.
  \item[连词] I'm afraid I'd never be able to see Jane again because I love
    her very much.
  \end{description}

\item (D) All that glitters is not gold.(会发亮的并不都是金子。)这是一句格言。
  关系从句 that glitters 之中的关系词应用 that,因为先行词 all 表示“全部”,
  是一个指示明确的范围,所以要用 that 来取代 which。

\item (A) 名词从句,不管是谁先举手。

\item (A) who live in the dorm 是wh- 名词性关系从句,主语 who 代表先行词 boys,是
  复数, 所以动词 live 不加 \emph{-s}。D选项如果去掉that也可以,非限制性从句作后
  置修饰语。

\item (A) Judge Clemens 是专有名词,不应该再接限制性关系分局,逗号后为非限制
  性关系从句,应用who/whom均可。有人认为宾格关系代词只能用whom,是纠枉过正的
  做法。

\item (B) 名词性关系从句。结合语意,不管他口袋里有什么,都不是一把枪。
  whatever不如what。

\item (C) 本句可还原为:Abandoned flower pots are places where mosquitoes
  thrive.(弃置的花盆是蚊虫孳生的地方。)省略掉 places 之后就是 C 的答案。


\item (D) 空格后面是表示时间的状语从句,while 即是状语从句连接词。

\item  (A) 后面的 fire extinguisher(灭火器)是单数,所以代词要用单数 it。从句 it is held… 需要连接词,故选 A。D 的 during 是介词。

\item (B) 从语法要求来看,A 和 B 都对。A 表示“图书馆开着这件事”,B 则是由疑问
  句变来,表示“图书馆什么时候开”,两者都是正确的名词从句。不过 B 的问题比较
  能配合上文 I need to know… 的语意。
\end{enumerate}

% \chapter{对等连接词与对等从句}


% \begin{enumerate}
% \item The Yangtze River, the most vital source of irrigation water across the
%   width of China \CJKunderline{and important as a transportation conduit as well,} has
%   nurtured the Chinese civilization for millennia. (误)


%   长江是横贯中国最重要的灌溉水源,同时也是重要的交通管道,数千年来孕育着中华文化。
% \end{enumerate}


% 主要从句的基本句型是:
% \begin{itemize}
% \item \unct{The Yangtze River}{S} \unct{has nurtured}{V} \unct{the Chinese
%     civilization}{O}.
% \end{itemize}

% 例句1错在对等连接词 and连接的两个部分在结构上并不对称:\textbf{左边的 the
%   most vital source是名词短语,右边的 important 却是形容词,词类不同,不适合
%   以对等连接词and连接。}底线部分的改法不只一种,但是最简单的改法就是把右边的
% 词类改为名词类以符合对称的要求,故应修正为:
% \begin{mybox}
% \begin{itemize}
% \item The Yangtze River, the most vital source of irrigation water across the
%   width of China \unbf{and an important transportation conduit}, has nurtured the
%   Chinese civilization for millennia. (正)
% \end{itemize}
% \end{mybox}

% \begin{enumerate}[resume]
% \item Scientists believe that hibernation is triggered \uline{by decreasing
%   environmental temperatures, food shortage, shorter periods of daylight,
%   and by hormonal activity}. (误)

%   科学家认为引发冬眠的因素包括环境的气温下降、食物短缺、白昼缩短以及荷尔蒙作用。
% \end{enumerate}

% 句中画底线的部分是以 by A、B、C and by D 的结构来修饰宾语从句中的动词 is
% triggered。由内容来看 A、B、C、D是平行的(都是引发冬眠的因素),应该以对等的
% 方式来处理。可是原句的处理方式中,by A、B、C 之间缺乏连接词,而 and 只能连接
% 两个 by 引导的介词短语(by this and by that),因此原句的结构有语法上的问题。
% 最佳的修改方式是把A、B、C、D 四项平行的因素并列,以连接词 and串连,共同置于单
% 一的介词之后成为 by A、B、C and D 的结构,故应修正为:
% \begin{mybox}

% \begin{itemize}
% \item Scientists believe that hibernation is triggered \ul{\textbf{by decreasing
%     environmental temperatures, food shortage,shorter periods of daylight,
%     and hormonal activity}}. (正)
% \end{itemize}
% \end{mybox}

% \begin{enumerate}[resume]
% \item Smoking by pregnant women may \ul{slow the growth and generally harm} the
%   fetus. (误)

%   孕妇吸烟可能妨碍胚胎发育,对胚胎造成一般性的伤害。
% \end{enumerate}
% 这个句子可视为以下的对等从句的省略:
% \begin{itemize}
% \item \unct{Smoking by pregnant women}{S} \unct{may slow}{V1} \unct{the growth
%     of the fetus}{O1}, and \unct{it}{S2} \unct{may generally harm the
%     fetus}{V2}.
% \end{itemize}

% 这两个对等从句的主语 smoking by pregnant women 相同,宾语 the fetus也相
% 同。\textbf{对等从句省略的原则就是,相对应位置如果是重复的元素就可以省略。}这是因为
% 对等从句有相当严格的对称要求,即使省略掉重复的元素依然能表达清楚。不过在上面
% 这个句子中,两个宾语扮演的角色不同:在前面的对等从句以fetus 为介词 of 的宾
% 语,在后面的对等从句则以 fetus 为动词 harm的宾语。所以固然可以省略前面的宾
% 语 fetus,但是介词 of却不能省略。故应修正为:
% \begin{mybox}

% \begin{itemize}
% \item Smoking by pregnant women \unbf{may slow the growth of and generally harm} the
%   fetus. (正)
% \end{itemize}
% \end{mybox}

% \begin{enumerate}[resume]
% \item Rapid advances in computer technology have enhanced \ul{the speed of
%     calculation, the quality of graphics, the fun with computer games, and
%     have lowered prices}. (误)

%   电脑技术的快速进展提高了计算的速度、图形的品质、电脑游戏的乐趣,也降低了价
%   格。
% \end{enumerate}
% 这个句子以 speed,quality 和 fun 三者为动词 have enhanced的宾语,三者在内容与
% 结构上都是对等的,可是却没有对等连接词来连接,反而在后面加上and 和 have
% lowered prices 连在一起,成为 A、B、C and D 的结构,其中A、B、C 都是名词短
% 语,D却是动词短语,这就犯了结构上不对称的毛病。内容上来说,A、B、C是所增加的
% 三样东西,D则是降低的东西,所以四者的内容也不对称,不适合并列。修改方法可以把
% 前面三个名词短语用A、B and C 的方式连接,第四项“降低价格”这项不对称的元素则
% 不必对等,而以从句简化(详见以后章节)的方式来处理,成为:
% \begin{mybox}

% \begin{itemize}
% \item Rapid advances in computer technology have enhanced \ul{\textbf{the speed of
%       calculation, the quality of graphics and the fun with computer games
%       while lowering prices}}. (正)
% \end{itemize}
% \end{mybox}

% \begin{enumerate}[resume]
% \item Population density is very low in Canada, \ul{the largest country in the
%   Western Hemisphere and it is the second largest in the whole world}. (误)

%   加拿大人口密度很低,它是西半球最大的国家,也是世界第二大国。
% \end{enumerate}
% 这个句子中,the largest country in the Western Hemisphere是关系从句省略
% 掉 which is 之后留下的名词补语,也就是所谓的同位语(作为Canada 的同位语),置
% 于对等连接词 and 的左边。但是连接词右边的 it is the second largest in the
% whole world在涵意上虽然和左边对称,可是却是主要从句的结构,所以结构上并不对称。
% 对等连接词的要求就是在涵意上、结构上都要尽量对称,所以可将it is the second
% largest in the whole world也改为名词短语以求结构对称工整,成为:
% \begin{mybox}
% \begin{itemize}
% \item Population density is very low in Canada, \ul{\textbf{the largest country in
%       the Western Hemisphere and the second largest in the whole
%       world}}. (正)
% \end{itemize}
% \end{mybox}

% \begin{enumerate}[resume]
% \item Once the safety concerns over the new production procedure were removed
%   and \ul{with its superiority to the old one being} proved, there was nothing to
%   stop the factory from switching over. (误)

%   新的生产程序一旦排除安全方面的顾虑,并且证明它比旧的生产程序更好,这家工厂
%   就没有理由不作改变了。
% \end{enumerate}

% 对等连接词 and 出现在底线之前。它的左边是一个从句,右边却是介词短语,造
% 成结构上的不对称。可以先把它还原为对等从句,成为:
% \begin{itemize}
% \item \unct{The safety concerns}{S1} over the new production procedure
%   \unct{were removed}{V1} and \unct{its superiority}{S2} to the old one
%   \unct{was proved}{V2}.
% \end{itemize}

% 这两个对等从句中,主语部分并不相同,动词部分是两个不同动词的被动态,只有
% be 动词是重复的元素,所以只能省略一个 be 动词,成为:
% \begin{itemize}
% \item The safety concerns over the new production procedure were removed and
%   its superiority to the old one proved.
% \end{itemize}

% 这个省略后的对等从句前面加上once(一旦)就成为表示条件的状语从句,若再附于主
% 要从句之上,就成为符合对称要求的从句:
% \begin{mybox}
% \begin{itemize}
% \item Once the safety concerns over the new production procedure were removed
%   and \unbf{its superiority to the old one proved}, there was nothing to stop the
%   factory from switching over. (正)
% \end{itemize}
% \end{mybox}

% \begin{enumerate}[resume]
% \item Worker bees in a honeybee hive assume various tasks, such as guarding the
%   entrance, \ul{serving as sentinel and to sound a warning at the slightest
%   threat}, and exploring outside the nest for areas rich in flowers and,
%   consequently, nectar. (误)

%   蜂窝中的工蜂担负各种任务,包括守卫入口、站哨并在威胁来临时发出警报,以及到
%   巢外寻找富有花朵及花蜜的地区。
% \end{enumerate}

% 句子中在 such as 之后列举工蜂担负的任务,基本上是 A、B and C的结构,其中
% B(画底线部分)又可以分成 B1 与B2——站哨并发出警报。这两个动作是一体的两面,
% 选择用对等的 and来连接本来十分恰当,只是所连接的两部分 serving as
% sentinel 与 to sound a warning 在结构上一是动名词,一是不定式,并不对称。再看
% 看 A(guarding the entrance)与 C(exploring outside the nest),都是动名词,
% 所以 B1 与 B2也应使用动名词才能对称,于是改为:
% \begin{mybox}
%   \begin{itemize}
%   \item Worker bees in a honeybee hive assume various tasks, such as guarding
%     the entrance, \unbf{serving as sentinel and sounding a warning at the
%       slightest threat}, and exploring outside the nest for areas rich in
%     flowers and, consequently, nectar. (正)
%   \end{itemize}
% \end{mybox}

% \begin{enumerate}[resume]
% \item Shi Huangdi of the Qin dynasty built the Great Wall of China in the 3rd
%   century BC, a gigantic construction that meanders from Gansu province in
%   the west through 2,400km to the Yellow Sea in the east and \ul{ranging}
%   from 4 to 12 m in width. (误)

%   秦始皇在公元前第三世纪修筑了长城,这是巨大的建筑,从西端的甘肃蜿蜒2,400 公
%   里到东端的黄海,宽度由 4 米至 12 米不等。
% \end{enumerate}


% 句中的 a gigantic construction 是 the Great Wall 的同位语,后面用 that
% meanders \ldots{} 的关系从句来修饰。对等连接词 and 的右边(底线部分)
% 是ranging,可是左边却找不到 Ving的结构可以与它对称。从意思上来看,右边是讲厚
% 度,左边讲长度的部分只有关系从句的动词meanders 可能与 ranging 对称,所以
% 把 ranging 改成动词 ranges以求对称,成为:
% \begin{mybox}
%   \begin{itemize}
%   \item Shi Huangdi of the Qin dynasty built the Great Wall of China in the 3rd
%     century BC, a gigantic construction that meanders from Gansu province in
%     the west through 2,400 km to the Yellow Sea in the east and
%     \unbf{ranges} from 4 to 12 m in width. (正)
%   \end{itemize}
% \end{mybox}

% \begin{enumerate}[resume]
% \item The large number of sizable orders suggests that factory operations are
%   thriving, \ul{but that the low-tech nature of the processing indicates
%     that} profit margins will not be as high as might be expected. (误)

%   从许多巨额订单来看,工厂的营运畅旺,可是加工程序属于低科技,显示利润幅度可
%   能不像预期那么高。
% \end{enumerate}

% 对等连接词 but 右边是 that 引导的名词从句,只能与左边的 that factory
% operations are thriving 对称。但是如此解释出来的句意不通。仔细对比 but的左右
% 边,发现意思上是另一种形式的对称:
% \begin{itemize}
% \item A. \unct{The large number}{S} of sizable orders \unct{suggests}{V} \unct{something good}{O}.
% \item B. \unct{The low-tech nature}{S} of the processing \unct{indicates}{V} \unct{something bad}{O}.
% \end{itemize}
% 这两句在形式与意思上都很对称。其中宾语部分的 something good 与 something bad
% 分别以一个 that引导的名词从句来表示。看出这层对称关系之后就可以明白: but的右
% 边应该与左边的主要从句对称,两句都是主要从句,不应以从属连接词 that来引导,所
% 以 把 but 右边的 that 拿掉,成为:
% \begin{mybox}
% \begin{itemize}
% \item The large number of sizable orders suggests that factory operations are
%   thriving, \unbf{but the low-tech nature of the processing indicates that} profit
%   margins will not be as high as might be expected. (正)
% \end{itemize}
% \end{mybox}

% \begin{enumerate}[resume]
% \item Not only is China the world's most populous state \ul{but also the largest
%   market} in the 21st century. (误)

%   中国不仅是世界人口最多的国家,也是 21 世纪最大的市场。
% \end{enumerate}
% 像 not only \ldots{} but also之类以相关词组(correlatives)出现的对等连接词,
% 在对称方面的要求更为严格:not only 与 but 之间所夹的部分要和 but 右边对称。原
% 句中把:not only移到句首成倒装句,造成的结果是它与 but 之间夹着一个完整的从句。
% 因此 but的右边只有名词短语 the largest market \ldots{}就不对称,应该改为完整
% 的从句,成为:
% \begin{mybox}
% \begin{itemize}
% \item Not only is China the world's most populous state \unbf{but it is also the
%   largest market} in the 21st century. (正)
% \end{itemize}
% \end{mybox}

% 注意 also 的位置不一定要和 but 放在一起。also 和 only一样有强调(focusing)的
% 功能。Not only 修饰动词 is,与其对称之下 also也和 be 动词放在一起才好,所以右
% 边是 but it is also 而不是 but also it is \ldots。

% \begin{enumerate}[resume]
% \item New radio stations are either overly partisan, resulting in lopsided
%   propaganda, or avoid politics completely, shirking the media's
%   responsibility as a public watchdog. (误)

%   新成立的广播电台不是党派色彩过于鲜明,造成一面倒的宣传,就是完全避谈政治,
%   推卸了媒体作为大众监察人的责任。)
% \end{enumerate}

% 相关词组 either \ldots{} or 之间所夹的部分也要与 or右边对称。原句中左边是形容
% 词 partisan,右边却是动词avoid,无法对称(两个简化从句 resulting
% \ldots{} 与 shirking \ldots{}在此先不讨论)。可将两边都改为形容词,成为:
% \begin{mybox}
% \begin{itemize}
% \item New radio stations \unbf{are either overly partisan,} resulting in lopsided
%   propaganda, \unbf{or completely apolitical,} shirking the media's responsibility
%   as a public watchdog. (正)
% \end{itemize}
% \end{mybox}
% 或者两边都用动词,成为:
% \begin{mybox}
% \begin{itemize}
% \item New radio stations \unbf{either take an overly partisan stance,} resulting in
%   lopsided propaganda, \unbf{or avoid politics completely,} thus shirking the
%   media's responsibility as a public watchdog. (正)
% \end{itemize}
% \end{mybox}


% \begin{enumerate}[resume]
% \item Many modern-day scientists are not atheists, to whom there is no such
%   thing as God; \ul{rather agnostics}, who refrain from conjecturing about the
%   existence of God, much less His properties. (误)

%   许多当代科学家并非无神论者,即不相信有神存在,而是不可知论者,即不愿妄加臆
%   测神的存在与否,更不愿推断神的属性。
% \end{enumerate}
% 这一句应该是以 not A but B 的相关词组来连接两个名词 atheists 和agnostics,后
% 面分别附上一个关系从句。但是原句中却选择用分号(;)和副词rather 来连接。分
% 号可以取代连接词来连接两个从句,例如:
% \begin{itemize}
% \item He's not an atheist; rather, he believes in agnosticism.

%   他不是无神论者,而是信奉不可知论。
% \end{itemize}
% 可是分号不能取代对等连接词来连接名词短语,更不能取代 not \ldots{} but
% 的相关词组,所以将相关词组还原成为:
% \begin{itemize}
% \item Many modern-day scientists are not atheists, to whom there no such thing
%   as God, \unbf{but} agnostics, who refrain from conjecturing about the existence
%   of God, much less His properties. (正)
% \end{itemize}


\chapter{比较从句}

\section{比较成分和比较基础}

在比较结构中,主句中的陈述与从句中的陈述进行比较。两句中共同的部分在从属分
句中可省略。
\begin{itemize}
\item \unct{Jane}{比较主体} is \unbf{as} \unct{healthy}{比较成分} \unbf{as}
  \unct{her sister}{比较对象} (is).
\item \unct{Jane}{比较主体} is \unct{healthier}{比较成分} \unbf{than}
  \unct{her sister}{比较对象} (is).
\end{itemize}

\section{比较成分的从句功能}

\textbf{比较成分可以是比较结果中除动词以外的任何一个成分}:
\begin{description}
\item[主语] \unbf{Most people} use this brand than (use) any other shampoo.

\item[直接宾语] She knows \unbf{more history} than most people (know).

\item[间接宾语] That toy has given \unbf{more children} happiness than any other (toy) (has).

\item[主语补语] Simo is \unbf{more relaxed} than he used to be.

\item[宾语补语] She thinks her children \unbf{more taller} than (they were) last year.
\item[状语] You've been working \unbf{much harder} than I (have).

\item[介词补足语] She's applied for \unbf{more jobs} than Joyce (has (applied for)).
\end{description}

由 more \ldots{} than, less \ldots{} than 和 as \ldots{} as等引导词(或称比较
标记)引导的\textbf{不一定是比较从句,后面可接续一个明显的比较标准或状态}。
\begin{itemize}
\item I weigh more than \unbf{200 pounds}.

\item It goes faster than \unbf{100 miles per hour}.

\end{itemize}

另一种不接比较从句的类型:
\begin{itemize}
\item I was more angry than \unbf{frightened}. [frightened:
  \doulos{/ˈfraɪtnd/} 受惊的,害怕的。]
\item I was angry more than \unbf{frightened}.

\item \sout{I was angrier than frightened}.
\end{itemize}
上述最后一句错误。因为angrier为屈折形式的比较级,frightened(害怕的)是过去分
词作形容词用,两者不对等。

more of a \ldots{} 和less of a \ldots{} 与可分等级的名词中心语连用:
\begin{itemize}
\item He's \unbf{more of a fool} than I thought (he was).

\item It was \unbf{less of a success} than I imagined (it would be).
\end{itemize}

当对比涉及\textbf{同一比较主体上的两个点}且一点高于另一点时,则than之后的部
分\textbf{不可以扩展成从句}。Than的功能是非从句比较中的\unbf{介词}:
\begin{itemize}
\item It's hotter \unbf{than} just warm. (或 It's hotter than 90°C.)
\item We drover farther \unbf{than} Henan.
\item She's wiser \unbf{than} merely clever.
\item We have to build it stronger \unbf{than} this.
\end{itemize}

\section{比较从句中的省略}

由于主句与从句在结构和内容上通常非常相似,因此\textbf{在比较从句中省略共同的
  部分是常规而不是例外}。以下是省略和代词、替代谓语和替代谓体的例子:

James and Susan often go to plays but
\begin{enumerate}
\item James enjoys the theater more than Susan enjoys the theater.
\item James enjoys the theater more than Susan enjoys it.
\item James enjoys the theater more than Susan does.
\item James enjoys the theater more than Susan.
\item James enjoys the theater more. [因为前半句已经说明了对象有两人,所以这
  里可以直接省略整个比较从句。]
\end{enumerate}

\textbf{宾语一般不可省略,除非主要动词也省略,如第3、4句,此时功能词可留可不留。}
\begin{itemize}
\item James enjoys the theater more than Susan \sout{enjoys}.

  误!比较从句中宾语省略,主要动词未省略。
\end{itemize}

但是,如果\textbf{宾语本来就是比较成分},那么\textbf{可以省略宾语,而不省略主要动词}:
\begin{itemize}
\item James knows more about the theater is more than Susan \unbf{knows}.
\end{itemize}


如作最大限度的省略,有可能造成歧义:
\begin{itemize}
\item He loves his dog more than his children.
\end{itemize}
上例的意思可能是他比他的孩子更爱狗(his children作从句主语),也可能是他爱狗
超过爱孩子(his children作从句宾语)。因此最好根据实际情况补充说明:
\begin{itemize}
\item He loves his dog more than his children \unbf{does} his dog.

  他比他的孩子更爱狗。
\item He loves his dog more than he loves his children.

  他爱狗超过爱孩子
\end{itemize}


\section{部分对比 (partial contrast)}

对比可能\textbf{只影响时态}或\textbf{加上了情态助动词}而已。在这种情况下,一
般是省略比较从句情态助动词之后的部分:

\begin{itemize}
\item I hear it more clearly than I \unbf{did}. [than I used to hear it]

\item I get up later than I \unbf{should}. [than I should get up]
\end{itemize}

如果只是\textbf{时态}上的对比,在比较从句中可能只用一个状语来表示:
\begin{itemize}
\item She'll enjoy it more than (she enjoyed it或 she did)last year.
\end{itemize}
这就为下列例句中从句全部省略提供了基础:
\begin{itemize}
\item You are slimmer (than you were).
\item You're looking better (than you were (looking)).
\end{itemize}

对一个隐含或实际表达的从句存在\textbf{逆向呼应}的省略:
\begin{itemize}
\item I caught the bus from the town: but Harry came home \unbf{even later}. [later than I came home]
\end{itemize}


话语之外\textbf{实境已包含被比较信息}的省略:
\begin{itemize}
\item You should have come home earlier. [earlier than you did]
\end{itemize}

部分对比可能是主句或对比从句中的\textbf{上位从句}:
\begin{itemize}
\item \unbf{She thinks} she's fatter than she (really) is.
\item He's a greater painter than \unbf{people suppose} (he is).
\item She enjoyed it more than \unbf{I expected} (her to (enjoy it)).
\end{itemize}

\section{等量比较 as \ldots{} as}

as \ldots{} as 结构在语法上与 more \ldots{} than 结构相似,只是 \textbf{as 不
  能像 more 那样作限定词、代词和下加伏语};这些功能由 as many (具数) 和 as
much(不具数)来弥补。因此我们可以在必要时用 as many 和 as much 替代 more:

as (much/many) 可作:
\begin{description}
\item[限定词] Isabelle has \unbf{as many} books as her brother (has).

\item[名词短语中心词] \unbf{As many} of his friends are in New York as (are) here.

\item[作下加状语] I agree with you \unbf{as much} as ((I agree) with) Robert.

\item[形容词中心语的修饰语] The article was \unbf{as objective} as I expected (it
  would be).

\item[前置修饰形容词的修饰语] It was \unbf{as lively} a discussion as we thought it would be.

  [形容词短语也可后置 It was a discussion \unbf{as lively} as \ldots{}]

\item[副词的修饰语] I am \unbf{as severely} handicapped as you (are).

  [副词也可后置 I am handicapped as severely as \ldots{}]
\end{description}

as ADJ a NOUN as \ldots{} 也是个常见的句式。
\begin{itemize}
\item  I did have a good time, but not \unbf{as good a time} as I should have had.
\end{itemize}

当母句是否定句时,可用 so \ldots{} as 取代 as \ldots{} as, \textbf{从句全部或大部
省略时尤其如此}:
\begin{itemize}
\item He's not \unbf{as naughty} as he was.
\item He's not \unbf{so naughty} as he was.
\item He's not \unbf{so naughty} (now).
\end{itemize}

\section{enough 和 too}

表达\textbf{足量或超越比较}的结构主要由enough或too + to不定式表示。
\begin{description}
\item[足量比较] They're rich \unbf{enough to} own a car.

  The book is simple \unbf{enough to} understand.

\item[超越比较] They're not \unbf{too poor to} own a car.

  他们还没有穷到买不起一辆车。

  The book is not \unbf{too difficult to} understand.

  这本书不是太难理解。
\end{description}

\textbf{too 有否定意义},表示\textbf{太、过于 \ldots{} 以致不能},比较:
\begin{itemize}
\item She's \unbf{old enough} to do some work.
\item She's \unbf{too old} to do any work.
\end{itemize}

如果语境许可,动词不定式从句可省略。sufficient(ly) 和 excessive(ly) 分别
是 enough 和 too 的较为正式的同义词,正式用法。
\begin{itemize}
\item The book is \unbf{sufficiently} simple to understand.
\item The book is not \unbf{excessively} difficult to understand.
\end{itemize}

\section{so \ldots{} that 和 such \ldots{} that}

So 是副词,前置修饰一个形容词或副词;such 是前限定词,与中后限定词一起修饰名
词中心语。

当that从句是\textbf{否定}时,so/such结构和too+to 不定式结构之间有一种对应关
系:
\begin{itemize}
\item It's \unbf{so} good a movie \unbf{that} we mustn't miss it.

  It's \unbf{too} good a movie \unbf{to} miss.

\item It was \unbf{such} a pleasant day \unbf{that} I didn't want to go to school.

  It was \unbf{too} pleasant a day \unbf{to} go to school.
\end{itemize}

当that从句是肯定的,so/such结构和enough+to 不定式结构之间有一种对应关系:
\begin{itemize}
\item It flies \unbf{so} fast \unbf{that} it can beat the speed record.

  It flies fast \unbf{enough to} beat the speed record.

\item I had \unbf{such} a bad headache \unbf{that} I needed two aspirins.

  I had a bad \unbf{enough} headache \unbf{to} need two aspirins.
\end{itemize}


当 \textbf{so} 单独与\textbf{动词}连用时,表示程度高;\textbf{such} 接续
的\textbf{名词}短语没有形容词前置修饰时,同样表示程度高:
\begin{itemize}
\item I \unbf{so} (much) enjoyed it \unbf{that} I'm determined to go
  again.
\item There was \unbf{such} a (large) crowd \unbf{that} we couldn't see a thing.
\end{itemize}

正式的结构so/such \ldots{} as+to不定式,有时替代so/such \ldots{} that从句:
\begin{itemize}
\item  We went early \unbf{so as to} get good seats.

\item I'm not \unbf{so} stupid \unbf{as to} believe that.

\item Would you be \unbf{so} kind \unbf{as to} lock the door when you leave?
\end{itemize}


\part{高级句型——简化从句}

\chapter{动词和形容词的补足关系}

动词和形容词的补足语就是接在动词或形容词后面,说明该词所隐含的意义关系的语法结
构。

\section{多词动词}

多词动词有两大类:
\begin{description}
\item[实义动词+小品词] 小品词 (PARTICLE)是一个中性名称,指一些空间副词与介词,
  有的小品词根据语境不同,其词性也不同。\index{概念!小品词 particle} 具体可分
  为:
  \begin{description}
  \item[短语动词 PHRASAL VERB] 小品词是空间副词。例如drink \textbf{up}, find
    \textbf{out}。\index{概念!短语动词 phrasal verb}

  \item[介词动词 PREPOSITIONAL VERB] 小品词是介词,例如 dispose \textbf{of},
    cope \textbf{with}。\index{概念!介词动词 prepositional verb}

  \item[短语--介词动词 PHRASAL--PREPOSITIONAL] 动词后接两个小品词,前为副词,后
    为介词。\index{概念!短语--介词动词 phrasal--prepositional}例如put
    \textbf{up} \textbf{with} \ldots{}
  \end{description}

\item[实义动词后不接小品词] 例如:cut short, put paid to。

\end{description}

在多词动词和自由结合之间并没有明确的界限。

小品词如下:
\begin{description}
\item[介词] against, among, as, at, beside, for, from, into, like, of, onto,
upon, with, etc.

\item[介词副词] about, above, across, after, along, around, by, down, in, off,
on, out (AmE),over, past, round, through, under, up, etc.

\item[空间副词] aback, ahead, apart, aside, astray, away, back, forward(s),
home, in front, on top, out (BrE), together, etc.

\end{description}

多次动词在语义上是一个整体,这常常表现在它\textbf{可用一个动词来替代}。

介词和空间副词之间最明显的差别在于:介词要求后面跟有一个名词短语作为补足语,而
副词则不要求这样。上述小品词只有\textbf{介词副词}类可作介词也可作副词:


\subsection{不及物短语动词}

不及物短语动词,一个动词接一个副词小品词。
\begin{itemize}
\item The plane has just \unbf{touched down}.
\item He is \unbf{playing around}.
\item I hope you'll \unbf{get by}.
\item How are you \unbf{getting on}?
\item Did he \unbf{catch on}?
\item The prisoner finally \unbf{broke down}.
\item She \unbf{turned up} unexpectedly.
\item When will they \unbf{give in}?
\item The tank \unbf{blew up}.
\item The two girls have \unbf{fallen out}.

\end{itemize}

短语动词和自由组合的差异:
\begin{itemize}
\item 诸如give in(投降)和(blow up)爆炸这样的短语动词,我们无法孤立地根据动
  词和小品词的意思来预测组合后习语的意思。但是在自由组合中(如: walk
  past),我们就可以做出预测。

\item 自由组合可替代可拆分。walk past 中的walk,我们可以用run, trot, swim,
  fly 等来替代;至于past,我们可用by, in, through, over 等来替代。


\item 通常不及物短语动词为固定搭配,动词和小品词之间不能插入其他内容且顺序固
  定;但在自由组合中就可以,如\textbf{go} straight \textbf{on}, 另外\textbf{自由组合
  中还可以副词前置},如\textbf{out} \textbf{came} the sun, \textbf{Up} you
  \textbf{come}等。

\end{itemize}

\subsection{及物短语动词}

很多短语动词可以带有直接宾语,因此是及物的:
\begin{itemize}
\item We will \unbf{set up} a new unit.
\item Shall I \unbf{put away} the dishes?
\item \unbf{Find out} if they are coming.
\item She's \unbf{bring up} two children.
\item Someone \unbf{turned on} the light.
\item They have \unbf{called off} the strike.
\item He can't \unbf{live down} his past.
\item I can't \unbf{make out} what he means.
\item She \unbf{looked up} her friends.
\item They may have \unbf{blown up} the bridge.
\end{itemize}

和同一种形式的自由组合一样,\textbf{及物短语动词的小品词既可以在直接宾语之前,
  也可以在其后面}:
\begin{itemize}
\item They \unbf{turned on} the light.
\item They \unbf{turned} the light \unbf{on}.
\item She \unbf{looked} her friends \unbf{up}.
\end{itemize}

但是,当\textbf{宾语是人称代词时,小品词必须位于宾语之后}:
\begin{itemize}
\item They \unbf{turned} it \unbf{on}.
\end{itemize}当宾语较长,或有意要使宾语成为末端的中心,小品词就往往放在宾语之前。

在惯用夸张语表达中,小品词只能放在最后:
\begin{itemize}
\item I was \unbf{crying} my eyes \unbf{out}.
\item I was \unbf{laughing} my head \unbf{off}.
\item I was \unbf{sobbing} my heart \unbf{out}.
\end{itemize}

\subsection{第一类介词动词}

第一类介词动词由实义动词后接介词构成,两者在语义上或句法上相关联。接在介词后面的
名词短语是\textbf{介词宾语},这个术语表示与直接宾语相区别。
\begin{itemize}
\item \unbf{Look at} these pictures.

\item I don't \unbf{care for} Jane's parties.

\item We must \unbf{go into} the problem.

\end{itemize}


\textbf{介词动词}也可以有\textbf{被动态};也可以轻松地在实义动词和介词之间
\textbf{插入一个副词}:
\begin{itemize}
\item This matter will have to \unbf{be dealt with} immediately.
\item The picture \unbf{was looked at} disdainfully by many people.
\item Many people \unbf{looked} disdainfully \unbf{at} the picture.
\end{itemize}

就\textbf{介词宾语发问}的wh- 疑问句是由\textbf{代词} who(m) 和what(用于直接宾语)引导,而不是\textbf{疑问副词}。

\subsection{第二类介词动词}

第二类介词动词是双宾语动词。也就是说,它们后面\textbf{接两个名词短语,通常由
  介词分开:后者为介词宾语},例如:

\begin{itemize}
\item They \unbf{plied} the young man \unbf{with} food.
\item Please \unbf{confine} your remarks \unbf{to} the matter under discussion.
\item This clothing will \unbf{protect} you \unbf{from} the worst weather.
\item Jenny \unbf{thanked} us \unbf{for} the present.
\item May I \unbf{remind} you \unbf{of} our agreement? They have \unbf{provided}
the child \unbf{with} a good education.
\end{itemize}

直接宾语在对应的被动态从句中变为主语:
\begin{itemize}
\item The gang \unbf{robbed} her \unbf{of} her necklace.
\item She was \unbf{robbed of} her necklace (by the gang).
\end{itemize}

\subsection{短语--介词短语}

短语--介词动词除实义动词外,还包含作小品词的副词和介词。他们多是非正式文体。

第一类短语--介词动词只包含一个介词宾语:
\begin{itemize}
\item We are all \unbf{looking forward to} your party on Saturday.

\item He had to \unbf{put up with} a lot of teasing at school. [忍受,容忍,包容]

\item Why don't you \unbf{look in on} Mrs. Johnson on your way back? [(短暂)探
访]

\item He things he can \unbf{get away with} everything.
\end{itemize}


第二类短语-介词动词是双宾语动词。他们需要两个宾语,第二个宾语是介词宾语(往往被
视为\textbf{受事}参与者):
\begin{itemize}
\item Don't \unbf{take} it \unbf{out on} me! [向…发泄;拿…撒气]

\item We \unbf{put our} success down \unbf{to} hard work. [to consider that sth.
is caused by sth. 把…归因于]

\item I'll \unbf{let} you \unbf{in on} a secret. [to allow sb. to share a secret
告知,透露(秘密)]
\end{itemize}

\section{动词补足语}

\subsection{不及物动词}

不及物动词除一般不及物用法以外,还有:
\begin{description}
\item[也可作及物而意思不变的动词] 可将其当作有一个“显明的被省略的宾语”。
  \begin{itemize}
  \item He \unbf{smokes} (a cigarette).

  \item I am \unbf{reading} (a book).

  \item He \unbf{drinks} (alcohol) heavily.

  \item Knock before you \unbf{enter} (the room).
  \end{itemize}

  此外还有drive, enter, help, pass, play, win, write。

\item[也可作及物但主被动变换] 不及物用法以受事参与者为主语;及物用法以施事者为主
语。
  \begin{itemize}
  \item The door \unbf{opened} slowly. 比较:Mary \unbf{opened} the door.

  \item The car \unbf{stopped}. 比较:He \unbf{stopped} the car.

  \item The door \unbf{closed} behind him: You can \unbf{close} the door
easily---it just \unbf{pulls}. [you just \unbf{pull} it"]
\end{itemize}

此外还有 begin, change, drop, increase, move, turn, unite, walk, work 等词。

也有些不及物动词变及物动词时,有使宾语被动的意思:
\begin{itemize}
\item \unbf{run} the water [cause the water to \unbf{run}]

\item \unbf{slide} the drawer shut [\unbf{slide} back the drawer 谓语+状语] 关
  上抽屉
\end{itemize}

\item[作不及物用时有互相参与意义] 如:
  \begin{itemize}
  \item We have \unbf{met}. 比较:I have \unbf{met} you.

  \item The bus and car \unbf{collided}. 比较:The bus \unbf{collided} with the
    car.(也是不及物)
  \end{itemize}
\end{description}


\subsection{动词补足关系的分类}

\begin{table}[p] \centering \small

  \begin{talltblr}[
    caption = {动词补足关系的类型},
    label = {tab:verbcop},
    note{a} = {$C_s$ 主语补语,$O_i$ indirect objects间接宾语,$O_d$ direct objects 直接宾语,
      $+S$ 含主语,$-S$ 不含主语,},
    ]{width=\linewidth,
      colspec = {ll},
      rowsep = 1pt, colsep = 2pt,
      row{1} = {font=\bfseries},
    }
    \toprule
    变体 & 例句 \\ \midrule
\textbf{连系动词(SVC和SVA)} & \\
形容词性 $C_s$ & The girl seemed restless. \\
名词性 $C_s$ & William is my friend. \\
状语补足语 & The kitchen is downstairs. \\ \midrule
\textbf{单宾语及物动词(SV$O$)} & \\
 {名词短语作O \\
 (有被动式)} & Tom caught the ball. \\
 {名词短语作O \\
 (无被动式)} & Paul lacks confidence. \\
 that- 从句作O & I think that we have met. \\
 wh- 从句作O & Can you guess what she said? \\
 wh- 不定式 (-S) 作O & I learned how to look after the cats. \\
 to- 不定式 (-S) 作O & We've decided to move house. \\
 -ing从句 (-S) 作O & She enjoys playing table tennis. \\
 to- 不定式 (+S) 作O & They want us to help. \\
 -ing从句 (+S) 作O & I hate the children picking a fight. \\ \midrule
 \textbf{复合及物动询 (SVOC和SVOA)} & \\
形容词性 $C_o$ & That music drives me mad. \\
名词性 $C_o$ & They named the ship ``Zeus''. \\
 O + 状语 & I left the key at home. \\
 O + to- 不定式 & They knew him to be a spy. \\
 O + 不带 to 不定式 & I saw her leave the room. \\
 O + -ing 从句 & I heard someone shouting. \\
 O + -ed 从句 & I get the watch repaired. \\ \midrule
 \textbf{双宾语及物动词 (SVOO)} & \\
名词短语作 $O_i$ 和 $O_d$ & Tom give me some food. \\
介词短语作 $O$ & Please say something to us. \\
$O_i$ + that- 从句 & They told me that I was ill. \\
 $O_i$ + wh- 从句 & He asked me what time it was. \\
 $O_i$ + wh- 不定式从句 & Mary showed us what to do. \\
$O_i$ + to- 不定式 & I advised Mark to see a doctor. \\
 \bottomrule
\end{talltblr}%
\end{table}


\subsection{系词补足关系}

seem, appear, look, sound, feel, smell, taste 等``seeming'' 感官系动词在下列
这类句子中用由 as if/though(似乎,好像) 开头的状语从句来补足。
\begin{itemize}
\item Jill \unbf{looked as if} she had seen a ghost.

\item It \unbf{seems as if} the weather is improving.
\end{itemize}

\subsection{单宾语及物补足关系}

在cost ten dollars; weight 20 kilos 之类的度量用语中可见到VO类型,但有同样理由将
其分析为 V + A,其中A为必要附加状语。因为除了用what外,还可以用how much 问句:
\begin{itemize}
\item How much / What does it cost/weight ?
\end{itemize}

宾语为that- 从句的句子变被动式,宾语变主语时,that 不能省略(见
\cref{subsubsec:thatclause}):
\begin{itemize}
\item Everybody hoped \unbf{(that)} she would sing.
\item \unbf{That} she would sing was hoped by everybody.
\end{itemize}

\begin{table}[htbp]
  \centering \small
  \begin{talltblr}[ caption = {作宾语的非限定性从句},
    label = {tab:obin},
    ]{
      width=\linewidth, colspec={lXX},
      rowsep=2pt, colsep=4pt,
      row{1} = {c, font=\bfseries},
    }
    \toprule
    & 不带主语 & 带主语 \\ \midrule
    to- 不定式 & Jack hates \emph{to miss the train}. & Jack hates \emph{her to
    miss the train}. \\
  -ing 从句 & Jack hates \emph{missing the train}. & Jack hates \emph{her missing
  the train}.\\
    \bottomrule
  \end{talltblr}%
\end{table}

SVO结构中,不带主语的不定式从句、-ing 从句\textbf{被省略的主语往往和母句的主语相同}。
\begin{itemize}
\item I love \unbf{listening to music}.
\end{itemize}也有例外:
\begin{description}
\item[被省略的主语独立且显明] 分词主语不确定,并且独立于前面母句主语。

  \begin{itemize}
  \item He recommended \unbf{introducing a wealth tax}.

    负责征收财产税的人是政府机关,而不是母句主语“他”。
  \end{itemize}
\end{description}

SVO中,\textbf{带主语的不定式从句}可以作补足语,但这一组中的动词为数极少,主要表
示(不)喜欢或(不)想要,如 desire, hate, like, love, prefer, want and wish:
\begin{itemize}
  \item They don't like \unbf{the house to be left empty}.
\end{itemize}在这些动词之后,不定式之前的名词短语不能转变为被动式中的主语。
\begin{itemize}
  \item \sout{The house isn't liked to be left empty (by them).}
\end{itemize}


SVO中,\textbf{带主语的 -ing 从句} 可以作补足语。\textbf{人称主语可用属格形式},
但常常使人感到别扭或不自然。
\begin{itemize}
  \item I dislike \unbf{him/his} driving my car.

  \item We look forward to \unbf{you/your} becoming our neighbour.
\end{itemize}

\textbf{that- 从句补语}:
\begin{itemize}
\item \unbf{It} seems \unbf{(that) you are mistaken.}
\item \unbf{It} appears \unbf{(that) you have lost your temper.}
\end{itemize}
以上两例句中 that- 从句不是动词的宾语而是\textbf{外置主语}。\index{概念!外置
  主语}
\begin{description}
\item[外置主语 (EXTRAPOSITION)] \index{概念!外置主语 extraposition} 句子
  中\textbf{通过形式主语 “it” 将真正的主语移到句末}的现象。这种结构通常用于
  使句子更流畅或避免过长的主语,使得句子的\textbf{重心更加突出}。外置主语常见
  于\textbf{名词性从句和不定式},除上面that- 从句外,还有:
  \begin{itemize}
  \item \unbf{It} is unclear \unbf{why} she told him.

  \item Would it be better \unbf{to pay now}?
  \end{itemize}
\end{description}
常用于这种类型的动词有: seem, appear, happen 和动词短语come about
[happen]和 turn out [transpire]。

\subsection{复合及物 (SVOC 和 SVOA) 的补足关系}

复合及物补足关系的一个明显特征是:\textbf{动词后面的两个成分(OC或OA)在意义
  上分别等同于一个名词性从句的主语和谓体。}
\begin{description}
\item[单宾语及物] She presumed \unbf{that her father was dead}.
\item[复合及物] She presumed \unbf{her father (to be) dead}.
\end{description}

\textbf{介词as 表示连系关系,特别是说明与直接宾语有关的角色或地位。}
\begin{itemize}
\item We considered him $ \left\{
    \begin{aligned}
      &\text{a genius} \\
      &\text{as a genius 补语} \\
      &\text{to be a genius}
    \end{aligned}
  \right. $
\end{itemize}
但在某些方面,介词as和引导比较从句的连词as相似,\textbf{一方面引导从句;另一方面又引
导和从句同位的名词短语}:
\begin{itemize}
\item Report me \unbf{as I am --- a superman}.

\item He described her \unbf{as he found her, a liar}.
\end{itemize}


\begin{table}[htbp]
  \centering
  \begin{talltblr}[ caption = {SVOC 中的非限定性从句},
    label = {tab:svocin},
    ]{
      width=\linewidth, colspec={ll},
      rowsep=2pt, colsep=4pt,
      row{1} = {c, font=\bfseries},
    }
    \toprule
    非限定性从句 & 例句\\ \midrule
    to- 不定式  & They knew him \emph{to be a spy}. \\
    不带to不定式 & I heard someone \emph{slam the door}. \\
    -ing 从句 & I caught Ann \emph{reading my diary}. \\
    -ed 从句 & We saw him \emph{beaten by the German in final}. \\
    \bottomrule
  \end{talltblr}%
\end{table}

\cref{tab:svocin} 中作为\textbf{宾语补语的的非限定性从句}(斜体表示)自身没有
主语,但\textbf{其隐含的主语总是前面的宾语},这样的宾语被叫做\textbf{上升宾
  语 (RAISED OBJECT)}\index{概念!上升宾语 raised object}。语义上,上升宾语是
\textbf{非限定型动词的主语};句法上,它从非限定性从句中上升出来作\textbf{母句的宾语}。
要\textbf{注意歧义},如:
\begin{itemize}
\item Tom left her $\left\{
    \begin{aligned}
      &\text{to finish the job.} \\
      &\text{finishing the job.} \\
    \end{aligned}
    \right.$

    Tom离开她,由她去完成工作。

    \textbf{her是上升宾语。}
\end{itemize}


\subsection{双宾语及物 (SVOO) 补足关系}

不同于SVOC中宾语与宾语补语的连系关系;SVOO中两个宾语之间没有连系关系。

\paragraph{宾语和介词宾语}

介词短语可做宾语,大体有以下句型:
\begin{table}[htbp]
  \centering \small
  \begin{talltblr}[ caption = {宾语和介词宾语},
    label = {tab:PrepObj},
    ]{
      width=\linewidth, colspec={},
      rowsep=2pt, colsep=4pt,
      row{1} = {font=\bfseries},
    }
    \toprule
    动词 & 双宾语 & 例句 \\ \midrule
   \SetCell[r=3]{l} told & $O_i + O_d$ & Mary told only John the secret. \\
   & $O_d + O_p$ & Mary told the secret only to John. \\
   & $O_i + O_p$ & Mary told only John about the secret. \\ \midrule
   \SetCell[r=2]{l} offer& $O_i + O_d$ & John offered Mary some help. \\
   & $O_d + O_p$ & John offered some help to Mary. \\ \midrule
   \SetCell[r=2]{l} envy & $O_i + O_d$ & She envied John his success. \\
   & $O_i + O_p$ & She envied John for his success. \\ \midrule
   wish & $O_i + O_d$ & They wished him good luck. \\ \midrule
   \SetCell[r=2]{l} blame & $O_d + O_p$ & He blamed the divorce on John. \\
   & $O_i + O_p$ & He blamed John for the divorce. \\ \midrule
   say &  $O_d + O_p$ & Why didn't anybody say this to me? \\ \midrule
   warn &  $O_i + O_p$ & Mary warned John of the dangers. \\
    \bottomrule
  \end{talltblr}%
\end{table}

\section{形容词的补足关系}

\textbf{名词不能做形容词补足语。}

和介词动词一样,形容词经常和后面的介词构成词汇单位:good at, fond of,
opposed to, angry with/about等等。


\begin{table}[htbp]
  \centering \small
  \begin{talltblr}[ caption = {形容词补足语类型},
    label = {tab:adjin},
    ]{
      width=\linewidth, colspec={lX},
      rowsep=2pt, colsep=4pt,
      row{1} = {c, font=\bfseries},
    }
    \toprule
    形容词补足语类型  & 例句   \\ \midrule
    介词短语 & She felt angry \emph{with herself}. \\
    that- 从句 & I am surprised \emph{(that) you didn't call the doctor
      before}.  \\
    wh- 从句 & It was unclear \emph{what they would do}. \\
    to- 不定式 & Bob is sorry \emph{to hear it}. \\
    -ing 从句 & I'm busy (with) \emph{getting the house redecorated}. \\
    \bottomrule
  \end{talltblr}%
\end{table}


\chapter{名词短语}

名词短语也可以非常复杂,因为句子本身可以被改写,以适用于名词短语结构。例如:
\begin{itemize}
\item The girl is Mary Smith.

\item The girl is tall.

\item The girl was standing in the corner.

\item You waved to the girl when you entered.

\item The girl became angry because you knocked over her glass.
\end{itemize}
以上句子可以组合为一个由很长的名词短语作主语的\textbf{简单句}:
\begin{itemize}

\item \emph{The tall girl standing in the corner who became angry because
    you knocked over her glass after you waved to her when you entered} is
  Mary Smith.
\end{itemize}

\section{名词短语的构成部分}
\label{subsec:nounimal}

\begin{description}
\item[中心成分(HEAD)] \index{概念!中心成分@中心成分 Head}被其他成分群集于周围,
  使之构成一致关系的成分,通常是一个名词或代词。中心成分确定了名词短语的句法
  角色(如主语、宾语等)以及与其他句子成分的关系。 如以上名词长句中的girl。

\item[限定成分 (DETERMINATIVE)] 它包括:前位、中心、后位限定词,见 \cref{tab:determ}。

\item[前置修饰 (PREMODIFICATION)] 位于中心词前,除限定词以外的所有成分,一般
  是形容词(短语)和作形容词用的名词,也称“\textbf{定语}”。如:
  \begin{itemize}
  \item some \textbf{expensive} equipment
  \item some \textbf{very very expensive office} equipment
  \end{itemize}

\item[后置修饰 (POSTMODIFICATION)] 位于中心词后的所有词项,有介词短语、非限定
  从句和关系从句和补足语。起后置修饰作用的从句也被称为\textbf{定语分
    句}。\index{概念!后置修饰@后置修饰 post-modification} \index{概念!定语分
    句@定语从句attributive clause}
  \begin{description}
  \item[介词短语] the car \emph{outside the station}
  \item[非限定性从句] the car \emph{standing outside the station}
  \item[关系从句] the car \emph{that stood outside the station}
  \item[补足语] a bigger car \emph{than that}
  \end{description}

\end{description}

whose是物主关系代词,像his,her, its, there 一样用在名词前作限定词。后置修饰
语(关系从句)中,不管是人或物都可以用whose。有人认为用whose指代物品不合适,
可以用of which 或that \ldots{} of替代:
\begin{itemize}
\item I saw a girl \unbf{whose beauty} took my breath away.

  我见到一个女孩,她的美貌让我十分惊异。(主语)

\item It was a meeting \unbf{whose purpose} I did not understand.

  这个会议的目的我不明白。(宾语)


\item I went to see my friends, the Jims, \unbf{whose children} I used to
  look after when they were small.

  我去看我的朋友,吉姆一家,他家孩子很小时曾由我照料过。


\item He's written a book \unbf{whose name}/\unbf{the name of which} I've forgotten.
\item He's written a book \unbf{that} I've forgotten \unbf{the name of}.
\item He's written a book \unbf{of which} I've forgotten \unbf{the name}.
\end{itemize}


\section{限制性和非限制性修饰语}
\label{sec:infmodifi}

修饰语可以是\index{概念!限制性修饰语}\index{概念!非限制性修饰语}
\begin{description}
\item[限制性修饰语] 限制性修饰语用来\textbf{限定或明确其修饰的名词或代词},提
  供关键性的信息。如果去掉限制性修饰语,句子的意思就不完整,或者会导致误解。
  它\textbf{不需要用逗号隔开},因为这些信息对理解句子至关重要。在阅读时,限
  制性修饰语通常紧跟在其所修饰的名词(先行词)之后,\textbf{没有间隔停顿}。
  \begin{itemize}

  \item Come and see my \unbf{younger} daughter.

    younger暗示是说话者两个女儿中的小女儿,是确定的,是限制性修饰语,也被称
    为“\textbf{定语}”。

  \item The students \unbf{who study hard} will pass the exam.

  \item I need the book \unbf{that you borrowed from the library}.

    上两句起限制性修饰语作用的从句也被叫做\textbf{限制性关系从句}。

  \end{itemize}


\item[非限制性修饰语] 非限制性修饰语提供的是\textbf{附加的、非关键性的信息}。
  即使去掉这些修饰语,句子的基本意思仍然成立。通常\textbf{用逗号隔开},因为信
  息不重要。
  \begin{itemize}
  \item My brother\unbf{, who lives in New York,} is coming to visit.
  \end{itemize}

  这句是\textbf{非限制性关系从句}。

\end{description}

\textbf{当名词短语中心语确定无疑义时(如专有名词、人名、隐含或明示的特指),
  其任何修饰语都将是非限制性的。}
\begin{itemize}
\item Mary Smith\unbf{, who is in the corner,} wants to meet you.
\item Come and meet my \unbf{beautiful} wife. [一夫一妻制度下]


\item He \unbf{likes dogs}, \unbf{which surprises me}.

  如上,\textbf{非名词性先行词不可能带有非限制性修饰语}:


\item \sout{I won't see \unbf{anyone}, \unbf{who has not made an
      appointment}.} [误,可以去修饰语前后的逗号,转限制性从句]

  \textbf{非断定的中心词}如any- 等\textbf{不能有非限制性关系从句},因
  为:any- 等非断定代词本就无法限定,再施加非限制性关系从句毫无意义。
  像 any, all 和 every 等非特指的限定词通常只有限制性修饰语。

  \begin{itemize}
  \item \sout{All the students\unbf{, who had failed the test,} wanted to
      try again.} [误,应去修饰语前后的逗号,转为限制性从句]

  \end{itemize}

  偶尔这样的句子里也可以使用(非)限制性修饰语。如下:
  \begin{itemize}
  \item All the students, \unbf{who had returned from their vacation}, wanted
    to take the exam.
    所有学生都是度假归来的,“度假归来”只是附加信息。
  \end{itemize}

\end{itemize}

\section{限制性从句作后置修饰语}

后置修饰的限制性从句可包括:
\begin{description}
\item[关系从句] that 和wh- 代词可以互相替代;that 本身就是\textbf{从句成分},
  如下例句分别作为主语和宾语:
  \begin{itemize}
  \item The news \unbf{that appeared in the papers this morning} was well received.
  \item The news \unbf{which we saw in the papers this morning} was well received.
  \end{itemize}

\item[同位从句] 同位语从句(Appositive clause)是一种名词性从句,它在复合句
  中充当同位语,通常用于进一步解释或说明前面的名词。

  同位从句中连接词\textbf{只能是that},不能用wh-代词替代,\textbf{that也不构
    成从句成分}。名词短语中心词应是一个\textbf{概括性抽象名词},例如fact,
  idea, proposition, reply, remark, answer 等等。
  \begin{itemize}
  \item \unbf{The news that the team had won} calls for a celebration.

  \item I agree with \unbf{the old saying that absence makes the heart grow fonder}.

    absence makes the heart grow fonder 习语,不相见,倍思念。
  \end{itemize}
  因为是同位关系,所以用be动词把它和被同位的单位连接起来后仍语法正确。
  \begin{itemize}
  \item The news \unbf{IS} that the team had won.

  \item The old saying \unbf{IS} that absence makes the heart grow fonder.
  \end{itemize}
\end{description}

\subsection{限制性的关系从句作后置修饰语}

关系代词可以有如下能力:
\begin{description}
\item[表明与它的先行词的一致关系] 先行词就是名词短语中的先行部分,关系从句是这
  个部分的后置修饰语。 [外在关系]
\item[表明它在关系从句中的作用] 或是作为从句结构中的一个成分 (S, O, C, A), 或
  是作为关系从句中一个成分的构成要素。 [内在关系]

\end{description}

\textbf{在非限制性的关系从句中,关系代词普遍是 wh­ 系列 (who, whom, which, whose)。}

\textbf{在限制性的关系从句中,除 wh- 代词外也常用 that 或零关系代词。}

\textbf{代词that 没有宾格(who/whom);也没有属格 (who 和 which 的 whose),不
  能在关系从句中充当个别成分的构成因素(详见\cref{tab:relaxpro})。}


\begin{table}[htbp!]
  \centering \small
  \begin{talltblr}[ caption = {关系代词可以在关系从句中充当的成分},
    label = {tab:relaxpro},
    ]{
      width=\linewidth, colspec={lXl},
      rowsep=2pt, colsep=4pt,
      row{1} = {font=\bfseries},
    }
    \toprule
    成分 & 例句 & 备注 \\ \midrule
    S & They are delighted with the person \emph{who/that} has been appointed. & 不可省略 \\
    O & They are delighted with the person \emph{who(m)/that} we have
    appointed. & 可省略 \\
    C & She is the perfect accountant \emph{which}/\sout{who/that} her
    predecessor was not. & 争议。 \\
    A &  He is the policeman \emph{at whom} the thief shot. & 介词后whom \\
    A & He is the policeman \emph{who(m)/that} the thief shot \emph{at}.   & 可省略 \\
    A & She arrived the day \emph{on which} I was ill. & 介词后不接that \\
    A & She arrived the day \emph{which/that} I was ill \emph{(on)}. &  \\
    \bottomrule
  \end{talltblr}%
\end{table}

就that/which/who 谁能做\textbf{关系从句中的补语},尚存不少争议。夸克认
为\textbf{当关系代词在关系从句中充当非介词的补足语时,人称或非人称先行词后都
  只能用which.} 我不知道其他专家的想法。对此暂不展开。

如果先行词是人称名词,关系代词可以表明 who 与 whom 的区别,这取决于关系代词
在关系从句中充当主语还是宾语,抑或介词补足语:
\begin{itemize}
\item The person $\left\{
    \begin{aligned}
      & \text{\emph{who} spoke to him \ldots} &\text{[主语]}  \\
      & \text{\unbf{to whom} he spoke \ldots} &\text{[介词补足语,过于正式]}  \\
      & \text{\emph{who(m)} he spoke \emph{to} \ldots} &\text{[介词补足语]} \\
      & \text{\emph{who(m)} he met \ldots} &\text{[宾语]} \\
    \end{aligned}
  \right. $
\end{itemize}


\textbf{夸克认为存在句和分裂句中可加可不加的that和wh- 从句并不是形容性关系从
  句。}\todo[inline]{补全存在句和分裂句意思}
\begin{itemize}
\item There's a table (that/which) stands in the corner. [存在句]

\item It was Simon (that/who) did it. [分裂句]
\end{itemize}

当先行词有一个\textbf{最高级形容词或后置限定词之一} (first, last, next,
only) 修饰时,充当关系从句\textbf{主语}的关系代词通常是 \textbf{that},充当宾
语的关系代词用 \textbf{that 或零形式}多于用 which 或 who(m):
\begin{itemize}
\item She must be one of the most remarkable women \unbf{that ever lived}.

  that 充当关系代词,引入关系从句,其先行词是“women”。
\end{itemize}

\textbf{英语中并不存在接在先行名词之后、与 where, when, why 平行并表示方式的
  关系词 how:}
\begin{itemize}

\item That's \unbf{the way} \sout{how/}\unbf{that} she spoke.

\item That's \unbf{how} she spoke. 正确,名词性从句。
\end{itemize}

在表示\textbf{时间}时,不管关系代词是 \textbf{that 还是零形式},通
常\textbf{省略介词}:
\begin{itemize}
\item What's the time \unbf{that} she normally arrives \unbf{(at)}?
\item What's the time \unbf{when} she normally arrives?
\end{itemize}

可是,当代词为 \textbf{which} (使用频率较低,更加正式)时,在下列三个例子中
都\textbf{必须用介词},而且通常\textbf{介词放在代词之前}:
\begin{itemize}
\item 5:30 is \unbf{the time at which} she usually arrives. [at which =
  when]
\item I don't remember \unbf{the day on which} she left.

\item He worked for three years \unbf{during which} he lived there.
\end{itemize}

表示方法和理由时,通常用零结构(偶尔用that),\textbf{不用介词}。
\begin{itemize}
\item That's the way \unbf{(that)} he did it.

  That's \unbf{how} he did it. [不含 the way]


\item Is that \unbf{the reason}/\unbf{why} they came?
\end{itemize}

然而,在类似 reason 和 way 表示状语关系意义的其他名词之后,\textbf{如用that 或
  零关系词就必须接介词}。
\begin{itemize}
\item This is \unbf{the style} he wrote it \unbf{in}.

\item Is this \unbf{the cause of} her coming?
\item Is this \unbf{motive of} her coming?
\end{itemize}



\subsection{非限定性从句作后置修饰语}

三种非限定性从句都可以充当名词短语的后置修伤语: -ing 分词从句, -ed 分词从句
和不定式从句。

\subsubsection{-ing 分词从句作后置修饰语}


\textbf{-ing 从句和关系从句的对应关系只限于以关系代词为主语的那些关系从句}:
\begin{itemize}
\item The person \emph{who}$\left\{
    \begin{aligned}
      &\text{\emph{will write}} \\
      &\text{\emph{will be writing}} \\
      &\text{\emph{writes}} \\
      &\text{\emph{is writing}} \\
      &\text{\emph{wrote}} \\
      &\text{\emph{was writing}} \\
    \end{aligned}
  \right\} $ \emph{reports} is my colleague.

\item The person \unbf{writing reports} is my colleague.
\end{itemize}

必须强调的是,后置修饰从句中的 -ing 形式不应该看作是关系从句中进行体的缩略形
式。例如,静态动词在限定性动词短语中不可能用进行式,但可以用于分词形式中。
\begin{itemize}
\item It was a mixture \unbf{consisting} of oil and vinegar.

  不能说是that was consisting(静态动词)的缩写。
\end{itemize}

其实我认为可以这样理解,现在分词从句可以是关系从句中进行体的缩略形式,也可以
是关系从句中(不带连系动词)一般式的缩略形式。


\subsubsection{ -ed 分词从句作后置修饰语}

和 -ing 从句一样, -ed 分词从句仅与\textbf{关系代词作主语}的关系从句相对应:
\begin{itemize}
\item The car \emph{that}$\left\{
    \begin{aligned}
      &\text{\emph{will be repaired}} \\
      &\text{\emph{is (being) repaired}} \\
      &\text{\emph{was (being) repaired}} \\
    \end{aligned}
  \right\} $ by that mechanic.

\item The car \unbf{(being) repaired}  by that mechanic.
\end{itemize}

\subsubsection{不定式从句作后置修饰语}

与 -ing 和 -ed从句不同,不定式从句所对应关系从句中的关系代词除了可以
是\textbf{从句主语}外,还可以是\textbf{宾语}或\textbf{状语},有限范围内还可以
是\textbf{补语}。
\begin{description}
\item[主语] The man \unct{to help you}{who can help you} is Mr Johnson.

\item[宾语] The man \unct{(for you) to see}{who(m) you should see} is Mr Johnson.

\item[补语] 不懂,略。\improve[inline]{大全17.30 看不懂,需补充}

\item[状语] The time \unct{(for you) to go}{when (you) should go} is July.

\item[状语] The place \unct{(for you) to stay}{where (you) should stay} is
  this room.  [where = at which,另从句主语可省]
\end{description}

因省略了连词、主语、be助动词等,-ing从句, -ed从句,不定式从句都存模糊性。

由不定式从句作同位后置修饰语\textbf{可能没有对应的同位限定性从句},而只有一个
介词短语和它相对应:
\begin{itemize}
\item He lost \unbf{the ability} \unbf{to use his hands}. [of using 可代替 to use]

  He lost the ability \sout{that he could use his hands.}

\item \unbf{Any attempt} \unbf{to leave early} is against regulations. [at
  leaving 可替代 to leave]

  Any attempt \sout{that one should leave early} is against regulations.
\end{itemize}

\todo[inline]{何时用to不定式,何时用介词+ing,夸克说的较乱,我理解困难,日后整理。}

\subsection{用介词短语作后置修饰语}

我们还可以用介词短语作后置修饰语,如下最后一句例句:
\begin{itemize}
\item The car \emph{was standing} outside the station.
\item the car \emph{which was standing} outside the station
\item the car \emph{standing} outside the station
\item the car \unbf{outside the station}
\end{itemize}

在英语中,介词短语肯定是后置修饰语中最常见的类型.

\begin{itemize}
\item the road \unbf{to Lincoln}
\item this book \unbf{on grammar}
\item a man \unbf{from the electricity company}
\item action \unbf{in case of fire}
\item the meaning \unbf{of this sentence}
\item the house \unbf{beyond the church}
\item two years \unbf{before the war}
\item some trees \unbf{along the river bank}
\item the university \unbf{as a political forum}
\end{itemize}

\section{名词化}

我们应该把介词短语作后置修饰语看作是其作状语的特殊例子。

\begin{description}
\item[名词化 (nominalization)] 在语法上将其他词性(如动词或形容词)派生为名词
  形式,使原本表达动作或状态的词汇变成了表示事物、概念或状态的名词,从而使句
  子的结构和信息传达方式发生变化。


  例句:
  \begin{taskitem}(2)
    * his \unbf{refusal} to help [n.]
    * he \unbf{refuses} to help [v.]
    * the \unbf{truth} of her statement [n.]
    * her statement is \unbf{true}  [v.]
    * her \unbf{friendship} for Jim [抽象名词]
    * she is a \unbf{friend} of Jim [具体名词]
  \end{taskitem}
\end{description}

在英语中,动词通常通过添加后缀如 ``-ment'' , ``-tion'', ``-ure'' 等来进行名词
化(详见\cref{tab:mainsuffix})。例如:
\begin{taskitem}(3)
* entertain → entertainment
* act → action
* fail → failure
\end{taskitem}
此外,一些动词和形容词可以直接作为名词使用,而不需要添加任何后缀,例
如``change''、``increase'' 和``use''。

\section{后置修饰的次要类型}

\subsection{副词短语作后置修饰}

略



\chapter{主位 (Theme)、中心 (Focus) 和信息处理 (Information processing)}

本章介绍的内容,如前置、倒装、分裂句等,都是为了调整句中某部分信息的权重。

\section{前置 (Fronting)}

\index{概念!前置 Fronting}
\begin{description}
\item[前置] 把一个非主语词项从它通常所在位置移到句首,使其成为中心内容——即
  使其不是主语。这也叫作主题化 (TOPICALIZATION).
\end{description}

前置成分可以是:
\begin{description}
\item[宾语]
  \begin{itemize}
  \item \unbf{People like that} I just can't stand.
  \item \unbf{This question } we have already discussed many times.

  \item  \unbf{What I'm going to do next} I just don't know.
  \end{itemize}
\item[补语] 较为少见,且现代英语中更少见。
  \begin{itemize}
  \item \unbf{Fool} that I was.
  \end{itemize}

\item[副词和状语]
  \begin{itemize}
  \item \unbf{Here} we go.

  \item \unbf{Once upon a time} there were three little pigs.

    once upon a time, “从前”(用于故事开头),习语。
  \item \unbf{Round the corner} came Mrs Porter.

  \end{itemize}
\item[as 和 though时] 形容词或副词前置
  \begin{itemize}
  \item \unbf{Young as I was}, I realized what was happening.

  \item \unbf{Tired though she was}, she went on working.

  \item \unbf{Much as I respect his work}, I cannot agree with him.

  \item \unbf{Genius as you may be}, you have no right to be rude to your teachers.

  \end{itemize}
\end{description}


\section{倒装 (Inversion)}
\label{subsec:inversion}

\index{概念!倒装 inversion} 一个成分的提前,往往牵涉到语序的倒装,有时涉及细
微的句法调整。
\begin{description}
\item[主语和动词的倒装] SVC 和 SVA 类型的从句大部分有必不可缺的第三成分,因为
  动词自身通常非常缺乏传达动力。如果只是将补语 C或状语A 提前,会把重心末在末短的
  动词上,有可能造成语义不清,如:
  \begin{description}
  \item[SVC] \unbf{The sound of the bell} \unbf{grew} faint.
  \item[CSV] \unbf{Faint} \unbf{the sound of the bell} grew. [不清晰,有可能误解为逐渐变小的铃声
    变大grew (louder)]
  \item[CVS] Faint \unbf{grew} \unbf{the sound of the bell}. [有矫揉造作感。]
  \end{description}
  还有:
  \begin{description}
  \item[SVA] \unbf{The Squidward’s hopes and dreams} \unbf{lies} here.

    章鱼哥的希望和梦想埋葬(躺)在这里。尾重,突出here。

  \item[AVS] Here \unbf{lies} \unbf{the Squidward’s hopes and dreams}. [尾重,
    在埋葬lies这样沉重悲伤的词之后\textbf{突出主语},渲染到位。]

    这里埋葬着\textbf{章鱼哥的希望和梦想}。
  \end{description}
  日常对话有些倒装句:
  \begin{description}
  \item[AVS] Here \unbf{comes} \unbf{my brother}.

  \item[AVS] Down \unbf{came} \unbf{the rain}.
  \item[AVS] Up \unbf{went} \unbf{the flag}.
  \end{description}

  主语和动词的倒装OVS中,前移的宾语O主要表示直接话语,并且主语往往不是人称代词:
  \begin{itemize}
  \item ``Please go away,'' said one child.

    ``And don't come back,'' pleaded another.
  \end{itemize}
  这个例子有点书面化,日常话语中一般用SV替代VS,如 ``one child said'',
  ``another pleaded''.

\item[主语和功能词的倒装]
  \begin{description}
  \item[so, neither, nor]
    \begin{itemize}
    \item John saw the accident and \unbf{so did Mary}. [Mary did so]

    \item He couldn't speak, \unbf{nor could he walk}.
    \item She wasn't angry and \unbf{neither was I}. [I wasn't, either]
    \item She must come and \unbf{so must you}. [you must come too]

    \item 如果想将重心放在动词上,也可只前置 so.

      You asked me to leave, \unbf{so} I did.
    \end{itemize}

    另外要注意到,so, neither, nor以上例句中主要动词均被省略,只有功能助动词。

  \item[否定形式或意义的短语前置] 正式文体中,形式和意义上否定整个从句的成分
    前置,且主语和功能词往往需要倒装。
    \begin{itemize}
    \item This door must be left unlocked at no time.

      \unbf{At no time} must \unbf{this door} be left unlocked.

    \item \unbf{Only in this way} is \unbf{it} possible to explain their actions.

    \item \unbf{Not a single book} had \unbf{he} read that month.

    \item \unbf{Not a word} whould he say.

    \item \unbf{No longer} are they staying with us.


    \item \unbf{N\'OT \`until yesterday} / did he CH\`ANGE his mind.
    \end{itemize}

  \item[主语不是人称代词的比较从句]
    \begin{itemize}

    \item \unbf{I} spend more than \unbf{do my friends}.

    \item \unbf{Oil} costs less than \unbf{would atomic energy}.

      \unbf{Oil} costs less than did \sout{it}. [\textbf{人称代词作主语的 than从句不能倒装}]

    \item She is a doctor, \unbf{as is he}. [\textbf{人称代词作主语
        的 as从句可以倒装}]

    \item They go to concerts frequently, \unbf{as do I}.

    \end{itemize}

  \item[条件和让步从句]
    \begin{itemize}
    \item \unbf{Were} \unbf{she} alive today, she would grieve at the changes.
    \item \unbf{Had} \unbf{I} known, I would have gone to her.
    \item \unbf{Should} \unbf{you} change your plans, please let me know.
    \end{itemize}
  \end{description}
\end{description}

\section{分裂句和假拟分裂句}
\label{subsec:cleftsen}

\index{概念!分裂句 cleft sentence} 分裂句是是将一个句子分裂成两个从
句,\textbf{其中一个从句的某成分为信息焦点,另一从句为处于支配地位、提供背景
  说明的关系从句},借此突出信息焦点的句式。通常是为了强调动作的执行者、动作本
身或是动作发生的时间、地点等。

\begin{description}
\item[it 子句分裂句] It + BE + \textbf{focus} + (that, who, whom,
  which, 零)名词性关系从句,如:
  \begin{itemize}
  \item It is \unbf{Simon} \uline{who’s gone down}.

  \item It was \unbf{here} \uline{that the young girl first fell in love}.


  \item It was \unbf{this matter} \uline{on which I consulted with Dr. Richard}.

    我就此事与理查德博士进行了磋商。
  \end{itemize}


\item[wh- 假拟分裂句] 名词性关系从句 (Wh- + 主语 + 谓语 + 其他成分) + BE + \textbf{focus}

  焦点可以是:
  \begin{description}
  \item[名词短语] \uline{What I saw} was \unbf{one of the most impressive government policies in years}.

    我所看到的是近年来最令人印象深刻的政府政策之一。
  \item[动词短语] \uline{What you do} is \unbf{wear it like that}.

    你要做的是像那样穿衣。

   \uline{What I did} was \unbf{send a complaint to Radio 2}.

   我要做的是像2号电台投诉。[send 动词原形]

 \item[关系从句]
   \uline{What you’ll find} is \unbf{that people who lie down with
     dogs will get up with fleas}, my boy.

   你会发现近墨者黑,我的儿子。(改编自西方谚语,字面意思是跟狗躺在一块儿的人,
   会和跳蚤一起起床。)

   \uline{What I didn’t like} was \unbf{leaving my mum}.

  \end{description}

  wh- 名词性关系从句可以和焦点的位置颠倒:
  \begin{itemize}
  \item \unbf{Send a complaint to Radio 2} is \uline{what I did}.

  \item \unbf{Leaving my mum} was \uline{what I didn’t like}.
  \end{itemize}
\end{description}

动词成分绝对不能充当分裂句的焦点,正如它不能以一个 wh- 成分出现一样。


\section{存在句}
\label{subsec:behave}

在默认结构中,一般默认主位为已知信息。如果期望听话者把主位理解为全新的信
息——与以前介绍过的任何事物没有关系,便利的做法是提供某种\textbf{假位主位},
使听者将整句话理解为全新信息。

除be 外,there也可以和appear, arise, arrive, begin, come, develop,
emerge, enter, escape, exist, live, loom, occur, remain, stand)等联动。另外还有其他假位主位。

\begin{itemize}
\item \unbf{There is} / \unbf{I have} a car blocking my way.

\item \unbf{There are} some people (that) I'd like you to meet.

\item $\left.
    \begin{aligned}
      \text{There are} \\
      \text{We have} \\
      \text{One finds} \\
      \text{It's a fact that} \\
    \end{aligned}
  \right\}$ many students are in financial trouble.
\end{itemize}


%%% Local Variables:
%%% mode: LaTeX
%%% TeX-master: "main"
%%% End:
