
\section{动词和形容词的补足关系}

动词和形容词的补语就是接在动词或形容词后面,说明该词所隐含的意义关系的语法结构。

\subsection{多词动词}

多词动词有两大类:
\begin{description}
\item[实义动词+小品词] 小品词是一个中性名称,指副词与介词词性重叠的词,根据语
  境不同,其词性也不同。
  \begin{description}
  \item[短语动词 PHRASAL VERB] 小品词是副词。例如drink \textbf{up}, find \textbf{out}。

  \item[介词动词 PREPOSITIONAL VERB] 小品词是介词,例如 dispose \textbf{of}, cope \textbf{with}。

  \item[短语--介词动词 PHRASAL-PREPOSITIONAL] 动词后接两个小品词,前为副 词,
    后为介词。例如put \textbf{up} \textbf{with} \ldots{}
  \end{description}

\item[实义动词后不接小品词] 例如:cut short, put paid to。

\end{description}

在多词动词和自由结合之间并没有明确的界限。


\subsubsection{不及物短语动词}

不及物短语动词,一个动词接一个副词小品词。
\begin{itemize}
\item The plane has just \unbf{touched down}.
\item He is \unbf{playing around}.
\item I hope you'll \unbf{get by}.
\item How are you \unbf{getting on}?
\item Did he \unbf{catch on}?
\item The prisoner finally \unbf{broke down}.
\item She \unbf{turned up} unexpectedly.
\item When will they \unbf{give in}?
\item The tank \unbf{blew up}.
\item The two girls have \unbf{fallen out}.

\end{itemize}

短语动词和自由组合的差异:
\begin{itemize}
\item 诸如give in(投降)和(blow up)爆炸这样的短语动词,我们无法孤立地根据动词
  和小品词的意思来预测组合后习语的意思。但是在自由组合中(如: walk past),我们
  就可以做出预测。

\item自由组合可替代可拆分。walk past中的walk,我们可以用run, trot, swim,
  fly等来替代;至于past,我们可用by, in, through, over等来替代。


\item 通常不及物短语动词为固定搭配,动词和小品词之间不能插入其他内容且顺序固
  定;但在自由组合中就可以,如\textbf{go} straight \textbf{on}, 另外自由组合
  中还可以副词前置,如\textbf{out} \textbf{came} the sun, \textbf{Up} you
  \textbf{come}等。

\end{itemize}

\subsubsection{及物短语动词}

很多短语动词可以带有直接宾语,因此是及物的:

















%%% Local Variables:
%%% mode: latex
%%% TeX-master: "main"
%%% End:
