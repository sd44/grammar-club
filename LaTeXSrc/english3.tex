
\section{动词和形容词的补足关系}

动词和形容词的补语就是接在动词或形容词后面,说明该词所隐含的意义关系的语法结构。

\subsection{多词动词}

多词动词有两大类:
\begin{description}
\item[实义动词+小品词] 小品词是一个中性名称,指副词与介词类有所重叠的词,根据语
  境不同,其词性也不同。
  \begin{description}
  \item[短语动词 PHRASAL VERB] 小品词是副词。例如drink \textbf{up}, find \textbf{out}。

  \item[介词动词 PREPOSITIONAL VERB] 小品词是介词,例如 dispose \textbf{of}, cope \textbf{with}。

  \item[短语--介词动词 PHRASAL-PREPOSITIONAL] 动词后接两个小品词,前为副 词,
    后为介词。例如put \textbf{up} \textbf{with} \ldots{}
  \end{description}

\item[实义动词后不接小品词] 例如:cut short, put paid to。

\end{description}

在多词动词和自由结合之间并没有明确的界限。

小品词如下:
\begin{description}
\item[介词] against, among, as, at, beside, for, from, into, like, of, onto, upon, with, etc.

\item[介词副词] about, above, across, after, along, around, by, down, in, off, on, out (AmE),over, past, round, through, under, up, etc.

\item[空间副词] aback, ahead, apart, aside, astray, away, back, forward(s), home, in front, on top, out (BrE), together, etc.

\end{description}

\subsubsection{不及物短语动词}

不及物短语动词,一个动词接一个副词小品词。
\begin{itemize}
\item The plane has just \unbf{touched down}.
\item He is \unbf{playing around}.
\item I hope you'll \unbf{get by}.
\item How are you \unbf{getting on}?
\item Did he \unbf{catch on}?
\item The prisoner finally \unbf{broke down}.
\item She \unbf{turned up} unexpectedly.
\item When will they \unbf{give in}?
\item The tank \unbf{blew up}.
\item The two girls have \unbf{fallen out}.

\end{itemize}

短语动词和自由组合的差异:
\begin{itemize}
\item 诸如give in(投降)和(blow up)爆炸这样的短语动词,我们无法孤立地根据动词
  和小品词的意思来预测组合后习语的意思。但是在自由组合中(如: walk past),我们
  就可以做出预测。

\item自由组合可替代可拆分。walk past中的walk,我们可以用run, trot, swim,
  fly等来替代;至于past,我们可用by, in, through, over等来替代。


\item 通常不及物短语动词为固定搭配,动词和小品词之间不能插入其他内容且顺序固
  定;但在自由组合中就可以,如\textbf{go} straight \textbf{on}, 另外自由组合
  中还可以副词前置,如\textbf{out} \textbf{came} the sun, \textbf{Up} you
  \textbf{come}等。

\end{itemize}

\subsubsection{及物短语动词}

很多短语动词可以带有直接宾语,因此是及物的:
\begin{itemize}
\item We will \unbf{set up} a new unit.
\item Shall I \unbf{put away} the dishes?
\item \unbf{Find out} if they are coming.
\item She's \unbf{bring up} two children.
\item Someone \unbf{turned on} the light.
\item They have \unbf{called off} the strike.
\item He can't \unbf{live down} his past.
\item I can't \unbf{make out} what he means.
\item She \unbf{looked up} her friends.
\item They may have \unbf{blown up} the bridge.

\end{itemize}

和同一种形式的自由组合一样,及物短语动词的小品词既可以在直接宾语之前,也可以在其后面:
\begin{itemize}
\item They \unbf{turned on} the light.
\item They \unbf{turned} the light \unbf{on}.
\item She \unbf{looked} her friends \unbf{up}.
\end{itemize}
但是,当\textbf{宾语是人称代词时,小品词必须位于宾语之后}:
\begin{itemize}
\item They \unbf{turned} it \unbf{on}.
\end{itemize}
当宾语较长,或有意要使宾语成为末端的中心,小品词就往往放在宾语之前。

在惯用夸张语表达中,小品词只能放在最后:
\begin{itemize}
\item I was crying my eyes out.

\item I was laughing my head off.
\item I was sobbing my heart out.
\end{itemize}

\subsubsection{第一类介词动词}

第一类介词动词由实义动词后接介词构成,两者在语义上或句法上相关联。接在介词后
面的名词短语是\textbf{介词宾语},这个术语表示与直接宾语相区别。
\begin{itemize}
\item \unbf{Look at} these pictures.

\item  I don't \unbf{care for} Jane's parties.

\item We must \unbf{go into} the problem.

\end{itemize}


\textbf{介词动词}也可以有\textbf{被动态};也可以轻松地在实义动词和介词之
间\textbf{插入一个副词}:
\begin{itemize}
\item This matter will have to \unbf{be dealt with} immediately.
\item The picture \unbf{was looked at} disdainfully by many people.
\item Many people \unbf{looked} disdainfully \unbf{at} the picture.
\end{itemize}

就介词宾语发问的wh- 疑问句是由代词who(m) 和what(用于直接宾语) 而不是疑问副
词。

\subsubsection{第二类介词动词}

第二类介词动词是双宾语动词。也就是说,它们后面\textbf{接两个名词短语,通常由
  介词分开:后者为介词宾语},例如:

\begin{itemize}
\item They \unbf{plied} the young man \unbf{with} food.
\item Please \unbf{confine} your remarks \unbf{to} the matter under discussion.
\item This clothing will \unbf{protect} you \unbf{from} the worst weather.
\item Jenny \unbf{thanked} us \unbf{for} the present.
\item May I \unbf{remind} you \unbf{of} our agreement?
They have \unbf{provided} the child \unbf{with} a good education.
\end{itemize}
直接宾语在对应的被动态分句中变为主语:
\begin{itemize}
\item The gang \unbf{robbed} her \unbf{of} her necklace.
\item She was \unbf{robbed of} her necklace (by the gang).
\end{itemize}

\subsubsection{短语--介词短语}

短语--介词动词除实义动词外,还包含作小品词的副词和介词。第一类短语--介词动词
只包含一个介词宾语:
\begin{itemize}
\item We are all \unbf{looking forward to} your party on Saturday.

\item He had to \unbf{put up with} a lot of teasing at school. [忍受,容忍,
  包容]

\item Why don't you \unbf{look in on} Mrs. Johnson on your way back? [(短暂)
  探访]

\item He things he can get away with everything.
\end{itemize}




%%% Local Variables:
%%% mode: latex
%%% TeX-master: "main"
%%% End:
