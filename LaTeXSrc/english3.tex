
\section{动词和形容词的补足关系}

动词和形容词的补语就是接在动词或形容词后面,说明该词所隐含的意义关系的语法结构。

\subsection{多词动词}

多词动词有两大类:
\begin{description}
\item[实义动词+小品词] 小品词 (PARTICLE)是一个中性名称,指副词与介词类有所重叠的词,根据语
  境不同,其词性也不同。\index{概念!小品词,particle}
  \begin{description}
  \item[短语动词 PHRASAL VERB] 小品词是副词。例如drink \textbf{up}, find
    \textbf{out}。\index{概念!短语动词@短语动词,phrasal verb}

  \item[介词动词 PREPOSITIONAL VERB] 小品词是介词,例如 dispose \textbf{of},
    cope \textbf{with}。\index{概念!介词动词@介词动词,prepositional verb}

  \item[短语--介词动词 PHRASAL--PREPOSITIONAL] 动词后接两个小品词,前为副词,后
    为介词。\index{概念!短语--介词动词,phrasal--prepositional}例如put
    \textbf{up} \textbf{with} \ldots{}
  \end{description}

\item[实义动词后不接小品词] 例如:cut short, put paid to。

\end{description}

在多词动词和自由结合之间并没有明确的界限。

小品词如下:
\begin{description}
\item[介词] against, among, as, at, beside, for, from, into, like, of, onto,
upon, with, etc.

\item[介词副词] about, above, across, after, along, around, by, down, in, off,
on, out (AmE),over, past, round, through, under, up, etc.

\item[空间副词] aback, ahead, apart, aside, astray, away, back, forward(s),
home, in front, on top, out (BrE), together, etc.

\end{description}

多次动词在语义上是一个整体,这常常表现在它\textbf{可用一个动词来替代}。
\subsubsection{不及物短语动词}

不及物短语动词,一个动词接一个副词小品词。
\begin{itemize}
\item The plane has just \unbf{touched down}.
\item He is \unbf{playing around}.
\item I hope you'll \unbf{get by}.
\item How are you \unbf{getting on}?
\item Did he \unbf{catch on}?
\item The prisoner finally \unbf{broke down}.
\item She \unbf{turned up} unexpectedly.
\item When will they \unbf{give in}?
\item The tank \unbf{blew up}.
\item The two girls have \unbf{fallen out}.

\end{itemize}

短语动词和自由组合的差异:
\begin{itemize}
\item 诸如give in(投降)和(blow up)爆炸这样的短语动词,我们无法孤立地根据动词
和小品词的意思来预测组合后习语的意思。但是在自由组合中(如: walk past),我们就可
以做出预测。

\item 自由组合可替代可拆分。walk past中的walk,我们可以用run, trot, swim, fly等来
替代;至于past,我们可用by, in, through, over等来替代。


\item 通常不及物短语动词为固定搭配,动词和小品词之间不能插入其他内容且顺序固定;
但在自由组合中就可以,如\textbf{go} straight \textbf{on}, 另外自由组合中还可以副
词前置,如\textbf{out} \textbf{came} the sun, \textbf{Up} you \textbf{come}等。

\end{itemize}

\subsubsection{及物短语动词}

很多短语动词可以带有直接宾语,因此是及物的:
\begin{itemize}
\item We will \unbf{set up} a new unit.
\item Shall I \unbf{put away} the dishes?
\item \unbf{Find out} if they are coming.
\item She's \unbf{bring up} two children.
\item Someone \unbf{turned on} the light.
\item They have \unbf{called off} the strike.
\item He can't \unbf{live down} his past.
\item I can't \unbf{make out} what he means.
\item She \unbf{looked up} her friends.
\item They may have \unbf{blown up} the bridge.

\end{itemize}

和同一种形式的自由组合一样,及物短语动词的小品词既可以在直接宾语之前,也可以在其
后面:
\begin{itemize}
\item They \unbf{turned on} the light.
\item They \unbf{turned} the light \unbf{on}.
\item She \unbf{looked} her friends \unbf{up}.
\end{itemize}但是,当\textbf{宾语是人称代词时,小品词必须位于宾语之后}:
\begin{itemize}
\item They \unbf{turned} it \unbf{on}.
\end{itemize}当宾语较长,或有意要使宾语成为末端的中心,小品词就往往放在宾语之前。

在惯用夸张语表达中,小品词只能放在最后:
\begin{itemize}
\item I was crying my eyes out.

\item I was laughing my head off.
\item I was sobbing my heart out.
\end{itemize}

\subsubsection{第一类介词动词}

第一类介词动词由实义动词后接介词构成,两者在语义上或句法上相关联。接在介词后面的
名词短语是\textbf{介词宾语},这个术语表示与直接宾语相区别。
\begin{itemize}
\item \unbf{Look at} these pictures.

\item I don't \unbf{care for} Jane's parties.

\item We must \unbf{go into} the problem.

\end{itemize}


\textbf{介词动词}也可以有\textbf{被动态};也可以轻松地在实义动词和介词之间
\textbf{插入一个副词}:
\begin{itemize}
\item This matter will have to \unbf{be dealt with} immediately.
\item The picture \unbf{was looked at} disdainfully by many people.
\item Many people \unbf{looked} disdainfully \unbf{at} the picture.
\end{itemize}

就介词宾语发问的wh- 疑问句是由代词who(m) 和what(用于直接宾语) 而不是疑问副词。

\subsubsection{第二类介词动词}

第二类介词动词是双宾语动词。也就是说,它们后面\textbf{接两个名词短语,通常由介词
分开:后者为介词宾语},例如:

\begin{itemize}
\item They \unbf{plied} the young man \unbf{with} food.
\item Please \unbf{confine} your remarks \unbf{to} the matter under discussion.
\item This clothing will \unbf{protect} you \unbf{from} the worst weather.
\item Jenny \unbf{thanked} us \unbf{for} the present.
\item May I \unbf{remind} you \unbf{of} our agreement? They have \unbf{provided}
the child \unbf{with} a good education.
\end{itemize}直接宾语在对应的被动态分句中变为主语:
\begin{itemize}
\item The gang \unbf{robbed} her \unbf{of} her necklace.
\item She was \unbf{robbed of} her necklace (by the gang).
\end{itemize}

\subsubsection{短语--介词短语}

短语--介词动词除实义动词外,还包含作小品词的副词和介词。他们多是非正式文体。

第一类短语--介词动词只包含一个介词宾语:
\begin{itemize}
\item We are all \unbf{looking forward to} your party on Saturday.

\item He had to \unbf{put up with} a lot of teasing at school. [忍受,容忍,包容]

\item Why don't you \unbf{look in on} Mrs. Johnson on your way back? [(短暂)探
访]

\item He things he can \unbf{get away with} everything.
\end{itemize}


第二类短语-介词动词是双宾语动词。他们需要两个宾语,第二个宾语是介词宾语(往往被
视为\textbf{受事}参与者):
\begin{itemize}
\item Don't \unbf{take} it \unbf{out on} me! [向…发泄;拿…撒气]

\item We \unbf{put our} success down \unbf{to} hard work. [to consider that sth.
is caused by sth. 把…归因于]

\item I'll \unbf{let} you \unbf{in on} a secret. [to allow sb. to share a secret
告知,透露(秘密)]
\end{itemize}

\subsection{动词补语}

\subsubsection{不及物动词}

除一般不及物动词以外,还有:
\begin{description}
\item[也可作及物而意思不变的动词] 可将其当作有一个“显明的被省略的宾语”。
  \begin{itemize}
  \item He \unbf{smokes} (a cigarette).

  \item I am \unbf{reading} (a book).

  \item He \unbf{drinks} (alcohol) heavily.

  \item Knock before you \unbf{enter} (the room).
  \end{itemize}此外还有drive, enter, help, pass, play, win, write等。

\item[也可作及物但主被动变换] 不及物用法以受事参与者为主语;及物用法以施事者为主
语。
  \begin{itemize}
  \item The door \unbf{opened} slowly. 比较:Mary \unbf{opened} the door.

  \item The car \unbf{stopped}. 比较:He \unbf{stopped} the car.

  \item The door \unbf{closed} behind him: You can \unbf{close} the door
easily---it just \unbf{pulls}. [you just \unbf{pull} it"]
  \end{itemize}此外还有 begin, change, drop, increase, move, turn, unite, walk,
work等词。

  也有些不及物动词变及物动词时,有使宾语被动的意思:
  \begin{itemize}
  \item \unbf{run} the water [cause the water to \unbf{run}]

  \item \unbf{slide} the drawer shut [\unbf{slide} back the drawer 谓语+状语] 关
上抽屉
  \end{itemize}

\item[作不及物用时有互相参与意义] 如:
  \begin{itemize}
  \item We have \unbf{met}. 比较:I have \unbf{met} you.

  \item The bus and car \unbf{collided}. 比较:The bus \unbf{collided} with the
car.(也是不及物)
  \end{itemize}
\end{description}

\subsubsection{接不定式和 -ing 分句皆可的动词}

\begin{description}
\item[remember doing] 记得\textbf{过去发生过}的事,可有\textbf{持续性}。
  \begin{itemize}
  \item I still remember buying my first bicycle. [remember that I had bought]
  \end{itemize}

\item[remember to do] 记得\textbf{将来应该做}的事,有\textbf{不确定性}(可能最终也未做此事)。
  \begin{itemize}
  \item You must remember to fetch Mr Lewis from the station tomorrow. [remember
    that you are to]
  \end{itemize}


\item[forget doing] 忘了\textbf{过去发生过}的事,可有\textbf{持续性}。
  \begin{itemize}
  \item I\textbf{'ll} never forget meeting the Queen \textbf{in 1988}.
  \end{itemize}

\item[forget to do] 忘做\textbf{应该做}的事,有\textbf{不确定性}(未发生)。
  \begin{itemize}
  \item I forgot to buy the soap.
  \end{itemize}


\item[go on doing] \textbf{持续}做\textbf{过去已经在做}的某事,可有\textbf{持续性}。
  \begin{itemize}
  \item She went on talking about her illness until midnight.
  \end{itemize}


\item[go on to do] (停止其他动作,)\textbf{继续}做某事,\textbf{动作非持续,有转折。}

  \begin{itemize}
  \item She stopped talking about that and went on to do her job.
  \end{itemize}


\item[regret dong] 后悔、遗憾\textbf{做过某事},有\textbf{持续性}。

  \begin{itemize}
  \item I regret leaving school at 16 –-- a big mistake. [ I regret \unbf{that I
      left} school at 16]
  \end{itemize}

\item[regret to do] 为\textbf{不得已发生的事情}感到抱歉、遗憾,并非真正后悔。

  \begin{itemize}
  \item We regret to say that we are unable to help you. [regret \unbf{that we
      should say} that]
  \end{itemize}

\item[see, watch, hear等] 可以接 doing(正在发生,还未结束,持续性);也可接不带to
  的不定式(事情完整结束)。
  \begin{itemize}
  \item I looked out of the window and saw Emily \unbf{crossing} the road.
  \item I saw Emily \unbf{cross} the road and \unbf{disappear} into the bank.
  \end{itemize}


\item[try] 接doing或不定式均可,尝试做困难的事。但try doing有期待一种结果的意思。
  \begin{itemize}
  \item I tried sending her flowers, but she still wouldn't speak to me.
  \item John isn't here. Try phoning his mobile.
  \end{itemize}
\item[like, love, hate, prefer] 接doing或不定式均可。不定式可以有尚未发生,或不得不的
  意思。
  \begin{itemize}
  \item I hate to tell you this, but we're going to miss the train.

  \item 'Can I give you a lift?''No thanks, I'd prefer to walk.'

    give you a lift: 载你一程
  \end{itemize}

\item[begin, start, continue] 接doing或不定式均可。

\item[stop] 停止正在做的事情,用doing;停下其他事情做某事,用不定式。
  \begin{itemize}
  \item We stopped taking pictures.

    We were no longer taking pictures.


  \item We stopped to take pictures.

    We stopped what we were doing so that we could start taking pictures.
  \end{itemize}

\item[advice, allow, permit, forbid] 不接宾语时,用doing (SVO, -ing 分句做宾语);接宾语
  时,一般接不定式 (SVOO, to- 不定式分句做直接宾语)。
  \begin{itemize}
  \item I wouldn't advise taking the car – there's nowhere to park.
  \item I wouldn't advise \unbf{you} to take the car …
  \item We don't allow/permit smoking in the lecture room.
  \item We don't allow/permit \unbf{people} to smoke in the lecture room.
  \end{itemize}
\end{description}

另外,go/get/walk/point/visit/down/be used 等位置副词后接的to为介词,不是不定式
标记to,如后接动词,需用doing。

\subsubsection{动词补足关系的分类}

\begin{table}[p] \centering \small

  \begin{talltblr}[
    caption = {动词补足关系的类型},
    label = {tab:verbcop},
    note{a} = {$C_s$ 主语补语,$O_i$ indirect objects间接宾语,$O_d$ direct objects 直接宾语,
      $+S$ 含主语,$-S$ 不含主语,},
    ]{width=\linewidth,
      colspec={ll},
      rowsep=1pt, colsep=2pt,
      row{1} = {font=\bfseries},
    }
    \toprule
    变体 & 例句 \\ \midrule
\textbf{连系动词(SVC和SVA)} & \\
形容词性 $C_s$ & The girl seemed restless. \\
名词性 $C_s$ & William is my friend. \\
状语补足语 & The kitchen is downstairs. \\ \midrule
\textbf{单宾语及物动词(SVO)} & \\
 {名词短语作O \\
 (有被动式)} & Tom caught the ball. \\
 {名词短语作O \\
 (无被动式)} & Pavl lacks confidence. \\
 that- 分句作O & I think that we have met. \\
 wh- 分句作O & Can you guess what she said? \\
 wh- 不定式 (-S) 作O & I learned how to look after the cats. \\
 to- 不定式 (-S) 作O & We've decided to move house. \\
 -ing分句 (-S) 作O & She enjoys playing table tenis. \\
 to- 不定式 (+S) 作O & They want us to help. \\
 -ing分句 (+S) 作O & I hate the children picking a fight. \\ \midrule
 \textbf{复合及物动询 (SVOC和SVOA)} & \\
形容词性 $C_o$ & That music drives me mad. \\
名词性 $C_o$ & They named the ship ``Zeus''. \\
 O + 状语 & I left the key at home. \\
 O + to- 不定式 & They knew him to be a spy. \\
 O + 不带 to 不定式 & I saw her leave the room. \\
 O + -ing 分句 & I heard someone shouting. \\
 O + -ed 分句 & I get the watch repaired. \\ \midrule
 \textbf{双宾语及物动词 (SVOO)} & \\
名词短语作 $O_i$ 和 $O_d$ & Tom give me some food. \\
有介词宾语O & Please say something to us. \\
$O_i$ + that- 分句 & They told me that I was ill. \\
 $O_i$ + wh- 分句 & He asked me what time it was. \\
 $O_i$ + wh- 不定式分句 & Mary showed us what to do. \\
$O_i$ + to- 不定式 & I advised Mark to see a doctor. \\
 \bottomrule
\end{talltblr}%
\end{table}


\subsubsection{系词补足关系}

seem, appear, look, sound, feel, smell, taste 等``seeming'' 感官系动词在下列
这类句子中用由 as if/though(似乎,好像) 开头的状语分句来补足。
\begin{itemize}
\item Jill \unbf{looked as if} she had seen a ghost.

\item It \unbf{seems as if} the weather is improving.
\end{itemize}

\subsubsection{单宾语及物补足关系}

在cost ten dollars; weight 20 kilos 之类的度量用语中可见到VO类型,但有同样理由将
其分析为 V + A,其中A为必要附加状语。因为除了用what外,还可以用how much 问句:
\begin{itemize}
\item How much / What does it cost/weight ?
\end{itemize}

宾语为that- 分句的句子变被动式,宾语变主语时,that 不能省略(见
\cref{subsub:thatclause}):
\begin{itemize}
\item Everybody hoped \unbf{(that)} she would sing.
\item \unbf{That} she would sing was hoped by everybody.
\end{itemize}

\begin{table}[htbp]
  \centering \small
  \begin{talltblr}[ caption = {作宾语的非限定性分句},
    label = {tab:obin},
    ]{
      width=\linewidth, colspec={lXX},
      rowsep=2pt, colsep=4pt,
      row{1} = {c, font=\bfseries},
    }
    \toprule
    & 不带主语 & 带主语 \\ \midrule
    to- 不定式 & Jack hates \emph{to miss the train}. & Jack hates \emph{her to
    miss the train}. \\
  -ing 分句 & Jack hates \emph{missing the train}. & Jack hates \emph{her missing
  the train}.\\
    \bottomrule
  \end{talltblr}%
\end{table}

SVO结构中,不带主语的不定式分句、-ing 分句\textbf{被省略的主语往往和领句的主语相同}。
\begin{itemize}
\item I love \unbf{listening to music}.
\end{itemize}也有例外:
\begin{description}
\item[被省略的主语独立且显明] 分词主语不确定,并且独立于前面领句主语。

  \begin{itemize}
  \item He recommended \unbf{introducing a wealth tax}.

    负责征收财产税的人是政府机关,而不是领句主语“他”。
  \end{itemize}
\end{description}

SVO中,\textbf{带主语的不定式分句}可以作补足语,但这一组中的动词为数极少,主要表
示(不)喜欢或(不)想要,如 desire, hate, like, love, prefer, want and wish:
\begin{itemize}
  \item They don't like \unbf{the house to be left empty}.
\end{itemize}在这些动词之后,不定式之前的名词短语不能转变为被动式中的主语。
\begin{itemize}
  \item \sout{The house isn't liked to be left empty (by them).}
\end{itemize}


SVO中,\textbf{带主语的 -ing 分句} 可以作补足语。\textbf{人称主语可用属格形式},
但常常使人感到别扭或不自然。
\begin{itemize}
  \item I dislike \unbf{him/his} driving my car.

  \item We look forward to \unbf{you/your} becoming our neighbour.
\end{itemize}

\textbf{that- 分句补语}:
\begin{itemize}
\item \unbf{It} seems \unbf{(that) you are mistaken.}
\item \unbf{It} appears \unbf{(that) you have lost your temper.}
\end{itemize}
以上两例句中 that- 分句不是动词的宾语而是\textbf{外置主语}。\index{概念!外置
  主语}
\begin{description}
\item[外置主语 (Extraposition)] 句子中\textbf{通过形式主语 “it” 将真正的主
    语移到句末}的现象。这种结构通常用于使句子更流畅或避免过长的主语,使得句子
  的\textbf{重心更加突出}。外置主语常见于\textbf{名词性分句和不定式},除上
  面that- 分句外,还有:
  \begin{itemize}
  \item \unbf{It} is unclear \unbf{why} she told him.

  \item Would it be better \unbf{to pay now}?
  \end{itemize}
\end{description}
常用于这种类型的动词有: seem, appear, happen 和动词短语come about
[happen]和 turn out [transpire]。

\subsubsection{复合及物 (SVOC 和 SVOA) 的补足关系}

复合及物补足关系的一个明显特征是:\textbf{动词后面的两个成分(OC或OA)在意义
  上分别等同于一个名词性分句的主语和谓体。}
\begin{description}
\item[单宾语及物] She presumed \unbf{that her father was dead}.
\item[复合及物] She presumed \unbf{her father (to be) dead}.
\end{description}

介词as 表示连系关系,特别是说明与直接宾语有关的角色或地位。
\begin{itemize}
\item We considered him $ \left\{
    \begin{aligned}
      &\text{a genius} \\
      &\text{as a genius 补语} \\
      &\text{to be a genius}
    \end{aligned}
  \right. $
\end{itemize}
但在某些方面,介词as和引导比较分句的连词as相似,一方面引导分句;另一方
面又引导和分句同位的名词短语:
\begin{itemize}
\item Report me \unbf{as I am --- a superman}.

\item He described her \unbf{as he found her, a liar}.
\end{itemize}


\begin{table}[htbp]
  \centering
  \begin{talltblr}[ caption = {SVOC 中的非限定性分句},
    label = {tab:svocin},
    ]{
      width=\linewidth, colspec={ll},
      rowsep=2pt, colsep=4pt,
      row{1} = {c, font=\bfseries},
    }
    \toprule
    非限定性分句 & 例句\\ \midrule
    to- 不定式  & They knew him \emph{to be a spy}. \\
    不带to不定式 & I heard someone \emph{slam the door}. \\
    -ing 分句 & I caught Ann \emph{reading my diary}. \\
    -ed 分句 & We saw him \emph{beaten by the German in final}. \\
    \bottomrule
  \end{talltblr}%
\end{table}

\cref{tab:svocin} 中作为\textbf{宾语补语的的非限定性分句}(斜体表示)自身没有
主语,但\textbf{其隐含的主语总是前面的宾语},这样的宾语被叫做\textbf{上升宾
  语 (raised object)}。语义上,上升宾语是非限定型动词的主语;句法上,它从非限
定性分句中上升出来作领句的宾语。要\textbf{注意歧义},如:
\begin{itemize}
\item Tom left her $\left\{
    \begin{aligned}
      &\text{to finish the job.} \\
      &\text{finishing the job.} \\
    \end{aligned}
    \right.$

    Tom离开她,由她去完成工作。

    \textbf{her是上升宾语,也是后面非限定性分句的主语。}
\end{itemize}




\subsubsection{双宾语及物 (SVOO) 补足关系}

不同于SVOC中宾语与宾语补语的连系关系;SVOO中两个宾语之间没有连系关系。

\paragraph{宾语和介词宾语}

介词短语可做宾语,大体有以下句型:
\begin{table}[htbp]
  \centering \small
  \begin{talltblr}[ caption = {宾语和介词宾语},
    label = {tab:PrepObj},
    ]{
      width=\linewidth, colspec={},
      rowsep=2pt, colsep=4pt,
      row{1} = {font=\bfseries},
    }
    \toprule
    动词 & 双宾语 & 例句 \\ \midrule
   \SetCell[r=3]{l} told & $O_i + O_d$ & Mary told only John the secret. \\
   & $O_d + O_p$ & Mary told the secret only to John. \\
   & $O_i + O-p$ & Mary told only John about the secret. \\ \midrule
   \SetCell[r=2]{l} offer& $O_i + O_d$ & John offered Mary some help. \\
   & $O_d + O_p$ & John offered some help to Mary. \\ \midrule
   \SetCell[r=2]{l} envy & $O_i + O_d$ & She envied John his success. \\
   & $O_i + O_p$ & She envied John for his success. \\ \midrule
   wish & $O_i + O_d$ & They wished him good luck. \\ \midrule
   \SetCell[r=2]{l} blame & $O_d + O_p$ & He blamed the divorce on John. \\
   & $O_i + O_p$ & He blamed John for the divorce. \\ \midrule
   say &  $O_d + O_p$ & Why didn't anybody say this to me? \\ \midrule
   warn &  $O_i + O_p$ & Mary warned John of the dangers. \\
    \bottomrule
  \end{talltblr}%
\end{table}

\subsection{形容词的补足关系}

\textbf{名词不能做形容词补足语。}

和介词动词一样,形容词经常和后面的介词构成词汇单位:good at, fond of,
opposed to, angry with/about等等。


\begin{table}[htbp]
  \centering \small
  \begin{talltblr}[ caption = {形容词补足语类型},
    label = {tab:adjin},
    ]{
      width=\linewidth, colspec={lXl},
      rowsep=2pt, colsep=4pt,
      row{1} = {c, font=\bfseries},
    }
    \toprule
    形容词补足语类型  & 例句   \\ \midrule
    介词短语 & She felt angry \emph{with herself}. \\
    that- 分句 & I am surprised \emph{(that) you didn't call the doctor
      before}.  \\
    wh- 分句 & It was unclear \emph{what they would do}. \\
    to- 不定式 & Bob is sorry \emph{to hear it}. \\
    -ing 分句 & I'm busy (with) \emph{getting the house redecorated}. \\
    \bottomrule
  \end{talltblr}%
\end{table}

\section{名词短语}

名词短语也可以非常复杂,因为句子本身可以被改写,以适用于名词短语结构。例如:
\begin{itemize}
\item The girl is Mary Smith.

\item The girl is tall.

\item The girl was standing in the corner.

\item You waved to the girl when you entered.

\item The girl became angry because you knocked over her glass.
\end{itemize}
以上句子可以组合为一个由很长的名词短语作主语的\textbf{简单句}:
\begin{itemize}

\item \emph{The tall girl standing in the corner who became angry because
    you knocked over her glass after you waved to her when you entered} is
  Mary Smith.
\end{itemize}

\subsection{名词短语的构成部分}

\begin{description}
\item[中心成分(Head)] 被其他成分群集于周围,使之构成一致关系的成分,通常是一
  个名词或代词。中心成分确定了名词短语的句法角色(如主语、宾语等)以及与其他
  句子成分的关系。 如以上名词长句中的girl。

\item[限定成分 (DETERMINATIVE)] 它包括:前位、中心、后位限定词,见 \cref{tab:determ}。

\item[前置修饰 (PREMODIFICATION)] 位于中心词前,除限定词以外的所有成分,一
  般是形容词短语和名词(作形容词用)。如:
  \begin{itemize}
  \item some \textbf{expensive} equipment
  \item some \textbf{very very expensive office} equipment
  \end{itemize}

\item[后置修饰 (POSTMODIFICATION)] 位于中心词后的所有词项,有介词短语、非限
  定分句和关系分句和补足语。
  \begin{description}
  \item[介词短语] the car \emph{outside the station}
  \item[非限定性分句] the car \emph{standing outside the station}
  \item[关系分句] the car \emph{that stood outside the station}
  \item[补足语] a bigger car \emph{than that}
  \end{description}
\end{description}












%%% Local Variables:
%%% mode: LaTeX
%%% TeX-master: "main"
%%% End:
