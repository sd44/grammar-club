\section{动词和形容词的补足关系}

动词和形容词的补足语就是接在动词或形容词后面,说明该词所隐含的意义关系的语法结
构。

\subsection{多词动词}

多词动词有两大类:
\begin{description}
\item[实义动词+小品词] 小品词 (PARTICLE)是一个中性名称,指一些空间副词与介词,
  有的小品词根据语境不同,其词性也不同。\index{概念!小品词 particle} 具体可分
  为:
  \begin{description}
  \item[短语动词 PHRASAL VERB] 小品词是空间副词。例如drink \textbf{up}, find
    \textbf{out}。\index{概念!短语动词 phrasal verb}

  \item[介词动词 PREPOSITIONAL VERB] 小品词是介词,例如 dispose \textbf{of},
    cope \textbf{with}。\index{概念!介词动词 prepositional verb}

  \item[短语--介词动词 PHRASAL--PREPOSITIONAL] 动词后接两个小品词,前为副词,后
    为介词。\index{概念!短语--介词动词 phrasal--prepositional}例如put
    \textbf{up} \textbf{with} \ldots{}
  \end{description}

\item[实义动词后不接小品词] 例如:cut short, put paid to。

\end{description}

在多词动词和自由结合之间并没有明确的界限。

小品词如下:
\begin{description}
\item[介词] against, among, as, at, beside, for, from, into, like, of, onto,
upon, with, etc.

\item[介词副词] about, above, across, after, along, around, by, down, in, off,
on, out (AmE),over, past, round, through, under, up, etc.

\item[空间副词] aback, ahead, apart, aside, astray, away, back, forward(s),
home, in front, on top, out (BrE), together, etc.

\end{description}

多次动词在语义上是一个整体,这常常表现在它\textbf{可用一个动词来替代}。

介词和空间副词之间最明显的差别在于:介词要求后面跟有一个名词短语作为补足语,而
副词则不要求这样。上述小品词只有\textbf{介词副词}类可作介词也可作副词:


\subsubsection{不及物短语动词}

不及物短语动词,一个动词接一个副词小品词。
\begin{itemize}
\item The plane has just \unbf{touched down}.
\item He is \unbf{playing around}.
\item I hope you'll \unbf{get by}.
\item How are you \unbf{getting on}?
\item Did he \unbf{catch on}?
\item The prisoner finally \unbf{broke down}.
\item She \unbf{turned up} unexpectedly.
\item When will they \unbf{give in}?
\item The tank \unbf{blew up}.
\item The two girls have \unbf{fallen out}.

\end{itemize}

短语动词和自由组合的差异:
\begin{itemize}
\item 诸如give in(投降)和(blow up)爆炸这样的短语动词,我们无法孤立地根据动
  词和小品词的意思来预测组合后习语的意思。但是在自由组合中(如: walk
  past),我们就可以做出预测。

\item 自由组合可替代可拆分。walk past 中的walk,我们可以用run, trot, swim,
  fly 等来替代;至于past,我们可用by, in, through, over 等来替代。


\item 通常不及物短语动词为固定搭配,动词和小品词之间不能插入其他内容且顺序固
  定;但在自由组合中就可以,如\textbf{go} straight \textbf{on}, 另外\textbf{自由组合
  中还可以副词前置},如\textbf{out} \textbf{came} the sun, \textbf{Up} you
  \textbf{come}等。

\end{itemize}

\subsubsection{及物短语动词}

很多短语动词可以带有直接宾语,因此是及物的:
\begin{itemize}
\item We will \unbf{set up} a new unit.
\item Shall I \unbf{put away} the dishes?
\item \unbf{Find out} if they are coming.
\item She's \unbf{bring up} two children.
\item Someone \unbf{turned on} the light.
\item They have \unbf{called off} the strike.
\item He can't \unbf{live down} his past.
\item I can't \unbf{make out} what he means.
\item She \unbf{looked up} her friends.
\item They may have \unbf{blown up} the bridge.
\end{itemize}

和同一种形式的自由组合一样,\textbf{及物短语动词的小品词既可以在直接宾语之前,
  也可以在其后面}:
\begin{itemize}
\item They \unbf{turned on} the light.
\item They \unbf{turned} the light \unbf{on}.
\item She \unbf{looked} her friends \unbf{up}.
\end{itemize}

但是,当\textbf{宾语是人称代词时,小品词必须位于宾语之后}:
\begin{itemize}
\item They \unbf{turned} it \unbf{on}.
\end{itemize}当宾语较长,或有意要使宾语成为末端的中心,小品词就往往放在宾语之前。

在惯用夸张语表达中,小品词只能放在最后:
\begin{itemize}
\item I was \unbf{crying} my eyes \unbf{out}.
\item I was \unbf{laughing} my head \unbf{off}.
\item I was \unbf{sobbing} my heart \unbf{out}.
\end{itemize}

\subsubsection{第一类介词动词}

第一类介词动词由实义动词后接介词构成,两者在语义上或句法上相关联。接在介词后面的
名词短语是\textbf{介词宾语},这个术语表示与直接宾语相区别。
\begin{itemize}
\item \unbf{Look at} these pictures.

\item I don't \unbf{care for} Jane's parties.

\item We must \unbf{go into} the problem.

\end{itemize}


\textbf{介词动词}也可以有\textbf{被动态};也可以轻松地在实义动词和介词之间
\textbf{插入一个副词}:
\begin{itemize}
\item This matter will have to \unbf{be dealt with} immediately.
\item The picture \unbf{was looked at} disdainfully by many people.
\item Many people \unbf{looked} disdainfully \unbf{at} the picture.
\end{itemize}

就\textbf{介词宾语发问}的wh- 疑问句是由\textbf{代词} who(m) 和what(用于直接宾语)引导,而不是\textbf{疑问副词}。

\subsubsection{第二类介词动词}

第二类介词动词是双宾语动词。也就是说,它们后面\textbf{接两个名词短语,通常由
  介词分开:后者为介词宾语},例如:

\begin{itemize}
\item They \unbf{plied} the young man \unbf{with} food.
\item Please \unbf{confine} your remarks \unbf{to} the matter under discussion.
\item This clothing will \unbf{protect} you \unbf{from} the worst weather.
\item Jenny \unbf{thanked} us \unbf{for} the present.
\item May I \unbf{remind} you \unbf{of} our agreement? They have \unbf{provided}
the child \unbf{with} a good education.
\end{itemize}

直接宾语在对应的被动态从句中变为主语:
\begin{itemize}
\item The gang \unbf{robbed} her \unbf{of} her necklace.
\item She was \unbf{robbed of} her necklace (by the gang).
\end{itemize}

\subsubsection{短语--介词短语}

短语--介词动词除实义动词外,还包含作小品词的副词和介词。他们多是非正式文体。

第一类短语--介词动词只包含一个介词宾语:
\begin{itemize}
\item We are all \unbf{looking forward to} your party on Saturday.

\item He had to \unbf{put up with} a lot of teasing at school. [忍受,容忍,包容]

\item Why don't you \unbf{look in on} Mrs. Johnson on your way back? [(短暂)探
访]

\item He things he can \unbf{get away with} everything.
\end{itemize}


第二类短语-介词动词是双宾语动词。他们需要两个宾语,第二个宾语是介词宾语(往往被
视为\textbf{受事}参与者):
\begin{itemize}
\item Don't \unbf{take} it \unbf{out on} me! [向…发泄;拿…撒气]

\item We \unbf{put our} success down \unbf{to} hard work. [to consider that sth.
is caused by sth. 把…归因于]

\item I'll \unbf{let} you \unbf{in on} a secret. [to allow sb. to share a secret
告知,透露(秘密)]
\end{itemize}

\subsection{动词补足语}

\subsubsection{不及物动词}

除一般不及物动词以外,还有:
\begin{description}
\item[也可作及物而意思不变的动词] 可将其当作有一个“显明的被省略的宾语”。
  \begin{itemize}
  \item He \unbf{smokes} (a cigarette).

  \item I am \unbf{reading} (a book).

  \item He \unbf{drinks} (alcohol) heavily.

  \item Knock before you \unbf{enter} (the room).
  \end{itemize}

  此外还有drive, enter, help, pass, play, win, write。

\item[也可作及物但主被动变换] 不及物用法以受事参与者为主语;及物用法以施事者为主
语。
  \begin{itemize}
  \item The door \unbf{opened} slowly. 比较:Mary \unbf{opened} the door.

  \item The car \unbf{stopped}. 比较:He \unbf{stopped} the car.

  \item The door \unbf{closed} behind him: You can \unbf{close} the door
easily---it just \unbf{pulls}. [you just \unbf{pull} it"]
\end{itemize}

此外还有 begin, change, drop, increase, move, turn, unite, walk, work 等词。

也有些不及物动词变及物动词时,有使宾语被动的意思:
\begin{itemize}
\item \unbf{run} the water [cause the water to \unbf{run}]

\item \unbf{slide} the drawer shut [\unbf{slide} back the drawer 谓语+状语] 关
  上抽屉
\end{itemize}

\item[作不及物用时有互相参与意义] 如:
  \begin{itemize}
  \item We have \unbf{met}. 比较:I have \unbf{met} you.

  \item The bus and car \unbf{collided}. 比较:The bus \unbf{collided} with the
    car.(也是不及物)
  \end{itemize}
\end{description}


\subsubsection{动词补足关系的分类}

\begin{table}[p] \centering \small

  \begin{talltblr}[
    caption = {动词补足关系的类型},
    label = {tab:verbcop},
    note{a} = {$C_s$ 主语补语,$O_i$ indirect objects间接宾语,$O_d$ direct objects 直接宾语,
      $+S$ 含主语,$-S$ 不含主语,},
    ]{width=\linewidth,
      colspec = {ll},
      rowsep = 1pt, colsep = 2pt,
      row{1} = {font=\bfseries},
    }
    \toprule
    变体 & 例句 \\ \midrule
\textbf{连系动词(SVC和SVA)} & \\
形容词性 $C_s$ & The girl seemed restless. \\
名词性 $C_s$ & William is my friend. \\
状语补足语 & The kitchen is downstairs. \\ \midrule
\textbf{单宾语及物动词(SV$O$)} & \\
 {名词短语作O \\
 (有被动式)} & Tom caught the ball. \\
 {名词短语作O \\
 (无被动式)} & Paul lacks confidence. \\
 that- 从句作O & I think that we have met. \\
 wh- 从句作O & Can you guess what she said? \\
 wh- 不定式 (-S) 作O & I learned how to look after the cats. \\
 to- 不定式 (-S) 作O & We've decided to move house. \\
 -ing从句 (-S) 作O & She enjoys playing table tennis. \\
 to- 不定式 (+S) 作O & They want us to help. \\
 -ing从句 (+S) 作O & I hate the children picking a fight. \\ \midrule
 \textbf{复合及物动询 (SVOC和SVOA)} & \\
形容词性 $C_o$ & That music drives me mad. \\
名词性 $C_o$ & They named the ship ``Zeus''. \\
 O + 状语 & I left the key at home. \\
 O + to- 不定式 & They knew him to be a spy. \\
 O + 不带 to 不定式 & I saw her leave the room. \\
 O + -ing 从句 & I heard someone shouting. \\
 O + -ed 从句 & I get the watch repaired. \\ \midrule
 \textbf{双宾语及物动词 (SVOO)} & \\
名词短语作 $O_i$ 和 $O_d$ & Tom give me some food. \\
介词短语作 $O$ & Please say something to us. \\
$O_i$ + that- 从句 & They told me that I was ill. \\
 $O_i$ + wh- 从句 & He asked me what time it was. \\
 $O_i$ + wh- 不定式从句 & Mary showed us what to do. \\
$O_i$ + to- 不定式 & I advised Mark to see a doctor. \\
 \bottomrule
\end{talltblr}%
\end{table}


\subsubsection{系词补足关系}

seem, appear, look, sound, feel, smell, taste 等``seeming'' 感官系动词在下列
这类句子中用由 as if/though(似乎,好像) 开头的状语从句来补足。
\begin{itemize}
\item Jill \unbf{looked as if} she had seen a ghost.

\item It \unbf{seems as if} the weather is improving.
\end{itemize}

\subsubsection{单宾语及物补足关系}

在cost ten dollars; weight 20 kilos 之类的度量用语中可见到VO类型,但有同样理由将
其分析为 V + A,其中A为必要附加状语。因为除了用what外,还可以用how much 问句:
\begin{itemize}
\item How much / What does it cost/weight ?
\end{itemize}

宾语为that- 从句的句子变被动式,宾语变主语时,that 不能省略(见
\cref{subsubsec:thatclause}):
\begin{itemize}
\item Everybody hoped \unbf{(that)} she would sing.
\item \unbf{That} she would sing was hoped by everybody.
\end{itemize}

\begin{table}[htbp]
  \centering \small
  \begin{talltblr}[ caption = {作宾语的非限定性从句},
    label = {tab:obin},
    ]{
      width=\linewidth, colspec={lXX},
      rowsep=2pt, colsep=4pt,
      row{1} = {c, font=\bfseries},
    }
    \toprule
    & 不带主语 & 带主语 \\ \midrule
    to- 不定式 & Jack hates \emph{to miss the train}. & Jack hates \emph{her to
    miss the train}. \\
  -ing 从句 & Jack hates \emph{missing the train}. & Jack hates \emph{her missing
  the train}.\\
    \bottomrule
  \end{talltblr}%
\end{table}

SVO结构中,不带主语的不定式从句、-ing 从句\textbf{被省略的主语往往和母句的主语相同}。
\begin{itemize}
\item I love \unbf{listening to music}.
\end{itemize}也有例外:
\begin{description}
\item[被省略的主语独立且显明] 分词主语不确定,并且独立于前面母句主语。

  \begin{itemize}
  \item He recommended \unbf{introducing a wealth tax}.

    负责征收财产税的人是政府机关,而不是母句主语“他”。
  \end{itemize}
\end{description}

SVO中,\textbf{带主语的不定式从句}可以作补足语,但这一组中的动词为数极少,主要表
示(不)喜欢或(不)想要,如 desire, hate, like, love, prefer, want and wish:
\begin{itemize}
  \item They don't like \unbf{the house to be left empty}.
\end{itemize}在这些动词之后,不定式之前的名词短语不能转变为被动式中的主语。
\begin{itemize}
  \item \sout{The house isn't liked to be left empty (by them).}
\end{itemize}


SVO中,\textbf{带主语的 -ing 从句} 可以作补足语。\textbf{人称主语可用属格形式},
但常常使人感到别扭或不自然。
\begin{itemize}
  \item I dislike \unbf{him/his} driving my car.

  \item We look forward to \unbf{you/your} becoming our neighbour.
\end{itemize}

\textbf{that- 从句补语}:
\begin{itemize}
\item \unbf{It} seems \unbf{(that) you are mistaken.}
\item \unbf{It} appears \unbf{(that) you have lost your temper.}
\end{itemize}
以上两例句中 that- 从句不是动词的宾语而是\textbf{外置主语}。\index{概念!外置
  主语}
\begin{description}
\item[外置主语 (EXTRAPOSITION)] 句子中\textbf{通过形式主语 “it” 将真正的主
    语移到句末}的现象。这种结构通常用于使句子更流畅或避免过长的主语,使得句子
  的\textbf{重心更加突出}。外置主语常见于\textbf{名词性从句和不定式},除上
  面that- 从句外,还有:
  \begin{itemize}
  \item \unbf{It} is unclear \unbf{why} she told him.

  \item Would it be better \unbf{to pay now}?
  \end{itemize}
\end{description}
常用于这种类型的动词有: seem, appear, happen 和动词短语come about
[happen]和 turn out [transpire]。

\subsubsection{复合及物 (SVOC 和 SVOA) 的补足关系}

复合及物补足关系的一个明显特征是:\textbf{动词后面的两个成分(OC或OA)在意义
  上分别等同于一个名词性从句的主语和谓体。}
\begin{description}
\item[单宾语及物] She presumed \unbf{that her father was dead}.
\item[复合及物] She presumed \unbf{her father (to be) dead}.
\end{description}

\textbf{介词as 表示连系关系,特别是说明与直接宾语有关的角色或地位。}
\begin{itemize}
\item We considered him $ \left\{
    \begin{aligned}
      &\text{a genius} \\
      &\text{as a genius 补语} \\
      &\text{to be a genius}
    \end{aligned}
  \right. $
\end{itemize}
但在某些方面,介词as和引导比较从句的连词as相似,\textbf{一方面引导从句;另一方面又引
导和从句同位的名词短语}:
\begin{itemize}
\item Report me \unbf{as I am --- a superman}.

\item He described her \unbf{as he found her, a liar}.
\end{itemize}


\begin{table}[htbp]
  \centering
  \begin{talltblr}[ caption = {SVOC 中的非限定性从句},
    label = {tab:svocin},
    ]{
      width=\linewidth, colspec={ll},
      rowsep=2pt, colsep=4pt,
      row{1} = {c, font=\bfseries},
    }
    \toprule
    非限定性从句 & 例句\\ \midrule
    to- 不定式  & They knew him \emph{to be a spy}. \\
    不带to不定式 & I heard someone \emph{slam the door}. \\
    -ing 从句 & I caught Ann \emph{reading my diary}. \\
    -ed 从句 & We saw him \emph{beaten by the German in final}. \\
    \bottomrule
  \end{talltblr}%
\end{table}

\cref{tab:svocin} 中作为\textbf{宾语补语的的非限定性从句}(斜体表示)自身没有
主语,但\textbf{其隐含的主语总是前面的宾语},这样的宾语被叫做\textbf{上升宾
  语 (RAISED OBJECT)}\index{概念!上升宾语 raised object}。语义上,上升宾语是
\textbf{非限定型动词的主语};句法上,它从非限定性从句中上升出来作\textbf{母句的宾语}。
要\textbf{注意歧义},如:
\begin{itemize}
\item Tom left her $\left\{
    \begin{aligned}
      &\text{to finish the job.} \\
      &\text{finishing the job.} \\
    \end{aligned}
    \right.$

    Tom离开她,由她去完成工作。

    \textbf{her是上升宾语。}
\end{itemize}


\subsubsection{双宾语及物 (SVOO) 补足关系}

不同于SVOC中宾语与宾语补语的连系关系;SVOO中两个宾语之间没有连系关系。

\paragraph{宾语和介词宾语}

介词短语可做宾语,大体有以下句型:
\begin{table}[htbp]
  \centering \small
  \begin{talltblr}[ caption = {宾语和介词宾语},
    label = {tab:PrepObj},
    ]{
      width=\linewidth, colspec={},
      rowsep=2pt, colsep=4pt,
      row{1} = {font=\bfseries},
    }
    \toprule
    动词 & 双宾语 & 例句 \\ \midrule
   \SetCell[r=3]{l} told & $O_i + O_d$ & Mary told only John the secret. \\
   & $O_d + O_p$ & Mary told the secret only to John. \\
   & $O_i + O_p$ & Mary told only John about the secret. \\ \midrule
   \SetCell[r=2]{l} offer& $O_i + O_d$ & John offered Mary some help. \\
   & $O_d + O_p$ & John offered some help to Mary. \\ \midrule
   \SetCell[r=2]{l} envy & $O_i + O_d$ & She envied John his success. \\
   & $O_i + O_p$ & She envied John for his success. \\ \midrule
   wish & $O_i + O_d$ & They wished him good luck. \\ \midrule
   \SetCell[r=2]{l} blame & $O_d + O_p$ & He blamed the divorce on John. \\
   & $O_i + O_p$ & He blamed John for the divorce. \\ \midrule
   say &  $O_d + O_p$ & Why didn't anybody say this to me? \\ \midrule
   warn &  $O_i + O_p$ & Mary warned John of the dangers. \\
    \bottomrule
  \end{talltblr}%
\end{table}

\subsection{形容词的补足关系}

\textbf{名词不能做形容词补足语。}

和介词动词一样,形容词经常和后面的介词构成词汇单位:good at, fond of,
opposed to, angry with/about等等。


\begin{table}[htbp]
  \centering \small
  \begin{talltblr}[ caption = {形容词补足语类型},
    label = {tab:adjin},
    ]{
      width=\linewidth, colspec={lX},
      rowsep=2pt, colsep=4pt,
      row{1} = {c, font=\bfseries},
    }
    \toprule
    形容词补足语类型  & 例句   \\ \midrule
    介词短语 & She felt angry \emph{with herself}. \\
    that- 从句 & I am surprised \emph{(that) you didn't call the doctor
      before}.  \\
    wh- 从句 & It was unclear \emph{what they would do}. \\
    to- 不定式 & Bob is sorry \emph{to hear it}. \\
    -ing 从句 & I'm busy (with) \emph{getting the house redecorated}. \\
    \bottomrule
  \end{talltblr}%
\end{table}

\section{名词短语}

名词短语也可以非常复杂,因为句子本身可以被改写,以适用于名词短语结构。例如:
\begin{itemize}
\item The girl is Mary Smith.

\item The girl is tall.

\item The girl was standing in the corner.

\item You waved to the girl when you entered.

\item The girl became angry because you knocked over her glass.
\end{itemize}
以上句子可以组合为一个由很长的名词短语作主语的\textbf{简单句}:
\begin{itemize}

\item \emph{The tall girl standing in the corner who became angry because
    you knocked over her glass after you waved to her when you entered} is
  Mary Smith.
\end{itemize}

\subsection{名词短语的构成部分}
\label{subsec:nounimal}

\begin{description}
\item[中心成分(HEAD)] \index{概念!中心成分@中心成分 Head}被其他成分群集于周围,
  使之构成一致关系的成分,通常是一个名词或代词。中心成分确定了名词短语的句法
  角色(如主语、宾语等)以及与其他句子成分的关系。 如以上名词长句中的girl。

\item[限定成分 (DETERMINATIVE)] 它包括:前位、中心、后位限定词,见 \cref{tab:determ}。

\item[前置修饰 (PREMODIFICATION)] 位于中心词前,除限定词以外的所有成分,一般
  是形容词(短语)和作形容词用的名词,也称“\textbf{定语}”。如:
  \begin{itemize}
  \item some \textbf{expensive} equipment
  \item some \textbf{very very expensive office} equipment
  \end{itemize}

\item[后置修饰 (POSTMODIFICATION)] 位于中心词后的所有词项,有介词短语、非限定
  从句和关系从句和补足语。起后置修饰作用的从句也被称为\textbf{定语分
    句}。\index{概念!后置修饰@后置修饰 post-modification} \index{概念!定语分
    句@定语从句attributive clause}
  \begin{description}
  \item[介词短语] the car \emph{outside the station}
  \item[非限定性从句] the car \emph{standing outside the station}
  \item[关系从句] the car \emph{that stood outside the station}
  \item[补足语] a bigger car \emph{than that}
  \end{description}

\end{description}


whose是物主关系代词,像his,her, its, there 一样用在名词前作限定词。后置修饰
语(关系从句)中,不管是人或物都可以用whose。有人认为用whose指代物品不合适,
可以用of which 或that \ldots{} of替代:
\begin{itemize}
\item I saw a girl \unbf{whose beauty} took my breath away.

  我见到一个女孩,她的美貌让我十分惊异。(主语)

\item It was a meeting \unbf{whose purpose} I did not understand.

  这个会议的目的我不明白。(宾语)


\item I went to see my friends, the Jims, \unbf{whose children} I used to
  look after when they were small.

  我去看我的朋友,吉姆一家,他家孩子很小时曾由我照料过。


\item He's written a book \unbf{whose name}/\unbf{the name of which} I've forgotten.
\item He's written a book \unbf{that} I've forgotten \unbf{the name of}.
\item He's written a book \unbf{of which} I've forgotten \unbf{the name}.
\end{itemize}


\subsection{限制性和非限制性修饰语}

修饰语可以是\index{概念!限制性修饰语}\index{概念!非限制性修饰语}
\begin{description}
\item[限制性修饰语] 限制性修饰语用来\textbf{限定或明确其修饰的名词或代词},提
  供关键性的信息。如果去掉限制性修饰语,句子的意思就不完整,或者会导致误解。
  它\textbf{不需要用逗号隔开},因为这些信息对理解句子至关重要。在阅读时,限
  制性修饰语通常紧跟在其所修饰的名词(先行词)之后,\textbf{没有间隔停顿}。
  \begin{itemize}

  \item Come and see my \unbf{younger} daughter.

    younger暗示是说话者两个女儿中的小女儿,是确定的,是限制性修饰语,也被称
    为“\textbf{定语}”。

  \item The students \unbf{who study hard} will pass the exam.

  \item I need the book \unbf{that you borrowed from the library}.

    上两句起限制性修饰语作用的从句也被叫做\textbf{限制性关系从句}。

  \end{itemize}


\item[非限制性修饰语] 非限制性修饰语提供的是\textbf{附加的、非关键性的信息}。
  即使去掉这些修饰语,句子的基本意思仍然成立。通常\textbf{用逗号隔开},因为信
  息不重要。
  \begin{itemize}
  \item My brother\unbf{, who lives in New York,} is coming to visit.
  \end{itemize}

\end{description}

\textbf{当名词短语中心语确定无疑义时(如专有名词、人名、隐含或明示的特指),
  其任何修饰语都将是非限制性的。}
\begin{itemize}
\item Mary Smith\unbf{, who is in the corner,} wants to meet you.
\item Come and meet my \unbf{beautiful} wife. [一夫一妻制度下]


\item He \unbf{likes dogs}, \unbf{which surprises me}.

  如上,\textbf{非名词性先行词不可能带有非限制性修饰语}:


\item \sout{I won't see \unbf{anyone}, \unbf{who has not made an
      appointment}.} [误,可以去修饰语前后的逗号,转限制性从句]

  \textbf{非断定的中心词}如any- 等\textbf{不能有非限制性关系从句},因
  为:any- 等非断定代词本就无法限定,再施加非限制性关系从句毫无意义。
  像 any, all 和 every 等非特指的限定词通常只有限制性修饰语。

  \begin{itemize}
  \item \sout{All the students\unbf{, who had failed the test,} wanted to
      try again.} [误,应去修饰语前后的逗号,转为限制性从句]

  \end{itemize}

  偶尔这样的句子里也可以使用(非)限制性修饰语。如下:
  \begin{itemize}
  \item All the students, \unbf{who had returned from their vacation}, wanted
    to take the exam.
    所有学生都是度假归来的,“度假归来”只是附加信息。
  \end{itemize}

\end{itemize}

\subsection{限定性从句作后置修饰语}

\begin{description}
\item[关系从句] that 和wh-  代词可以互相替代;that 本身就是\textbf{从句成分},
  如下例句分别作为主语和宾语:
  \begin{itemize}
  \item The news \unbf{that appeared in the papers this morning} was well received.
  \item The news \unbf{which we saw in the papers this morning} was well received.
  \end{itemize}

\item[同位从句] 同位语从句(Appositive clause)是一种名词性从句,它在复合句
  中充当同位语,通常用于进一步解释或说明前面的名词。

  同位从句中连接词\textbf{只能是that},不能用wh-代词替代,\textbf{that也不构
    成从句成分}。名词短语中心词应是一个\textbf{概括性抽象名词},例如fact,
  idea, proposition, reply, remark, answer 等等。
  \begin{itemize}
  \item \unbf{The news that the team had won} calls for a celebration.

  \item I agree with \unbf{the old saying that absence makes the heart grow fonder}.

    absence makes the heart grow fonder 习语,不相见,倍思念。
  \end{itemize}
  因为是同位关系,所以用be动词把它和被同位的单位连接起来后仍语法正确。
  \begin{itemize}
  \item The news \unbf{IS} that the team had won.

  \item The old saying \unbf{IS} that absence makes the heart grow fonder.
  \end{itemize}
\end{description}

\subsection{限制性关系从句}

关系代词可以有如下能力:
\begin{description}
\item[表明与它的先行词的一致关系] 先行词就是名词短语中的先行部分,关系从句是这
  个部分的后置修饰语。 [外在关系]
\item[表明它在关系从句中的作用] 或是作为从句结构中的一个成分 (S, O, C, A), 或
  是作为关系从句中一个成分的构成要素。 [内在关系]

\end{description}

在非限制性关系从句中,关系代词普遍是 wh­ 系列 (who, whom, which, whose)。

在限制性从句中,除 wh- 代词外也常用 that 或零关系代词。that 没有宾格
(who/whom);也没有属格 (who 和 which 的 whose),不能在关系从句中充当某一个成
分的一个构成因素。


\begin{table}[htbp!]
  \centering \small
  \begin{talltblr}[ caption = {关系代词可以在关系从句中充当的成分},
    label = {tab:relaxpro},
    ]{
      width=\linewidth, colspec={lXl},
      rowsep=2pt, colsep=4pt,
      row{1} = {font=\bfseries},
    }
    \toprule
    成分 & 例句 & 备注 \\ \midrule
    S & They are delighted with the person \emph{who/that} has been appointed. & 不可省略 \\
    O & They are delighted with the person \emph{who(m)/that} we have
    appointed. & 可省略 \\
    C & She is the perfect accountant \emph{which}/\sout{who/that} her
    predecessor was not. & 争议。 \\
    A &  He is the policeman \emph{at whom} the thief shot. & 介词后whom \\
    A & He is the policeman \emph{who(m)/that} the thief shot \emph{at}.   & 可省略 \\
    A & She arrived the day \emph{on which} I was ill. & 介词后不接that \\
    A & She arrived the day \emph{which/that} I was ill \emph{(on)}. &  \\
    \bottomrule
  \end{talltblr}%
\end{table}

就that/which/who 谁能做\textbf{关系从句中的补语},尚存不少争议。夸克认为\textbf{当关系
  代词在关系从句中充当非介词的补足语时,人称或非人称先行词后都只能用which.}
我不知道其他专家的想法。对此暂不展开。

如果先行词是人称名词,关系代词可以表明 who 与 whom 的区别,这取决于关系代词
在关系从句中充当主语还是宾语,抑或介词补足语:
\begin{itemize}
\item The person $\left\{
    \begin{aligned}
      & \text{\emph{who} spoke to him \ldots} &\text{[主语]}  \\
      & \text{\unbf{to whom} he spoke \ldots} &\text{[介词补足语,过于正式]}  \\
      & \text{\emph{who(m)} he spoke \emph{to} \ldots} &\text{[介词补足语]} \\
      & \text{\emph{who(m)} he met \ldots} &\text{[宾语]} \\
    \end{aligned}
  \right. $
\end{itemize}


夸克认为存在句和分裂句中可加可不加的that和wh- 从句并不是形容性关系从句。
\todo[inline]{补全存在句和分裂句意思}
\begin{itemize}
\item There's a table (that/which) stands in the corner. [存在句]

\item It was Simon (that/who) did it. [分裂句]
\end{itemize}

当先行词有一个\textbf{最高级形容词或后置限定词之一} (first, last, next,
only) 修饰时,充当关系从句\textbf{主语}的关系代词通常是 \textbf{that},充当宾
语的关系代词用 \textbf{that 或零形式}多于用 which 或 who(m):
\begin{itemize}
\item She must be one of the most remarkable women \unbf{that ever lived}.

  that 充当关系代词,引入关系从句,其先行词是“women”。
\end{itemize}

\textbf{英语中并不存在接在先行名词之后、与 where, when, why 平行并表示方式的
  关系词 how:}
\begin{itemize}

\item That's \unbf{the way} \sout{how/}\unbf{that} she spoke.

\item That's \unbf{how} she spoke. 正确,名词性从句。
\end{itemize}

在表示\textbf{时间}时,不管关系代词是 \textbf{that 还是零形式},通常\textbf{省略介词}:
\begin{itemize}
\item What's the time \unbf{that} she normally arrives \unbf{(at)}?
\item What's the time \unbf{when} she normally arrives?
\end{itemize}

可是,当代词为 \textbf{which} (使用频率较低,更加正式)时,在下列三个例子中
都\textbf{必须用介词},而且通常\textbf{介词放在代词之前}:
\begin{itemize}
\item 5:30 is \unbf{the time at which} she usually arrives. [at which =
  when]
\item I don't remember \unbf{the day on which} she left.

\item He worked for three years \unbf{during which} he lived there.
\end{itemize}

表示方法和理由时,通常用零结构(偶尔用that),\textbf{不用介词}。
\begin{itemize}
\item That's the way \unbf{(that)} he did it.

  That's \unbf{how} he did it. [不含 the way]


\item Is that \unbf{the reason}/\unbf{why} they came?
\end{itemize}

然而,在类似 reason 和 way 表示状语关系意义的其他名词之后,\textbf{如用that 或
  零关系词就必须接介词}。
\begin{itemize}
\item This is \unbf{the style} he wrote it \unbf{in}.

\item Is this \unbf{the cause of} her coming?
\item Is this \unbf{motive of} her coming?
\end{itemize}



\subsection{用非限定性从句作后置修饰语}

\subsubsection{用 -ing 分词从句作后置修饰语}

三种非限定性从句都可以充当名词短语的后置修伤语: ing 分词从句, -ed 分词从句和
不定式从句。

\textbf{-ing 从句和关系从句的对应关系只限于以关系代词为主语的那些关系从句}:
\begin{itemize}
\item The person \emph{who}$\left\{
    \begin{aligned}
      &\text{\emph{will write}} \\
      &\text{\emph{will be writing}} \\
      &\text{\emph{writes}} \\
      &\text{\emph{is writing}} \\
      &\text{\emph{wrote}} \\
      &\text{\emph{was writing}} \\
    \end{aligned}
  \right\} $ \emph{reports} is my colleague.

\item The person \unbf{writing reports} is my colleague.
\end{itemize}

必须强调的是,后置修饰从句中的 -ing 形式不应该看作是关系从句中进行体的缩略形
式。例如,静态动词在限定性动词短语中不可能用进行式,但可以用于分词形式
中。
\begin{itemize}
\item It was a mixture \unbf{consisting} of oil and vinegar.

  不能说是that was consisting(静态动词)的缩写。
\end{itemize}
其实我认为可以这样理解,现在分词从句既可以可是关系从句中进行体的缩略形式,也
可以是关系从句中(不带连系动词)一般式的缩略形式。


\subsubsection{用 -ed 分词从句作后置修饰语}

和 -ing 从句一样, -ed 分词从句仅与\textbf{关系代词作主语}的关系从句相对应:
\begin{itemize}
\item The car \emph{that}$\left\{
    \begin{aligned}
      &\text{\emph{will be repaired}} \\
      &\text{\emph{is (being) repaired}} \\
      &\text{\emph{was (being) repaired}} \\
    \end{aligned}
  \right\} $ by that mechanic.

\item The car \unbf{(being) repaired}  by that mechanic.
\end{itemize}

\subsubsection{用不定式从句作后置修饰语}

与 -ing 和 -ed从句不同,不定式从句所对应关系从句中的关系代词除了可以是\textbf{从句主
语}外,还可以是\textbf{宾语}或\textbf{状语},有限范围内还可以是\textbf{补语}。
\begin{description}
\item[主语] The man \unct{to help you}{who can help you} is Mr Johnson.

\item[宾语] The man \unct{(for you) to see}{who(m) you should see} is Mr Johnson.

\item[补语] 不懂,略。\improve[inline]{大全17.30 看不懂,需补充}

\item[状语] The time \unct{(for you) to go}{when (you) should go} is July.

\item[状语] The place \unct{(for you) to stay}{where (you) should stay} is
  this room.  [where = at which,另从句主语可省]
\end{description}

因省略了连词、主语、be助动词等,-ing从句, -ed从句,不定式从句都存模糊性。

\subsubsection{带 to 不定式或 of 短语的结构}

由不定式从句作同位后置修饰语\textbf{可能没有对应的同位限定性从句},而只有一个
介词短语和它相对应:
\begin{itemize}
\item He lost \unbf{the ability} \unbf{to use his hands}. [of using 可代替 to use]

  He lost the ability \sout{that he could use his hands.}

\item \unbf{Any attempt} \unbf{to leave early} is against regulations. [at
  leaving 可替代 to leave]

  Any attempt \sout{that one should leave early} is against regulations.
\end{itemize}

\todo[inline]{何时用to不定式,何时用介词+ing,夸克说的较乱,理解困难,日后整理。}

\subsection{用介词短语作后置修饰语}

我们还可以用介词短语作后置修饰语,如下最后一句例句:
\begin{itemize}
\item The car \emph{was standing} outside the station.
\item the car \emph{which was standing} outside the station
\item the car \emph{standing} outside the station
\item the car \unbf{outside the station}
\end{itemize}

在英语中,介词短语肯定是后置修饰语中最常见的类型.

\begin{itemize}
\item the road \unbf{to Lincoln}
\item this book \unbf{on grammar}
\item a man \unbf{from the electricity company}
\item action \unbf{in case of fire}
\item the meaning \unbf{of this sentence}
\item the house \unbf{beyond the church}
\item two years \unbf{before the war}
\item some trees \unbf{along the river bank}
\item the university \unbf{as a political forum}
\end{itemize}

\subsection{名词化}

我们应该把介词短语作后置修饰语看作是其作状语的特殊例子。

\begin{description}
\item[名词化 (nominalization)] 在语法上将其他词性(如动词或形容词)派生为名词
  形式,使原本表达动作或状态的词汇变成了表示事物、概念或状态的名词,从而使句
  子的结构和信息传达方式发生变化。


  例句:
  \begin{taskitem}(2)
    * his \unbf{refusal} to help [n.]
    * he \unbf{refuses} to help [v.]
    * the \unbf{truth} of her statement [n.]
    * her statement is \unbf{true}  [v.]
    * her \unbf{friendship} for Jim [抽象名词]
    * she is a \unbf{friend} of Jim [具体名词]
  \end{taskitem}
\end{description}

在英语中,动词通常通过添加后缀如 ``-ment'' , ``-tion'', ``-ure'' 等来进行名词
化(详见\cref{tab:mainsuffix})。例如:
\begin{taskitem}(3)
* entertain → entertainment
* act → action
* fail → failure
\end{taskitem}
此外,一些动词和形容词可以直接作为名词使用,而不需要添加任何后缀,例
如``change''、``increase'' 和``use''。

\subsection{后置修饰的次要类型}

\subsubsection{副词短语作后置修饰}

略




\section{主位 (Theme)、中心 (Focus) 和信息处理 (Information processing)}


\subsection{前置 (Fronting)}

\index{概念!前置 Fronting}
\begin{description}
\item[前置] 把一个非主语词项从它通常所在位置移到句首,使其成为中心内容——即
  使其不是主语。
\end{description}

前置成分可以是:
\begin{description}
\item[宾语]
  \begin{itemize}
  \item \unbf{People like that} I just can't stand.
  \item \unbf{This question } we have already discussed many times.

  \item  \unbf{What I'm going to do next} I just don't know.
  \end{itemize}
\item[补语] 较为少见,且现代英语中更少见。
  \begin{itemize}
  \item \unbf{Fool} that I was.
  \end{itemize}

\item[副词和状语]
  \begin{itemize}
  \item \unbf{Here} we go.

  \item \unbf{Once upon a time} there were three little pigs.

    once upon a time, “从前”(用于故事开头),习语。
  \item \unbf{Round the corner} came Mrs Porter.

  \end{itemize}
\item[as 和 though时] 形容词或副词前置
  \begin{itemize}
  \item \unbf{Young as I was}, I realized what was happening.

  \item \unbf{Tired though she was}, she went on working.

  \item \unbf{Much as I respect his work}, I cannot agree with him.

  \item \unbf{Genius as you may be}, you have no right to be rude to your teachers.

  \end{itemize}
\end{description}


\subsection{倒装 (Inversion)}
\label{subsec:inversion}

\index{概念!倒装 inversion} 一个成分的提前,往往牵涉到语序的倒装,有时涉及细
微的句法调整。
\begin{description}
\item[主语和动词的倒装] SVC 和 SVA 类型的从句大部分有必不可缺的第三成分,因为
  动词自身通常非常缺乏传达动力。如果只是将补语 C或状语A 提前,会把重心末在末短的
  动词上,有可能造成语义不清,如:
  \begin{description}
  \item[SVC] \unbf{The sound of the bell} \unbf{grew} faint.
  \item[CSV] \unbf{Faint} \unbf{the sound of the bell} grew. [不清晰,有可能误解为逐渐变小的铃声
    变大grew (louder)]
  \item[CVS] Faint \unbf{grew} \unbf{the sound of the bell}. [有矫揉造作感。]
  \end{description}
  还有:
  \begin{description}
  \item[SVA] \unbf{The Squidward’s hopes and dreams} \unbf{lies} here.

    章鱼哥的希望和梦想埋葬(躺)在这里。尾重,突出here。

  \item[AVS] Here \unbf{lies} \unbf{the Squidward’s hopes and dreams}. [尾重,
    在埋葬lies这样沉重悲伤的词之后\textbf{突出主语},渲染到位。]

    这里埋葬着\textbf{章鱼哥的希望和梦想}。
  \end{description}
  日常对话有些倒装句:
  \begin{description}
  \item[AVS] Here \unbf{comes} \unbf{my brother}.

  \item[AVS] Down \unbf{came} \unbf{the rain}.
  \item[AVS] Up \unbf{went} \unbf{the flag}.
  \end{description}

  主语和动词的倒装OVS中,前移的宾语O主要表示直接话语,并且主语往往不是人称代词:
  \begin{itemize}
  \item ``Please go away,'' said one child.

    ``And don't come back,'' pleaded another.
  \end{itemize}
  这个例子有点书面化,日常话语中一般用SV替代VS,如 ``one child said'',
  ``another pleaded''.

\item[主语和功能词的倒装]
  \begin{description}
  \item[so, neither, nor]
    \begin{itemize}
    \item John saw the accident and \unbf{so did Mary}. [Mary did so]

    \item He couldn't speak, \unbf{nor could he walk}.
    \item She wasn't angry and \unbf{neither was I}. [I wasn't, either]
    \item She must come and \unbf{so must you}. [you must come too]

    \item 如果想将重心放在动词上,也可只前置 so.

      You asked me to leave, \unbf{so} I did.
    \end{itemize}

    另外要注意到,so, neither, nor以上例句中主要动词均被省略,只有功能助动词。

  \item[否定形式或意义的短语前置] 正式文体中,形式和意义上否定整个从句的成分
    前置,且主语和功能词往往需要倒装。
    \begin{itemize}
    \item This door must be left unlocked at no time.

      \unbf{At no time} must \unbf{this door} be left unlocked.

    \item \unbf{Only in this way} is \unbf{it} possible to explain their actions.

    \item \unbf{Not a single book} had \unbf{he} read that month.

    \item \unbf{Not a word} whould he say.

    \item \unbf{No longer} are they staying with us.


    \item \unbf{N\'OT \`until yesterday} / did he CH\`ANGE his mind.
    \end{itemize}

  \item[主语不是人称代词的比较从句]
    \begin{itemize}

    \item \unbf{I} spend more than \unbf{do my friends}.

    \item \unbf{Oil} costs less than \unbf{would atomic energy}.

      \unbf{Oil} costs less than did \sout{it}. [\textbf{人称代词作主语的 than从句不能倒装}]

    \item She is a doctor, \unbf{as is he}. [\textbf{人称代词作主语
        的 as从句可以倒装}]

    \item They go to concerts frequently, \unbf{as do I}.

    \end{itemize}

  \item[条件和让步从句]
    \begin{itemize}
    \item \unbf{Were} \unbf{she} alive today, she would grieve at the changes.
    \item \unbf{Had} \unbf{I} known, I would have gone to her.
    \item \unbf{Should} \unbf{you} change your plans, please let me know.
    \end{itemize}
  \end{description}
\end{description}

\subsection{分裂句和假拟分裂句}
\label{subsec:cleftsen}

\index{概念!分裂句 cleft sentence} 分裂句是是将一个句子分裂成两个从句,其中一
个从句的某成分为\textbf{信息焦点},另一从句为处于支配地位、提供背景说明
的\textbf{关系从句},借此突出信息焦点的句式。通常是为了强调动作的执行者、动作
本身或是动作发生的时间、地点等。

\begin{description}
\item[it 子句分裂句] It + BE + \textbf{focus} + (that, who, whom, which, 零)关系从句,
  如:
  \begin{itemize}
  \item It is \unbf{Simon} \uline{who’s gone down}.

  \item It was \unbf{here} \uline{that the young girl first fell in love}.


  \item It was \unbf{this matter} \uline{on which I consulted with Dr. Richard}.

    我就此事与理查德博士进行了磋商。
  \end{itemize}


\item[wh- 假拟分裂句] Wh+ 主语 + 谓语 + 其他成分 + BE + \textbf{focus}


  焦点可以是:
  \begin{description}
  \item[名词短语] \uline{What I saw} was \unbf{one of the most impressive government policies in years}.

    我所看到的是近年来最令人印象深刻的政府政策之一。
  \item[动词短语] \uline{What you do} is \unbf{wear it like that}.

    你要做的是像那样穿衣。

   \uline{What I did} was \unbf{send a complaint to Radio 2}.

   我要做的是像2号电台投诉。[send 动词原形]

 \item[关系从句]
   \uline{What you’ll find} is \unbf{that people who lie down with
     dogs will get up with fleas}, my boy.

   你会发现近墨者黑,我的儿子。(改编自西方谚语,字面意思是跟狗躺在一块儿的人,
   会和跳蚤一起起床。)

   \uline{What I didn’t like} was \unbf{leaving my mum}.

  \end{description}

  wh- 从句可以和焦点的位置颠倒:
  \begin{itemize}
  \item \unbf{Send a complaint to Radio 2} is \uline{what I did}.

  \item \unbf{Leaving my mum} was \uline{what I didn’t like}.
  \end{itemize}
\end{description}

动词成分绝对不能充当分裂句的焦点,正如它不能以一个 wh- 成分出现一样。


\subsection{存在句}
\label{subsec:behave}

在默认结构中,一般默认主位为已知信息。如果期望听话者把主位理解为全新的信
息——与以前介绍过的任何事物没有关系,便利的做法是提供某种\textbf{假位主位},
使听者将整句话理解为全新信息。

除there be 外,还有其他假位主位。

\begin{itemize}
\item \unbf{There is} / \unbf{I have} a car blocking my way.

\item\unbf{ There are} some people (that) I'd like you to meet.

\item $\left.
    \begin{aligned}
      \text{There are} \\
      \text{We have} \\
      \text{One finds} \\
      \text{It's a fact that} \\
    \end{aligned}
  \right\}$ many students are in financial trouble.
\end{itemize}



%%% Local Variables:
%%% mode: LaTeX
%%% TeX-master: "main"
%%% End:
