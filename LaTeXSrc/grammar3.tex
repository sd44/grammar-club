
\chapter{简化从句练习}

\textbf{一、对等从句中,相对应位置(主语与主语,动词与动词等)如果重复,择一弹性省略。}

\textbf{二、从属从句(名词从句、关系从句与状语从句)中,省略主语与 be动词两部分,
  留下补语。不过主语若非重复或空洞的元素,就应设法保留,以免句意改变。}

这两项原则的共同目的都是为了增强句子的精简性:尽量删除两个从句间重复或空洞的
元素,但以不伤害清楚性为前提。现在就借一些例句的组合来练习如何写作高难度的句
型。

\paragraph{例一}

\begin{enumerate}
\item The patient had not responded to the standard treatment.

  病人对标准疗法没有反应。
\item This fact greatly puzzled the medical team.

  医疗小组对此深感不解。
\end{enumerate}

这两个简单句中,句 2 的主语 this fact 指的就是整个句 1叙述的那件事。两句经由
这个交叉建立了关系,可以考虑用关系从句(即关系从句)连结起来。亦即把句2 的
交叉点 this fact 改写为关系词,附于句 1 上作关系从句,成为:
\begin{itemize}
\item The patient had not responded to the standard treatment, which greatly
  puzzled the medical team. (不够清楚)
\end{itemize}

如此组合这两句话,短语上看来可以,但修辞上有严重的缺点:关系词 which固然可以
代表逗点前的整句话(表示病人缺乏反应这一点令人困惑),但是它也可以代表逗号前
面的名词the standard treatment(表示标准治疗方式本身令人困惑)。如此一来,一
个句子有两种可能的解释,犯了模棱两可(ambiguous)的毛病,也就是没有把意义表达
清楚,不如尝试另一种组合方式。

既然整个句 1 是句 2 主语 this fact的内容,不妨把它改成名词从句(前面加上连接
词 that 即可),然后直接置于句2 中 this fact 的位置当主语使用,成为复句:
\begin{itemize}
\item \unbf{That the patient had not responded to the standard treatment} greatly
  puzzled the medical team.
\end{itemize}
这个句子中的名词从句(that引导的从句)可再进一步简化,一般做法是删除主语
与 be动词。但这个从句中主语是 the patient,在主要从句中并无重复,无法省略。动
词 had not responded其中也没有 be 动词可以省略,那么该怎么做?首先,动词简化
的通用原则是:

\textbf{一、有 be 动词即省略 be 动词;}

\textbf{二、有情态助动词(can、must、should 等)则改为不定式(to V);}

\textbf{三、除此之外的动词一律加上 \emph{-ing} 保留下来。}

以 had not responded 这个动词短语而言,符合第三种情形,所以改写为 not having
responded,以取代原先的名词从句。原来的主语 the patient改为所有格(the
patient's)置于前面,再删除无意义的连接词 that即完成了简化的动作,成为:
\begin{itemize}
\item \unbf{The patient's not having responded to the standard treatment}
  greatly puzzled the medical team.
\end{itemize}

另外,也可以直接进行词类变化,把动词改写为名词后,成为:
\begin{itemize}
\item \unbf{The patient's failure to respond to the standard treatment} greatly
  puzzled the medical team.
\end{itemize}

这种讲法读起来会比上一种讲法更自然一点。

\paragraph{例二}

\begin{enumerate}
\item The summer tourists are all gone.

  夏季的观光客都走光了。
\item The resort town has resumed its air of tranquility.

  这个度假小镇又恢复了平静。
\end{enumerate}

这两句话之间没有重复的元素,但有逻辑关系存在:在观光客走了之后,或是因为观光
客都走了,小镇才得以恢复平静。这时可以用状语从句的方式,选择恰当的连接词
(after、because、now that 等)附在句 1 前面,再把句 1 与句 2 并列即可:
\begin{itemize}
\item \unbf{Now that the summer tourists are all gone}, the resort town has
  resumed its air of tranquility.
\end{itemize}
Now that 引导的状语从句若要进一步简化,关键在主语、动词两个部分。主语the
summer tourists与主要从句并无重复,必须保留下来以免损害句意。动词部分有 be动
词(are),后面还有补语(gone)。这时若去掉 be动词,留下主语与补语,就破坏了
这个状语从句的结构,可以省略连接词 now that,成为:
\begin{itemize}
\item \unbf{(With) the summer tourists all gone}, the resort town has resumed its air of tranquility.
\end{itemize}
如果最前面没有加上 with,而是以 the summer tourists all gone直接代表一个简化
的状语从句,这种讲法比较文诌诌,不够口语化。

较口语化的做法是,用介词 with 来取代连接词 now that 的意义,而把 the
tourists 放在 with 后面作它的宾语, all gone仍然作补语,即成为上句中多一
个 with 在前面的句型。

\paragraph{例三}

\begin{enumerate}
\item Confucius must have written on pieces of bamboo.

  孔子当年一定是在竹简上写字。
\item Confucius lived in the Eastern Zhou Dynasty.

  孔子是东周时代的人。
\item Paper was not available until the Eastern Han Dynasty.

  纸到东汉时期才有。
\end{enumerate}

这三句话中,句 1 和句 2有一个交叉:Confucius。经由这个交叉点建立关系,可用关
系从句的方式连结,将句2 的 Confucius 改写为关系词 who,成为:
\begin{itemize}
\item (1+2) Confucius, \unbf{who lived} in the Eastern Zhou Dynasty, must have
  written on pieces of bamboo.
\end{itemize}
这个关系从句(who lived in the Eastern Zhou Dynasty)可以进行简化,省略重复的
主语 who,再把普通动词 lived 改写为living,即成为简化关系从句:
\begin{itemize}
\item Confucius, \unbf{living} in the Eastern Zhou Dynasty, must have written on
  pieces of bamboo.
\end{itemize}
东周时代的孔子为什么要用竹简写字?是因为句 3:纸到东汉时期才有。句 3的内容表
示原因,所以用状语从句的方式——外加连接词 because成为状语从句,与主要从句并
列,即得到:
\begin{itemize}
\item (+3) Confucius, living in the Eastern Zhou Dynasty, must have written on
  pieces of bamboo, \unbf{because paper was not available until} the Eastern Han Dynasty.
\end{itemize}

句中的状语从句(because之后的部分)如要进一步简化,又要观察主语与动词部分。主
语 paper没有重复,必须留下来。动词虽然是 be动词,可是\textbf{状语从句的简化中,一旦留
下主语,就得有个分词配合(传统语法称为分词构句)},所以使用be 动词来制造分
词 being,并省略连接词 because,即成为简化的状语从句:

\begin{itemize}
\item Confucius, living in the Eastern Zhou Dynasty, must have written on
  pieces of bamboo, \unbf{paper not being available} until the Eastern Han Dynasty.
\end{itemize}

\paragraph{例四}

\begin{enumerate}
\item The movable-type press was invented by Gutenberg.

  古登堡发明活版印刷。
\item The movable-type press was introduced to England in 1485.

  活版印刷在 1485 年引进英国。
\item This event marked the end of the Dark Ages there.

  这件事标示英国黑暗时期的结束。
\end{enumerate}

这个例子中的句 1 和句 2 也有一个交叉:the movable-type press,可以将它改写为
关系词 which,以关系从句方式连接:
\begin{itemize}
\item (1+2) The movable-type press, \unbf{which was} invented by Gutenberg, was
  introduced to England in 1485.
\end{itemize}
这个关系从句(which 引导的部分)可以直接简化,省略主语 which 和 be 动词
was,只保留补语 invented 这个部分,即成为简化的关系从句:
\begin{itemize}
\item The movable-type press, \unbf{invented} by Gutenberg, was introduced to England
  in 1485.
\end{itemize}
句 3 中的主语 this event(这个事件)指的就是上面整句话的那个事件。这时候因为
上面的句子比较长,可以先加个同位语an event,再用它和句 3 主语 the event 的交
叉构成关系从句,成为:
\begin{itemize}
\item (+3) The movable-type press, invented by Gutenberg, was introduced to
  England in 1485, \unbf{an event which marked} the end of the Dark Ages there.
\end{itemize}
要进一步简化这个句子,可以把重复部分 an event 删除,再省略关系从句的主
语which,把动词 marked 改成分词 marking:
\begin{itemize}
\item   The movable-type press, invented by Gutenberg, was introduced to
  England in 1485, \unbf{marking} the end of the Dark Ages there.
\end{itemize}

\paragraph{例五}

\begin{enumerate}
\item Ben Kook was educated in an art college.

  本·库克曾在一所美术学院念书。
\item Ben Kook acts unusual at times.

  本·库克有时表现得与众不同。
\item Ben Kook deals with economic matters at these times.

  这时本·库克处理经济事务。
\end{enumerate}

句 1 和句 2之间有因果关系:因为在艺术学院读过书,所以才有与众不同的表现。那么
就在句1 前面加上连接词 because 成为状语从句,与句 2 的主要从句并列,成为:
\begin{itemize}
\item (1+2) \unbf{Because he was educated in an art college}. Ben Kook acts unusual at
  times.
\end{itemize}
这个句子中,简化 because 引导的状语从句,可以直接省略 he was,再把连接
词because 删去,只保留补语 educated 部分,成为:
\begin{itemize}
\item \unbf{Educated} in an art college, Ben Kook acts unusual at times.
\end{itemize}
这个句子要与句 3 连结,可以观察到句尾的 at times 就是句 3 结尾部分的 at
these times。以这个交叉改写为关系词 when,构成关系从句的形态:
\begin{itemize}
\item (+3) Educated in an art college, Ben Kook acts unusual \unbf{(at times)
    when he deals} with economic matters.
\end{itemize}
句中括弧部分的 at times 是副词类,属于次要元素,又与后面的 when重复,可以先行
省略。进一步的简化做法仍是一样:把主语 he 省略,动词 deals改成 dealing。不过,
由于原先的 at times 已经省略,所以与它重复的 when不宜省略。把 when 留下来,即
成为:

\begin{itemize}
\item Educated in an art college, Ben Kook acts unusual \unbf{when dealing} with
  economic matters.
\end{itemize}

\paragraph{例六}

\begin{enumerate}
\item   I'd like something.

  我希望一件事。
\item   You will meet some people.

  你去见见一些人。
\item   Then you can leave.

  然后你就可以走了。
\end{enumerate}

句 1 中的宾语 something 就是整个句 2 叙述的那件事,所以在句 2前面加上一个连接
词 that,成为名词从句,然后放入句 1 中 something的位置作为 like 的宾语:
\begin{itemize}
\item (1+2) I'd like \unbf{that you (will) meet} some people.
\end{itemize}
附带提一下,1+2 合并时,that 从句的语气成为祈使句的语气,所以助动词 will应省
略成动词原形,但简化时仍变成不定式。以下的例子若看到助动词上加个括弧都是同样
的原因。这里的名词从句要简化时,因主语you 与主要从句并无重复,所以要留下来,
把它放在 like后面的宾语位置。简化从句的做法是把助动词简化为不定式 to V,因为
情态助动词 must、should、will(would)、can(could)、 may(might)等都可以改写
成 be+to 的形式。省略 be 动词后就剩下to,所以上面这个从句中的 will meet 就改
成 to meet 当补语用,成为:
\begin{itemize}
\item I'd like \unbf{you to meet} some people.
\end{itemize}
再把句 3 加上去。句 3 是表示时间,可以用连接词 before 把它改成状语从句:
\begin{itemize}
\item (+3) I'd like you to meet some people \unbf{before you (can) leave}.
\end{itemize}
这个状语从句若进一步简化,得把 before 留下才能表达“在……之前”的意思。
但 before这个连接词也可当介词用,一旦后面的从句简化了,它就成为介词,只能
接名词形态。因此把重复的主语you 省略后,原来的动词 leave 要改成动名
词 leaving 的形态,成为:
\begin{itemize}
\item I'd like you to meet some people \unbf{before leaving}.
\end{itemize}

\paragraph{例七}

\begin{enumerate}
\item I have not practiced very much.

  我练习得不多。
\item I should have practiced very much.

  我应该多练习。
\item I am worried about something.

  我担心一件事。
\item I might forget something.

  我可能忘记什么事。
\item What should I say during the speech contest?

  在演讲比赛中我该说些什么?
\end{enumerate}

句 1 和句 2 可以用比较级 as \ldots{} as 的连接词合成复句:
\begin{itemize}
\item (1+2) I have not practiced \unbf{as much as I should} (have practiced).
\end{itemize}
因为“练习不够”,才会造成句 3 “我很担心”的结果。表示这种因果关系,可以使
用 because的状语从句来连接:
\begin{itemize}
\item (+3) \unbf{Because I have not practiced} as much as I should, I am
  worried about something.
\end{itemize}
Because 引导的状语从句,简化时可把重复的主语 I 省略。动词部分 have not
practiced 因为没有 be 动词,也没有情态助动词,就只能加上 \emph{-ing},成
为 not having practiced,再把连接词 Because 删去,成为:
\begin{itemize}
\item \unbf{Not having practiced} as much as I should, I am worried about something.
\end{itemize}
这个句子中,“担心的事情” something,就是句 4的内容“我可能会忘记什么事”。
因为 something 是放在介词 about的后面,要连成复句的话可以先改成 about the
possibility,再把句 4加上连接词 that,形成名词从句,作为 possibility 的同位语,
成为:
\begin{itemize}
\item (+4) Not having practiced as much as I should, \unbf{I am worried (about the
  possibility)} that I might forget something.
\end{itemize}
这个句子中的介词短语 about the possibility 意思和下文的 that从句重复,可以
省略。但是如果要简化其后的 that 从句,就得把介词 about留下来,简化的结果才
有地方安置。that 从句的简化,省去重复的主语 I之后,动词 might forget 的简化一
般是改成不定式 to forget。可是现在要放在介词 about 后面,不能用不定式的形态,
只能改成forgetting:
\begin{itemize}
\item Not having practiced as much as I should, I am worried \unbf{about forgetting}
  something.
\end{itemize}
现在,这个句子中“担心会忘记的”那件 something,就是句 5的问题:“演讲比赛该
说什么?”只要将这个疑问句改成非疑问句,就是一个名词从句,可直接取代上句中
的something,作为 forget 的宾语:
\begin{itemize}
\item (+5) Not having practiced as much as I should, I am worried about
  \unbf{forgetting what I should say} during the speech contest.
\end{itemize}
最后一步是简化 what 引导的名词从句。做法一样:省略主语 I,动词 should
say 改为不定式 to say:
\begin{itemize}
\item Not having practiced as much as I should, I am worried about forgetting
  \unbf{what to say} during the speech contest.
\end{itemize}

\paragraph{例八}

\begin{enumerate}
\item A. Fries was the leader of the College football team then.

  A.弗赖斯当时是学院足球队队长。
\item A. Fries is the director of a football club now.

  A.弗赖斯现在是一家足球俱乐部的主管。
\item A. Fries saw something.

  A.弗赖斯当时见到一件事。
\item The College football team lost in the important game.

  学院足球队在重要的球赛中失利。
\item A. Fries offered something.

  A.弗赖斯提议做一件事。
\item He would assume responsibility.

  弗赖斯愿意负责。
\item He would tender his resignation.

  弗赖斯将提出辞呈。
\end{enumerate}

这里一共有七个句子,要合并在一起,而且其中六个都得简化,只许留下一个完整的从
句。这可能是个不小的挑战,请读者仔细观察如何逐步完成整个动作。

首先,句 1 和句 2 分别叙述A.弗赖斯当时与现在的身份。这两句在内容与句型上对仗
工整,适合以对等从句方式表现,故加上对等连接词and 来连接:
\begin{itemize}
\item (1+2) A. Fries was the leader of the College football team then and
  \unbf{he is the director} of a football club now.
\end{itemize}
对等从句的简化方法是:两从句间相对应位置如有重复,则省略一个。因此把 and
右边那个从句重复的 he is 去掉,成为:
\begin{itemize}
\item (A) A. Fries was the leader of the College football team then \unbf{and
    the director} of a football club now.
\end{itemize}
这个描述弗赖斯身份的句子,我们称作句 A,先放着备用。下一步来组合 3 和 4两句。
句 3 中“弗赖斯见到”的 something 就是整个句 4的内容:“学院足球队比赛失利”。
所以把句 4 冠上连接词 that成为名词从句,置于句 3 中取代 something,作为 saw
的宾语:
\begin{itemize}
\item (3+4) A. Fries saw that \unbf{the College football team lost} in the important
  game.
\end{itemize}
that 引导的这个名词从句可以如此简化:主语 the College football team改为所有格
留下,动词 lost 直接改为名词的 lost,成为:
\begin{itemize}
\item (B) A. Fries saw \unbf{the College football team's loss} in the important
  game.
\end{itemize}
“弗赖斯眼见学院足球队失利。”这句话我们称作句 B,也先放着暂时不用。

接下来组合 5 和 6 两句。句 5 “弗赖斯提出”的 something,就是句 6的“他要负起
责任”。所以如法炮制把句 6 改成名词从句置入句 5 来取代something,成为:
\begin{itemize}
\item (5+6) A. Fries offered \unbf{that he (would) assume} responsibility.
\end{itemize}
这个句子可再将助动词简化为不定式 to V 的简化从句 he be to assume,而 be
动词可再省略成为:
\begin{itemize}
\item A. Fries offered \unbf{to assume} responsibility.
\end{itemize}

现在就用这个句子来把前面整理的结果堆砌上去。先把句 A 拿出来。句 A内容是描述弗
赖斯的职位,有补充形容A.弗赖斯身份的功能,所以拿它来做关系从句,将 A. Fries
改为关系词who,附于上句的主语 A. Fries 之后,成为:
\begin{itemize}
\item (+A) A. Fries, \unbf{who was the leader of the College football team then and
  the director of a football club now}, offered to assume responsibility.
\end{itemize}
句中这个 who 引导的关系从句可以简化,省略主语 who 和 be 动词was,留下名词类补
语(一般所谓的同位语),成为:
\begin{itemize}
\item A. Fries, \unbf{the leader of the College football team then and the
    director of a football club now}, offered to assume responsibility.
\end{itemize}
“当时的学院足球队队长,现今一家足球俱乐部的主管弗赖斯,表示要负责。”为什么?
因为句B:“他目睹学院足球队比赛失利。”现在把句 B拿出来用,它和上句的关系是因
果关系,所以加上连接词because,做成状语从句与上句并列:
\begin{itemize}
\item (+B) \unbf{Because he saw} the College football team's loss in the important
  game, A. Fries, the leader of the College football team then and the
  director of a football club now, offered to assume responsibility.
\end{itemize}
句子越来越长了,现在来简化一下。上句中 because 引导的状语从句,主语 he和主要
从句的 A. Fries 重复,可以省略。动词 saw 因无 be动词与助动词,可直接改
成 seeing,再把多余的 because 去掉,成为:
\begin{itemize}
\item \unbf{Seeing} the College football team's loss in the important game, A. Fries,
  the leader of the College football team then and the director of a
  football club now, offered to assume responsibility.
\end{itemize}

别忘了,一直未动用到句7:“弗赖斯打算提出辞呈。”从内容来看,它是说明上句
中“负责”(assume responsibility)的方式。也就是句 7 应拿来修饰上句中的原形
动词 assume一词。\textbf{“以……方式”的最佳表达是用 by 的介词短语},所以把
句7(He would tender his resignation.)直接放入 by 的后面,不过,by是介词,后
面只能接受名词短语,所以要将句 7简化为名词短语的形态。省略主语 he,动
词 would tender因为要放在介词后面,只能改成动名词 tendering,成为:
\begin{itemize}
\item (+7) Seeing the College football team's loss in the important game, A.
  Fries, the leader of the College football team then and the director of a
  football club now, offered to assume responsibility \unbf{by tendering his
    resignation}.

  眼见学院足球队在重大的比赛中失利,当时的学院足球队队长,也现在一家足球俱乐
  部的主管弗赖斯,表示要提出辞呈以示负责。
\end{itemize}
终于大功告成。读者经过这一番演练,当可了解上面这个句子实际上隐含多达七句话。
然而经过简化的过程,删掉了一切多余的元素,最后的结果并不显得太长或太复杂,这
就是简化从句的功效。

如开场白中所述,简化从句是高难度句型,颇富挑战性。读者若看到这里都能大致了解,
那么句型观念可说已相当完整,欠缺的只是大量的阅读功夫,那要靠日积月累的培养。
有清晰的句型观念,再加上大量的阅读,日后自然能写出一手好文章。

下面再附上一篇练习,请读者先自行尝试组合、简化其中的句子,再比对附在后面的参
考——只是参考,因为简化从句没有一定的做法,也没有标准答案。在告别句型之前,
还有一个问题要处理:倒装句。下一章我们就来研究这个也很重要的问题。

\section{Test}

\paragraph{将下列各题中的句子写在一起成为复句或合句,然后再简化到最精简的地步:}

\begin{enumerate}
\item Ben Book was educated in an art college. (because)

  Ben Book acts unusual.

  Ben Book deals with economic matters. (while)

\item I'd like something.

  You will meet some people. (that)


\item I'm not sure.

  What should I do?


\item He worked late into the night.

  He was trying to finish the report. (because)

\item The soldier was wounded in the war. (after)

  He was sent home.

\item He used to smoke a lot.

  He got married. (before)

\item I am afraid.

  The Democratic Party might win a majority. (that)

\item I have nothing better to do. (when)

  I enjoy something.

  I play poker. (that)

\item Mike won the contest. (when)

  Mike was awarded ten thousand dollars.

\item The motorcyclist was pulled over by the police car.

The motorcyclist did not wear a safety helmet. (who)

\item The mayor declined.

 The mayor was a very busy person. (who)

 The mayor was asked to give a speech at the opening ceremony. (when)

\item Tax rates are already very high. (although)

  Tax rates might be raised further to rein in inflation.

\item The resort town is crowded.

  There has been an influx of tourists for the holiday season. (because)

\item The student had failed in two tests. (though)

  The student was able to pass the course.

\item The president avoided the issue. (that)

  This was obvious to the audience.

\item Anyone could tell he was upset.

  He had the look on his face. (because)

\item Michael Crichton is in town.

 He is author of Jurassic Park. (who)

 He could promote his new novel. (so that)

\item I am a conservative. (although)

 I'd like to see something.

 The conservative party is chastised in the next election. (that)

\item The man found a fly in his soup. (when)

  The man called to the waiter.

\item It is a warm day. (because)

 We will go to the beach.

\end{enumerate}

\section{Answer}
\begin{enumerate}
\item Because he was educated in an art college. Ben Book acts unusual while he
  deals with economic matters. 简化为:

  Educated in an art college, Ben Book acts unusual while dealing with
  economic matters.

\item I'd like that you will meet some people. 简化为:

  I'd like you to meet some people.

\item I'm not sure what I should do. 简化为:

  I'm not sure what to do.

\item He worked late into the night because he was trying to finish the report.
  简化为:

  He worked late into the night trying to finish the report.

\item After the soldier was wounded in the war, he was sent home. 简化
  为:

  (After being) wounded in the war, the soldier was sent home.

\item He used to smoke a lot before he got married. 简化为:

  He used to smoke a lot before getting married.

\item I am afraid that the Democratic Party might win a majority. 简化为:

  I am afraid of the Democratic Party winning a majority.

\item When I have nothing better to do, I enjoy that I play poker. 简化为:

  When I have nothing better to do, I enjoy playing poker.
\item When Mike won the contest, he was awarded ten thousand dollars. 简化为:

  (Upon) winning the contest. Mike was awarded ten thousand dollars.

\item The motorcyclist who did not wear a safety helmet was pulled over by the
  police car. 简化为:

  The motorcyclist not wearing a safety helmet was pulled over by the police
  car.
\item The mayor, who was a very busy person, declined when he was asked to give
  a speech at the opening ceremony. 简化为:

  The mayor, a very busy person, declined when asked to give a speech at the
  opening ceremony.

\item The mayor, who was a very busy person, declined when he was asked to give
  a speech at the opening ceremony. 简化为:

  The mayor, a very busy person, declined when asked to give a speech at the
  opening ceremony.

\item Although tax rates are already very high, they might be raised further to
  rein in inflation. 简化为:

  Although already very high, tax rates might be raised further to rein in
  inflation.

\item The resort town is crowded because there has been an influx of tourists
  for the holiday season. 简化为:

  The resort town is crowded with an influx of tourists for the holiday
  season.

\item Though the student had failed in two tests, he was able to pass the
  course. 简化为:

  Though having failed in two tests, the student was able to pass the
  course.

\item That the president avoided the issue was obvious to the audience. 或

  It was obvious to the audience that the president avoided the issue. 简化
  为:

  The president's avoiding the issue was obvious to the audience. 或

  The president's avoidance of the issue was obvious to the audience.
\item Anyone could tell he was upset because he had the look on his face. 简化
  为:

  Anyone could tell he was upset, with the look on his face.

\item Michael Crichton, who is author of Jurassic Park, is in town so that he
  could promote his new novel. 简化为:

  Michael Crichton, author of Jurassic Park, is in town to promote his new
  novel.
\item Although I am a conservative, I'd like to see that the conservative party
  is chastised in the next election. 简化为:

  Although (being) a conservative, I'd like to see the conservative party
  chastised in the next election.
\item When the man found a fly in his soup, he called to the waiter. 简化
  为:

  Finding a fly in his soup, the man called to the waiter.
\item Because it is a warm day, we will go to the beach. 简化为:

  It being a warm day, we will go to the beach.
\end{enumerate}
\chapter{倒装句}

\textbf{倒装句是一种把动词(或助动词)移到主语前面的句型。}以这个定义来看,一般的疑
问句都可以算是倒装句。

撇开疑问句这种只具有语法功能的倒装句不谈,比较值得研究的是具有修辞功能的倒装
句。恰当地运用倒装句,可以强调语气、增强清楚性与简洁性,以及更流畅地衔接前后
的句子。以下分别就几种重要的倒装句来看看它倒装的条件,以及可达到的修辞效果。

\section{比较级的倒装}

在开始谈比较级的倒装前,有一些关于比较级的修辞问题应先弄清楚,请看这个例子:
\begin{enumerate}
\item Girls like cats more than boys. (不清楚)
\end{enumerate}
这个句子可能有两种意思:
\begin{enumerate}[resume]
\item Girls like cats more than boys do.

  女孩比男孩更喜欢猫。
\item Girls like cats more than they like boys.

  女孩比较喜欢猫,比较不喜欢男孩。
\end{enumerate}

比较级的句型通常会牵涉到两个从句互相比较。这两个从句间应有重复的部分才能比较。
一旦有重复,就有省略的空间。但是如果省略不当,就会伤害句子的清楚性。就像上面
的例1,可以作例 2 和例 3 两种不同的解释。修辞学上称这种句子为ambiguous(模棱
两可)。如果要表达例 2 的意思,那么句尾的 do就不能省略,否则读者有可能把它当
作例 3 来理解。

如果把例 2 修改一下,成为:
\begin{itemize}
\item Girls like cats more than boys, who as a rule are a cruel lot, \ul{do}. (不
  佳)
\end{itemize}
这个句子在 boys后面加上一个修饰它的关系从句。从刚才的分析中可了解到,句尾
的 do不能省略,否则读者无从判断 boys是主语还是宾语——是喜欢猫的人,还是被喜
欢的对象。

do 这个词既不能省略,把它留在句尾却又有修辞上的毛病。首先,do这个助动词和它的
主语 boys之间,因为关系从句的阻隔,距离太远,会伤害句子的清楚性。另外,助动
词 do所代表的是前面从句中的 like cats,但同样也因为距离太远而不够清楚。

要解决这个修辞上的问题,有个方法——倒装句。将 do 挪到主语 boys前面,成为:
\begin{itemize}
\item Girls like cats more than \unct{do boys}{倒装句}, who as a rule are a cruel lot.

  女孩比男孩更喜欢猫——男孩通常都很残酷。
\end{itemize}

如此一来,助动词 do 和主语 boys 放在一起了,而且 do 和它所代表的 like cats的
距离也减到最小,解决了所有的修辞问题。比较级需要用到倒装句的情形大抵都是这
样:

\textbf{一、从属从句中的助动词或 be 动词不宜省略。}

\textbf{二、主语后面有比较长的修饰语。}

\section{关系从句的倒装}

关系从句中的关系词,如果不是原来就在句首位置,就要向前移到句首让它发挥连接词
的功能。

例如:
\begin{enumerate}
\item The President is \unbf{a man}.

  总统是一个人。
\item A heavy responsibility, whether he likes it or not, falls on \unbf{him}.

  不论他喜不喜欢,他负有重大的责任。
\end{enumerate}
例 2 中的 him 就是例 1 的 a man,由这个交叉建立起关系,可以制造一个关系从句:
\begin{itemize}
\item The President is a man on \unnormal{whom}{关系词} \unnormal{a heavy
    responsibility}{关系从句主语}, whether he likes it or not, falls. (不
  佳)
\end{itemize}
介词短语 on whom因为内含关系词,要移到句首的位置。然而一经移动,就产生了修
辞上的问题。

首先,on whom 这个介词短语是当做副词使用,修饰动词falls。但是移到句首之后,
它与修饰的对象 falls之间隔了颇长的距离,这就会伤害修辞的清楚性。另外,关系从
句主语 a heavy responsibility 与它的动词 falls 之间也隔了一个插入的副词从
句whether \ldots{},主语动词间的距离过长又是一个不清楚性的来源。要解决这两个
问题还是得靠倒装句,把动词移到主语前面:
\begin{itemize}
\item The President is a man \unct{on whom falls a heavy responsibility}{倒装句},
  whether he likes it or not.

  总统负有重大责任,不论他喜不喜欢。
\end{itemize}
如此一来,关系词 whom 与先行词 a man 在一起,介词 on whom与它修饰的对
象 falls 在一起,而且动词 falls 又与它的主语 a heavy responsibility 在一起,
一举解决了所有问题。这就是倒装句的妙用。

要注意的是,\textbf{关系词必须先向句首移动,造成顺序的反常,才有倒装的可能。}如果关系
词没有移动就不能倒装。例如:
\begin{itemize}
\item The President is a man who bears a lot of responsibility.
\end{itemize}
这句话的意思和原来的句子差不多,不过它无法倒装。因为里面的关系从句原来是He
bears a lot of responsibility,主语 he 改成关系词who,由于原本就在句首,没有
移动位置,所以也就不能倒装。

\section{假设语句的倒装}

这种倒装比较单纯。在虚拟语气的状语从句中(往往是由 if 引导的),如果有be 动词
或助动词,就可以考虑倒装。做法是把连接词(例如 if)省略掉,把 be动词或助动词
移到主语前面来取代连接词的功能。例如:
\begin{itemize}
\item \unct{If I had been there}{状语从句}, I could have done something to help.

  如果当时我在场,就可以帮得上忙。
\end{itemize}
为了加强简洁性,可以把连接词 if 省略掉,用倒装句来取代,成为:
\begin{itemize}
\item \unct{Had I been there}{倒装句}, I could have done something to help.
\end{itemize}

但状语从句中若没有 be动词或助动词,就缺乏可倒装的工具,因而不能使用倒装。

\section{引用句的倒装}

\textbf{在直接引句(用到双引号者)与间接引句(没有用双引号者)中,都可以选择使用倒
  装句来凸显出句中的重点。}例如:
\begin{itemize}
\item \unct{The police}{S} \unct{said}{V}, \unct{“None was killed in the
    accident.”}{O 直接引句}

  警方说:“这桩车祸无人死亡。”
\end{itemize}
引用句往往出现在宾语位置,上面这个例子就是如此。不过,引用句的内文才是读者急
于知道的事情,至于是“谁说的”倒不那么关心。然而引用句的构造偏偏是“谁说
的”作为主语、动词,出现在前面,宾语从句拖在后面。选择倒装句就可以解决这个问
题:
\begin{itemize}
\item \unct{“None was killed in the accident.”}{O} \unct{said}{V} \unct{the police}{S}.
\end{itemize}
把读者最关心的引用句内文移到句首,可以达到强调语气的效果。因为宾语从句挪到句
首,与它关系密切的动词said也可以移到主语前面,成为倒装句。\textbf{不过在直接引句的情
况下,主语、动词也可以选择不必倒装},像上面这个例子,句尾部分可以维持the
police said(S+V)的顺序不必倒过来。接下来看间接引句:
\begin{itemize}
\item \unct{The WHO}{S} \unct{warns}{V}, \unct{that cholera is coming back}{O
    间接引句}.

  世界卫生组织警告:霍乱已死灰复燃。
\end{itemize}
这句话有一个间接引句,除了选择把整个宾语从句移到句首之外,也可以选择只把引用句的主语移到句首来加强语气,主要从句倒装,成为:
\begin{itemize}
\item Cholera, \unct{warns}{V} \unct{the WHO}{S}, is coming back.
\end{itemize}

不论直接引句还是间接引句,选择倒装的修辞原因都是为了凸显引用句的内容,把它摆
在句首最显著的地位。

\section{类似 there is/are的倒装}

这种倒装句是把地方副词挪到句首,句型和 there is/are的句型很接近,修辞功能在于
强调语气,以及衔接上下文。例如:
\begin{itemize}
\item \unct{There}{地方副词} \unct{goes}{V} \unct{the train}{S}!

  你看,火车开走了!
\end{itemize}
这个句子以倒装句处理,可以强调动词 goes,表示“正在开走”。再如:
\begin{itemize}
\item \unct{Here}{地方副词} \unct{is}{V} \unct{your ticket}{S} for the opera!

  你的歌剧票,拿去吧!
\end{itemize}
除了 here,there 之外,其他的地方副词也可以倒装,例如:
\begin{itemize}
\item \unct{In Loch Ness}{地方副词} \unct{dwells}{V} \unct{a mysterious monster}{S}.

  尼斯湖里住着一头神秘的水怪。
\end{itemize}
这个倒装句可以同时加强句首地方副词与句尾主语两个部分的语气。

有时候可以使用这种倒装句来加强上下文的衔接。例如:
\begin{itemize}
\item To the west of Taiwan lies Southern China.

\item To the east spreads the expanse of the Pacific.

\item 台湾西方是华南,东方是浩瀚的太平洋。
\end{itemize}
为了以空间顺序(spatial order)来组织上下文,这两个句子都用地方副词(To the
west \ldots,To the east \ldots)开头,也都动用倒装句来达到这个目的。

\section{否定副词开头的倒装}

如果把表示否定意味的副词(not、never, hardly)挪到句首来强调语气,就得使用倒
装句。例如:
\begin{itemize}
\item We \unbf{don't} have such luck \unbf{every day}.

  我们不是每天都能有这种运气。
\end{itemize}
如果为了强调“不是每天”,而把 not every day挪到句首,就要用倒装句。因为 not
和 every day 都是修饰动词的,而且 not是用来作否定句的副词,和助动词 do 不能分
开。一旦移到句首,助动词 do也要往前移来配合否定句的需要,就成为倒装句:
\begin{itemize}
\item \unbf{Not every day do} we have such luck.
\end{itemize}

再看一个例子:
\begin{itemize}
\item I will \unbf{not} stop waiting for you \unbf{until you are married}.

  除非你结婚,否则我会一直等你。
\end{itemize}
同样的,如果把 not until you are married移到句首来强调语气,就得把助动
词 will 倒装到主语前面来配合否定句的要求:
\begin{itemize}
\item \unbf{Not until you are married will} I stop waiting for you.
\end{itemize}

另外有一些副词,像 hardly,barely 等等,虽然不是一般否定句用的
not,不过功能与用法都类似,移到句首时也要倒装。例如:
\begin{itemize}
\item I had \unbf{hardly} sat down to work when the phone rang.

  我刚坐下来要做事,电话就响了。
\end{itemize}
把 hardly 移到句首也是为了加强语气,这时就要倒装:
\begin{itemize}
\item \unbf{Hardly had} I sat down to work when the phone rang.
\end{itemize}

不过,下面这个句子就不要倒装:
\begin{itemize}
\item \unbf{Hardly anyone} knew him.

  几乎没有人认识他。
\end{itemize}
这是因为 hardly 虽然在句首,不过它是用来修饰主语anyone,句首是它正常的位置,
没有经过调动,因而也不需要倒装。

同样的情形也见于 only 一字的变化。请看这个例子:
\begin{itemize}
\item \unbf{Only I} saw him yesterday.

  昨天只有我见到他。
\end{itemize}
Only 原本就是修饰主语I,放在它前面是正常位置,不需倒装。下面这个句子则不同:
\begin{itemize}
\item I saw him \unbf{only yesterday}.

  我见到他,不过是昨天的亊。
\end{itemize}
如果把 only yesterday调到句首来强调“不过是昨天而已”,意思是“不是更早以前的
事”,也有否定的意味,所以可以视同表示否定的副词移到句首的变化,需要倒装:
\begin{itemize}
\item \unbf{Only yesterday did I} see him.
\end{itemize}

再比较一下这两个句子:
\begin{enumerate}
\item \unbf{Gradually} they became close friends.
\item \unbf{Only gradually did they} become close friends.
\end{enumerate}
例 1 中的副词 gradually放在句首,是语法上许可的位置,而且没有否定意味,不必倒
装。可是例 2 中的only gradually 就带有强烈的否定意味,表示 not at
once 或是 not very fast,这时就得动用倒装句型了。

not only 和 but also 配合时,如果选择倒装,变化比较复杂。请看这个例子:
\begin{itemize}
\item He \unbf{not only} passed the exam \unbf{but also} scored at the top.

  他不但及格了,还考了第一。
\end{itemize}
句中的 but 是对等连接词。形成 not only \ldots{} but also
的相关词组(correlative)时,严格要求连接的对称。上例中的 passed the
exam 和 scored at the top 都是动词短语,符合对称的要求。

如果要把 not only 移到句首来强调语气,因为 not only是有否定功能的副词,所以要
用倒装句型。先直接倒装成为:
\begin{itemize}
\item \ul{Not only} did he pass the exam \ul{but also} scored at the top. (误)
\end{itemize}

前半句用倒装句是对的,错在对等连接词 but 的左右不对等。 左边 he passed
the exam 是从句,而右边的 scored at the top 却是动词短语。

修正的方法是把右边的动词短语也改成能对称的从句:
\begin{itemize}
\item   Not only did he pass the exam but also he scored at the top. (不佳)
\end{itemize}
这样改过来,but 的左右都是从句,满足了语法的要求,不过还是有缺憾。因
为also 和 only 一样都是属于 focusing adverbs,是一种有强调功能的副词。许多学
习者把 but also连在一起来背,不知它有时也该拆开。在 but 右边的 also 不应用来
强调he,而应用来强调 scored at the top(而且还考第一),这样才能呼应左边 not
only did he pass \ldots(不仅考及格)的语气。所以最佳的作法是
把 also 移到scored 的前面:
\begin{itemize}
\item Not only did he pass the exam \unbf{but he also} scored at the top.
\end{itemize}

这样才算满足了所有的语法修辞要求。

\section{结语}

以上的整理涵盖了英语中重要的倒装句型。另有一些简单的倒装句,例如:
\begin{itemize}
\item Mary is pretty. So \unct{is}{V} \unct{her sister}{S}.

  玛丽很美,她妹妹也很美。
\end{itemize}

以及不常用的倒装句,像某些祈使句的句型:
\begin{itemize}
\item Long \unct{live}{V} \unct{the King}{S}!

  国王万岁!
\end{itemize}
这些也是倒装句,可是不需要深入探讨。

看完了倒装句,整个英语句型问题至此总算尘埃落定。恭喜本书读者,至此你们已经建
立了相当完整的句型观念,对英语句型有了深入的理解。

\section{Test}

\paragraph{请选出最适当的答案填入空格内,以使句子完整。}

\begin{enumerate}

\item The students were warned that on no account \ttu to cheat.
\begin{tasks}(2)
  \task they were
  \task were they
  \task they should
  \task they can
\end{tasks}

\item \ttu make up for lost time.
\begin{tasks}
  \task Only by working hard we can
  \task By only working hard we can
  \task Only by working hard can we
  \task By only working hard can we
\end{tasks}

\item Rarely \ttu such nonsense.
\begin{tasks}(2)
  \task I have heard
  \task have I heard
  \task I do hear
  \task don't I hear
\end{tasks}

\item \ttu perched a large black bird.
\begin{tasks}(2)
  \task Often
  \task Suddenly
  \task On the wire
  \task It
\end{tasks}

\item Only just now \ttu to him about the things to heed while riding a
  motorcycle.
\begin{tasks}(2)
  \task I talked
  \task was I talking
  \task talked I
  \task I was talked
\end{tasks}

\item John was as confused about the rules \ttu.
\begin{tasks}
  \task as were the other contestants
  \task as the other contestants had
  \task than were the other contestants
  \task than the other contestants had
\end{tasks}

\item An IBM PC 286 is as powerful \ttu on NASA's Voyager II.
\begin{tasks}
  \task than the mainframe computer is
  \task than is the mainframe computer
  \task as the mainframe computer is powerful
  \task as is the mainframe computer
\end{tasks}

\item The New Testament is a book \ttu the life and teachings of Jesus.
\begin{tasks}(2)
  \task which can be found
  \task in which can be found
  \task which can find
  \task in which can find
\end{tasks}

\item Not until the doctor was sure everything was all right \ttu the emergency room.
\begin{tasks}(2)
  \task he left
  \task left he
  \task did he leave
  \task he did leave
\end{tasks}

\item \ttu, man could die out.
\begin{tasks}
  \task World War III should ever break out
  \task If should World War III ever break out
  \task If World War III should have broken out
  \task Should World War III ever break out
\end{tasks}

\item The results, \ttu, the leading journal of science, indicate that the experimental procedure is flawed.
\begin{tasks}(2)
  \task says Nature
  \task Nature says
  \task which says Nature
  \task which Nature says
\end{tasks}

\item Across the street from the station \ttu.
\begin{tasks}
  \task stood an old drugstore
  \task it stood an old drugstore
  \task where an old drugstore stood
  \task which stood an old drugstore
\end{tasks}

\item I tried to call some friends but \ttu.
\begin{tasks}(2)
  \task none could I reach
  \task could I reach none
  \task I could none reach
  \task I none could reach
\end{tasks}

\item \ttu trouble you again.
\begin{tasks}(2)
  \task Never will I
  \task Not I will ever
  \task Will not ever I
  \task Never I will
\end{tasks}

\item Not until you paint your first oil color \ttu the difference between
  theory and practice.
\begin{tasks}(2)
  \task you find out
  \task and find out
  \task finding out
  \task do you find out
\end{tasks}

\item \ttu a baby deer is born, it struggles to stand on its own feet.
\begin{tasks}(2)
  \task No sooner
  \task As soon as
  \task So soon as
  \task Not sooner that
\end{tasks}

\item \ttu the invention of the movable print, books were mostly copied by hand
  and cost far more than ordinary people could afford.
\begin{tasks}(2)
  \task After
  \task Until
  \task Not until
  \task Because of
\end{tasks}

\item \ttu did I find out that he was dead.
\begin{tasks}(2)
  \task A moment ago
  \task Only a moment ago
  \task An only moment ago
  \task For a moment
\end{tasks}

\item Henry James is \ttu is his philosopher brother William.
\begin{tasks}(2)
  \task famous and also
  \task as famous as
  \task famous so
  \task equally famous
\end{tasks}

\item \ttu does the recluse venture out of his hermitage.
\begin{tasks}(2)
  \task Seldom
  \task Often
  \task Occasionally
  \task Sometimes
\end{tasks}

\end{enumerate}

\section{Answer}
\begin{enumerate}
\item (B) on no account 是否定副词短语,移至 that 从句句首即需倒装。

\item (C) only by working hard 因有 only 修饰,在句首要倒装。

\item (B) rarely 有否定功能,置于句首要倒装。

\item (C) 地方副词置于句首,类似 there is/are 的句型,方可倒装,故选 C。

\item (B) 因有 only just now 在句首,要倒装。

\item (A) 前有 as confused,后面要有 as(A 或 B)。因为前面是 John was confused,
  有 be 动词,后面不能用 had 来代表,应用 be 动词,故选 A,这是比较级的倒装。

\item (D) 上文 as 要求用 as 作连接,C 错在 powerful 不应重复。

\item (B) 原句是 The life and teachings of Jesus can be found in the book,改成关系从句再倒装,即是 B。


\item (C) not until 移到句首要用倒装句型。
\item (D) 原句是 If World War \Ronum{3} should ever break out… 省略 If 后倒装即
  是 D。

\item (A) 原句是间接引句,Nature says the results… 改成倒装句成为 A 会比不倒装
  的 B 好,因为空格后的 the leading journal of science 是 Nature 的同位语,两
  者应该在一起。

\item (A) 地方副词 across the street from the station 移到句首而成倒装句,类
  似 there is/are 的句型。

\item (A) 是 I could reach none 的倒装。

\item (A) 是 I will never trouble you again 的倒装句。

\item (D) not until 移到句首要倒装。

\item (B) 答案 A 要用倒装句,C 和 D 都不是正确的连接词,只有 B 能引导后面那个没有倒装的从句。

\item (B) “活版印刷发明前,书原来都是用手抄,一般人根本买不起。”从句意来看,只
  有 until 符合。

\item (B) 下文是倒装句,所以选择要求倒装的 only。

\item (B) 比较级后面倒装了。

\item (A) 下文是倒装句,所以选择要求倒装的 seldom。
\end{enumerate}
