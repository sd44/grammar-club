\chapter{英语}

对于英语来说并不存在语法学家编纂的学院式语法,更多是\textbf{惯用法(USAGE)},属于
\textbf{社会语言学范畴}。

《郎文英语语法大全》中所述语法指:
\begin{description}
\item [句法(SYNTAX)] 如陈述句变疑问句,句子简化或复合等。
\item [词法(MORPHOLOGY)] 单词的曲折变化(INFLECTIONS,也称词态变化 ACCIDENCE),
  如动词的过去分词、现在分词、过去式、第三人称单数等变化。
\end{description}

英语语法单位(从大到小)可分为:句
子 (SENTENCES), 分句 (CLAUSES), 短语 (PHRASES), 单词 (WORDS), 词素
(MORPHEME,如词缀、后缀、词态变化等)。

只有一个分句CLASUSES构成的句子被称为简单句 SIMPLE SENTENCES,两个或以上分句构成复句
COMPLEX 或 COMPOUND SENTENCES。

从属分句 SUBORDINATE CLASUSES 通常由一个从属连接词 (CONJUNCTION) 引导,
如 since。而且事实上也被称为状语从句(adverbial clause)。

语法等级体系中相等地位的两个或两个以上的单位,可构成一个与之性质相同的单位。这
种结构称为并列关系 (COORDINATION), 而且像从属关系一样,由一个称为连词的连接词
明确表示出来.这种连词叫并列 (COORDINATING) 连词。最常用的并列连词有 and, or
和 but。





%%% Local Variables:
%%% mode: latex
%%% TeX-master: "main"
%%% End:
