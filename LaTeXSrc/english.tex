\chapter{英语}

对于英语来说并不存在语法学家编纂的学院式语法,更多是\textbf{惯用法(USAGE)},属于
\textbf{社会语言学范畴}。

《郎文英语语法大全》中所述语法指:
\begin{description}
\item [句法(SYNTAX)] 如陈述句变疑问句,句子简化或复合等。
\item [词法(MORPHOLOGY)] 单词的曲折变化(INFLECTIONS,也称词态变化 ACCIDENCE),
  如动词的过去分词、现在分词、过去式、第三人称单数等变化。
\end{description}

英语语法单位(从大到小)可分为:句
子 (SENTENCES), 分句 (CLAUSES), 短语 (PHRASES), 单词 (WORDS), 词素
(MORPHEME,如词缀、后缀、词态变化等)。

只有一个分句CLASUSES构成的句子被称为简单句 SIMPLE SENTENCES,两个或以上分句构成复句
COMPLEX 或 COMPOUND SENTENCES。

从属分句 SUBORDINATE CLASUSES 通常由一个从属连接词 (CONJUNCTION) 引导,
如 since。而且事实上也被称为状语从句(adverbial clause)。

语法等级体系中相等地位的两个或两个以上的单位,可构成一个与之性质相同的单位。这
种结构称为并列关系 (COORDINATION), 而且像从属关系一样,由一个称为连词的连接词
明确表示出来.这种连词叫并列 (COORDINATING) 连词。最常用的并列连词有 and, or
和 but。


\section{动词}

\subsection{动词功能分类}

根据动词在动词短语中的功能,可分为三类:
\begin{description}
\item[全义动词 FULL VERBS] 又叫实义动词 lexical verbs,如play, grow, jump等,只
  能用作主要动词。
\item[基本动词 PRIMARY VERBS] be, have, do,既可以作主要动词,也可作助动词。虚
  拟语气中没有was这个形式。
\item[情态助动词 MODAL AUXILIARY VERB] may, might, will, would, can, could,
  shall, should, must,只可作为助动词,并且必须是\textbf{谓语部分第一个动词},
  它们能表达所谓情态 (MODALITY, 包括意愿、可能性、义务等)这一领域中的含义。

\end{description}

动词的 分词 partiple 这个名称,也反映了这一形式既带有动词特征,又带有形容词特征。


\subsection{动词的第三人称单数及名词复数 -s }

\begin{enumerate}
\item 以清、浊咝声结尾的原型的-s 形式,结尾应是 es,读作  \doulos{/ɪz/},如以
  \doulos{/s z ʧ ʤ/} 等音。
\item 以清辅音结尾的原型后读作 \doulos{/s/},如 \doulos{/p t k f/} 等音。
\item 除咝声外,以浊音(包括元音)结尾的原型后,读作 \doulos{/z/}。
\item 以o结尾的一些单词要加 es,如go  \Rightarrow goes, echo  \Rightarrow echoes
\item 以 辅音 + -y 结尾的原形, 把 -y 变成 -i,后加 es : try \Rightarrow tries,carry  \Rightarrow carries。
\end{enumerate}

\subsection{规则动词的过去式和过去分词}

规则动词的过去式和过去分词:
\begin{enumerate}
\item 在以 \emph{t} 和 \emph{d} 结尾的原型后面读作 \doulos{/ɪd/}。 padded, patted
\item 在以浊音(包括元音)结尾的原形后面读作 \doulos{/d/}。buzzed, towed, called
\item 除 \emph{t} 外,在以清音结尾的原形后面读作 \doulos{/t/}。passed, packed
\item 以 辅音 + -y 结尾的原形, 把 -y 变成 -i,后加 ed:
  \begin{taskitem}(4)
    * try \Rightarrow tried
    * carry  \Rightarrow carried
  \end{taskitem}

\item 如果动词原形以单个辅音字母结尾,之前只有一个发元音的字母并且重读,那么它的现在分词和过去分词形式中要加双拼。
  \begin{taskitem}(2)
    * bar \Rightarrow barring \Rightarrow barred
    * beg \Rightarrow begging \Rightarrow begged
    * permit \Rightarrow permitting \Rightarrow permitted
    * patrol \Rightarrow patrolling \Rightarrow patrolled
  \end{taskitem}

\item 以 元音+c 结尾的动词原形,其现在分词和过去分词形式要加 k。如
  \begin{taskitem}(2)
    * panic \Rightarrow panicking \Rightarrow panicked
    * traffic \Rightarrow trafficking \Rightarrow traficked
  \end{taskitem}

\item 如果原形以不发音的 -e 结尾,他的过去和现在分词形式,总是先删去 -e。
  \begin{taskitem}(2)
    * create \Rightarrow creating \Rightarrow created
    * type \Rightarrow typing \Rightarrow typed
  \end{taskitem}
\end{enumerate}

\subsection{不规则动词曲折变化}

请见\ccref{tab:irrverb} 。

\subsection{(半)情态助动词}

\begin{table}[htbp]
  \centering \small
  \begin{talltblr}[ caption = {情态助动词到主要动词的递差度表},
    label = {tab:auxverb},
    note{a} = {ought to用在肯定句中,否定和疑问句中则去掉to.}
    ]{
      width=\linewidth, colspec={X[-1,l]X[l]},
      rowspec={Q[t]Q[t]Q[t]}, rowsep=2pt, colsep=4pt,
      row{1} = {font=\bfseries},
    }
    \toprule
    动词类别 & 助动词或主要动词短语 \\ \midrule
    \textsf{主要情态动词} &  may, might, will, would, can, could, shall, should, must \\
    \textsf{临界情态动词} &  dare, need, ought to, used to \\
    \textsf{情态动词习语} &  had better, would rather/sooner, be to, have got to等 \\
    \textsf{半助动词} &  have to, be about to, be able to, be allow to, be bound to, be going to, be likely to, be
    obliged to, be supposed to, be willing to 等 \\
    \textsf{链接动词} &  appear to, happend to, seem to, get + -ed分词, keep + -ing分词等 \\
    {\textsf{主要动词+} \\\textsf{非限定性分句}} &  begin + -ing分词等 \\ \bottomrule
  \end{talltblr}%
\end{table}

除主要情态动词只可作助动词以外,其他兼具情态助动词功能的动词还可作为主要动词
使用,因此要注意区分,如:
\begin{itemize}
\item  \unbf{Need} we \unbf{escape}? We \unct{needn't escape}{V}. (need作为情态助动词)

\item She \unbf{needs} \unbf{to practise}{A} and so \unbf{do} I. (need 作为主
  要动词)
\end{itemize}

有情态助动词功能的动词短语示例:
\begin{itemize}
\item No one \unct{dare tell}{V} the king this bad news.

\item We \unbf{ought to} give him another chance. \unbf{Ought} we have done it?

\item  You\unbf{'d better} lock the door.

\item I\unbf{'d rather/sooner} live in the country \ul{than} in the city.
\item No one \unbf{is likely to} \unbf{be able to} recognize her.
\item \unbf{Has} he \unbf{to} answer the letter this week?
\end{itemize}

\section{名词}

\subsection{不规则名词复数}

以下是规律总结,详表请见\ccref{tab:irrnoun} 。

\begin{description}
\item[以f或fe结尾] 大多数以f或fe结尾的名词的复数形式时将其转为ves:
  \begin{taskitem}(3)
    *  calf -- calves
    *  elf -- elves
    *  half -- halves
    *  hoof -- hooves
    *  knife -- knives
    *  leaf -- leaves
    *  life -- lives
    *  loaf -- loaves
    *  scarf -- scarfs/scarves
    *  self -- selves
    *  sheaf -- sheaves
    *  shelf -- shelves
    *  thief -- thieves
    *  wife -- wives
    *  wolf -- wolves
  \end{taskitem}

\item[元音] 有些名称的复数形式是改变它们的元音声:
  \begin{taskitem}(3)
    *  fireman -- firemen
    *  foot -- feet
    *  goose -- geese
    *  louse -- lice
    *  man -- men
    *  mouse -- mice
    *  tooth -- teeth
    *  woman -- women
  \end{taskitem}


\item[古英语] 有些是沿用古英语:
  \begin{taskitem}(3)
    *  child -- children
    *  ox -- oxen
  \end{taskitem}


\item[以o结尾] 见下文:

  \textbf{有的加``s''}

  \begin{taskitem}(3)
    *  auto -- autos
    *  kangaroo -- kangaroos
    *  kilo -- kilos
    *  memo -- memos
    *  photo -- photos
    *  piano -- pianos
    *  pimento -- pimentos
    *  pro -- pros
    *  solo -- solos
    *  soprano -- sopranos
    *  studio -- studios
    *  tattoo -- tattoos
    *  video -- videos
    *  zoo -- zoos
  \end{taskitem}

  \textbf{有的则加``es''}

  \begin{taskitem}(3)
    *  echo -- echoes
    *  embargo -- embargoes
    *  hero -- heroes
    *  potato -- potatoes
    *  tomato -- tomatoes
    *  torpedo -- torpedoes
    *  veto -- vetoes
  \end{taskitem}

  \textbf{有的两种都可以}
  \begin{taskitem}(2)
    *  buffalo -- buffalos/buffaloes
    *  cargo -- cargos/cargoes
    *  halo -- halos/haloes
    *  mosquito -- mosquitos/mosquitoes
    *  motto -- mottos/mottoes
    *  no -- nos/noes
    *  tornado -- tornados/tornadoes
    *  volcano -- volcanos/volcanoes
    *  zero -- zeros/zeroes
  \end{taskitem}

\item [不变] 单复数同型:
  \begin{taskitem}(3)
    *  cod -- cod
    *  deer -- deer
    *  fish -- fish
    *  offspring -- offspring
    *  perch -- perch
    *  sheep -- sheep
    *  trout -- trout
  \end{taskitem}

  注:很多鱼类的复数形式都是不变的,但有例外

\item[不变] 本身就是复数,只有复数形式
  \begin{taskitem}(4)
    *  barracks
    *  crossroads
    *  dice
    *  gallows
    *  headquarters
    *  means
    *  series
    *  species
  \end{taskitem}

\item[借用] 借用自其他语言,部分借用单词的复数跟随外语形式:
  \begin{taskitem}(3)
    *  alga -- algae
    *  larva -- larvae
    *  vertebra -- vertebrae
  \end{taskitem}

\item[借用] 以``us''结尾的转为``a''(适用于专业术语):
  \begin{taskitem}(3)
    *  corpus -- corpora
    *  genus -- genera
  \end{taskitem}

\item[借用] 以``us''结尾的转为``i'':
  \begin{taskitem}(3)
    *  alumnus -- alumni
    *  bacillus -- bacilli
    *  focus -- foci
    *  nucleus -- nuclei
    *  radius -- radii
    *  stimulus -- stimuli
    *  syllabus -- syllabuses
    *  terminus -- termini
  \end{taskitem}

\item[借用] 以``um''结尾的转为``a'':
  \begin{taskitem}(3)
    *  addendum -- addenda
    *  bacterium -- bacteria
    *  datum -- data
    *  erratum -- errata
    *  medium -- media
    *  ovum -- ova
    *  stratum -- strata
  \end{taskitem}


\item[借用] 以``is''结尾的转为``es'':
  \begin{taskitem}(2)
    *  analysis -- analyses
    *  axis -- axes
    *  basis -- bases
    *  crisis -- crises
    *  diagnosis -- diagnoses
    *  emphasis -- emphases
    *  hypothesis -- hypotheses
    *  neurosis -- neuroses
    *  oasis -- oases
    *  parenthesis -- parentheses
    *  synopsis -- synopses
    *  thesis -- theses
  \end{taskitem}


\item[借用] 以``on''结尾的转为``a'':
  \begin{taskitem}(2)
    *  criterion -- criteria
    *  phenomenon -- phenomena
    *  automaton -- automata
  \end{taskitem}


\item[沿用其他语言规则] 意大利语
  \begin{taskitem}(3)
    *  libretto -- libretti
    *  tempo -- tempi
    *  virtuoso -- virtuosi
  \end{taskitem}

\item[沿用其他语言规则] 希伯来语
  \begin{taskitem}(3)
    *  cherub -- cherubim
    *  seraph -- seraphim
  \end{taskitem}

\item[沿用其他语言规则] 希腊语
  \begin{taskitem}(3)
    *  schema -- schemata
  \end{taskitem}
\end{description}

\subsection{限定词}

名词短语中,限定词位置大体可以分为前位、中位、后位(见\cref{tab:determ})。

\begin{table}[htbp]
  \centering \small
  \begin{talltblr}[ caption = {名词短语中限定词的位置},
    label = {tab:determ},
    ]{
      width=0.9\linewidth, colspec={lX},
      rowsep=2pt, colsep=4pt,
    }
    \toprule
    \SetCell[c=2]{l} \textbf{前位限定词\qquad (互斥,只选其一)}& \\
    感叹 & such, what\\
    倍数词 & double, twice, three times\\
    分数词 & one-third, one-fifth\\
    数量词 & all, both, half\\
    \midrule
    \SetCell[c=2]{l} \textbf{中位限定词\qquad (互斥,只选其一)} & \\
    冠词 & a, an, the\\
    物主代词 &  my, our, your, his, her, its, their\\
    名词所有格 & the rabbit's, the wolf's\\
    关系代词 & whose,which\\
    指示代词 & this, that‍‍, these, those\\
    wh-ever 限定词 & whichever, whatever, whoever\\
    疑问代词 & what, whose,which\\
    量词和其他 & {enough, each, every, some, any, either, neither,\\
      lot(s)/piece/few/plenty of, no}\\
    \midrule
    \SetCell[c=2]{l} \textbf{后位限定词}  \\
    基数词 & one,two,three\\
    数量词 & {few, little, many, much, several\\ large/great/good
      number of} \\
    (一般)序数词 & {first, second, fourth, twentieth,\\ next, last, past, (an)other}\\
    \bottomrule
  \end{talltblr}%
\end{table}

\textbf{限定词互斥的例外:}
\begin{itemize}
\item 中位限定词\textbf{every}有时可在属格后面,例:

  His every action shows that he is a very determined young man.
\item 前位限定词 such 用作代用式 (pro-form)时,也能接在数量词 any, no 和 many 以
  及基数词的后面:

  no/any/several/many/forty-one such incidents \ldots
\end{itemize}


除作前位限定词外,all, both 和 half 作为代词还能带 of- 短语 (partitive
of-phtase)表示“部分”。 \textbf{与名词连用时, of- 短语可有可无,与代词连用则非
  用 of短语不可}:
\begin{taskitem}(2)
\item all (of) the students
\item all of them/whom
\item both (of) his eyes
\item both of them/which
\item half (of) the time/cost
\item half of it/this
\end{taskitem}

以下用法可表示类指:
\begin{itemize}
\item a + 可数名词单数,如 a tiger.
\item (the) + 可数名词复数,如 the tigers,tigers.
\item 零冠词 + 不可数名词,如 milk.
\end{itemize}

\subsection{属格}

\begin{description}
\item[the genitive] 属格(所有格):名词或者形容词,used to show possession or
  close connection between two things. 展示两者之间的所属或紧密关系。
\item[of-construction] of介词 + 名词性短语,belonging to sb/sth; relating to sb/sth 属于(某人/某物);关
  于(某人/某物)。
\end{description}

\begin{itemize}
\item What is \unbf{the ship's} name?

  What is the name \unbf{of the ship}?

\item Some people's opnions

  the options of some people (不很清晰,少用)
\end{itemize}
许多情况下,这两种形式意义相同且完全能够成立。

属格和of结构应根据如下侧重点,结合实际情况加以选择:
\begin{enumerate}
\item \textbf{具有人性特点的名词类别常常用属格。}例如人、高等动物、集体(多个个人组
  成)、地理位置(人类生活区域)、时间、人的感官活动等。
  \begin{taskitem}(2)
    * the nations resources
    * Europe's future
    * China's development
    * the school's history
    * today's paper
    * a day's work
    * the body's needs
    * the game's history
  \end{taskitem}

\item \textbf{属格有特指的意思,带有限定性;of结构有泛指的意思。}
  \begin{itemize}
  \item Susan's son (苏珊的儿子,单看短语本身她也只有一个儿子)

  \item a son of Susan (苏珊有多儿子,其中之一)
  \end{itemize}

\item 根据\textbf{末尾焦点 (end foucs) 和末尾重心 (end-weight)原则。}更复杂和重要的单
  位应放在名词短语末尾。这样一来,\textbf{属格倾向于将信息中心放在名词中心词上,of结构倾向于将
    中心放在介词补语上。}

  \begin{itemize}
  \item The explosion damaged \unct{the ship's funnel}{焦点在funnel}.
  \item The explosion damaged \unct{the funnel of the ship}{焦点在ship}.

    爆炸损坏了船上的烟囱。
  \end{itemize}

\item 两者皆可的情况下,属格往往比较简明清晰,优先考虑。
\end{enumerate}

\improve[inline]{关于of结构还可参见 17.38f}

\subsection{独立属格 (the independent genitive)}

如果上下文中已交代清楚属格后面的中心词,则中心词可以省略。省略的结果就构成了
所谓“\textbf{独立属格}”。
\begin{itemize}
\item  My \unbf{car} is faster than \unbf{John's}.  [省略 car]
\item  Her \unbf{memory} is like \unbf{an elephant's}. [省略memory]
\item \unbf{Mary's} was the prettiest \unbf{dress}. [省略dress]
\item If you can't afford a \unbf{sleeping bag}, why not borrow \unbf{somebody
    else's}? [省略sleeping bag]
\item The New York's \unbf{population} is greater than \unbf{Chicago's}.
\end{itemize}

需要注意,of结构如果出现在可比较语境 (comparable environments)中,of前面通常
要加指示代词\textbf{that/those}。上句如采用of结构,应是
\begin{itemize}
\item The \unbf{population} of New York is greater than \unbf{that of Chicago}.
\end{itemize}

\subsection{后置属格(双重属格)}

of结构可以与属格结合起来产生一种称为\textbf{后置属格(双重属格)}的结构。将of结构
后置。

我个人认为,后置属格


%%% Local Variables:
%%% mode: latex
%%% TeX-master: "main"
%%% End:
