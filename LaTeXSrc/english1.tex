\chapter{英语}

对于英语来说并不存在语法学家编纂的学院式语法,更多是\textbf{惯用法(USAGE)},属于
\textbf{社会语言学范畴}。

《郎文英语语法大全》中所述语法指:
\begin{description}
\item [句法(SYNTAX)] 如陈述句变疑问句,句子简化或复合等。
\item [词法(MORPHOLOGY)] 单词的曲折变化(INFLECTIONS,也称词态变化 ACCIDENCE),
  如动词的过去分词、现在分词、过去式、第三人称单数等变化。
\end{description}

英语语法单位(从大到小)可分为:句
子 (SENTENCES), 分句 (CLAUSES), 短语 (PHRASES), 单词 (WORDS), 词素
(MORPHEME,如词缀、后缀、词态变化等)。

只有一个分句CLASUSES构成的句子被称为\index{概念!简单句@简单句,simple
  sentences)}\textbf{简单句} (SIMPLE SENTENCES),两个或以上分句构成\textbf{复
句}COMPLEX 或 COMPOUND SENTENCES。\index{概念!复合句@复合句,complex}

\index{概念!从属分句@从属分句,subordinate clauses}\textbf{从属分
  句 SUBORDINATE CLASUSES 通常由一个从属连接词 (CONJUNCTION) 引导,如 since。
  而且事实上也被称为状语从句(adverbial clause)。}

语法等级体系中相等地位的两个或两个以上的单位,可构成一个与之性质相同的单位。这
种结构称为\textbf{并列关系} (COORDINATION), 而且像从属关系一样,由一个称为连词
的连接词明确表示出来.这种连词叫\textbf{并列 (COORDINATING) 连词}。最常用的并
列连词有 and, or和 but。


\section{动词}

\subsection{动词功能分类}

根据动词在动词短语中的功能,可分为三类:
\begin{description}
\item[全义动词 FULL VERBS] 又叫实义动词 lexical verbs,如play, grow, jump等,只
  能用作主要动词。
\item[基本动词 PRIMARY VERBS] be, have, do,既可以作主要动词,也可作助动词。虚
  拟语气中没有was这个形式。
\item[情态助动词 MODAL AUXILIARY VERB] may, might, will, would, can, could,
  shall, should, must,只可作为助动词,并且必须是\textbf{谓语部分第一个动词},
  它们能表达所谓情态 (MODALITY, 包括意愿、可能性、义务等)这一领域中的含义。

\end{description}

动词的 分词 participle 这个名称,也反映了这一形式既带有动词特征,又带有形容词特征。


\subsection{动词的第三人称单数及名词复数 -s }

\begin{enumerate}
\item 以清、浊咝声结尾的原形的-s 形式,结尾应是 es,读作  \doulos{/ɪz/},如以
  \doulos{/s z ʧ ʤ/} 等音。
\item 以清辅音结尾的原形后读作 \doulos{/s/},如 \doulos{/p t k f/} 等音。
\item 除咝声外,以浊音(包括元音)结尾的原形后,读作 \doulos{/z/}。
\item 以o结尾的一些单词要加 es,如go  \Rightarrow goes, echo  \Rightarrow echoes
\item 以 辅音 + -y 结尾的原形, 把 -y 变成 -i,后加 es : try \Rightarrow tries,carry  \Rightarrow carries。
\end{enumerate}

\subsection{规则动词的过去式和过去分词}

规则动词的过去式和过去分词:
\begin{enumerate}
\item 在以 \emph{t} 和 \emph{d} 结尾的原形后面读作 \doulos{/ɪd/}。 padded, patted
\item 在以浊音(包括元音)结尾的原形后面读作 \doulos{/d/}。buzzed, towed, called
\item 除 \emph{t} 外,在以清音结尾的原形后面读作 \doulos{/t/}。passed, packed
\item 以 辅音 + -y 结尾的原形, 把 -y 变成 -i,后加 ed:
  \begin{taskitem}(4)
    * try \Rightarrow tried
    * carry  \Rightarrow carried
  \end{taskitem}

\item 如果动词原形以单个辅音字母结尾,之前只有一个发元音的字母并且重读,那么它的现在分词和过去分词形式中要加双拼。
  \begin{taskitem}(2)
    * bar \Rightarrow barring \Rightarrow barred
    * beg \Rightarrow begging \Rightarrow begged
    * permit \Rightarrow permitting \Rightarrow permitted
    * patrol \Rightarrow patrolling \Rightarrow patrolled
  \end{taskitem}

\item 以 元音+c 结尾的动词原形,其现在分词和过去分词形式要加 k。如
  \begin{taskitem}(2)
    * panic \Rightarrow panicking \Rightarrow panicked
    * traffic \Rightarrow trafficking \Rightarrow trafficked
  \end{taskitem}

\item 如果原形以不发音的 -e 结尾,他的过去和现在分词形式,总是先删去 -e。
  \begin{taskitem}(2)
    * create \Rightarrow creating \Rightarrow created
    * type \Rightarrow typing \Rightarrow typed
  \end{taskitem}
\end{enumerate}

\subsection{不规则动词曲折变化}

请见\ccref{tab:irrverb} 。

\subsection{(半)情态助动词}

\begin{table}[htbp]
  \centering \small
  \begin{talltblr}[ caption = {情态助动词到主要动词的递差度表},
    label = {tab:auxverb},
    note{a} = {ought to用在肯定句中,否定和疑问句中则去掉to.}
    ]{
      width=\linewidth, colspec={X[-1,l]X[l]},
      rowspec={Q[t]Q[t]Q[t]}, rowsep=2pt, colsep=4pt,
      row{1} = {font=\bfseries},
    }
    \toprule
    动词类别 & 助动词或主要动词短语 \\ \midrule
    \textsf{主要情态动词} &  may, might, will, would, can, could, shall, should, must \\
    \textsf{临界情态动词} &  dare, need, ought to, used to \\
    \textsf{情态动词习语} &  had better, would rather/sooner, be to, have got to等 \\
    \textsf{半助动词} &  have to, be about to, be able to, be allow to, be bound to, be going to, be likely to, be
    obliged to, be supposed to, be willing to 等 \\
    \textsf{链接动词} &  appear to, happened to, seem to, get + -ed分词, keep + -ing分词等 \\
    {\textsf{主要动词+} \\\textsf{非限定性分句}} &  begin + -ing分词等 \\ \bottomrule
  \end{talltblr}%
\end{table}

除主要情态动词只可作助动词以外,其他兼具情态助动词功能的动词还可作为主要动词
使用,因此要注意区分,如:
\begin{itemize}
\item  \unbf{Need} we \unbf{escape}? We \unct{needn't escape}{V}. (need作为情态助动词)

\item She \unbf{needs} \unbf{to practice}{A} and so \unbf{do} I. (need 作为主
  要动词)
\end{itemize}

有情态助动词功能的动词短语示例:
\begin{itemize}
\item No one \unct{dare tell}{V} the king this bad news.

\item We \unbf{ought to} give him another chance. \unbf{Ought} we have done it?

\item  You\unbf{'d better} lock the door.

\item I\unbf{'d rather/sooner} live in the country \ul{than} in the city.
\item No one \unbf{is likely to} \unbf{be able to} recognize her.
\item \unbf{Has} he \unbf{to} answer the letter this week?
\end{itemize}

\subsection{限定性和非限定性动词}
\label{subsec:iffinite}

在英语语法中,动词可以分为限定性动词(finite verbs)和非限定性动词
(non-finite verbs)。\index{概念!限定性动词@限定性动词,finite
  verbs}\index{概念!非限定性动词@非限定性动词,non-finite verbs}
\begin{description}
\item[限定性动词] 受\textbf{主语、时态和语气}等因素影响的动词。这类动词
  \textbf{可以独立构成谓语},可以明确表明时态、人称和数的不同。具体特点是:
  \begin{description}
  \item[时态明确] 表示动作发生的时间(如过去、现在、将来)。
  \item[与主语一致] 动词形式会随着主语的人称和数的不同而变化。
  \item[可单独作为谓语] 在句子中,限定性动词通常是谓语动词。且\textbf{只有谓语动
      词中的第一个动词是限定性的}。
  \end{description}

\item[限定性分句] 带有一个限定性动词短语作谓语成分的分句叫做“限定性动词分
  句”,简称“限定性分句”。

\item[非限定性动词] \textbf{不受时态、人称和数等因素影响的动词形式。它们不能
    单独作为句子的谓语},通常需要与其他动词搭配使用。有以下三种形式:
  \begin{description}
  \item[不定式(Infinitive)] 通常以 to 开头,如 to eat、to run。不带 to 的不
    定式谓语较少见,但也有:
    \begin{itemize}
    \item \unbf{Rather than you do the job}, I'd prefer to finish it myself.
    \end{itemize}
  \item[分词(Participles)] 包括现在分词 (eating) 和过去分词 (eaten)。
  \item[动名词(Gerund)] 动词加 -ing 构成,具有名词的功能,如 running。
  \end{description}

\item[非限定性分句] 带有一个非限定性动词作谓语成分的分句就叫做“\textbf{非限
    定性(动词)分句}”
\end{description}



\section{名词}

\subsection{不规则名词复数}

以下是规律总结,详表请见\ccref{tab:irrnoun} 。

\begin{description}
\item[以f或fe结尾] 大多数以f或fe结尾的名词的复数形式时将其转为ves:
  \begin{taskitem}(3)
    *  calf -- calves
    *  elf -- elves
    *  half -- halves
    *  hoof -- hooves
    *  knife -- knives
    *  leaf -- leaves
    *  life -- lives
    *  loaf -- loaves
    *  scarf -- scarfs/scarves
    *  self -- selves
    *  sheaf -- sheaves
    *  shelf -- shelves
    *  thief -- thieves
    *  wife -- wives
    *  wolf -- wolves
  \end{taskitem}

\item[元音] 有些名称的复数形式是改变它们的元音声:
  \begin{taskitem}(3)
    *  fireman -- firemen
    *  foot -- feet
    *  goose -- geese
    *  louse -- lice
    *  man -- men
    *  mouse -- mice
    *  tooth -- teeth
    *  woman -- women
  \end{taskitem}


\item[古英语] 有些是沿用古英语:
  \begin{taskitem}(3)
    *  child -- children
    *  ox -- oxen
  \end{taskitem}


\item[以o结尾] 见下文:

  \textbf{有的加``s''}

  \begin{taskitem}(3)
    *  auto -- autos
    *  kangaroo -- kangaroos
    *  kilo -- kilos
    *  memo -- memos
    *  photo -- photos
    *  piano -- pianos
    *  pimento -- pimentos
    *  pro -- pros
    *  solo -- solos
    *  soprano -- sopranos
    *  studio -- studios
    *  tattoo -- tattoos
    *  video -- videos
    *  zoo -- zoos
  \end{taskitem}

  \textbf{有的则加``es''}

  \begin{taskitem}(3)
    *  echo -- echoes
    *  embargo -- embargoes
    *  hero -- heroes
    *  potato -- potatoes
    *  tomato -- tomatoes
    *  torpedo -- torpedoes
    *  veto -- vetoes
  \end{taskitem}

  \textbf{有的两种都可以}
  \begin{taskitem}(2)
    *  buffalo -- buffalos/buffaloes
    *  cargo -- cargos/cargoes
    *  halo -- halos/haloes
    *  mosquito -- mosquitos/mosquitoes
    *  motto -- mottos/mottoes
    *  no -- nos/noes
    *  tornado -- tornados/tornadoes
    *  volcano -- volcanos/volcanoes
    *  zero -- zeros/zeroes
  \end{taskitem}

\item [不变] 拼写不变

  \textbf{单复数同型}:
  \begin{taskitem}(3)
    *  cod -- cod
    *  deer -- deer
    *  fish -- fish
    *  offspring -- offspring
    *  perch -- perch
    *  sheep -- sheep
    *  trout -- trout
  \end{taskitem}

  注:很多鱼类的复数形式都是不变的,但有例外

  \textbf{本身就是复数,只有复数形式}:
  \begin{taskitem}(4)
    *  barracks
    *  crossroads
    *  dice
    *  gallows
    *  headquarters
    *  means
    *  series
    *  species
  \end{taskitem}

\item[借用] 单词和其复数形式都借用自其他语言:

  \begin{taskitem}(3)
    *  alga -- algae
    *  larva -- larvae
    *  vertebra -- vertebrae
  \end{taskitem}

  \textbf{以``us''结尾的转为``a''(适用于专业术语)}:
  \begin{taskitem}(3)
    *  corpus -- corpora
    *  genus -- genera
  \end{taskitem}

  \textbf{以``us''结尾的转为``i''}:
  \begin{taskitem}(3)
    *  alumnus -- alumni
    *  bacillus -- bacilli
    *  focus -- foci
    *  nucleus -- nuclei
    *  radius -- radii
    *  stimulus -- stimuli
    *  syllabus -- syllabuses
    *  terminus -- termini
  \end{taskitem}

  \textbf{以``um''结尾的转为``a''}:
  \begin{taskitem}(3)
    *  addendum -- addenda
    *  bacterium -- bacteria
    *  datum -- data
    *  erratum -- errata
    *  medium -- media
    *  ovum -- ova
    *  stratum -- strata
  \end{taskitem}

  \textbf{以``is''结尾的转为``es''}:
  \begin{taskitem}(2)
    *  analysis -- analyses
    *  axis -- axes
    *  basis -- bases
    *  crisis -- crises
    *  diagnosis -- diagnoses
    *  emphasis -- emphases
    *  hypothesis -- hypotheses
    *  neurosis -- neuroses
    *  oasis -- oases
    *  parenthesis -- parentheses
    *  synopsis -- synopses
    *  thesis -- theses
  \end{taskitem}


  \textbf{以``on''结尾的转为``a''}:
  \begin{taskitem}(2)
    *  criterion -- criteria
    *  phenomenon -- phenomena
    *  automaton -- automata
  \end{taskitem}


  \textbf{意大利语,变``o'' 为 ``i''}:
  \begin{taskitem}(3)
    *  libretto -- libretti
    *  tempo -- tempi
    *  virtuoso -- virtuosi
  \end{taskitem}

  \textbf{希伯来语,末尾加 ``im''}:
  \begin{taskitem}(3)
    *  cherub -- cherubim
    *  seraph -- seraphim
  \end{taskitem}

  \textbf{希腊语,末尾加ta}:
  \begin{taskitem}(3)
    *  schema -- schemata
  \end{taskitem}
\end{description}

\subsection{限定词}
\label{sub:determin}

名词短语中,限定词位置大体可以分为前位、中位、后位(见\cref{tab:determ})。
\index{概念!限定词!前位}\index{概念!限定词!中位}\index{概念!限定词!后位}

\begin{table}[htbp]
  \centering \small
  \begin{talltblr}[ caption = {名词短语中限定词的位置},
    label = {tab:determ},
    ]{
      width=0.9\linewidth, colspec={lX},
      rowsep=2pt, colsep=4pt,
    }
    \toprule
    \SetCell[c=2]{l} \textbf{前位限定词\qquad (互斥,只选其一)}& \\
    感叹 & such, what\\
    倍数词 & double, twice, three times\\
    分数词 & one-third, one-fifth\\
    数量词 & all, both, half\\
    \midrule
    \SetCell[c=2]{l} \textbf{中位限定词\qquad (互斥,只选其一)} & \\
    冠词 & a, an, the\\
    物主代词 &  my, our, your, his, her, its, their\\
    名词所有格 & the rabbit's, the wolf's\\
    关系代词 & whose,which\\
    指示代词 & this, that‍‍, these, those\\
    wh-ever 限定词 & whichever, whatever, whoever\\
    疑问代词 & what, whose,which\\
    不定代词 & {enough, each, every, some, any, either, neither,\\
      lot(s)/piece/few/plenty of, no}\\
    \midrule
    \SetCell[c=2]{l} \textbf{后位限定词}  \\
    基数词 & one,two,three\\
    数量词 & {few, little, many, much, several\\ large/great/good
      number of} \\
    (一般)序数词 & {first, second, fourth, twentieth,\\ next, last, past, (an)other}\\
    \bottomrule
  \end{talltblr}%
\end{table}

\textbf{限定词互斥的例外:}
\begin{itemize}
\item 中位限定词\textbf{every}有时可在属格后面,例:

  His every action shows that he is a very determined young man.
\item 前位限定词 such 用作代用式 (pro-form)时,也能接在数量词 any, no 和 many 以
  及基数词的后面:

  no/any/several/many/forty-one such incidents \ldots
\end{itemize}


除作前位限定词外,all, both 和 half 作为代词还能带 of- 短语 (partitive
of-phrase)表示“部分”。 \textbf{与名词连用时, of- 短语可有可无,与代词连用则非
  用 of短语不可}:
\begin{taskitem}(2)
  * all (of) the students
  * all of them/whom
  * both (of) his eyes
  * both of them/which
  * half (of) the time/cost
  * half of it/this
\end{taskitem}

以下用法可表示类指:
\begin{itemize}
\item a + 可数名词单数,如 a tiger (\textbf{不定指}).
\item (the) + 可数名词复数,如 the tiger(\textbf{定指}),tigers(\textbf{不定指})。
\item 零冠词 + 不可数名词,如 milk.
\end{itemize}

\textbf{如果泛指整个群体里的所有成员,就不能与a/an 。}
\begin{itemize}
\item \unbf{The tiger} is in danger of becoming extinct.

  整个老虎种族,不能用a tiger,只有the tiger is 或者 tigers are。

\item Do you like horses?
\end{itemize}

\subsection{平行结构}

如果两个名词一起放在同一平行结构里,即使是单数具数名词,也有\textbf{省略冠词}的倾向:
\begin{taskitem}(4)
  * face to face
  * day by day
  * hand in hand
  * eye to eye
  * arm in arm
  * mile upon mile
  * back to back
  * side by side
\end{taskitem}

有时一个名词与另一个具有相反意义的名词相平衡,如;
\begin{taskitem}(2)
* from father to son
* husband and wife
* from (the) right to (the)left
* from (the) beginning to (the) end
* both mother and child
* neither child nor adult
\end{taskitem}

这些平衡结构中的名词基本上部会有数的变化,也不可能有限定词和修饰
语,\textbf{实际上是习语},是冠词“固定“用法的例子。



\textbf{带有名词重复的短语通常具有副词功能。}

\subsection{属格}

\begin{description}
\item[the genitive] 属格(所有格):名词或者形容词,used to show possession or
  close connection between two things. 展示两者之间的所属或紧密关系。
\item[of-construction] of介词 + 名词性短语,belonging to sb/sth; relating to sb/sth 属于(某人/某物);关
  于(某人/某物)。
\end{description}

\begin{itemize}
\item What is \unbf{the ship's} name?

  What is the name \unbf{of the ship}?

\item Some people's opinions

  the options of some people (不很清晰,少用)
\end{itemize}
许多情况下,这两种形式意义相同且完全能够成立。

属格和of结构应根据如下侧重点,结合实际情况加以选择:
\begin{enumerate}
\item \textbf{具有人性特点的名词类别常常用属格。}例如人、高等动物、集体(多个个人组
  成)、地理位置(人类生活区域)、时间、人的感官活动等。
  \begin{taskitem}(2)
    * the nations resources
    * Europe's future
    * China's development
    * the school's history
    * today's paper
    * a day's work
    * the body's needs
    * the game's history
  \end{taskitem}

\item \textbf{属格有特指的意思,带有限定性;of结构有泛指的意思。}
  \begin{itemize}
  \item Susan's son (苏珊的儿子,单看短语本身她也只有一个儿子)

  \item a son of Susan (苏珊有多儿子,其中之一)
  \end{itemize}

\item 根据\textbf{末尾焦点 (end focus) 和末尾重心 (end-weight)原则。}更复杂和
  重要的单位应放在名词短语末尾。这样一来,\textbf{属格倾向于将信息中心放在名
    词中心词上,of结构倾向于将中心放在介词补足语上。}

  \begin{itemize}
  \item The explosion damaged \unct{the ship's funnel}{焦点在funnel}.
  \item The explosion damaged \unct{the funnel of the ship}{焦点在ship}.

    爆炸损坏了船上的烟囱。

  \item the ears of the man in the deckchair. 根据尾重,避免隔断产生歧义

    帆布躺椅上那个男人的耳朵。
  \item \sout{the man's ears} in the deckchair. [歧义,惊悚]

    帆布躺椅上有那个男人的耳朵。
  \end{itemize}

\item 两者皆可的情况下,属格往往比较简明清晰,优先考虑。
\end{enumerate}

\improve[inline]{关于of结构还可参见 17.38f}

\subsection{独立属格 (the independent genitive)}

如果上下文中已交代清楚属格后面的中心词,则中心词可以省略。省略的结果就构成了
所谓“\textbf{独立属格}”。
\begin{itemize}
\item  My \unbf{car} is faster than \unbf{John's}.  [省略 car]
\item  Her \unbf{memory} is like \unbf{an elephant's}. [省略memory]
\item \unbf{Mary's} was the prettiest \unbf{dress}. [省略dress]
\item If you can't afford a \unbf{sleeping bag}, why not borrow \unbf{somebody
    else's}? [省略sleeping bag]
\item The New York's \unbf{population} is greater than \unbf{Chicago's}.
\end{itemize}

需要注意,of结构如果出现在可比较语境 (comparable environments)中,of前面通常
要加指示代词\textbf{that/those}。上句如采用of结构,应是
\begin{itemize}
\item The \unbf{population} of New York is greater than \unbf{that of Chicago}.
\end{itemize}

\subsection{后置属格(双重属格)}

of结构可以与属格结合起来产生一种称为\textbf{后置属格(双重属格)}的结构,且后
置属格基本上只能是人的属格。

简单理解的话就是\textbf{将本应前置的属格后置,并在其前加of。}

\begin{itemize}
\item some friends \unbf{of Jim's}

  等同于 some \unbf{Jim's} friends,将Jim's 后置并在前面加of

\item several students \unbf{of his}

  等同于several \unbf{his} students,将 his 后置并在前面加of
\end{itemize}

后置属格可能是古英语传承自其他语言的遗留,与当前语法有些不同之处,很难简明扼
要说清楚。

当中心语为人称,即人称 + of + 后置修饰语 时,后置修饰语可以不用属格。
\begin{itemize}
\item A friend of \unbf{Jim('s)} is coming to the party.
\end{itemize}

试分析下组例子中的不同含义:
\begin{itemize}
\item a painting of my sister's. [我姐妹画的,或者属于我姐妹的,一幅画]
\item a painting of my sister. [画有我姐妹的一幅画]
\item a painting by my sister. [我姐妹画的一幅画]
\item a painting of my sister by my brother. [我兄弟画有我姐妹的一副画像]
\end{itemize}

后置属格在中国英语应试教育中时有出现。


\section{代词}

代词更多情况下不是简单“代替”名词,而是代替“名词性短语”。

\subsection{人称代词}

\begin{table}[htbp]
  \centering \small
  \begin{talltblr}[ caption = {人称代词的主格、宾格、属格以及反身代词},
    label = {tab:whoself},
    note{a} = {限定式属格:在名词短语中起限定作用,修饰其后的中心语。},
    note{b} = {独立式属格:省略场景中已知要修饰的名词,独立使用。},
    ]{
      width=\linewidth, colspec={llllllllll},
      rowsep=1pt, colsep=2pt,
      row{1} = {font=\bfseries, c},
    }
    \toprule
    & \SetCell[c=4]{m}单数 & & & & \SetCell[c=2]{m}复数 & & \SetCell[c=2]{m}单复数 & \\ \midrule
    主格 & I & he & she & \SetCell[r=2]{m} it & we & they & \SetCell[r=2]{m} you & who \\
    宾格 & me & him & her & & us & them & & who(m) \\
    反身代词 & myself & himself & herself & itself & ourselves & themselves
    & {yourself \\ yourselves} & -- \\ \midrule
   \SetCell[c=3]{m} \textbf{属格(物主代词)} &&&&&&&&&\\
    限定式 & my & \SetCell[r=2]{m} his & her & \SetCell[r=2]{m} its & our & their & your &  \SetCell[r=2]{m} whose \\
    独立式 & mine & & hers & & ours & theirs & yours & \\
    \bottomrule
  \end{talltblr}%
\end{table}

宾语位置如是人称代词,则使用人称代词宾格,这个大家一般都了解。但\textbf{关于主语补
  语,在正式文体中使用主格,非正式文体中使用宾格且越来越流行。}

\begin{enumerate}
\item \unct{It}{S} \unct{was}{V} \unct{he}{$\mathbf{C_S}$}.

  非正式文体中,不用he,而用him。

\item  \unct{It}{S}\unct{'s}{V} \unct{I}{$\mathbf{C_S}$} who's to blame.

  我负有(不好的)责任。非正式文体中,变 I为me.
\end{enumerate}

\begin{sdbig4}{He is}{more intelligent than}{as intelligent as}{\unbf{she (is)}.}
\end{sdbig4}
用主格she的话,加is更符合并列连接的语义。非正式文体中,变she (is)为her.

\subsection{代用的it}

由于 it 是人称代词中最中性的而且在词义上是无标记的,它就用来作为“\textbf{虚主
  语}”或“\textbf{代用主语}”,尤其用在那些表示时间、距离或天气情况的词组中:
\begin{itemize}
\item What time is it?
\item How far is it to York?
\item It's warm today.
\item It's getting dark.
\end{itemize}

也许完全虚指或无所指it 的最好的例子是在下列习语中. it在这些习语中接在动词后
面,笼统地泛指“生活”等:
\begin{itemize}
\item At last we've made \unbf{it}. [achieved success]
\item have a hard time of \unbf{it}. [to find life difficult]
\item How's \unbf{it} going?
\item Go \unbf{it} alone.
\item You're in for \unbf{it}. ["You're going to be in trouble."]
\end{itemize}

it 也能用来\textbf{代替谓体 (predication)},尤其是\textbf{代替表示特征的补语}。
\begin{sdbig4}{She was}{a rich woman}{rich}{and she looked \unbf{it}.}
\end{sdbig4}

上句it 指代前面所说a rich woman 或者 rich。

\subsection{反身代词}
反身代词以 \emph{-self} (单数) 和 \emph{-selves} (复数)结尾(见\cref{tab:whoself})。
顾名思义,反身代词“反映“分句或句子中另一个名词性成分(通常是主语),并与它形成
互指关系:
\begin{itemize}
\item She allowed herself a rest.
\item He is not himself today.
\end{itemize}

% Please add the following required packages to your document preamble:

% \usepackage{tabularray}
% \UseTblrLibrary{booktabs}
% \UseTblrLibrary{nameref}

\begin{table}[htbp]
  \centering \small
  \begin{talltblr}[ caption = {反身代词的功能},
    label = {tab:reflexive},
    ]{
      width=\linewidth, colspec={ccl},
      rowsep=2pt, colsep=4pt,
      row{1} = {c, font=\bfseries},
    }
    \toprule
    先行词        & 反身代词        & 举例            \\ \midrule
    \SetCell[c=2]{l,m} \textbf{基本用法} &                    &         \\
    主语         & 直接宾语        & \unbf{They} helped \unbf{themselves}.     \\
    主语         & 间接宾语        & \unbf{She} allowed \unbf{herself} a rest. \\
    主语         & 主语补语        & \unbf{He} is not \unbf{himself} today.    \\
    主语         & 介词补足语       & \unbf{Jim} pay for \unbf{himself}.        \\\midrule
    \SetCell[c=2]{l,m} \textbf{强调用法} &                   &          \\
    主语         & 同位语短语       & \unbf{We} couldn't come \unbf{ourselves}. \\
    主语         & 同位语短语       & \unbf{We} \unbf{ourselves} couldn't come.\\
    \bottomrule
  \end{talltblr}%
\end{table}

当一个非限定性分句 (non-finite clause) 或一个名词化短语有一个以\textbf{最不确定的
  人 (``someone or other'') }作隐含主语时,可用反身代词 \textbf{oneself}(或用它在非
正式文体中的等同词 \textbf{yourself})。
\begin{itemize}
\item Voting for \unbf{oneself} is unethical.

  给自己投票是不道德的。(非正式文体中可用yourself)
\end{itemize}

如涉及到自身,以下介词后面必须用反身代词:
\begin{taskitem}
  * look at/after
  * do with
  * thinks of
  * take upon
  * a story about
  * portraits/photograph of
\end{taskitem}

\subsection{前指、后指、实景所指和先行词}
\label{sub:anacata}

\begin{description}
\item[实景所指situational reference] 指语言外的实际环境,如双方都心照不宣或被指向的人和
  物。
\item[前指anaphoric reference] 代词或其他指代词与\textbf{前面}话语中的名词性话语形成互指关系。
\item[后指cataphoric reference] 代词或其他指代词与\textbf{后面}话语中的名词性话语形成互指关系。
\item[先行词antecedent reference] 与代词或其他指代词形成互指关系的那部分名词性话语。
\item[近指] 既可以前指也可以后指的指示代词,如this/these.
\item[远指] 只可以前指的指示代词,如that/those.
\end{description}

\begin{itemize}

\item He told the story like \unbf{this}: ``Once upon a time \ldots{} ''

  因直接引述的话语在句子后面,只能使用近指的this,而不能用that.

\item What do you think of TH\`{A}T! Bob smashes up my car, and then expects me to
  play for the repairs.

  在极为有限的语境中,例如表示愤慨等强烈负面情绪时,that可用于后指。
\end{itemize}

\subsection{名词性关系分句}

\begin{itemize}
\item The book \unbf{which} you ordered last month has arrived.

  You ordered the book last month. It has arrived.

  你上月订购的这本书到货了。
\end{itemize}

\subsubsection{限制性和非限制性关系从句}

\textbf{关系代词引导名词性关系分句,总是放在关系分句开头。有which who whom that
  zero零关系代词,他们都可以作限制性关系代词,但只有that或zero可做限制性关系
  代词。}

关系分句可根据其与先行词的关系,分为限制性和非限制性两种。

\textbf{1. 限制性关系分句与它们的先行词或中心词在读音上是一气呵成的,以示对先行词所指
  对象的限制。}
\begin{itemize}
\item This is not something \unbf{that/which} would disturb me \`{A}NYway.

  无论如何,这都不会打扰我。(something that \ldots{} me连读)
\item I'd like to see the car \unbf{that/which/( )} you bought last week. (zero由( )表
  示)

  我想看看你上周买的那辆车。(the car that/which \ldots{} last week连读)
\item The lady \unbf{whose} daughter you met is Mrs. Brown.

  你见过她女儿的那位女士是布朗夫人。(The lady whose \ldots{} met连读)
\end{itemize}

\textbf{2. 非限制性关系分句是一种插入性说明,它通常是对先行词加以描述而不是对先行词作进一
  步的限定,可在关系代词前面(先行词或中心词之后)停顿。}
\begin{itemize}
\item  They operated like poliT\`{I}cians | \unbf{who notoriously have no sense of humour
  at \`{A}LL}.
\end{itemize}

\subsubsection{whose和of which}

与who和whom不同,whose可以指人称,也可以指非人称,但人们不太原意用whose来指
非人称的先行词,可以用of which来表示,但也常显得有些别扭。
\begin{itemize}
\item \unbf{The lady whose} daughter you met is Mrs. Brown.
\item \unbf{The house whose} roof was damaged has now been repaired.
\item The house \unbf{of which} the roof was damaged the roof of which was damaged.
\item The house the roof \unbf{of which} was damaged the roof of which was damaged.
\end{itemize}

\subsection{不定代词}

不定代词缺少人称代词、反身代词、物主代词、指示代词、(某种程度上)wh- 代词所
具有的特指成分。

不定代词在逻辑意义上是量词,与其相同或类似形式的限定词密切对应(见
\cref{tab:someany})。

\begin{table}[hbtp]
\centering
\caption{主要的不定代词和限定词}
\label{tab:someany}
\resizebox{\textwidth}{!}{%
\begin{tabular}{|c|c|c|ccc|}
\hline
\multirow{2}{*}{}    & \multirow{2}{*}{\textbf{数}}  & \multirow{2}{*}{\textbf{功能}} & \multicolumn{2}{c|}{\textbf{可数}}                                               & \multirow{2}{*}{\textbf{不可数}}  \\ \cline{4-5}
                     &                     &                     & \multicolumn{1}{c|}{人称的}             & \multicolumn{1}{c|}{非人称的}      &                       \\ \hline
\multirow{5}{*}{\textbf{通用}} &
  \multirow{3}{*}{\textbf{单数}} &
  \multirow{2}{*}{代词} &
  \multicolumn{1}{c|}{everyone, everybody} &
  \multicolumn{1}{c|}{everything} &
  \multirow{2}{*}{(it(...)) all} \\ \cline{4-5}
                     &                     &                     & \multicolumn{2}{c|}{each}                                             &                       \\ \cline{3-6}
                     &                     & 限定词                 & \multicolumn{2}{c|}{every, each}                                        & all                   \\ \cline{2-6}
                     & \multirow{2}{*}{\textbf{复数}} & 代词                  & \multicolumn{2}{c|}{(they(...)) all/both}                             & \multirow{2}{*}{}     \\ \cline{3-5}
                     &                     & 限定词                 & \multicolumn{2}{c|}{all/both}                                         &                       \\ \hline
\multirow{3}{*}{\textbf{断定}}  & \multirow{2}{*}{\textbf{单数}} & 代词                  & \multicolumn{1}{c|}{someone, somebody} & \multicolumn{1}{c|}{something} & \multirow{3}{*}{some} \\ \cline{3-5}
                     &                     & 限定词                 & \multicolumn{2}{c|}{a(n)}                                             &                       \\ \cline{2-5}
                     & \textbf{复数}                  & 代词和限定词              & \multicolumn{2}{c|}{some}                                             &                       \\ \hline
\multirow{3}{*}{\textbf{非断定}} & \multirow{2}{*}{\textbf{单数}} & 代词                  & \multicolumn{1}{c|}{anyone, anybody}  & \multicolumn{1}{c|}{anything}  & \multirow{3}{*}{any}  \\ \cline{3-5}
                     &                     & 限定词                 & \multicolumn{2}{c|}{either, any}                                        &                       \\ \cline{2-5}
                     & \textbf{复数}                  & 代词和限定词              & \multicolumn{2}{c|}{any}                                              &                       \\ \hline
\multirow{5}{*}{\textbf{否定}} &
  \multirow{3}{*}{\textbf{单数}} &
  \multirow{2}{*}{代词} &
  \multicolumn{1}{c|}{no one, nobody} &
  \multicolumn{1}{c|}{nothing} &
  \multirow{3}{*}{none} \\ \cline{4-5}
                     &                     &                     & \multicolumn{2}{c|}{none}                                             &                       \\ \cline{3-5}
                     &                     & 代词和限定词              & \multicolumn{2}{c|}{neither}                                          &                       \\ \cline{2-6}
                     & \textbf{复数}                  & 代词                  & \multicolumn{2}{c|}{none}                                             &                       \\ \cline{2-6}
                     & \textbf{单数或复数}               & 限定词                 & \multicolumn{3}{c|}{no}                                                                       \\ \hline
\end{tabular}%
}
\end{table}

后置修饰语 else 可加在复合代词的后面.它的词义在下面例子的
括号中作了解释:
\begin{itemize}
\item everyone else [every other person]
\item nobody else [no other person]
\item anything else [any other thing]
\end{itemize}

\textbf{如用属格,'s 要加在else的后面},而不直接加在代词后面:
\begin{itemize}
\item I must be drinking \unbf{someone else's} coffee.
\item His hair is longer than \unbf{anybody else's}.
\end{itemize}

\textbf{除 not 之外,以下否定范围也要用非肯定形式的否定词 any 等}
\begin{enumerate}
\item 具有否定形式的词: never, no, neither; nor
\item 具有否定意义的词:
  \begin{enumerate}
  \item 副词和限定词 barely, little, few, only, seldom 等
  \item “隐含的否定词” just, before; fail, prevent; reluctant, hard, difficult
    等;以及带 too 的比较用法
    \begin{itemize}
    \item Jean will always manage to do \unbf{something} useful. manage to do:
      设法做某事
    \item Jean will never manage to do \unbf{anything} useful.
    \item There was \unbf{a good} chance \unbf{somebody} would come.
    \item There was \unbf{little} chance \unbf{anybody} would come.
    \item John was \unbf{eager} to read \unbf{some} of the books.
    \item John was \unbf{reluctant}/\unbf{too lazy} to read \unbf{any} (of the) books.
    \end{itemize}
  \end{enumerate}
\end{enumerate}

\textbf{最终选用some组合词还是any组合词,取决于整个句子的内在含义或基本含义。}
\begin{itemize}
\item Did \unbf{somebody} telephone last nigh?

  暗示说话人期待有来电,anybody则不暗示这种期待。

\item Would you like \unbf{some} tea?

  期待对方接受,礼貌。

\item Will \unbf{somebody} please open the door?


\item But \unct{what if}{要是…会怎样} \unbf{somebody} decides to break the
  rules?
\end{itemize}

\todo[inline]{either与 both, neither与no/none的差别单独成章}

\subsection{数词}
美国计算法中,billion是one thousand million (1,000,000,000),10亿;此外还
有trillion(万亿),quadrillion(1000万亿)。老式英国计算方法与此不同。

\begin{table}[tp!]
  \centering \footnotesize
  \caption[skip=2pt]{基数词和序数词}
  \label{tab:onefirst}
  \begin{tabular}[hp!]{rlrl}
    \toprule
    \multicolumn{2}{c}{基数词 } & \multicolumn{2}{c}{序数词 } \\ \midrule
    1         & \textbf{one}                   & 1st         & \textbf{first}                   \\
    2         & \textbf{two}                   & 2nd         & \textbf{second}                  \\
    3         & \textbf{three}                 & 3rd         & \textbf{third}                   \\
    4         & \textbf{four}                  & 4th         & fourth                     \\
    5         & \textbf{five}                  & 5th         & fi\textbf{f}th                   \\
    6         & \textbf{six}                   & 6th         & sixth                      \\
    7         & \textbf{seven}                 & 7th         & seventh                    \\
    8         & \textbf{eight}                 & 8th         & eigh\textbf{t}h                  \\
    9         & \textbf{nine}                  & 9th         & nin\textbf{t}h                   \\
    10        & \textbf{ten}                   & 10th        & tenth                      \\
    11        & \textbf{eleven}                & 11th        & eleventh                   \\
    12        & \textbf{twelve}                & 12th        & twel\textbf{f}th                 \\
    13        & \textbf{thirteen}              & 13th        & thirteenth                 \\
    14        & fourteen                 & 14th        & fourteenth                 \\
    15        & fifteen                  & 15th        & fifteenth                  \\
    16        & sixteen                  & 16th        & sixteenth                  \\
    17        & seventeen                & 17th        & seventeenth                \\
    18        & eighteen                 & 18th        & eighteenth                 \\
    19        & nineteen                 & 19th        & nineteenth                 \\
    20        & \textbf{twenty}                & 20th        & twentieth                  \\
    21        & twenty-one               & 21st        & twenty-first               \\
    22        & twenty-two               & 22nd        & twenty-second              \\
    23        & twenty-three             & 23rd        & twenty-third               \\
    24        & twenty-four              & 24th        & twenty-fourth              \\
    25        & twenty-five              & 25th        & twenty-fifth               \\
    26        & twenty-six               & 26th        & twenty-sixth               \\
    27        & twenty-seven             & 27th        & twenty-seventh             \\
    28        & twenty-eight             & 28th        & twenty-eighth              \\
    29        & twenty-nine              & 29th        & twenty-ninth               \\
    30        & \textbf{thirty}                & 30th        & thirtieth                  \\
    40        & \textbf{forty}                 & 40th        & fortieth                   \\
    50        & \textbf{fifty}                 & 50th        & fiftieth                   \\
    60        & sixty                    & 60th        & sixtieth                   \\
    70        & seventy                  & 70th        & seventieth                 \\
    80        & eighty                   & 80th        & eightieth                  \\
    90        & ninety                   & 90th        & ninetieth                  \\
    100       & a/one \textbf{hundred}         & 100th       & (one) hundredth            \\
    101       & a/one hundred and one    & 101st       & (one) hundred and first    \\
    102       & a/one hundred and two    & 102nd       & (one) hundred and second   \\
    1 000     & a/one \textbf{thousand}        & 1 000th     & (one) thousandth           \\
    1 001     & a/one thousand (and) one & 1 001st     & (one) thousand (and) first \\
    2 000     & two thousand             & 2 000th     & two thousandth             \\
    10 000    & ten thousand             & 10 000th    & ten thousandth             \\
    100 000   & a/one hundred thousand   & 100 000th   & one hundred thousandth     \\
    1 000 000 & a/one \textbf{million}         & 1 000 000th & (one) millionth            \\
    \bottomrule
  \end{tabular}
\end{table}

a/one 的不同,数词后面只能用one.
\begin{itemize}
\item 1100: \unbf{a/one} thousand \unbf{one} hundred

  thousand后面不能用a
\end{itemize}

与名词和动词中的 -y 改为 -ie(s) 不同,以 -y 结尾的基数词变为以 -ie(th) 结尾的
序数词时,要增加一个音节。试比较:

sixty \qquad the sixties\doulos{/'sɪkstiz/} \qquad the sixtieth\doulos{/ˈsɪkstiəθ/}

\subsubsection{时间}

我们总是以“百”为单位来读年份:
\begin{itemize}
\item in 1985: nineteen hundred and eighty-five
\item in 1600s: sixteen hundreds (17世纪)
\end{itemize}
其他例子:
\begin{itemize}
\item in the 17th century "seventeenth century

\item in the 1980s 读作(但很少写成): "nineteen-eighties"

  在20世纪80年代
\end{itemize}

月、日通常的表示形式是:
\begin{itemize}
\item 7(th) February 或 February 7(th)

  读作 the seventh of February, February (the) seventh或 February seven。
\end{itemize}

在日期的缩写中,数词通常由斜线或句点分隔开:
\begin{itemize}
\item 7/2/82 或 7.2.82

  英国英语日期顺序日月年 The 7(th) February 1982

  美国英语日期顺序月日年 July 2(nd), 1982。
\end{itemize}

表示时刻缩写形式的数词中用冒号(尤其在美国英语中),或英文句号(英国英语),如:
\begin{itemize}
\item 6:30 或 6.30

  读作six-thirty 或 half past six。
\end{itemize}

\subsubsection{分数}

 普通分数的书写和朗读形式如下:
 \begin{taskitem}(2)
   * $\frac{1}{2}$ a/one half
   * $\frac{2}{3}$ two-third\textbf{s}
   * $\frac{1}{3}$ a/one third
   * $\frac{7}{8}$ seven-eighth\textbf{s}
   * $\frac{1}{4}$ a/one quarter
   * $3\frac{3}{4}$ three and three-quarter\textbf{s}
   * $\frac{1}{5}$ a/one fifth
   * $\frac{8}{76}$ eight seventy-sixth\textbf{s}
   * $\frac{8}{76}$ eight over seventy-six
 \end{taskitem}

\textbf{连字符号不与不定冠词连用},如 one-third 用连字符号,但 a third 就不用。

\begin{itemize}
\item He won the race by \unbf{a/one hundredth} of a second. $\frac{1}{100}$
\item He won the race by \unbf{a/one two-hundredth} of a second. $\frac{1}{100}$
\item He got \unbf{three hundredths} of the money $\frac{1}{300}$
\end{itemize}

在小数中,整数部分按通常基数词的读法读出 seventy-one 等,小数点以后的数作为单
个的数字读出five three 等:
 \begin{itemize}
 \item 71.53 seventy-one \unbf{point} five three

 \item 0.426  zero \unbf{point} four two six
 \end{itemize}

 大部分欧洲国家、南非习惯用逗号(读作 comma) 而不用句点来书写小数点:
 \begin{itemize}
 \item 1,2\% one \unbf{comma} two per cent
 \end{itemize}

 \subsubsection{数学符号}
 \begin{taskitem}(2)
 * = equals
 * + plus
 * - minus
 * x times or multiplied by
 * + divided by
 * $\sqrt{}$ the (square) root of
 \end{taskitem}

 \begin{itemize}
 \item $(17 - \sqrt{9} + \frac{65}{5}) - (4X3) =15$
 \end{itemize}
读成 seventeen minus the square root of nine, plus sixty-five over five,
minus four times three, equals fifteen. (数学符号使数字之间的关系很明确.)



%%% Local Variables:
%%% mode: LaTeX
%%% TeX-master: "main"
%%% End:
