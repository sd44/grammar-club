\chapter{副词从句}

副词从句是三种从属从句(名词、定语、副词从句)中最简单的一种。它与主要从句之
间,有点\textbf{像对等从句}的关系,很容易了解]。

\section{副词从句与对等从句的比较}

请看下例:

\begin{enumerate}
\item \unct{Because}{从属连接词} \unct{he needs the money}{副词从句}, \unct{he
    works hard}{主要从句}.

  因为他缺钱,所以他勤奋工作。
\item \unct{He needs the money}{对等从句
  }, \unct{and}{对等连接词} \unct{he
    works hard}{对等从句}.

  他需要钱,也勤奋工作。
\end{enumerate}

例 1 是分成主、从的复句结构。其中副词从句 He needs the money 和主要从句He works
hard 分别都是完整、独立的简单句,以一个连接词连起来。这和例 2中两个对等从句的情形
完全相同。唯一的差别是对等从句使用对等连接词(例 2中的and),连接起来的两个从句地
位相等,没有主从之分,也不需互相解释。副词从句则使用从属连接词(例1 中的
because),使得 because he needs the money成为从属地位的从句,当作副词使用,用来
修饰主要从句中的动词 works(交待works hard的原因)。除了这一层修饰关系之外,副词
从句和对等从句同样单纯。

\section{副词从句与名词从句的比较}

副词从句和名词从句就有较大的差别了。请看下例:

\begin{enumerate}
\item \unct{The witness}{S} \unct{said}{V} \unct{that}{连接词} \unct{he saw the
    whole thing}{名词从句}.

  证人说他目睹了事情发生的全过程。
\item \unct{The witness}{S} \unct{said}{V} \unct{this}{O}, \unct{though}{连接词}
  \unct{he didn't really see it}{副词从句}.

  证人这样说,尽管他没有真正看到。
\end{enumerate}

先来观察一下名词从句和副词从句的共同点。首先,两者原来都是完整、独立的简单句
(例1 中的 He saw the whole thing 与例 2 中的 He didn't really see it)。然后,
两者都是加上从属连接词构成从属从句,但是由此开始有了差别。名词从句加的连接词
是that,表示“那件事情”,此外没有别的意义。副词从句加的连接词,如
例 2 的though,以及上节例子中的 because等等,都是有意义的连接词,表达两句话之
间的逻辑关系:\textbf{though表示让步,because 表示原因,if表示条件。使用的连接词不
同,一个有意义,一个没有意义,这是副词从句和名词从句第一个重要的差别。}

第二个差别是:名词从句属于名词类,要放在主要从句中的名词位置使用,副词从句则
不然。例1 中主要从句 The witness said 部分尚不完整,在及物动词 said之后还要有
个名词当宾语,构成 S+V+O 的句型才算完成。取一个独立的简单句 He saw the whole
thing 外加上没有意义的连接词that,造成一个名词从句,就可以放入主要从句 The
witness said后面的宾语位置使用,成为例 1 的形状。

副词从句情况不同。\textbf{它是修饰语的词类,要附在一个完整的主要从句上作修饰语使用。}
如例2 He didn't really see it 是完整的单句,外面加上表示让步的连接词 though构
成副词从句。主要从句 The witness said this已经是完整的句子(S+O+V),把副词从
句 though he didn't really see it直接附上去,当作副词,用来修饰动词said。因为
两个从句都是完整的简单句,所以说其间的关系\textbf{很像对等从句}的关系。这是副词从句与
名词从句第二个重要的差别。

\section{副词从句的种类}

副词从句因为结构十分单纯,所以学习副词从句的重点只是在认识各种连接词,以便写
作时可以选择贴切的连接词来表达各种逻辑关系。以下按照各种逻辑关系把副词从句的
连接词大略分类。

\subsection{时间、地方}

\begin{enumerate}
\item He became more frugal \unct{after}{连接词} \unct{he got married}{副词从
    句}.

  他结婚以后变得比较节俭。
\end{enumerate}
副词从句修饰动词 became 的时间。

\begin{enumerate}[resume]
\item I'll be waiting for you \unct{until}{连接词} \unct{you're married}{副词从
    句}.

  我会等你,直到你结婚为止。
\end{enumerate}
副词从句修饰动词 will be waiting 的时间。

附带说明一下:\textbf{未来时间的副词从句,}虽然还没有到发生的时间,可是语气上必须当
作“到了那个时候”来说,所以\textbf{时态要用现在式来表示}(如例2 中的 are married)。
这是属于语气的问题,在从前介绍语气的单元中曾说明过。

\begin{enumerate}[resume]
\item It was all over \unct{when}{连接词} \unct{I got there}{副词从句}.

  我赶到的时候事情都结束了。
\end{enumerate}
副词从句修饰动词 was 的时间。

when这个\textbf{连接词},也可以当做\textbf{关系词}来使用,这点留待下一章讲到关系从句时再
详细说明。

\begin{enumerate}[resume]
\item A small town grew \unct{where}{连接词} \unct{three roads met}{副词从句}.

  一个小镇在三条路交会处发展起来。
\end{enumerate}
副词从句修饰动词 grew 的地方。

同样的,where 这个连接词也可以当作关系词来解释。

\subsection{条件}

\begin{enumerate}
\item \unct{If}{连接词} \unct{he calls}{副词从句}, I'll say you're
  sleeping.

  如果他打电话来,我就说你在睡觉。
\end{enumerate}
副词从句修饰动词 will say 的条件——如果打来就会说,不打来就不说了。

在表示条件的副词从句中,如果时间是未来,也必须以“当作真正发生”的语气来说,
所以要用现在式的动词。同时请注意say 的宾语(名词从句)you're sleeping也用现在
式,因为这是当作已经打来了,自然要说“在睡觉”,而不是“要去睡觉”(will be
sleeping)。只有主要从句 I'll say用未来式的动词,因为如果打来了“就会”说,这
表示现在还没说!

\begin{enumerate}[resume]
\item He won't have it his way, \unct{as long as}{连接词} \unct{I'm here}{副词从
    句}.

  只要我在,不会让他称心如意。
\end{enumerate}
副词从句修饰动词 won't have 的条件。

as long as 也可以用比较级来诠释。
\begin{enumerate}[resume]
\item \unct{Suppose}{连接词} \unct{you were ill}{副词从句}, where would you go?

  假定你生病了,你会到哪里去?
\end{enumerate}
副词从句修饰动词 would go 的条件。

\textbf{suppose 本来是动词,这个副词从句原来是 supposing that you were ill的句型,经过
  省略后才成为只剩 suppose 一字当连接词用。}同时请注意例 3中两个动词都是非事实的
假设语句。

\subsection{原因、结果}

\begin{enumerate}
\item \unct{As}{连接词} \unct{there isn't much time left}{副词从句}, we might as
  well call it a day.

  既然时间所剩无几,我们不妨就此结束好了。
\end{enumerate}
副词从句修饰动词 might call 的原因。

\begin{enumerate}[resume]
\item There's nothing to worry about, \unct{now that}{连接词} \unct{Father is
    back}{副词从句}.

  既然父亲回来了,就没什么好担心了。
\end{enumerate}
副词从句修饰动词 is 的原因。

请注意:简单句前面加上一个单独的、没有意义的that,会成为名词从句(指“那件
事”)。\textbf{可是 that一旦配合其他字眼当作连接词、具有表达逻辑关系的功能时,就成了副
  词从句的连接词,引导的是副词从句。now that 解释为“既然”,用来表达原因,所以它
  后面的 Father is back就成了副词从句。}

\begin{enumerate}[resume]
\item He looked \unct{so}{连} sincere \unct{that}{接词} \unct{no one doubted his
  story}{副词从句}.

他看起来是那么诚恳,所以没有人怀疑他说的话。
\end{enumerate}
副词从句修饰形容词 sincere 造成什么结果。

连接词 so \ldots{} that 表示因果关系,所以引导的是副词从句。

\begin{enumerate}[resume]
\item The mother locked the door from the outside, \unct{so that}{连接词} \unct{the
    kids couldn't}{副}\\
  \unct{get out when they saw fire}{词从句}.

  这位妈妈把门反锁,所以小孩看到火起时也跑不出去。
\end{enumerate}

副词从句修饰动词 locked 造成什么结果。

连接词 so that亦表示因果关系,所以引导的是副词从句。请注意这个副词从句中又有一个
表示时间的副词从句when they saw fire。

\subsection{目的}

\begin{enumerate}
\item The mother locked away the drugs \unct{so that}{连接词} \unct{the kids
    wouldn't swallow}{副词}\\
  \unct{any by mistake}{从句}.

  这位妈妈把药锁好,目的是不让小孩误食。
\end{enumerate}
副词从句修饰动词 locked 有什么目的。

同样是 so that 连接词,同样引导副词从句,但是这里用来表示目的。

\begin{enumerate}[resume]
\item I've typed out the main points in boldface, \unct{in order that}{连接词}
  \unct{you won't miss them}{副词从句}.

  我用黑体字把重点打出来,好让你们不会遗漏掉。
\end{enumerate}
副词从句修饰动词 type out有何目的。同样的,这里的连接词不是单独、无意义的 that,
而是表示目的的 in order that,所以引导的是副词从句。

\begin{enumerate}[resume]
\item I've underlined the key points, \unct{lest}{连接词} \unct{you miss them}{副词
    从句}.

  我已把重点画了线,以免你们把它们漏掉。
\end{enumerate}
副词从句修饰动词 have underlined 有何目的。

\begin{enumerate}[resume]
\item You'd better bring more money, \unct{in case}{连接词} \unct{you should need
    it}{副词从句}.

  你最好多带点钱,万一要用。
\end{enumerate}
副词从句修饰动词 bring 的目的。

\subsection{让步}

\begin{enumerate}
\item \unct{Although}{连接词} \unct{you may object}{副词从句}, I must give it a
  try.

  虽然你可能会反对,我仍然必须试试看。
\end{enumerate}
副词从句修饰动词 must give。

\begin{enumerate}[resume]
\item \unct{While}{连接词} \unct{the disease is not fatal}{副词从句}, it can be
  very dangerous.

  这虽然不是要命的病,不过也很危险。
\end{enumerate}
副词从句修饰动词 can be。

\begin{enumerate}[resume]
\item \emph{Wh-} 拼法的连接词,若解释为 No matter \ldots(不论),就表示让步语气,引导副
  词从句。

\begin{itemize}
\item \unct{Whether (=No matter)}{连接词} \unct{you agree or not}{副词从句}, I want
  to give it a try.

  无论你是否同意,我都想试一试。
\item \unct{Whoever (=No matter who) calls}{连接词 + 副词从句}, I won't answer.

  不管谁来电话,我都不接。
\item \unct{Whichever (=No matter which) way you go}{连接词 + 副词从句}, I'll
  follow.

  不论你走到哪里,我都跟定你了。
\item \unct{However (=No matter how) cold it is}{连接词 + 副词从句}, he's always
  wearing a shirt only.

  不管多冷,他总是只穿件村衫。
\item \unct{Wherever (=No matter where) he is}{连接词 + 副词从句}, I'll get
  him!

  不管他躲到哪儿,我都会抓到他!
\item \unct{Whenever (=No matter when) you like}{连接词 + 副词从句}, you can call me.

  你随时给我来电话都可以。
\end{itemize}
\end{enumerate}


\subsection{限制}

\begin{enumerate}
\item \unct{As far as}{连接词} \unct{money is concerned}{副词从句}, you needn't
  worry.

  钱的方面你不必担心。
\end{enumerate}
副词从句修饰动词 needn't worry,表示不必担心的事情是在某一方面,暗示也许是别的方
面才要担心。

\begin{enumerate}[resume]
\item Picasso was a revolutionary \unct{in that}{连接词} \unct{he broke all
    traditions}{副词从句}.

  毕加索是革命派,原因是他打破了一切传统。
\end{enumerate}
副词从句修饰动词was,把“是革命派”的意思加以限制:在于打破传统,并非真的举枪起义。
连接词 in that 是由 in the sense that(从某种意义来说)省略而来。

\subsection{方法,状态}

\begin{enumerate}
\item He played the piano \unct{as}{连接词} \unct{Horowitz would have}{副词从
    句}.

  他弹钢琴有如大师霍洛维兹。
\end{enumerate}
副词从句修饰动词 played——如何弹法。

\begin{enumerate}[resume]
\item 他写字像左撇子
  \begin{itemize}
  \item He writes \unbf{as if} \unbf{he is left handed} .
  \item He writes \unbf{as if} \unbf{he were left handed} .
  \item He writes \unbf{as if} \unbf{he was left handed} .
  \end{itemize}
\end{enumerate}

上面三句中,用 is 表示他应该真的是左撇子,用 were表示他不是,只是冒充左撇子,
用 was则表示不一定——可能是,也可能不是。三句话都是用连接词 as if引导后面的副词
从句,修饰动词 writes——“如何写法”。

\section{结语}

副词从句是最简单的从属从句,由完整的简单句外加有意义的、表达逻辑关系的连接词
(because、if、although等等)所构成。它附在一个完整的主要从句上作修饰语(修饰动词
最为常见),两个从句(主、从)之间的关系类似对等从句。以上整理的副词从句连接词并
不完整,只挑出有代表性和值得注意的副词从句连接词来作分类。读者可利用以上的分类去
观察,加上广泛的阅读,以扩充自己认识的、会用的副词从句连接词。

介绍完名词从句与副词从句后,相信读者已充分掌握了这两种从句的特色。如果再征服比较
困难的定语从句(又称关系从句),就可以全盘了解复句结构了。

\section{Test}

\paragraph{请选出最适当的答案填入空格内,以使句子完整。}

\begin{enumerate}
\item Please come back \ttu you finish your work.
\begin{tasks}(2)
  \task as soon as
  \task as soon as possible
  \task as possibly soon
  \task as soon possible
\end{tasks}

\item Which of the following is correct?
\begin{tasks}
  \task He is very smart; moreover, he is diligent.
  \task He is very smart, moreover, he is diligent.
  \task He is very smart, Moreover, he is diligent.
  \task He is very smart; and moreover, he is diligent.
\end{tasks}

\item It is not safe to get off a car \ttu.
\begin{tasks}
  \task unless it is in motion
  \task until it has come to a stop
  \task after you have opened the window
  \task before the traffic light turns red
\end{tasks}

\item If you sell your rice now, you will be playing your hand very badly. Wait \ttu the price goes up.
\begin{tasks}(2)
  \task until
  \task still
  \task for
  \task that
\end{tasks}

\item (The rain is over. You must not stay any longer.) You must not stay any longer \ttu the rain is over.
\begin{tasks}(2)
  \task when
  \task that
  \task now that
  \task as for
\end{tasks}

\item It is such a good opportunity \ttu you should not miss it.
\begin{tasks}(2)
  \task as
  \task that
  \task which
  \task of which
\end{tasks}

\item Tom is dull. He works hard. He will surely pass the exam.
\begin{tasks}
  \task Though Tom is dull, he works so hard that he will surely pass the exam.
  \task Despite his dullness, Tom will surely pass the exam by work hard.
  \task Tom will surely pass the exam because he works hard with his dullness.
  \task Dull as Tom is, he will surely pass the exam with work hard.
\end{tasks}

\item She had worked several years \ttu she could continue her studies in France.
\begin{tasks}(2)
  \task as
  \task while
  \task before
  \task then
\end{tasks}

\item \ttu, he never begged for money.
\begin{tasks}(2)
  \task Despite he was poor
  \task Because he was poor
  \task Poor as he was
  \task In spite of he was poor
\end{tasks}

\item \ttu the typhoon warnings, several fishing boats set sail.
\begin{tasks}(2)
  \task Because
  \task According
  \task Despite
  \task Although
\end{tasks}

\item I knew I would never have what I needed \ttu it myself.
\begin{tasks}(2)
  \task even made
  \task without me making
  \task except making
  \task unless I made
\end{tasks}

\item Which of the following is correct?
\begin{tasks}
  \task I shall either go back to Taiwan or my family will come to England.
  \task I shall go back either to Taiwan or my family will come to England.
  \task Either I shall go back to Taiwan or my family will come to England.
\end{tasks}

\item \ttu unwilling to do so, he had to follow the traditional ways.
\begin{tasks}(2)
  \task After
  \task Although
  \task Since
  \task Once
\end{tasks}

\item Which of the following is correct?
\begin{tasks}
  \task Not only the money but also three paintings was stolen.
  \task Not only the money but also three paintings were stolen.
  \task Not only the money was stolen but also were the paintings.
\end{tasks}

\item No one was sure \ttu was going to happen.
\begin{tasks}(2)
  \task what
  \task who
  \task when
  \task where
\end{tasks}

\item \ttu she studied hard, but she didn't succeed.
\begin{tasks}(2)
  \task Though
  \task Although
  \task Indeed
  \task While
\end{tasks}

\item "You seem angry at Martha." "I am. \ttu I'm concerned, she can go away forever."
\begin{tasks}(2)
  \task As like as
  \task As many as
  \task As such as
  \task As far as
\end{tasks}

\item I'm going to tell you the number once more, \ttu you forget.
\begin{tasks}(2)
  \task don't
  \task that
  \task so that
  \task lest
\end{tasks}

\item The mother's warning \ttu there be no contact with boys was generally ignored.
\begin{tasks}(2)
  \task which
  \task that
  \task if
  \task wherever
\end{tasks}

\item Don't go away \ttu you have told me what actually happened.
\begin{tasks}(2)
  \task since
  \task then
  \task after
  \task until
\end{tasks}

\end{enumerate}

\section{Answer}

\begin{enumerate}

\item (A) 空格前后分别是完整的独立从句,中间只需要连接词,如 A,把后面的从句改为副词
  从句。B 的 as soon as possible 已经是一个从句(as soon as it is possible 的简
  化),不再是连接词。C 和 D 都不是完整的连接词。

\item (A) \textbf{moreover 是副词,不具连接词的语法功能,所以要用分号(;)来取代连接词。}

\item (B) 四个答案在语法上都对,句意则只有 B 合理:“除非车子停稳了,否则下车不安全。”

\item (A) wait 一词构成一个命令句,与右边的 the price goes up 之间要有连接词,故可排除非连接词的 B。答案 D 的 that 会把从句变成名词从句,不合词类要求。C 的 for 可以当连接词,不过要解释为 because,在此不通,只有 A 这个连接词是引导时间副词从句用的,符合要求。

\item (C) now that 解释为“既然”,符合原意。

\item (B) 上文有 such,因而要有 that 来配合,表示因果关系。

\item (A) 句一和句二有相反关系,句二和句三有因果关系,因而分别用 though 和 so…that 来连接。B 中的 by work hard 错在以动词 work 直接放在介词后面。C 中的 he works hard with his dullness 句意十分牵强。D 与 B 相同,也是错在把动词(work)直接放在介词(with)后面。

\item (C) had worked 是过去完成式,could continue 是过去简单式,这是时间先后顺序,因
  而用 before。

\item  (C) Though he was poor 可改写为 Poor as he was,注意连接词现在要用 as。A 和 D 都是用介词(despite 和 in spite of)来引导从句,属语法错误。B 的句型正确,但逻辑关系不通顺。

\item (C) 名词短语 the typhoon warnings 前面应有介词(只有 C 是)。

\item  (D) 空格中要表示“条件”,因而 C 不适合。A 多一个动词,文法错误。B 应该省略掉与主语重复的 me。D 是以 unless 的副词从句表示条件,符合要求。

\item  (C) either 和 or 之间的部分要和 or 之后的部分对称。符合条件的只有 C(从句对从句),其余答案在词类上都不对称。

\item (B) unwilling 和 had to 意思上相反,只有 although 可表示相反的关系。答
  案 B 是 although he was unwilling… 的简化。

\item (B) not only 和 but also 亦要求对称。A 虽然有对称,但是动词 was 和主语 three
  paintings 在单复数上有冲突,而 C 中应倒装的是 not only was the money stolen,不
  是后面的从句。


\item (A) 这个位置要用连接词,又要能当 was 的主语,所以要用关系代名词类(A 或 B)。
  因为它前面没有先行词,不能用 who,只能用 what,故选 A。what was going to
  happen 亦可作疑问句类的名词从句看待。

\item  (C) 两个从句间已有连接词 but,不能再用连接词(A、B 和 D 都是),只剩下一个副词类的 C。

\item (D) 这个位置要用连接词。D 是表示限度的从属连接词,符合要求。B 的 as many as 则要配合复数名词才能使用。
\item (D) 这个位置要用连接词,故排除 A。B 会造成名词从句,不合句型要求。C 和 D 都是副词从句连接词,但只有 D 的 lest(以免……)符合逻辑关系。
\item (B) 从下文的 there be no contact… 来看,是间接命令句语气,应为名词从句,故选择 B。

\item (D) 这个位置连接两个从句,要用连接词(A、C 或 D),从意思判断用 D 较合理。
\end{enumerate}
\chapter{关系从句}

从属从句有三种,除了名词从句和副词从句之外,还有就是形容词从句,又称关系从句。如
同名词从句和副词从句一样,关系从句也有它明显的特色。

\section{关系从句的特色}

关系从句如果没有经过任何省略,都应该以形容词从句看待。它的特色有以下几点:

\subsection{两个句子要有交叉}

也就是:两个句子间要有一个\textbf{重复的元素},由此建立“关系”,才可以用关系从句的方式
来合成复句。例如:

\begin{itemize}
\item For boyfriend I'm looking for \unbf{a man}.

  找男朋友,我想找个男人。
\item \unbf{He} is tall, rich, and well-educated.

  他身材高、收入高、教育水准高。
\end{itemize}
这两个句子中的 a man 和 he 是重复的:a man 就是代名词 he所代表的对象(即先行词)。
因为有这个交叉存在,两个句子有关系,才可以进行下一步的动作——制造关系从句。

\subsection{把交叉点改写为关系词的拼法 (\emph{wh-}),让它产生连接词的功能}

在上例中就是把 he 改写为 who,成为:
\begin{itemize}
\item \unct{who}{S} \unct{is}{V} tall, rich, and well-educated
\end{itemize}
这就是一个关系从句。who 仍然具有 he的功能,也就是作为这个从句的主语,但是它同时也
有连接词的功能。

\subsection{将关系从句附于主要子句的交叉点(名词),后面来修饰它(做形容词使用)}

上例中就是把 who 从句附在 a man 之后成为:
\begin{itemize}
\item For boyfriend I'm looking for \unct{a man}{先行词} \unct{who is tall,
    rich, and well-educated}{关系从句\ 形容词类}.

  找男朋友,我想找一个身材高、收入高、教育水准高的人。
\end{itemize}

由以上的分析可以看出,关系从句有一个很重要的特色:关系从句的连接词是从句中内含字
眼的改写,而名词从句与副词从句的连接词都是外加的。请比较下列三句:

\begin{enumerate}
\item I know \unct{that I am right}{O连接词 + 名词从句}.

  我知道我对。
\item I know \unct{this}{O} \unct{because}{连接词} \unct{I have proof}{副词从句}.

  我知道,因为我有证据。
\item I don't trust \unct{people}{O} \unct{who talk too much}{关系从句}.

  我不信任话太多的人。
\end{enumerate}

例 1 中的名词从句是由完整的简单句 I am right 外加连接词 that 所构成。例2 的副词从
句也是完整的单句 I have proof 外加连接词 because 构成。只有例3 的关系从句没有外加
连接词,而是直接由 They talk too much 的单句,把they 改写成 who 而构成。产生的关
系从句 who talk too much属于形容词的功能,用来修饰先行词 people。

\section{关系代名词与关系副词}

关系从句中与主要从句的交叉点,可能是代名词,也可能是副词。改变为 \emph{wh-}的拼法后,
分别称为关系代名词与关系副词。请观察下列拼法上的变化:

\begin{longtable}[]{@{}ll@{}}
  \toprule
  代名词 & 关系代名词 \\
  \midrule
  he (she, they) & who \\
  it (they) & which \\
  his (her, their, its) & whose \\
  him (her, them) & whom \\
  \bottomrule
\end{longtable}

\begin{longtable}[]{@{}ll@{}}
  \toprule
  副词 & 关系副词 \\
  \midrule
  then & when \\
  there & where \\
  so & how \\
  for a reason & why \\
  \bottomrule
\end{longtable}

关系词除了以上这些,还有一些变化的拼法,后面再谈。

\section{关系代名词的省略}

语法书列出规则:\textbf{关系代名词作宾语使用时可以省略。}这条规则没错,就是不太好背!本书
号称从头到尾没有一条规则要背,包括关系代名词的省略,其实都是可以理解的。

关系词所以常会省略,主要是因为它在句子中是重复的元素:和主要从句中的先行词重复。
可是它除了代名词的功能之外,还有连接词的功能,用来标示另外一个从句的开始。假如两
个从句的断句很清楚,把关系词省掉也不会影响句子的清楚性,就可以省略。例如:

\begin{enumerate}
\item \unct{The man}{S} \unct{is}{V} \unct{my uncle}{C}.

  那个人是我叔叔。
\item \unct{You}{S} \unct{saw}{V} \unct{him}{O} just now.

  你刚刚看到他。
\end{enumerate}

例 2 中的 him 是宾语,与例 1 中的 man 重复,可以改成关系代名词whom,变成:
\begin{itemize}
\item \unct{you}{S} \unct{saw}{V} \unct{whom}{O} just now
\end{itemize}

请观察一下:\textbf{关系代名词是宾语的话,位置应该在动词后面。可是它要标示关系从句的开
  始,所以要调到句首(这个调动和它的省略大有关系)},成为:
\begin{itemize}
\item \unct{whom}{O} \unct{you}{S} \unct{saw}{V} just now
\end{itemize}

再把关系从句和主要从句合起来成为复句:
\begin{enumerate}[resume]
\item \unct{The man}{S1} [\unct{whom}{O} \unct{you}{S2} \unct{saw}{V2} just now] \unct{is}{V1} \unct{my uncle}{C}.

  你刚看到的那个人是我叔叔。
\end{enumerate}

如果把 whom 省略掉,读者仍然看得出来 you saw just now是另一个从句,不会和主要从
句 The man\ldots{} is my uncle混淆。这就是为什么可以省掉它的原因。

反之,\textbf{如果关系代名词是主语,就不适合省略},否则会造成断句上的困难。例如:

\begin{enumerate}
\item \unct{The man}{S} \unct{is}{V} \unct{my uncle}{C}.
\item \unct{He}{S} \unct{was}{V} \unct{here}{C} just now.
\end{enumerate}

例 2 中的 he 是主语,改成关系代名词 who 之后位置不会调动,直接成为:
\begin{itemize}
\item \unct{who}{S} \unct{was}{V} \unct{here}{C} just now
\end{itemize}

再和句 1 合并成为复句:
\begin{enumerate}[resume]
\item \unct{The man}{S1} [\unct{who}{S2} \unct{was}{V2} \unct{here}{C2} just
  now] \unct{is}{V1} \unct{my uncle}{C1}.

  刚才在这儿的那个人是我叔叔。
\end{enumerate}

这时候如果要省掉 who(主语),会造成断句上的困难:
\begin{itemize}
\item The man was here just now is my uncle. (误)
\end{itemize}

这个句子语法有错误,因为读者无从判断它的句型。看到 The man was here just now
为止都还好:\textbf{读者的印象是一个简单句}。可是后面再加上 is my uncle的部分,就不
知所云了。

经由以上的比较当可发现:关系代名词当宾语时,因为要往前移,即使省略掉它,后面
还是有S+V的构造,可以和主要从句区分清楚,因而可以省略。请看看下面这个不同的例
子:
\begin{itemize}
\item \unct{He}{S} \unct{is}{V1} not \unct{the man}{C} [\unct{he}{S2}
  \unct{used to be}{V2}].

  他现在和从前不一样了。
\end{itemize}

中括号里原来是:
\begin{itemize}
\item \unct{He}{S} \unct{used to be}{V} \unct{the man}{C}.
\end{itemize}
其中的 the man 是补语,和另一句中的 the man 重复,改写成关系词who(其实
用 that 比较恰当,这点稍后再谈),成为关系从句:
\begin{itemize}
\item \unct{who}{C} \unct{he}{S} \unct{used to be}{V}
\end{itemize}

who 是补语,不是宾语,所以不能拼成宾格的whom。但是它一样可以省略,原因是它也
要向前挪,所以和宾语一样,省掉并不会造成断句上的困难。因此,光是背规则:“关
系代名词作宾语时可以省略”,一方面不好背,一方面也不够周延,还是经过理解比较
能够变通。

\section{何时该使用that?}

关系代名词 who 和 which 有时可用 that来取代。这中间的选择有差别,需要说明一
下。that是借自指示代名词,具有指示的功能。所以,\textbf{关系从句如果有指示的作用时才
适合借用that 作关系词}。例如:

\begin{itemize}
\item \unct{Man}{S1} \unct{is}{V1} an animal \unct{that}{S2} \unct{is}{V2}
  capable of reason.

  人类是有理性能力的动物。
\end{itemize}

主要从句中的先行词 an animal本来可以代表任何一种动物,范围极大。后面加上一个
条件:The animal is capable of reason(有理性能力的那种),明确指出是哪种动物
才能算人,具有指示的功能。所以关系词可以选择不用which 而借用that。事实上,上
面这个句型常被用在各种下定义的句子中,这时候因为指示的功能明确,多半都是
用that,例如:

\begin{itemize}
\item \unct{Meteorology}{S1} \unct{is}{V1} a science \unct{that}{S2}
  \unct{deals}{V2} with the behavior of the atmosphere.

  气象学是处理大气变化的枓学。
\end{itemize}

上面两个例子如果用 which也不算错误。另外有些情况,因为指示的要求很强烈,一般
都选择用that,如果用 who 或 which 反而不恰当。例如:
\begin{enumerate}
\item \unct{Money}{S1} \unct{is}{V1} [\unct{the only thing}{先行词}]
  \unct{that}{S2} \unct{interests}{V2} him.

  钱是唯一能让他感兴趣的东西。
\item \unct{He}{S1}\unct{'s}{V1} [\unct{the best man}{先行词}] that
  \unct{I}{S2} \unct{can}{V2} recommend.

  他是我能推荐的最佳人选。
\item Spaceman \unct{Armstrong}{S1} \unct{was}{V1} [\unct{the first man}{先行词}]
  \unct{that}{S2} \unct{set}{V2} foot on the moon.

  太空人阿姆斯特朗是第一个踏上月球的人。
\end{enumerate}

这几个例子中的先行词,都需要关系从句配合做相当明确的指示,所以关系词都要选择
that,用 who 或 which 并不适合。

\section{何时不该使用 that?}

同样的道理,如果关系从句缺乏指示的功能,就不该借用指示代名词 that当关系词。

\textbf{如果关系从句没有指示的作用,只是补充说明的性质;应该用逗点和先行词隔开。这
  时逗点的功能和括弧类似。}例如:

\begin{itemize}
\item For boyfriend, \unct{I}{S1}\unct{'m considering}{V1} [\unct{your brother
    John}{先行词}], \unct{who}{S2} \unct{is}{V2} tall, rich, etc.

  找男朋友,我在考虑你哥哥约翰,他个子高、收入高等等。
\end{itemize}

这个句子中的先行词 your brother John是个专有名词,只代表一个人,听的人听到这
个名字也已经知道是谁了,所以后面的关系从句不再具有指出是谁或是哪种人的功能。
因而这个关系从句只有补充说明的功能,说明“为什么”在考虑John这个对象。这种补
充说明的句子应该放在括弧性的逗点后面——括弧就是用来作补充说明的。同时也
就\textbf{不能再借用指示代名词that 了,因为 that是用来指出“哪个”或“哪种”的。语
  法书上把加上逗号的关系从句称为“非限定用法”},并且有一条规则说非限定用法的
关系从句不能用that 做关系词,道理就在这里。再看下面的例子:
\begin{itemize}
\item \unct{I}{S1} \unct{like}{V1} [\unct{books}{先行词}], whatever the
  subject, \unct{that}{S2} \unct{have}{V2} illustrations.

  我喜欢的书,不论什么主题,都是附有插图的。
\end{itemize}

关系从句 that have illustrations虽然放在逗号后面,可是这个逗号是和前一个逗号
构成一组括弧,把 whatever the subject括在里面,关系从句本身并不是放在括弧中作
补充说明,它仍然是一个具有指示功能的从句,指出喜欢的书是哪一种,所以还是可以
借用指示代名词that。

另外,关系从句如果是放在括弧性的逗号中作补充说明,就不再适合省略宾语位置的关
系词了。请看下例:
\begin{itemize}
\item \unct{I}{S1} \unct{like}{V1} [\unct{\textit{Time Classic Words}}{先行词}],
  \unct{which}{O} many \unct{people}{S2} \unct{like}{V2}, too.

  我喜欢《时代经典用字》,很多人都喜欢这套书。
\end{itemize}

先行词是个书名,听的人已经知道是哪套书,所以后面的关系从句属于补充说明的性质,
应该放在括弧性的逗号后,关系词which 没有指示功能,不能借用 that来取代。而且,
一旦打了逗号,和主要从句隔开,关系词 which虽然是宾语也不适合省略了。这是因为
两个从句已断开来,不能再共用先行词这个重复点,所以关系从句要有自己的which 作
宾语。

\section{先行词的省略}

关系代名词与先行词重复,有时候可以省略掉关系词。同样的,有时候可以选择把先行
词省略掉。\textbf{如果要省略掉先行词,首先这个先行词得是空的、没有内容的字
  眼,}像thing、people等等空泛的字眼。\textbf{其次,关系从句与先行词之间不能有逗号
  断开。}而且,因为先行词是名词类,属于重要元素,不是可有可无的修饰语,所以一
旦省略掉先行词,在关系词的部分要有所表示。表示的方式可分成以下几种情形。

\subsection{what}

把关系代名词改写为 what 来表示前面省掉的先行词。例如:

\begin{enumerate}
\item \unct{I}{S} \unct{have}{V} \unct{the thing}{O}.
\item \unct{You}{S} \unct{need}{V} \unct{it}{O}.
\end{enumerate}
这两个句子中的 the thing 和 it重复,建立了两句间的关系,可以用关系从句的方式
合成复句:

\begin{enumerate}[resume]
\item \unct{I}{S1} \unct{have}{V1} [\unct{the thing}{O1先行词}]
  [\unct{that/which}{O2 关系词}] \unct{you}{V2} \unct{need}{O2} .
\end{enumerate}

接下来可以有两种变化。首先,关系代名词(that 或 which)在关系从句中是
need 的宾语,可以省略掉,成为:

\begin{enumerate}[resume]
\item \unct{I}{S1} \unct{have}{V1} the thing \unct{you}{S2} \unct{need}{V2}.
\end{enumerate}

另外,先行词 the thing 是空的字眼,也可以选择省略它。可是句子中的 the thing
省掉以后,主要从句 I have 就缺了宾语,关系从句(that you need)也失去了它修饰
的名词,所以要修改成为:

\begin{enumerate}[resume]
\item \unct{I}{S1} \unct{have}{V1} what \unct{you}{S2} \unct{need}{V2}.

  我有你需要的东西。
\end{enumerate}

把关系词由 that 改成what,表示前面有一个省略掉的先行词。在句型分析时也可以
说 what you need是名词从句,作为 have 的宾语。

\subsection{whoever}

如果把关系词 who 变成\textbf{whoever},表示不管先行词是谁,那么就可以省略掉先行词了。
例如:
\begin{enumerate}
\item \unct{I}{S}\unct{'ll shoot}{V} \unct{any person}{O}.
\item \unct{He}{S} \unct{moves}{V}.
\end{enumerate}
这两句中的 any person 与 he 重复,可以用关系从句合成:
\begin{enumerate}[resume]
\item \unct{I}{S1}\unct{'ll shoot}{V1} \unct{any person}{O先行词}
  \unct{that}{S2关系词} \unct{moves}{V2}.
\end{enumerate}

如果要省略掉先行词 any person,那么关系词首先要用who(代表是人),然后再改
成 whoever(以取代 any)。whoever是“不论是谁”,所以前面的先行词就不必交待,
可以省略成为:
\begin{enumerate}[resume]
\item \unct{I}{S1}\unct{'ll shoot}{V1} \unct{whoever}{S2} \unct{moves}{V2}.

  谁动我就开枪打谁。
\end{enumerate}
whoever 表示前面省掉一个先行词。句型分析上也可以直接把 whoever moves看成名词
从句,作为 \textbf{shoot 的宾语}。

\subsection{whichever}

如果是“任意选哪一个”的意思,可以用 whichever来代表先行词的省略。例如:

\begin{enumerate}
\item \unct{You}{S} \unct{can take}{V} \unct{any car}{O}.
\item \unct{You}{S} \unct{like}{V} \unct{it}{O}.
\end{enumerate}
这两句中的 any car 和 it 重复,可以用关系从句合成:
\begin{enumerate}[resume]
\item \unct{You}{S1} \unct{can take}{V1} \unct{any car}{O1} \unct{that}{O2 关系
    词} \unct{you}{S2} \unct{like}{V2}.
\end{enumerate}
因为关系词 that 在关系从句(that you like)中是宾语,可以选择省略掉,成为:
\begin{enumerate}[resume]
\item \unct{You}{S1} \unct{can take}{V1} \unct{any car}{O先行词} \unct{you}{S2}
  \unct{like}{V2}.
\end{enumerate}
但是也可以选择省略先行词 any car。这时关系词 that 要改成 whichever
来表示“不论哪一个”:
\begin{enumerate}[resume]
\item \unct{You}{S1} \unct{can take}{V1} \unct{whichever}{O} \unct{you}{S2}
  \unct{like}{V2}.

  你爱哪个就拿哪个。
\end{enumerate}
或者也可以说:
\begin{enumerate}[resume]
\item \unct{You}{S1} \unct{can take}{V1} \unct{whichever car}{O} \unct{you}{S2}
  \unct{like}{V2}.

你爱哪个就拿哪个。(你爱哪辆车就拿哪辆车。)
\end{enumerate}

whichever 表示省略掉先行词。分析句型的时候也可以把 whichever (car) you like
直接视为名词从句,作为 take 的宾语。

\section{关系从句的位置}

一般语法书都列出一条规则:关系从句要放在先行词的后面。这是因为关系从句是形容
词类,是修饰语的性质,它修饰的对象就是先行词位置的名词。一般说来,修饰语与其
修饰的对象应该尽量接近,以增强明确性。可是列成规则来背,就会有违反规则的情况
出现,亦即所谓的例外。例外丛生的规则,不仅没有指导的功能,甚至会妨碍判断,所
以判断关系从句的位置,应该回到原点:放在哪个位置最清楚?以此作为判断的准则,
远胜于死背规则。例如:

\begin{enumerate}
\item There are two apples in \unbf{the basket}.
\item \unbf{The basket} is lying on the table.
\end{enumerate}
这两句中有 the basket的重复关系,可以改成以关系词来连接。形成的关系从句就直接
放在先行词的后面,成为:
\begin{enumerate}[resume]
\item There are two apples in \unct{the basket}{先行词} \unct{which is lying on
    the table}{关系从句}.

  桌上的篮子里有两个苹果。
\end{enumerate}
这个位置够清楚,因为关系从句中的动词 is 是单数形式,表示主语 which是单数,所
以 which 的先行词只能是单数的 basket,不会是 two apples。可是如果是下面这两个
句子的连接就不同了:
\begin{enumerate}
\item You can find \unbf{two apples} in the basket.
\item I bought \unbf{the apples}.
\end{enumerate}

两句话在 apples上建立关系,用关系词作成关系从句后,如果套用语法规则,把关系从
句和先行词放在一起,就会成为:
\begin{itemize}
\item You can find \unct{\textnormal{two apples}}{先行词}
  \unct{\textnormal{which I bought}}{关系从句} in the basket. (误)
\end{itemize}
这个句子就没有把意思交待清楚。关系从句 Which I bought 插到先行词 two apples
后面,造成一个结果:地方副词 in the basket 和它所修饰的动词 can find 之问距离
过远,而且现在它更接近另一个动词bought(关系从句的动词),所以这个句子读起来
不像是“可以在篮子里找到我买的两个苹果”,反而像是“你可以找到我在篮子里买的
两个苹果”。这就失去原来的意思了。

如果把 in the basket 向前移,把关系从句挪后,就成为:
\begin{itemize}
\item You can find \unct{\textnormal{two apples}}{先行词} in the basket
  \unct{\textnormal{which I bought}}{关系从句}. (误)
\end{itemize}
这个句子还是有问题。因为关系词 which不是主语,看不出应该是单数还是复数,所以
它的先行词可能是apples,也可能是更接近的basket。也就是说,整个句子可以解释
为“我买的两个苹果在篮子里”,也可以解释为“两个苹果在我买的篮子里”。一句话
有两种不同的解释,就是没有把话说清楚。

那么,到底该怎么说才算清楚?要避免混淆,最好的办法是把 in the basket
这个短语移开,成为:
\begin{itemize}
\item In the basket you can find \unct{two apples}{先行词} \unct{which I
    bought}{关系从句}.

  你会在篮子里找到我买的两个苹果。
\end{itemize}
这时候 in the basket 只能修饰 can find,而关系从句也只有单一的先行词,意思才
清楚。再看下面的例子:
\begin{enumerate}
\item \unbf{A plague} broke out.

  一场瘟疫爆发。
\item \unbf{It} lasted 20 years.

  它延续了 20 年。
\end{enumerate}

如果照规则处理,把例 2 改成关系从句,置于例 1 的先行词之后,就成为:
\begin{itemize}
\item \unct{A plague}{\textnormal{先行词}} \unct{which lasted 20
    years}{\textnormal{关系从句}} broke out. (误)
\end{itemize}
这个句子虽不能说错,可是颠三倒四,不合逻辑。“一场延续了 20年的瘟疫爆发了”。
这就是套用语法规则而不知思考的结果。合理的说法是先说爆发,然后再说延续多久,
也就是:

\begin{enumerate}[resume]
\item \unct{A plague}{先行词} broke out \unct{which lasted 20 years}{关系从句}.

  一场瘟疫爆发,延续了 20 年。
\end{enumerate}
这个关系从句与先行词虽然有距离,然而距离不远,而且中间只有动词相隔,没有别的
名词来妨碍判断先行词的问题,所以应让它有距离,以换取表达顺序的合理性。

总之,关系从句与其他的修饰语相同,应该尽量靠近修饰的对象,这是为了表达清楚起
见。假如关系从句直接放在先行词后面会引起误解,就要把它移开或者进一步更动句型,
不能一味硬套规则。

\section{关系副词}

如果关系从句中是以副词和主要从句中的先行词重复,就会改写为关系副词。关系副词
因为是副词类,不像关系代名词是重要的名词类,所以关系副词可以比较自由省略。但
是它与关系代名词一样,如果有括弧性的逗号隔开,就不能省略了。详见下述。

\subsection{when}

关系副词 when 就是时间副词 then 的改写,有连接词的功能。请看下例:

\begin{enumerate}
\item The rain came at \unbf{a time}.
\item The farmers needed it most \unbf{then}.
\end{enumerate}
这两个句子以 a time 和 then 的重复建立关系(then 就是 at that time)。把时间
副词 then 改写为关系副词 when,借以连接两句,即成为:

\begin{enumerate}[resume]
\item The rain came (at \unct{a time}{先行词}) \unct{(when) the farmers needed
    it most}{关系从句}.
\end{enumerate}
如果认定 when 的先行词是 a time(名词),那么关系从句形容这个名词,依旧是形容
词类。这样的诠释比较统一,也比较单纯。也就是:在省略之前,关系从句全部都是形
容词从句,所有的形容词从句也都是关系从句,两者间可划等号。

谈到省略,观察例 3 当可发现:

一、at a time 和 when 都是空洞、无内容的字眼(不像 in 1964, last January之类
有明确内容的时间);

二、at a time 和 when 重复;

三、at a time 和 when 都是可有可无的副词类。

基于这三点观察,at a time 和 when应该择一省略来避免重复,让句子紧凑些。也就是
例 3可以省略而变成以下两种状况:

\begin{enumerate}[resume]
\item The rain came \unct{when the farmers needed it most}{关系从句}.
\end{enumerate}

关系从句
\begin{enumerate}[resume]
\item The rain came at a time \unct{the farmers needed it most}{关系从句}.

  这场雨下得正是时候,农夫们这时最需要它。
\end{enumerate}

一般语法书说例 4 中的关系从句(when the farmers needed it most)是副词从句,
这是把 when视为外加的连接词看待。这也可以讲得通,但是分析得不够透彻。因
为 when不是外加的连接词,而是内含的关系副词,只不过 at a time被省略掉了,所以
看不到先行词。反之,如果选择保留 at a time而省略关系副词 when,就成为例 5 的
结果。例 4 和例 5是同一个句子的不同省略方式,应该同样解释为关系从句比较合理。

下面这组句子又有不同的变化:
\begin{enumerate}
\item I need \unbf{some time}.
\item I can be with my daughter \unbf{then}.
\end{enumerate}
这两句由名词的 some time 和副词的 then(代表 at that time)产生交叉而建立关系,
可改写为关系词 when 来连接:
\begin{enumerate}[resume]
\item I need \unct{some time}{先行词} \unct{(when) I can be with my daughter}{关
    系从句}.
\end{enumerate}
本句如果要省略,仍可省去重复的关系副词 when,成为:
\begin{enumerate}[resume]
\item \unct{I}{S} \unct{need}{V} \unct{some time}{O} \unct{I can be with my
    daughter}{关系从句}.
\end{enumerate}
(我需要点时间陪陪女儿。)

但是如果省去同样重复的先行词 some time 就不行了:
\begin{itemize}
\item \unct{I}{\textnormal{S}} \unct{need}{\textnormal{V}} \unct{I can be with
    my daughter}{\textnormal{关系从句}}. (误)
\end{itemize}
这是因为 some time虽然没有内容,而且重复,可是它属于名词类,不是可有可无的副
词类,不能随便省略成上面这句;省略之后,及物动词need 没有宾语,就是错误的句
子。

如果说名词类的先行词不能随便省略,读者对下面这组句子的变化可能会有疑惑:
\begin{enumerate}
\item \unct{I}{S} \unct{know}{S} \unct{the time}{O}.
\item He will arrive \unbf{then}.
\end{enumerate}
这两句也是由 the time 和 then 的重复建立关系,可以改写为关系从句,成为:
\begin{enumerate}[resume]
\item I know \unct{the time}{先行词} \unct{(when) he will arrive}{关系从句}.
\end{enumerate}
这个句子当然也可以省掉关系副词 when,成为:
\begin{enumerate}[resume]
\item I know \unct{the time}{先行词} \unct{he will arrive}{关系从句}.
\end{enumerate}
但是另一个重复元素,先行词 the time,属于名词类,应该是不能省略的,可是好像省
掉也没错。请看下例:
\begin{enumerate}[resume]
\item\label{timewhen} I know when he will arrive.

  我知道他什么时候会到。
\end{enumerate}

\ref{timewhen} 怎么看都是正确的句子,那么是不是表示名词的先行词也可以省略?如果可以的话,
再前面的例子为什么又不能省略(I need some time (when) I can be with my
daughter)?

笔者认为:名词的先行词不能省略!至于例5,并不是省略名词的结果,甚至它根本不是
关系从句,而是名词从句。这一点需要说明。

在讲解名词从句的那一章中曾提到,名词从句有两种:

一、一个直述句外加连接词 that 所构成,表示 that thing(那件事)。例如:
\begin{itemize}
\item \unct{He}{S} \unct{said}{V} \unct{(that) he would call}{O名词从句}.

  他说过要打电话来。
\end{itemize}

二、由疑问词引导的疑问句改造而成,表示 a question(一个问题)。例如:
\begin{itemize}
\item \unct{He}{S} \unct{asked}{V} \unct{how much it was}{O名词从句}.

  他问它多少钱。
\end{itemize}
这个例子中的名词从句就是由疑问句 How much is it 改造而成。

在此稍为补充一下,还有一种名词从句,是由 Yes/No questions,也就是由不具疑问词
的疑问句改造而成。例如:
\begin{itemize}
\item Will the stock go up?

  这支股票会不会涨?
\end{itemize}
这个疑问句没有疑问词,要如何改造成名词从句?只加 that
是不成的。首先要把它改写为:
\begin{itemize}
\item Either the stock will go up or it will not.
\end{itemize}
然后就可以制造名词从句了。请看下例:
\begin{itemize}
\item \unct{No one}{S} \unct{knows}{V} \unct{whether the stock will go up (or
    not)}{O 名词从句}.

  没有人知道这支股票会不会涨。
\end{itemize}
这个句子原本是 No one knows the question,而 the question 就是 Will the
stock go up 这个 Yes/No question。先改成 either \ldots{} or的构造,再与表示“何
者”的 which 合并,就成了 whether。这就是 Yes/No questions 改成名词从句的做
法。

现在再回来看看刚才那个有疑问的句子。如果说:
\begin{itemize}
\item I know \unct{the time}{O} \unct{he will arrive}{关系从句}.
\end{itemize}
那么 he will arrive 是关系从句,前面省掉关系副词 when,用来形容先行词
the time。因为 the time 是名词类,它不可以省略!读者看到这一句:
\begin{itemize}
\item   I know when he will arrive.
\end{itemize}
它并不是关系从句省掉 the time,因为名词的先行词不能就这样省略掉。而是下面的变
化:
\begin{enumerate}
\item \unct{I}{S} \unct{know}{V} \unct{the question}{O}.
\item \unct{When}{疑问词} will he arrive?
\end{enumerate}

由此变成:
\begin{enumerate}[resume]
\item I know when he will arrive.
\end{enumerate}
这是由疑问句改造而成的名词从句,与关系从句无关,并不是 the time
的省略。when 是疑问词,不是关系词。

\subsection{where}

关系副词 where 就是地方副词 there 的改写,它的变化与 when大同小异,故不赘述,
只看看几个例句:

\begin{enumerate}
\item The car stopped at \unbf{a place}.
\item Three roads met \unbf{there}.

\reitem\label{3road} (A) The car stopped (at \unct{a
  place}{先行词}) \unct{(where) three roads met}{关系从句}.

车子在三岔路口停了下来。
\end{enumerate}
(A) 中的 at a place 和 where都是副词类,应该择一省略以更为简洁。可是下例则不
同:
\begin{itemize}
\item \unct{The Johnsons}{S} \unct{have}{V} \unct{a place}{先行词}
  \unct{(where) they can get away from other people.}{关系从句}

  约翰逊一家有个地方可以躲开其他人。
\end{itemize}
这个句子中的 where 是副词类,省掉无妨,可是省掉名词类的 a place就会产生错误,
读者可自行观察。至于下面这个句子就要视为疑问句的名词从句,而非关系从句的省
略:

\begin{itemize}
\item Please \unct{tell}{V} \unct{me}{O} \unct{where you were last night}{名词从
    句}.

  请告诉我昨晚你在何处。
\end{itemize}
名词从句由 Where were you last night 改造而来。

\subsection{how/why}

关系副词另外还有两个:由 so 改写的 how 与由 for a reason 改成的why。它们的变
化也没什么特殊之处,只要注意关系从句与名词从句的差别即可。例如:

\begin{enumerate}
\item Can you show me \unbf{the way}?
\item You pulled off that trick \unbf{in that way} (=so).

  \reitem (A) Can you show me \unct{the way}{先行词} \unct{(how) you pulled
    off that trick}{关系从句}?

  能不能教我你那套把戏是怎么变的?
\end{enumerate}
(A)中的关系副词 how 应省略掉比较精简,但是名词类的 the way不能省略。所以:
\begin{itemize}
\item Can \unct{you}{S} \unct{show}{V} \unct{me}{O} \unct{how you pulled off
    that trick}{O 名词从句}?
\end{itemize}
这句中的 how \ldots{} 从句应视为 How did you pull off that trick? 这个疑问句
改造成的名词从句。how 是疑问词,而不是关系词。又如:
\begin{enumerate}
\item I've forgotten \unbf{the reason}.
\item I called \unbf{for a reaseon}.

  \reitem (A) I've forgotten \unct{the reason}{先行词} \unct{(why) I
    called}{关系从句} .

  我忘了我为什么打这个电话。
\end{enumerate}

同样的,副词类的 why 省略为佳,名词类的 the reason 则不宜省略,所以:
\begin{itemize}
\item \unct{I}{S}\unct{'ve forgotten}{V} \unct{why I called}{O 名词从句}.
\end{itemize}
这句中的 why I called 应视为由 Why did I call 改造而成的名词从句。

\section{有逗点隔开的关系从句}

关系副词引导的关系从句,如果要\textbf{用逗号与主要从句隔开},原因与关系代名词时的情形
完全相同:将逗号视为一组括弧,括弧中的关系从句为\textbf{补充说明}的功能,\textbf{失去了指示
的功能}。所以关系副词不能用指示代名词that 来替代,同时也不能够省略。请看下例:

\begin{enumerate}
\item Shakespeare was born in \unbf{1564}.
\item Queen Elizabeth I was on the throne \unbf{then} .
\end{enumerate}
例 2 的 then 和例 1 的 1564 重复,建立关系,改写成关系词when。然后,因为先行
词 1564 是一个明确的年代,不是模糊的时间(像 a time等),所以只能补充说明那一
年有什么特别的事情,而不是进一步指出时间。这种性质的先行词,后面要用括弧性的
逗号把关系从句括起来,成为:
\begin{enumerate}[resume]
\item Shakespeare was born in \unct{1564}{先行词} \unct{when Queen Elizabeth I
    was on the throne}{关系从句}.

  莎士比亚出生于 1564 年,当时伊莉莎白女王一世在位。
\end{enumerate}

再看下例:
\begin{enumerate}
\item   The best museum in Taiwan is \unbf{the Palace Museum} .
\item   You can see our national treasures \unbf{there}.
\end{enumerate}
故宫博物院是个明确的地名,已无法进一步指认,所以关系从句要用括弧性的逗号隔开,
当作补充说明,成为:
\begin{enumerate}[resume]
\item The best museum in Taiwan is \unct{the Palace Museum}{先行词}, where you
  can see our national treasures.

  台湾最好的博物馆是故宮,那里可以看到我们的国宝。
\end{enumerate}
当然,\textbf{这里的 where不能够省略,因为括弧与正文切断开来,不能借用,括弧中要重
  新作完整的交待。}

\section{Wh-ever 与副词从句}

\emph{wh-ever} 解释为 \emph{no matter wh-},是表示让步、条件的语气,它的功能相当于副词从
句的连接词,引导的就是副词从句。例如:

\begin{itemize}
\item Whenever (=No matter when) he gets upset, he turns on the radio.

  只要他心情不好,他就会打开收音机。
\end{itemize}

这个句子中 whenever 当作 no matter when(不论何时)来解释,是一种表让步或条件
的语气,和 if的语气近似。\textbf{它引导的从句就当副词从句看待,也就是直接附加在主要
  从句上,修饰其动词turns on。}另外,wherever,however等由关系副词变出来的连
接词,同样都分别解释为 no matter where,no matter how,后而引导的也都是副词从
句。可是由关系代名词变出来的whoever,whatever 与 whichever 的变化就比较复杂,
请看下例:
\begin{itemize}
\item Whoever (=No matter who) stole the money, it can't be John.

  不论钱是谁偷的,总之不会是约翰。
\end{itemize}
这个 whoever 当作 no matter who(不论谁)来用,仍是让步的语气,所以引导的从句
还是作副词从句诠释,直接附在主要从句上。下面这个例子就不同了:
\begin{itemize}
\item \unct{I}{S}\unct{'ll fire}{V} \unct{whoever (=anyone that) stole the
    money}{O 关系从句省略先行词变名词从句}.

  偷钱的人我一定要开除。
\end{itemize}
这个句子中的 whoever 不能解释为 no matter who,因为这样一来后面的从句成了副词
从句,那么动词 fire就没有宾语了。whoever 应该解释为 anyone that,这样 fire才
有宾语(anyone)。这种区分对 whatever 、 whichever这两个关系代名词也要注意,
试比较下面句子:
\begin{enumerate}
\item \unct{Whatever (= No matter what) he may say}{副词从句}, \unct{I}{S} \unct{won't change}{V} \unct{my mind}{O}.

  不管他会怎么说,我主意已定。
\item \unct{Whatever (= Anything that) he may say}{S 关系从句省略先行词成为名词
    从句} \unct{won't be}{V} \unct{true}{C}.

  他再说什么都是騙人的。
\end{enumerate}

再看看 whichever:
\begin{enumerate}
\item \unct{Whichever (=No matter which) way you go}{副词从句},
  \unct{I}{S}\unct{'ll follow}{V}.

  不论你走哪条路,我都跟定你了。
\item \unct{Whichever way (=Any way that) you go}{S 关系从句省略先行词成为名词
    从句} \unct{is}{S} \unct{fine}{C} with me.

  无论你走哪条路.我都乐意奉陪。
\end{enumerate}

在此可以归纳一下:\emph{wh-ever} 的构造,如果解释为 \emph{no matter wh-},近似让步、
条件的语气,其后的从句要当副词从句解释,直接附在主要从句上作修饰语。但是如果
解释为\emph{anyone/anything that},就是关系从句省略掉先行词,后面的从句因而要当名词
从句解释,在主要从句中扮演主语、宾语……等的名词角色。

\section{结语}

随着关系从句的结束,本书也已处理完名词从句、形容词从句、副词从句等所有的从属
从句,也就是介绍完了复句结构。下一章开始要处理合句(Compound Sentence),亦即
对等从句。对等从句本身的观念不难,但变化仍然很多,尤其是牵涉到省略时,写起来
很容易出错,是写作必须克服的一关。世界上最难的语法测验——GMAT语法修辞——在
这方面的题目就相当多。这是我们下一章要探讨的范围。下面先做一做关系从句的练习
题,回忆一下本章的要点。

\section{Test}

\paragraph{请选出最适当的答案填入空格内,以使句子完整。}

\begin{enumerate}
\item Not long ago I wrote a letter to a friend, \ttu almost got us into a quarrel.
\begin{tasks}(2)
  \task whom
  \task where
  \task which
  \task what
\end{tasks}

\item England, \ttu is justly proud of her poets, is today ranked behind the continent in poetic achievement.
\begin{tasks}(2)
  \task which
  \task that
  \task where
  \task whom
\end{tasks}

\item You are the only friend \ttu he will listen to at all.
\begin{tasks}(2)
  \task where
  \task whom
  \task which
  \task that
\end{tasks}

\item Choose the correct sentence:
\begin{tasks}
  \task I have bought a book, the cover of which bears a picture of The Hague.
  \task I have bought a book; the cover of which bears a picture of The Hague.
  \task I have bought a book, the cover of which, bears a picture of The Hague.
  \task I have bought a book, of which bears a picture of The Hague.
\end{tasks}

\item This is the one encyclopedia upon \ttu I can depend.
\begin{tasks}(2)
  \task that
  \task which
  \task what
  \task it
\end{tasks}

\item \ttu likes good food and cheerful service would like the Regent Hotel.
\begin{tasks}(2)
  \task Who that
  \task Someone
  \task Whoever
  \task Who
\end{tasks}

\item This custom, \ttu, is slowly disappearing.
\begin{tasks}
  \task of many centuries ago origin
  \task which originated many centuries ago
  \task with many centuries origin
\end{tasks}

\item I find it very unfair when \ttu I do is considered mediocre or a failure. I can be depressed for days because of \ttu happens.
I.
\begin{tasks}(2)
  \task that
  \task those
  \task which
  \task what
\end{tasks}

II.
\begin{tasks}(2)
  \task who
  \task what
  \task that
  \task where
\end{tasks}

\item \ttu is elected President, corruption won't cease.
\begin{tasks}(2)
  \task Whatever
  \task Who
  \task How
  \task Whoever
\end{tasks}

\item Neither success nor money, to me at least, is the criterion \ttu we are to be judged.
\begin{tasks}(2)
  \task which
  \task under which
  \task under which
  \task since which
\end{tasks}

\item I'm afraid I'd never be able to see Jane again, \ttu very much.
\begin{tasks}(2)
  \task that I love
  \task I love
  \task I love her
  \task whom I love
\end{tasks}

\item Didn't you know that all \ttu is not gold?
\begin{tasks}(2)
  \task which glitters
  \task glitters
  \task who glitters
  \task that glitters
\end{tasks}

\item I have a present for \ttu his hand first.
\begin{tasks}(2)
  \task whoever raises
  \task whomever raises
  \task anyone raises
  \task whoever that raises
\end{tasks}

\item Boys \ttu in the dorm make a lot of friends.
\begin{tasks}(2)
  \task who live
  \task who lives
  \task live
  \task that living
\end{tasks}

\item The final decision will be up to \ttu everyone trusts.
\begin{tasks}(2)
  \task Judge Clemens, whom
  \task Judge Clemens, who
  \task Judge Clemens whom
  \task Judge Clemens who
\end{tasks}

\item \ttu he has in his pocket, it's not a gun.
\begin{tasks}(2)
  \task What
  \task Whatever
  \task When
  \task How
\end{tasks}

\item Abandoned flower pots are \ttu.
\begin{tasks}
  \task where do mosquitoes thrive
  \task mosquitoes thrive there
  \task where mosquitoes thrive
  \task what mosquitoes thrive
\end{tasks}

\item The author wrote his first novel \ttu he was working as a hotel clerk.
\begin{tasks}(2)
  \task which
  \task during
  \task what
  \task while
\end{tasks}

\item \ttu held upside down, the fire extinguisher begins to spray bubbles.
\begin{tasks}(2)
  \task When it is
  \task When they are
  \task Whenever they are
  \task During it is
\end{tasks}

\item I need to know \ttu the library is open.
\begin{tasks}(2)
  \task that
  \task when
  \task which
  \task if it
\end{tasks}

\end{enumerate}

\section{Answer}
\begin{enumerate}
\item (C) 这个位置要能作 got 的主语,又要作连接词,因而是关系代名词
  (A、C 或 D)。 其中 A 是宾格不行,D 又需省去先行词,所以只剩 C 的 which,
  表示“写信这件事”或“这封信”险些引起争吵。


\item (A) 空格中是 is 的主语(A 或 B),而在括弧性的逗号中不应用指示性的 that,
  故选 A。

\item (D) 先行词是 the only friend,有明显的指示功能,故关系词应用 that。

\item (A) B 错在以分号区分关系从句和先行词,C 错在以逗号分隔主词 the cover of which 和动词 bears,D 错在用介词短语 of which 作主语。


\item  (B) 虽然先行词 the one encyclopedia 也有明显的指示功能,但关系词出现在介词后面(在此为 upon)时就不能借用 that,所以选 B。

\item (C) 空格前面没有先行词,因而要选 whoever 这个省略先行词的关系词。

\item (B) A 和 C 都在名词 origin 前面加上了短语(many centuries ago 和 many
  centuries)来修饰,可是名词前面只能用单词的形容词来修饰,所以错误。B 是正确
  的形容词从句。

\item \Ronum{1} (D) \Ronum{2} (B) 两个位置都省掉了先行词,所以只能选择 what。
\item (D) 空格前无先行词,只能选 A 或 D。而“当选总统”者应为“人”,故选 D。


\item (C) 关系从句可还原为 We are to be judged under the criterion of \ldots(我们应以此
  标准来被衡量。),因而改成关系从句要用 under which。


\item (D) 空格以下原来是这一句:I love Jane very much,改成关系从句即为 whom I
  love very much。因为关系从句前面有逗号,所以 whom 不能省略。\footnote{
    “I'm afraid I'd never be able to see Jane again, I love her very much.”
    有一个小的语法错误。两个独立的分句应该用正确的标点符号连接。这里可以用分
    号或者句号来分隔两个句子,或者使用连词来使句子更流畅。

    改正后的句子还可以是:
    \begin{enumerate}

    \item 使用分号: “I'm afraid I'd never be able to see Jane again; I love
      her very much.”
    \item 使用句号:“I'm afraid I'd never be able to see Jane again. I love
      her very much.”
    \item 使用连词:“I'm afraid I'd never be able to see Jane again because I
      love her very much.”
    \end{enumerate}
  }

\item (D) All that glitters is not gold.(会发亮的并不都是金子。)这是一句格言。
  关系从句 that glitters 之中的关系词应用 that,因为先行词 all 表示“全部”,
  是一个指示明确的范围,所以要用 that 来取代 which。

\item (A) whoever 这种关系词不需要先行词,功能相当于 anyone that。因为要作动
  词 raises 的主语,所以要用主格 whoever。

\item  (A) who live in the dorm 是形容词从句,主语 who 代表先行词 boys,是复数, 所以动词 live 不加 \emph{-s}。
\item (A) Judge Clemens 是专有名词,作为先行词时要用逗号隔开来,因为这种关系从句不能有指示性。逗号后面的关系词应用宾格的 whom,因为作 trusts 的宾语使用。

\item (B) whatever 一字作为 no matter what 解释时,是表示让步的语气,后面引导的从句应作为表示让步的副词从句解释。

\item (C) 本句可还原为:Abandoned flower pots are places where mosquitoes thrive.(弃置的花盆是蚊虫孳生的地方。)省略掉 places 之后就是 C 的答案。


\item (D) 空格后面是表示时间的副词从句,while 即是副词从句连接词。

\item  (A) 后面的 fire extinguisher(灭火器)是单数,所以代名词要用单数 it。从句 it is held… 需要连接词,故选 A。D 的 during 是介词。
\item (B) 从语法要求来看,A 和 B 都对。A 表示“图书馆开着这件事”,B 则是由疑问
  句变来,表示“图书馆什么时候开”,两者都是正确的名词从句。不过 B 的问题比较
  能配合上文 I need to know… 的语意。
\end{enumerate}

\chapter{对等连接词与对等从句}

对等连接词(主要是 and,or 与 but三个)用来连接句子中两个对等的部分(单词或短
语),也可以连接两个对等的从句。所谓对等,指的是\textbf{结构与内容两方面都要对称},而
且对得越工整越好。这个要求很容易理解,但是在写作时却常常被忽略而产生错误或不
佳的句子。尤其是在有主、从关系的复句中,或者是简化从句中,若再出现对等连接词,
稍有不慎就会出错。以“相关词组”(correlatives)出现的对等连接词(如not
\ldots{} but;not only \ldots{} but also;both \ldots{} and;either \ldots{} or等等)
也很容易造成错误。再者,对等连接词所连接的对等从句中常会为了避免重复而进行省
略,这又是一个容易出错的地方。所以,对等连接词本身固然很单纯,但它在句中的运
用却是变化万千。全世界最难的语法考试——GMAT(美国管理研究所入学测验)的语法
修辞(Sentence Correction)部分,有关对等连接词的题目就占了不小的比例。

以下不再赘述简单的观念,直接提供十二则例子来说明对等连接词与对等从句需注意的
地方。这些例子部分模仿GMAT考题的形态,每一句中都有一部分画了底线,其中包含对
等连接词使用不当所造成的错误。读者可以自我测验一下:先找找看错在哪里,试着改
改看,然后再看后面的说明以及建议的改法。这些例句的性质相当接近GMAT考题,句型
结构多半较长,也比较复杂,其中包含了本书下一章才会讲解的“简化从句”。如果读
者一时无法全部了解,或是不知如何修改,可以先看一下翻译再尝试改改看。

\begin{enumerate}
\item The Yangtze River, the most vital source of irrigation water across the
  width of China \CJKunderline{and important as a transportation conduit as well,} has
  nurtured the Chinese civilization for millennia. (误)


  长江是横贯中国最重要的灌溉水源,同时也是重要的交通管道,数千年来孕育着中华文化。
\end{enumerate}


主要从句的基本句型是:
\begin{itemize}
\item \unct{The Yangtze River}{S} \unct{has nurtured}{V} \unct{the Chinese
    civilization}{O}.
\end{itemize}

主语与动词中间的两个逗号当一对括弧来看,括弧中放的是主语 The Yangtze River 的
同位语(就是形容词从句简化,省略 which is的结果。详情将于“简化从句”单元中介
绍)。这个句子就错在对等连接词 and连接的两个部分在结构上并不对称:左边的 the
most vital source是名词短语,右边的 important 却是形容词,词类不同,不适合以
对等连接词and连接。底线部分的改法不只一种,但是最简单的改法就是把右边的词类改
为名词类以符合对称的要求,故应修正为:
\begin{mybox}
\begin{itemize}
\item The Yangtze River, the most vital source of irrigation water across the
  width of China \unbf{and an important transportation conduit}, has nurtured the
  Chinese civilization for millennia. (正)
\end{itemize}
\end{mybox}

\begin{enumerate}[resume]
\item Scientists believe that hibernation is triggered \uline{by decreasing
  environmental temperatures, food shortage, shorter periods of daylight,
  and by hormonal activity}. (误)

  科学家认为引发冬眠的因素包括环境的气温下降、食物短缺、白昼缩短以及荷尔蒙作用。
\end{enumerate}

句中画底线的部分是以 by A、B、C and by D 的结构来修饰宾语从句中的动词 is
triggered。由内容来看 A、B、C、D是平行的(都是引发冬眠的因素),应该以对等的
方式来处理。可是原句的处理方式中,by A、B、C 之间缺乏连接词,而 and 只能连接
两个 by 引导的介词短语(by this and by that),因此原句的结构有语法上的问题。
最佳的修改方式是把A、B、C、D 四项平行的因素并列,以连接词 and串连,共同置于单
一的介词之后成为 by A、B、C and D 的结构,故应修正为:
\begin{mybox}

\begin{itemize}
\item Scientists believe that hibernation is triggered \ul{\textbf{by decreasing
    environmental temperatures, food shortage,shorter periods of daylight,
    and hormonal activity}}. (正)
\end{itemize}
\end{mybox}

\begin{enumerate}[resume]
\item Smoking by pregnant women may \ul{slow the growth and generally harm} the
  fetus. (误)

  孕妇吸烟可能妨碍胚胎发育,对胚胎造成一般性的伤害。
\end{enumerate}
这个句子可视为以下的对等从句的省略:
\begin{itemize}
\item \unct{Smoking by pregnant women}{S} \unct{may slow}{V1} \unct{the growth
    of the fetus}{O1}, and \unct{it}{S2} \unct{may generally harm the
    fetus}{V2}.
\end{itemize}

这两个对等从句的主语 smoking by pregnant women 相同,宾语 the fetus也相
同。\textbf{对等从句省略的原则就是,相对应位置如果是重复的元素就可以省略。}这是因为
对等从句有相当严格的对称要求,即使省略掉重复的元素依然能表达清楚。不过在上面
这个句子中,两个宾语扮演的角色不同:在前面的对等从句以fetus 为介词 of 的宾
语,在后面的对等从句则以 fetus 为动词 harm的宾语。所以固然可以省略前面的宾
语 fetus,但是介词 of却不能省略。故应修正为:
\begin{mybox}

\begin{itemize}
\item Smoking by pregnant women \unbf{may slow the growth of and generally harm} the
  fetus. (正)
\end{itemize}
\end{mybox}

\begin{enumerate}[resume]
\item Rapid advances in computer technology have enhanced \ul{the speed of
    calculation, the quality of graphics, the fun with computer games, and
    have lowered prices}. (误)

  电脑技术的快速进展提高了计算的速度、图形的品质、电脑游戏的乐趣,也降低了价
  格。
\end{enumerate}
这个句子以 speed,quality 和 fun 三者为动词 have enhanced的宾语,三者在内容与
结构上都是对等的,可是却没有对等连接词来连接,反而在后面加上and 和 have
lowered prices 连在一起,成为 A、B、C and D 的结构,其中A、B、C 都是名词短
语,D却是动词短语,这就犯了结构上不对称的毛病。内容上来说,A、B、C是所增加的
三样东西,D则是降低的东西,所以四者的内容也不对称,不适合并列。修改方法可以把
前面三个名词短语用A、B and C 的方式连接,第四项“降低价格”这项不对称的元素则
不必对等,而以从属从句简化(详见以后章节)的方式来处理,成为:
\begin{mybox}

\begin{itemize}
\item Rapid advances in computer technology have enhanced \ul{\textbf{the speed of
      calculation, the quality of graphics and the fun with computer games
      while lowering prices}}. (正)
\end{itemize}
\end{mybox}

\begin{enumerate}[resume]
\item Population density is very low in Canada, \ul{the largest country in the
  Western Hemisphere and it is the second largest in the whole world}. (误)

  加拿大人口密度很低,它是西半球最大的国家,也是世界第二大国。
\end{enumerate}
这个句子中,the largest country in the Western Hemisphere是形容词从句省略
掉 which is 之后留下的名词补语,也就是所谓的同位语(作为Canada 的同位语),置
于对等连接词 and 的左边。但是连接词右边的 it is the second largest in the
whole world在涵意上虽然和左边对称,可是却是主要从句的结构,所以结构上并不对称。
对等连接词的要求就是在涵意上、结构上都要尽量对称,所以可将it is the second
largest in the whole world也改为名词短语以求结构对称工整,成为:
\begin{mybox}
\begin{itemize}
\item Population density is very low in Canada, \ul{\textbf{the largest country in
      the Western Hemisphere and the second largest in the whole
      world}}. (正)
\end{itemize}
\end{mybox}

\begin{enumerate}[resume]
\item Once the safety concerns over the new production procedure were removed
  and \ul{with its superiority to the old one being} proved, there was nothing to
  stop the factory from switching over. (误)

  新的生产程序一旦排除安全方面的顾虑,并且证明它比旧的生产程序更好,这家工厂
  就没有理由不作改变了。
\end{enumerate}

对等连接词 and 出现在底线之前。它的左边是一个从属从句,右边却是介词短语,造
成结构上的不对称。可以先把它还原为对等从句,成为:
\begin{itemize}
\item \unct{The safety concerns}{S1} over the new production procedure
  \unct{were removed}{V1} and \unct{its superiority}{S2} to the old one
  \unct{was proved}{V2}.
\end{itemize}

这两个对等从句中,主语部分并不相同,动词部分是两个不同动词的被动态,只有
be 动词是重复的元素,所以只能省略一个 be 动词,成为:
\begin{itemize}
\item The safety concerns over the new production procedure were removed and
  its superiority to the old one proved.
\end{itemize}

这个省略后的对等从句前面加上once(一旦)就成为表示条件的副词从句,若再附于主
要从句之上,就成为符合对称要求的从句:
\begin{mybox}
\begin{itemize}
\item Once the safety concerns over the new production procedure were removed
  and \unbf{its superiority to the old one proved}, there was nothing to stop the
  factory from switching over. (正)
\end{itemize}
\end{mybox}

\begin{enumerate}[resume]
\item Worker bees in a honeybee hive assume various tasks, such as guarding the
  entrance, \ul{serving as sentinel and to sound a warning at the slightest
  threat}, and exploring outside the nest for areas rich in flowers and,
  consequently, nectar. (误)

  蜂窝中的工蜂担负各种任务,包括守卫入口、站哨并在威胁来临时发出警报,以及到
  巢外寻找富有花朵及花蜜的地区。
\end{enumerate}

句子中在 such as 之后列举工蜂担负的任务,基本上是 A、B and C的结构,其中
B(画底线部分)又可以分成 B1 与B2——站哨并发出警报。这两个动作是一体的两面,
选择用对等的 and来连接本来十分恰当,只是所连接的两部分 serving as
sentinel 与 to sound a warning 在结构上一是动名词,一是不定词,并不对称。再看
看 A(guarding the entrance)与 C(exploring outside the nest),都是动名词,
所以 B1 与 B2也应使用动名词才能对称,于是改为:
\begin{mybox}
  \begin{itemize}
  \item Worker bees in a honeybee hive assume various tasks, such as guarding
    the entrance, \unbf{serving as sentinel and sounding a warning at the
      slightest threat}, and exploring outside the nest for areas rich in
    flowers and, consequently, nectar. (正)
  \end{itemize}
\end{mybox}

\begin{enumerate}[resume]
\item Shi Huangdi of the Qin dynasty built the Great Wall of China in the 3rd
  century BC, a gigantic construction that meanders from Gansu province in
  the west through 2,400km to the Yellow Sea in the east and \ul{ranging}
  from 4 to 12 m in width. (误)

  秦始皇在公元前第三世纪修筑了长城,这是巨大的建筑,从西端的甘肃蜿蜒2,400 公
  里到东端的黄海,宽度由 4 米至 12 米不等。
\end{enumerate}


句中的 a gigantic construction 是 the Great Wall 的同位语,后面用 that
meanders \ldots{} 的形容词从句来修饰。对等连接词 and 的右边(底线部分)
是ranging,可是左边却找不到 Ving的结构可以与它对称。从意思上来看,右边是讲厚
度,左边讲长度的部分只有形容词从句的动词meanders 可能与 ranging 对称,所以
把 ranging 改成动词 ranges以求对称,成为:
\begin{mybox}
  \begin{itemize}
  \item Shi Huangdi of the Qin dynasty built the Great Wall of China in the 3rd
    century BC, a gigantic construction that meanders from Gansu province in
    the west through 2,400 km to the Yellow Sea in the east and
    \unbf{ranges} from 4 to 12 m in width. (正)
  \end{itemize}
\end{mybox}

\begin{enumerate}[resume]
\item The large number of sizable orders suggests that factory operations are
  thriving, \ul{but that the low-tech nature of the processing indicates
    that} profit margins will not be as high as might be expected. (误)

  从许多巨额订单来看,工厂的营运畅旺,可是加工程序属于低科技,显示利润幅度可
  能不像预期那么高。
\end{enumerate}

对等连接词 but 右边是 that 引导的名词从句,只能与左边的 that factory
operations are thriving 对称。但是如此解释出来的句意不通。仔细对比 but的左右
边,发现意思上是另一种形式的对称:
\begin{itemize}
\item A. \unct{The large number}{S} of sizable orders \unct{suggests}{V} \unct{something good}{O}.
\item B. \unct{The low-tech nature}{S} of the processing \unct{indicates}{V} \unct{something bad}{O}.
\end{itemize}
这两句在形式与意思上都很对称。其中宾语部分的 something good 与 something bad
分别以一个 that引导的名词从句来表示。看出这层对称关系之后就可以明白: but的右
边应该与左边的主要从句对称,两句都是主要从句,不应以从属连接词 that来引导,所
以 把 but 右边的 that 拿掉,成为:
\begin{mybox}
\begin{itemize}
\item The large number of sizable orders suggests that factory operations are
  thriving, \unbf{but the low-tech nature of the processing indicates that} profit
  margins will not be as high as might be expected. (正)
\end{itemize}
\end{mybox}

\begin{enumerate}[resume]
\item Not only is China the world's most populous state \ul{but also the largest
  market} in the 21st century. (误)

  中国不仅是世界人口最多的国家,也是 21 世纪最大的市场。
\end{enumerate}
像 not only \ldots{} but also之类以相关词组(correlatives)出现的对等连接词,
在对称方面的要求更为严格:not only 与 but 之间所夹的部分要和 but 右边对称。原
句中把:not only移到句首成倒装句,造成的结果是它与 but 之间夹着一个完整的从句。
因此 but的右边只有名词短语 the largest market \ldots{}就不对称,应该改为完整
的从句,成为:
\begin{mybox}
\begin{itemize}
\item Not only is China the world's most populous state \unbf{but it is also the
  largest market} in the 21st century. (正)
\end{itemize}
\end{mybox}

注意 also 的位置不一定要和 but 放在一起。also 和 only一样有强调(focusing)的
功能。Not only 修饰动词 is,与其对称之下 also也和 be 动词放在一起才好,所以右
边是 but it is also 而不是 but also it is \ldots。

\begin{enumerate}[resume]
\item New radio stations are either overly partisan, resulting in lopsided
  propaganda, or avoid politics completely, shirking the media's
  responsibility as a public watchdog. (误)

  新成立的广播电台不是党派色彩过于鲜明,造成一面倒的宣传,就是完全避谈政治,
  推卸了媒体作为大众监察人的责任。)
\end{enumerate}

相关词组 either \ldots{} or 之间所夹的部分也要与 or右边对称。原句中左边是形容
词 partisan,右边却是动词avoid,无法对称(两个简化从句 resulting
\ldots{} 与 shirking \ldots{}在此先不讨论)。可将两边都改为形容词,成为:
\begin{mybox}
\begin{itemize}
\item New radio stations \unbf{are either overly partisan,} resulting in lopsided
  propaganda, \unbf{or completely apolitical,} shirking the media's responsibility
  as a public watchdog. (正)
\end{itemize}
\end{mybox}
或者两边都用动词,成为:
\begin{mybox}
\begin{itemize}
\item New radio stations \unbf{either take an overly partisan stance,} resulting in
  lopsided propaganda, \unbf{or avoid politics completely,} thus shirking the
  media's responsibility as a public watchdog. (正)
\end{itemize}
\end{mybox}


\begin{enumerate}[resume]
\item Many modern-day scientists are not atheists, to whom there is no such
  thing as God; \ul{rather agnostics}, who refrain from conjecturing about the
  existence of God, much less His properties. (误)

  许多当代科学家并非无神论者,即不相信有神存在,而是不可知论者,即不愿妄加臆
  测神的存在与否,更不愿推断神的属性。
\end{enumerate}
这一句应该是以 not A but B 的相关词组来连接两个名词 atheists 和agnostics,后
面分别附上一个形容词从句。但是原句中却选择用分号(;)和副词rather 来连接。分
号可以取代连接词来连接两个从句,例如:
\begin{itemize}
\item He's not an atheist; rather, he believes in agnosticism.

  他不是无神论者,而是信奉不可知论。
\end{itemize}
可是分号不能取代对等连接词来连接名词短语,更不能取代 not \ldots{} but
的相关词组,所以将相关词组还原成为:
\begin{itemize}
\item Many modern-day scientists are not atheists, to whom there no such thing
  as God, \unbf{but} agnostics, who refrain from conjecturing about the existence
  of God, much less His properties. (正)
\end{itemize}

\section{Test}

\paragraph{请选出最适当的答案填入空格内,以使句子完整。}

\begin{enumerate}
\item Gold not only looks beautiful \ttu lasts forever.
\begin{tasks}(2)
  \task and
  \task nevertheless
  \task but also
  \task besides
\end{tasks}

\item \ttu to militarism nor the imposition of a totalitarianism could long guarantee Japan victory in war.
\begin{tasks}(2)
  \task The devotion is neither
  \task Neither is the devotion
  \task The devotion, neither
  \task Neither the devotion
\end{tasks}

\item Democracy is not the ideal political institution, \ttu it is an optimal one.
\begin{tasks}(2)
  \task where
  \task and
  \task so
  \task but
\end{tasks}

\item War is destructive, wasteful, and \ttu.
\begin{tasks}
  \task ultimately futile
  \task an ultimately futile exercise
  \task it is ultimately futile
  \task ultimate futility
\end{tasks}

\item To succeed in this business, you must be either talented \ttu hard working.
\begin{tasks}(2)
  \task or be
  \task or
  \task nor
  \task and
\end{tasks}

\item Not only is fruit cheap in Thailand \ttu.
\begin{tasks}
  \task but it also comes in many varieties
  \task but also in many varieties
  \task but also comes in many varieties
  \task and also various
\end{tasks}

\item Oil painting began with the Flemish artists, \ttu watercolor has been around since ancient cavemen first dug out colored earth from the ground and mixed it with water.
\begin{tasks}(2)
  \task so
  \task and
  \task or
  \task but
\end{tasks}

\item Her boyfriend is tall, handsome, and \ttu .
\begin{tasks}(2)
  \task intelligence
  \task intelligent
  \task intelligently
  \task he is intelligent
\end{tasks}

\item They plan to shop the whole afternoon and \ttu the evening through.
\begin{tasks}(2)
  \task dance
  \task dancing
  \task have danced
  \task will dancing
\end{tasks}

\item Not only \ttu but he also drinks heavily.
\begin{tasks}(2)
  \task he smokes a lot
  \task he does smoke a lot
  \task does he smoke a lot
  \task does smoke a lot
\end{tasks}

\item The origin of "go" and \ttu was in ancient China.
\begin{tasks}
  \task the place of its development
  \task it was developed
  \task it was developed which
  \task the development was there
\end{tasks}

\item Hawaii is famous for its spectacular volcanoes, friendly people, and \ttu.
\begin{tasks}(2)
  \task pleasant
  \task to have pleasant beaches
  \task its beaches are pleasant
  \task pleasant beaches
\end{tasks}

\item When the eye of a typhoon passes through, the air is still, the humidity high, \ttu low.
\begin{tasks}(2)
  \task with air pressure
  \task air pressure being
  \task that the air pressure is
  \task and the air pressure
\end{tasks}

\item A password consisting of both letters and numerals cannot be easily guessed, \ttu be easily cracked by a decoding expert.
\begin{tasks}(2)
  \task nor can it
  \task and cannot it
  \task nor it cannot
  \task it cannot
\end{tasks}

\item The police detective tried to find clues by \ttu and repeatedly questioning the suspect.
\begin{tasks}(2)
  \task careful
  \task carefully
  \task he is careful
  \task to be careful
\end{tasks}

\item Meteorological satellites help make weather forecasts more accurate and \ttu.
\begin{tasks}(2)
  \task more reliably
  \task more reliability
  \task more reliable
  \task it is reliable
\end{tasks}

\item Controlling the way you spend money is often a more effective way to meet a budget than \ttu.
\begin{tasks}
  \task try to make more money
  \task you try to make more money
  \task trying to make more money
  \task you are trying to make more money
\end{tasks}

\item Allowing children to make small decisions for themselves may contribute to harmony, efficiency and \ttu.
\begin{tasks}(2)
  \task happiness
  \task they are happy
  \task happily
  \task to happy
\end{tasks}

\item Contrary to common belief, the pencil uses \ttu.
\begin{tasks}(2)
  \task lead nor graphite
  \task but lead not graphite
  \task not lead but graphite
  \task graphite but lead
\end{tasks}

\item Dr. Sun Yat-sen is remembered by Chinese \ttu the Ching Dynasty but also for laying down the foundations for a new China.
\begin{tasks}
  \task not only overthrew
  \task only not overthrew
  \task not only for overthrowing
  \task for not only overthrowing
\end{tasks}

\end{enumerate}

\section{Answer}
\begin{enumerate}
\item (C) not only 必须有 but also 配合使用。

\item (D) neither…nor 之间要求对称。nor 的右边是名词短语 the imposition of a totalitarianism(强加以集权统治),最符合对称要求的是 D 中的名词短语 the devotion to militarism(奉献于军国主义)。

\item (D) 上文有 not,可看出下文要有 but,来表示:“并非前者,而是后者”。

\item (A) 对等连接词 and 要求对称。它前面的 destructive 和 wasteful 都是形容词,所以后面也要选形容词(futile 是“徒劳的”)。

\item  (B) either 要和 or 配合使用,而且要求对称。

\item (A) not only 要与 but also 配合使用,而且要求对称。Not only 后面是一个\textbf{倒装的从句},所以 but 后面也要选从句的构造,故选 A。
\item  (D) 空格前后分别是一个完整的句子,这两句话的内容有相反之处,所以要选表示相反的对等连接词 but。

\item (B) 对等连接词 and 要求对称。tall、handsome 和 intelligent 都是形容词,可以对称。


\item (A) and 右边用原形动词 dance 和左边的原形动词 shop 对称。

\item (C) Not only 移至句首时要用倒装句型。
\item  (A) and 左边是名词 the origin,右边也要求名词来对称,故选 A。go 指“围棋”,源自日文。

\item (D) 同样是着眼于对称要求。只有 D 的 pleasant beaches 可以和 and 左边的 spectacular volcanoes 和 friendly people 对称。

\item (D) 未省略前是 the air is still,the humidity is high,and the air pressure is low 这三个以 and 连接的对等从句,省略掉重复的 be 动词之后即得出 D。

\item(A) nor 置于句首时要用倒装句型。

\item (B) and 的右边有副词 repeatedly,因而左边选副词 carefully 来对称。

\item(C) 形容词 more reliable 可和形容词 more accurate 对称。

\item (C) 比较级也要求对称。比较的一方是动名词短语 controlling the way…,所以在 than 后面与它比较的另一方应选 C,trying to… 也是动名词短语。
\item (A) 因为对等连接词 and 的要求,所以选名词 happiness 来和名词 harmony、efficiency 对称。

\item  (C) not… but 表示“非前者,是后者”。铅笔用的不是铅,是石墨。

\item (C) 下文有 but also for…,所以空格中要选 not only for… 来配合。
\end{enumerate}


\part{高级句型——简化从句}

\chapter{从属从句简化的通则}

\section{简化从句}

英语语法以句子为研究对象。英语句型有结构较单纯的简单句与结构较复杂的复句、合
句之分,在前面已分别探讨过。简单句的结构比较单纯,只有五种基本句型的变化。作
文中若只用简单句,除了风格不够成熟外,表达力亦嫌薄弱。间杂复句、合句于文中,
则有助于表达较为复杂的观念,亦可丰富句型的变化,是风格趋于成熟。

然而,复句、合句包含两个以上的从句,其间往往有重复的元素,因而有进一步精简的
空间。若剔除重复或空洞的元素,让复句、合句更加精简,又不失清楚,这就是简化从
句。如果说简单句是初级句型,复句、合句是中级句型,那么精简的简化从句就是高级
句型。这种句型可以浓缩若干句子的意思于一句,同时符合修辞学对清楚与简洁的要求,
是讲究修辞的TIME 大量使用的句型。

\textbf{合句的简化方式是删除对等从句间相对应位置(主语与主语、动词与动词等等)重复
  的部分,}第十五章已经以例句的方式介绍过其简化方法。\textbf{复句的简化包括名词从句、
  形容词从句、副词从句三种的简化。}一般语法书称这三种从属从句的简化为\textbf{“非限
  定从句”(Nonfinite Clauses)},并称其中的 Ving(动名词或现在分词
)、Ven(过去分词)与 to V(不定词)为\textbf{“非限定动词”(Nonfmite Verbs)}。

\section{为何不称“非限定从句”?}

读者可能会感到奇怪,为何本书不沿用行之有年的“非限定从句”观念,而要提出新
的“简化从句”(Reduced Clauses)概念,原因有二:

第一,“非限定从句”的概念固然很好,但是对于各种非限定动词的由来、变化及如何
选择等等问题所提出的说明,似乎不易让学习者很快通盘了解,至少从笔者接触的学习
者及教学经验中观察是如此。

第二,非限定从句往往被与非限定动词划上等号。亦即,许多学生只知有 to V、Ving
与Ven,而不知还有许多其他的变化。因此,笔者尝试建立一套统一、易懂的结构,来诠
释比较复杂的高级句型变化。简化从句的观念就是如此产生的。这个观念回溯到修辞的
根源,以\textbf{修辞的两大要求——清楚(clear)与简洁(concise)}——为出发点,借着探
讨如何由完整的从句简化为非限定从句等等的过程,帮助学习者了解各种句型变化的道
理。

简化从句的观念可以说是笔者对修辞学的观察与教学经验结合的成果,并已经过长期实
际教学所验证,能在短期内大幅促进学习者对英语句型的掌握。

\section{从属从句简化的通则}

不论是名词类、形容词类还是副词类的从属从句,\textbf{简化的共同原则是省略主语与be动
  词,只保留补语部分}。这当中还有一些变化,例如若省略从属从句的主语会造成主语
不清时该如何处理?剩下的补语部分如果词类与原来的从属从句词类不同时要怎么办?
还有,连接词是否应一并省略?这些问题在不同词类的从属从句中,处理的方式不尽相
同,当分别探讨。不过,“省略主语与be动词,只留补语”,可以视为所有从属从句简
化的共通原则,是相当重要的观念,学习简化从句不妨从这个观念着手。

\subsection{为何省略主语?}

如果从属从句的主语是空洞的字眼(one、everybody、people等),或者从属从句主语
在主要从句中重复出现,从修辞的角度来看皆有违精简的原则,如果能省略会更简洁。
例如:

\begin{enumerate}
\item \unct{It}{S} \unct{is}{V} \unct{common courtesy}{C} that one should wear
  black while one attends a funeral.

  参加丧礼时应该穿黑衣,这是基本礼貌。
\end{enumerate}
这个句子的主要从句是 It is common courtesy,至于由连接词 that引导的名词从
句 that one should wear black 及连接词 while 引导的副词从句while one attends
a funeral 这两个从属从句的主语都是 one,代表anyone(任何人),所以是空洞的字
眼,可以省略掉,成为:
\begin{itemize}
\item It is common courtesy to wear black while attending a funeral.
\end{itemize}
去除这两个空洞的主语,句子的意思还是一样,但是变得紧凑多了,修辞效果就比原来
的句子好。再看下例:
\begin{enumerate}[resume]
\item Whether it is insured or not, \unct{your house}{S}, which is a wooden
  building, \unct{needs}{V} \unct{a fire alarm}{O}.

  你的房子是木造建筑,不论有没有保险都应该装个火警警钤。
\end{enumerate}
这个句子的主要从句是 Your house needs a fire alarm,至于由 whether引导的副词
从句 whether it is insured or not 与由 which 引导的形容词从句which is a
wooden building 这两句的主语(it 与which)指的都是主要从句中的主语 your
house,虽然用了代名词 it与关系代名词 which 来避免重复,但是仍嫌累赘,所以不如
省略,成为:
\begin{itemize}
\item Whether insured or not, your house, a wooden building, needs a fire
  alarm.
\end{itemize}
省掉重复的部分并没有更改句意,但是结构就变得比较精简,比原来的句子漂亮。

当然,从属从句的简化不能只省略主语,否则会造成句型的错误。读者应该已经看出来
了,上面两个例子中除了主语省略,连动词也经过改变。动词的改变一律可视为be 动词
的省略,包括例 1 中的 one should wear black 变成 to wear black,与 one
attends a funeral 变成 attending a funeral 都算是省略 be动词。这一点容后补述。
现在先来看看从属从句简化通则的第二个部分。

\subsection{为何省略 be动词?}

一个句子可分成主语部分 (Subject) 与动词引导的部分 (Predicate)。在简单句的五
种基本句型中,有四种都是由动词来作最重要的叙述——告诉别人这个主语在做什么。
例如:

\begin{itemize}
\item \unct{Birds}{S} \unct{fly}{V}.

  鸟儿飞。
\item \unct{Birds}{S} \unct{cat}{V} \unct{worms}{O}.

  鸟儿吃虫。
\item \unct{Birds}{S} \unct{give}{V} \unct{us}{O} \unct{songs}{O}.

  鸟为我们歌唱。
\item \unct{Birds}{S} \unct{make}{V} \unct{the morning}{O} \unct{beautiful}{C}.

  鸟儿令清晨无比美妙。
\end{itemize}
这四个不同句型的句子,同样都靠动词告诉别人,鸟做了什么事:“飞、吃虫、为我们
歌唱、让清晨美丽”。只有另一种句型——S+V+C不然,它的动词没有意义(尤其是be动
词),不能告诉别人鸟在做什么,反而要靠补语来做全部的叙述,告诉别人“鸟怎么
样”。be动词只扮演串连主语与补语的角色(所以叫做 Linking Verb 连缀动词)。例
如:
\begin{itemize}
\item \unct{Birds}{S} \unct{are}{V} \unct{lovely}{C}.

  鸟很可爱。
\end{itemize}
这句中,be动词完全不需翻译,因为它完全没有意义,只用来串连“鸟”和“很可爱”,
是由补语来负责表达对于主语的叙述。

如果 Birds are lovely 是主要从句,那么 be动词不可缺乏;可是如果这个句子是从属
从句,依附在主要从句上,再加上主语birds 如果与主要从句重复出现,那么这个从属
从句中需要保留的就只有 lovely一个词而已!重复的主语与无意义的 be动词都是多余
的,徒然浪费文字。这个从属从句去掉了主语与动词两个部分,已经不是完整的句子,
所以不再需要连接词。剩下的补语部分,如果词类与原来的从属从句的词类没有冲突,
就可以直接保留下来以取代从属从句,这就是简化从句。所以,为什么省略be 动词?我
们的答复是:因为 be 动词没有意义,省略不会影响原句的意思。

\subsection{没有 be动词怎么办?}

如果从属从句中没有 be 动词可为省略,那么可分为两种情形来处理。

\paragraph{有助动词时,变成不定词}

这是因为\textbf{所有的情态助动词都可以改写成 be 动词加不定词},例如:
\begin{itemize}
\item   You \unbf{must} go at once.

  \reitem You \unbf{are to} go at once.

  你必须马上离开。
\item The train \unbf{will} leave in 10 minutes.

\reitem The train \unbf{is to} leave in 10 minutes.

火车 10 分钟后开动。

\item   He \unbf{should} do as I say.

  \reitem He \unbf{is to} do as I say.

  他该按我说的去做。

\item   You \unbf{may} call me “Sir.”

  \reitem You \unbf{are to} call me “Sir.”

  你可以叫我“先生”。

\item Children \unbf{can't} watch this movie.

  \reitem Children \unbf{are not to} watch this movie.

  这部电影儿童不能看。
\end{itemize}

当然,助动词改写成 be加不定词,表达的意思不如原来的精确。这是为求简洁所作的牺
牲。不过也可以用going to、willing to、able to、likely to、in order to、so as
to、free to、bound to等等来补充(另可见\cref{tab:modalinf}),况且依附于主要
从句中又可以靠上下文来暗示,所以不会偏离原意。例如:

\begin{itemize}
\item He studied hard \unct{so that}{连接词} he could get a scholarship.

  他努力学习以获得奖学金。
\end{itemize}
从属连接词 so that 所引导的副词从句中,主语 he与主要从句的主语重复,可以省略。
动词 could get 可以改写为 was (able) to get,如此可省去 be 动词,留下补语部分
的 to get a scholarship,连接词也不再需要,就成为:
\begin{itemize}
\item He studied hard \unbf{to} get a scholarship.
\end{itemize}
如果怕 so that he could 省略后意思不清楚,也可如此补充:
\begin{itemize}
\item   He studied hard \unbf{so as to} get a scholarship.
\item   He studied hard \unbf{in order to} get a scholarship.
\end{itemize}

所以,从属从句中如果有助动词,简化从句时只要直接改成不定词就可以了。

\subsection{没有助动词时,变成 Ving}
\label{subsec:toing}

从属从句中若无 be 动词,也无助动词,可以如此思考:先加个 be动词进去,原来的动
词就加上 \emph{-ing},使它成为进行式的形态。如此一来就有了 be动词,Ving 之后的部
分则视为补语而保留下来。然后同样把主语和 be动词这两个没有意义的部分省略,就完
成了简化的动作。例如:
\begin{itemize}
\item \unct{John}{S} \unct{remembers}{V} \unct{that}{连接词} \unct{he saw the
    lady before}{O}.

  约翰记得以前见过这位女士。
\end{itemize}

从属连接词 that 所引导的宾语从句 that he saw the lady before 中,主语 he就是
主要从句的主语 John,可省略。可是动词 saw 不是 be动词,又没有助动词,所以无法
省略。但是简化从句中不能留下这种动词,否则句型错误(John remembers saw the
lady before 是错的,因为有两个动词)。这时候只要先把he saw the lady before 改
成 he was seeing the lady before,就有 be动词了。当然,这里用进行式并不恰当,
可是只要把 he was省略就可避免这个问题:
\begin{itemize}
\item \unct{John}{S} \unct{remembers}{V} \unct{seeing the lady}{O} before.
\end{itemize}
原来的 that he saw the lady before 是名词从句,作为主要从句中 remembers的宾语。
现在变成 seeing the lady before,可以当动名词看待,仍是名词类,同样作宾语使用,
符合词类要求又完整保留原意,这就是成功的简化从句。

所以,\textbf{从属从句简化时,如果没有 be动词可省略、也没有助动词可改成不定词,一律加
上 \emph{-ing},使动词成为 Ving的形态留下来即可。}

\section{结语}

从属从句简化,是了解复杂句型的关键,也是进入高级句型的阶梯。综上所述,从属从
句简化的通则是把主语与be动词省略,留下补语。这是简化从句最重要的观念。另外,
各种词类的从属从句在简化时各有一些细部的变化要注意,接下来几章就按词类不同分
别介绍形容词、名词及副词三种从句的简化。

\chapter{形容词从句简化}

形容词从句就是关系从句,主、从两个从句间一定有重复的元素以建立关系。既然有重
复,就可省略。\textbf{如果重复的元素(关系词)是关系从句的宾语,通常只是把关系词本
  身省略},例如:

\begin{enumerate}
\item \unct{The man}{S} \unct{is}{V} \unct{here}{C}.
\item \unct{You}{S} \unct{asked}{V} about \unct{him}{O}.
\reitem (A) The man \unct{whom you asked about}{关系从句} is here.
\reitem (B) The man \unct{you asked about}{关系从句} is here.
你要找的人在这儿。
\end{enumerate}
句 2 中的 him 就是句 1 中的 the man,借由这个重复来建立两句间的关系。将him 改
写为关系词 whom 即可将两句连接起来,成为 (A) 的形状。关系词 whom是介系
词 about 的宾语,挪到句首后可以省略,成为 (B)。

关系词是宾语而省略掉的情况,只是一般性的省略。关系从句中仍有主语、动词
((B)的 you asked about),所以这种省略不算是真正的简化从句。

如果关系词是关系从句的主语,那么简化起来,省略主语就势必也要省略 be动词,这就
是典型的形容词从句简化。以下就简化之后所留下的不同补语来加以分类介绍。

\section{补语为 Ven}

\textbf{如果关系从句中是被动态,就会简化成为过去分词的补语部分。}例如:
\begin{enumerate}
\item \unct{Beer}{S} is most delicious.
\item \unct{It}{S} is chilled to 6°C.

  \reitem (A) Beer \unct{which is chilled}{关系从句} to 6°C is most delicious.

  啤酒冰到摄氏六度最可口。
\end{enumerate}
例 2 的主语和例 1 重复,改成关系词 which来连接两句,即成(A)的形状。在(A)
中,主语 which 与先行词的 beer重复,动词部分因为是被动态,有 be 动词在。这时
只要将主语与 be动词(which is)省略,就成为:
\begin{itemize}
\item Beer \unct{chilled to 6°C}{简化形容词从句} is most delicious.
\end{itemize}

关系从句简化后剩下的补语是过去分词短语,属于形容词类,而原来的从句也是形容词类,
所以没有词类的冲突,可以取代关系从句来形容beer,而且意思不变,这就是成功的简
化从句。再举一个有逗号的关系从句为例:
\begin{itemize}
\item \unct{Your brother John}{先行词}, \unct{who was wounded in war}{关系从句}, will soon be sent home.

你哥哥约翰作战受伤,即将被送回家。
\end{itemize}
这个句子中,先行词 your brother John是专有名词,后面的关系从句因而没有“指出
是谁”的功能,只有“补充说明”的功能,所以应置于括弧性的逗号中——一对逗号当
括弧使用,用来作补充说明。放在逗号中的关系从句,简化方式仍然一样,只要把主语
与be 动词省略即可:
\begin{itemize}
\item \unct{Your brother John}{先行词}, \unct{wounded in war}{关系从句}, will soon be sent home.
\end{itemize}

\section{补语为 Ving}

如果关系从句中的动词是 be+Ving 的形状(进行式),只要省略主语与 be动词即可。
例如:
\begin{itemize}
\item \unct{The ship}{先行词} \unct{which is coming to shore}{关系从句} is from Gaoxiong.

  正在靠岸的那条船是从高雄来的。
\end{itemize}
关系从句中的主语 which 就是 the ship,又有 be
动词,只要省去这两个部分,就成为:
\begin{itemize}
\item The ship \unct{coming to shore}{关系从句} is from Gaoxiong.
\end{itemize}
剩下的补语部分是现在分词短语,属于形容词类,与原来的关系从句词类相同,这就是成功的简化从句。

如果关系从句中没有 be 动词,也没有助动词,就要把动词改成 Ving的形状。例如:
\begin{itemize}
\item \unct{My old car}{先行词}, \unct{which breaks down every other week}{关系
    从句}, won't last much longer.

  我那辆老爷车,每隔一个星期总要抛锚一次,大概开不了多久了。
\end{itemize}
这个关系从句,动词是 breaks down,既无 be动词也无助动词,无法省略,所以要先改
成有 be 动词的形态:is breaking down,有了 be 动词,breaking down 就可成为补
语部分保留下来,只省略主语与be 动词,成为:
\begin{itemize}
\item My old car, \unct{breaking down every other week}{简化形容词从句}, won't last much longer.
\end{itemize}

\section{补语为 to V}

\textbf{如果关系从句的动词有情态助动词存在,就会成为不定词补语留下来。}例如:
\begin{itemize}
\item John is \unct{the one}{先行词} \unct{who should go this time}{关系从句}.

  这次是约翰走人。
\end{itemize}
关系从句中的 who should go 固然没有 be 动词,只要将其改成 who is to go就有了,
且意思相近,再把 who is 省略,即成为:
\begin{itemize}
\item   John is the one \unct{to go this time}{简化关系从句}.
\end{itemize}

不定词的词类是“不一定”什么词类,也就是当名词、形容词、副词使用皆可。所以也
符合原来关系从句的词类,可以形容先行词the one,是正确的简化从句。

\subsection{不定词的主动、被动判断}

不定词也有主动与被动之分。其间的选择如果还原成关系从句就可以看得很清楚。例如:
\begin{enumerate}
\item John is not a man \unct{to trust}{简化形容词从句}.

  约翰这人不可信。
\item John is not a man \unct{to be trusted}{简化形容词从句} .
\end{enumerate}
例 1 和例 2都对。为什么?这得看看原来的关系从句是什么。如果原先是这两句:
\begin{itemize}
\item John is not \unct{a man}{O}.
\item \unct{One}{S} \unct{can trust}{V} \unct{the man}{O}.
\end{itemize}
后面这一句的宾语 the man 就是前一句的 a man,可以改为关系词,合成:
\begin{itemize}
\item John is not \unct{a man}{先行词} \unct{whom one can trust}{关系从句}.
\end{itemize}

因为关系从句中的关系词 whom 是宾语,可以省略,成为:
\begin{itemize}
\item John is not \unct{a man}{先行词} \unct{one can trust}{关系从句}.
\end{itemize}

这个关系从句中的主语是空洞的 one,可以简化,再把 can trust 简化为 to trust,
即成为例 1 John is not a man to trust。反之,如果原先是这两句:
\begin{itemize}
\item John is not \unct{a man}{O}.
\item \unct{The man}{S} \unct{can be trusted}{V}.
\end{itemize}
就会成为这个复句:
\begin{itemize}
\item John is not \unct{a man}{先行词} \unct{who can be trusted}{关系从句}.
\end{itemize}
从这个关系从句简化出来(省略主语 who,助动词改为不定词),即可得出例 2 John
is not a man to be trusted的结果。所以在这个例子中,不定词采主动或被动皆可。
至于该用主动还是被动,要看上下文决定,不可一概而论。

\subsection{不定词有无宾语的判断}

不定词中如果是及物动词,又有加不加宾语的差别。这也要看原来关系从句的句型来判断。例如:
\begin{enumerate}
\item This is exactly \unct{the thing}{先行词} \unct{to do}{简化形容词从句}.

  这正是该做的事。
\item   This is exactly the time to do it.

  是做这件事的时候了。
\end{enumerate}

例 1 可视为由这两句变化而来:
\begin{itemize}
\item This is exactly \unct{the thing}{O}.
\item \unct{We}{S} \unct{should do}{V} \unct{the thing}{O}.
\end{itemize}
后一句中的 the thing 是宾语,改写为关系词后成为:
\begin{itemize}
\item This is exactly \unct{the thing}{先行词} \unct{which we should do}{关系从
    句}.
\end{itemize}
因为关系词 which是宾语,可径行省略(这就是为什么到最后不定词中缺了宾语),成
为:
\begin{itemize}
\item This is exactly \unct{the thing}{先行词} \unct{we should do}{关系从句}.
\end{itemize}

再把关系从句中的主语 we省略(因为对方知道你在说谁),把助动词改为不定词,就得
出例 1 This is exactly the thing to do。如果原来是这两句话:
\begin{itemize}
\item This is exactly \unct{the time}{O}.
\item \unct{We}{S} \unct{should do}{V} \unct{it}{O} \unct{at this time}{时间副词}.
\end{itemize}

后一句中是以时间副词和先行词 the time 重复,因而改写成关系副词 when来连接:
\begin{itemize}
\item This is exactly \unct{the time}{先行词} \unct{when we should do}{关系从句}.
\end{itemize}

关系副词非主要词类,在前面没有逗号的情况下可以径行省略,成为:
\begin{itemize}
\item This is exactly \unct{the time}{先行词} \unct{we should do it}{关系从句}.
\end{itemize}
再将关系从句以同样方法简化,于是得出例 2 This is exactly the time to do it 的
结果。

\subsection{不定词后面有无介词的判断}

有些不定词宾语后面会跟个介词,像 to talk to、to deal with、to get into等。
这是因为介词后面的宾语就是关系词,径行省略之故,因而只见介词不见宾语。例
如:
\begin{enumerate}
\item He will be the toughest \unct{guy}{O}.
\item \unct{You}{S} \unct{must deal}{V} \unct{with}{介词} \unct{the guy}{O}.
\end{enumerate}
例 2 中的 the guy 是介词 with 的宾语,它和例 1 的 guy
重复而建立关系,改写成关系词来连接两句:
\begin{itemize}
\item He will be the toughest \unct{guy}{先行词} \unct{whom you must deal
    with}{关系从句}.

  他会是你得对付的家伙中最难缠的一个。
\end{itemize}

关系从句中的关系词因为是宾语,可以径行省略,成为:
\begin{itemize}
\item He will be the toughest \unct{guy}{先行词} \unct{you must deal with}{关系
    从句}.
\end{itemize}

如果对方知道你的意思,那么关系从句的主语 you 就可省略,再把 must 简化为
to,即成为:
\begin{itemize}
\item He will be the toughest \unct{guy}{先行词} \unct{to deal with}{简化关系从
    句}.
\end{itemize}

不定词后面如果跟有介词,大多是这个道理,只要还原成关系从句即可明白。

\subsection{不定词的主语不清时如何处理}

\textbf{如果主语省略会造成意思不清楚,可以安排主语于介词短语中以宾语形态出现。}最常
用的介词是for。例如:
\begin{itemize}
\item I have \unct{a job}{先行词} \unct{that your brother can do}{关系从句}.

  我有件差事想请你哥哥来做。
\end{itemize}
关系从句的关系词 that 是宾语,可以径行省略,成为:
\begin{itemize}
\item I have \unct{a job}{先行词} \unct{your brother can do}{关系从句}.
\end{itemize}
这个关系从句的动词 can do 照样可简化为 to do,但是主语 your brother不宜省略,
不然会变成 I have a job to do(我自己有件差事要做)。碰到这种主语不能省略的情
形,可以用介词短语来安插主语(这是配合不定词时的选择,若非不定词则另当别
论),成为:
\begin{itemize}
\item I have \unct{a job}{先行词} \unct{for your brother to do}{简化形容词从句}.
\end{itemize}

\section{补语为一般形容词}

若关系从句动词是 be
动词,后面是单纯的形容词类作补语,可直接简化主语(即关系词)和 be
动词,只留下补语。例如:

\begin{itemize}
\item \unct{Hilary Clinton}{先行词}, \unct{who is pretty and intelligent}{关系
    从句}, is a popular First Lady.

  希拉里·克林顿又漂亮又聪明,是相当受欢迎的第一夫人。
\end{itemize}
关系从句中的主语 who 与 be 动词省略后,剩下的部分 pretty and intelligent
还是形容词,与原来的关系从句词类相同,所以可简化取代:
\begin{itemize}
\item \unct{Hilary Clinton}{先行词}, \unct{pretty and intelligent}{关系
    从句}, is a popular First Lady.
\end{itemize}

了解形容词从句的简化,就可以了解 pretty and intelligent是简化从句的补语部分。
由此观之,\textbf{形容词只有两种位置:名词短语中(a pretty woman)及补语位置(the
  woman is pretty)。如果乍看之下两个位置都不是,那么多半就是简化形容词从句的
  残留补语。}

\section{补语为名词}

\textbf{关系从句是形容词类,如果简化主语和 be动词,剩下的是名词补语,其词类虽与原来的
关系从句词类有冲突,但仍然可以使用。}传统语法则为此取了个名称:\textbf{同位语},来避开
词类的冲突。例如:
\begin{itemize}
\item \unct{Bill Clinton}{先行词}, \unct{who is President of the U.S.}{关系从句},
  is a Baby Boomer.

  比尔·克林顿,美国总统,是生育高峰期出生的。
\end{itemize}
由 who 引导的关系从句以名词短语 President of the U.S. 为补语,简化主语与
be 动词后就剩下它。这就是传统语法所谓的同位语:
\begin{itemize}
\item Bill Clinton, President of the U.S., is a Baby Boomer.
\end{itemize}

\section{Test}

\subsection{练习一}

\paragraph{将下列各句中的关系从句(即画底线部分)改写为简化从句:}

\begin{enumerate}

\item Medieval suits of armor, \ul{which were developed for protection during
battle}, are now placed in castles for decoration.

\item The change of style in these paintings should be obvious to anyone
\ul{that is familiar with the artist's works}.

\item Islands are actually tips of underwater mountain peaks \ul{that rise above
  water}.

\item John Milton, \ul{who was author of Paradise Lost}, was a key member of
Oliver Cromwell's cabinet.

\item The secretary thought that it might not be the best time \ul{that she
should ask her boss for a raise}.

\item Gold is one of the heaviest metals \ul{that are known to man}.

\item Here are some books \ul{that your brother can use}.

\item Sexual harassment, \ul{which is a hotly debated issue in the work place}, will
  be the topic of the intercollegiate debate next week.

\item There's nothing left \ul{that I can say now}.

\item People \ul{that live along the waterfront} must be evacuated before the storm
  hits.

\end{enumerate}

\subsection{练习二}

\paragraph{请选出最适当的答案填入空格内,以使句子完整。}

\begin{enumerate}
\item \ttu often found in fruit and vegetables.
\begin{tasks}
  \task Vitamin C, a trace element that is
  \task For vitamin C, a trace element to be
  \task Vitamin C, a trace element, is
  \task Vitamin C, is that trace element
\end{tasks}

\item The most important fossil \ttu in East Africa was that of an ancient female, dubbed Lucy.
\begin{tasks}(2)
  \task excavated
  \task was excavated
  \task to excavate
  \task excavating
\end{tasks}

\item Steve Jobs' vision of the personal computer greatly expanded the number of people \ttu the computer for business and for pleasure.
\begin{tasks}(2)
  \task actively used
  \task were using actively
  \task actively using
  \task who actively using
\end{tasks}

\item The Amazon rain forests, \ttu the earth's lungs, convert carbon dioxide in the atmosphere back into oxygen.
\begin{tasks}(2)
  \task functioning as
  \task which functioning as
  \task functions as
  \task functioned as
\end{tasks}

\item Through a process \ttu coalescence, water droplets in clouds grow to a size large enough to fall to earth.
\begin{tasks}(2)
  \task calls
  \task to be called
  \task calling
  \task called
\end{tasks}

\item If you are looking for investment advice, I know just the place \ttu.
\begin{tasks}(2)
  \task going
  \task to go
  \task you to go
  \task for you going
\end{tasks}

\item Penicillin, \ttu in the early 20th century, brought in the golden age of chemotherapy.
\begin{tasks}(2)
  \task to be discovered
  \task discovering
  \task discovery was
  \task discovered
\end{tasks}

\item Those are not words \ttu.
\begin{tasks}
  \task to be taken seriously
  \task to take them seriously
  \task taking seriously
  \task are taken seriously
\end{tasks}

\item The mouse, like the keyboard, is a control device \ttu to a computer.
\begin{tasks}(2)
  \task connected
  \task to connect it
  \task and connect
  \task that connect
\end{tasks}

\item An amendment to the Constitution \ttu in Harry Truman's tenure limits the US presidency to two terms.
\begin{tasks}(2)
  \task passing
  \task to pass
  \task passed
  \task was passed
\end{tasks}

\end{enumerate}

\section{Answer}
\subsection{练习一答案}
\begin{enumerate}
\item Medieval suits of armor, \ul{developed for protection during battle}, are now
  placed in castles for decoration.

\item The change of style in these paintings should be obvious to anyone
  \ul{familiar with the artist's works}.

\item Islands are actually tips of underwater mountain peaks \ul{rising above
    water}.

\item John Milton, \ul{author of Paradise Lost}, was a key member of Oliver
  Cromwell's cabinet.
\item The secretary thought that it might not be the best time \ul{to ask her
    boss for a raise}.
\item Gold is one of the heaviest metals \ul{known to man}.
\item Here are some books \ul{for your brother to use}.
\item Sexual harassment, \ul{a hotly debated issue in the work place}, will be the
  topic of the intercollegiate debate next week.
\item There's nothing left \ul{(for me) to say now}.
\item People \ul{living along the waterfront} must be evacuated before the storm
  hits.
\end{enumerate}

\subsection{练习二答案}
\begin{enumerate}
\item (C) 答案 C 的句型是 Vitamin C is often found in fruit and vegetables,中间
  再加上同位语 a trace element(微量元素),也就是关系从句 which is a trace
  element 的简化。

\item (A) 空格以下原为关系从句 that was excavated in East Africa,简化后即得 A。

\item (C) 空格以下原为关系从句 who were actively using the computer \ldots{} 简化成为 C。

\item (A) 空格以下原为关系从句 which functions as the earth's lungs,简化为 A。

\item (D) 空格以下原为关系从句 that is called coalescence,简化为 D.

\item (B) 空格以下原为关系从句 where yon can go,简化为 B。

\item  (D) 空格以下原为关系从句 which was discovered in the early 20th century,简化为 D。

\item (A) 空格以下原为关系从句 that should be taken seriously,简化为 A。

\item  (A) 空格以下原为关系从句 that is connected to a computer,简化为 A。

\item (C) 空格以下原为关系从句 that was passed in Harry Truman's tenure,简化为 C。

\end{enumerate}



\chapter{名词从句简化}

名词从句的简化与其他词类的从属从句相同,都是省略主语与 be动词,只留下补语。因
为主语与主要从句中的元素重复,或主语原本就没有明确的内容(像someone,people
等),所以将主语省略。而省略 be动词是因为它只是连缀动词,本身没有意义。由于省
略主语与动词之后,已经不再需要连接词,所以名词从句的连接词that也一并省略。\textbf{如
果名词从句是由疑问句演变而来的,以疑问词(who、what、where等)充当连接词,那
么疑问词就要保留,因为它和 that 不同,是有意义的字眼。}

名词从句简化之后,剩下来的补语有两种常见的形态:Ving 与 to V(分别称为动名词
与不定词)。这两种形态都可以当名词使用,所以可以取代原先的名词从句,不会有词
类上的冲突。至于第三种常见的补语Ven(过去分词),因为是形容词,不能取代名词类
的从句,所以不能使用。因此名词从句中如果是被动态(be+Ven),简化时不能只是省
略be 留下 Ven,而要在词类上进一步改造,这部分详见后述。现在分别就 Ving 与to
V 这两种补语形态来探讨名词从句的简化。

\section{简化后剩下的补语是 Ving形态时}

和形容词从句简化的做法相同,如果名词从句中没有 be动词,也没有助动词,一律把动
词加上 \emph{-ing}。以下就名词从句常出现的位置分别举例说明。

\subsection{主语位置}

\begin{itemize}
\item \unct{That I drink good wine with friends}{S名词从句} \unct{is}{V} my greatest \unct{enjoyment}{C}.

  和好友一起喝美酒是我最大的享受。
\end{itemize}
典型的名词从句是由一个直述句(如 I drink good wine with friends)外加连接
词 that而构成,表示“那件事”。上例中这个名词从句置于主要从句的主语位置当主语
使用。简化的做法是省去里面的主语I(因为主要从句中有“my” greatest
enjoyment可表示是谁在喝酒)。但因为这个名词从句没有 be动词,也没有助动词,所
以得先把它改成进行式的形态:
\begin{itemize}
\item That I am drinking good wine with friends is my greatest enjoyment.
\end{itemize}
然后就可省略主语 I 与 be 动词,以及已经没有作用的连接词
that,成为较紧凑的句子:
\begin{itemize}
\item \unct{Drinking good wine with friends}{简化名词从句} is my greatest enjoyment.
\end{itemize}

\subsection{宾语位置}

\subsection{动词的宾语}

\begin{itemize}
\item \unct{Many husbands}{S} \unct{enjoy}{V} \unct{that they do the
    cooking}{O名词从句}.

  许多丈夫喜欢下厨做菜。
\end{itemize}
名词从句的主语 they 与主要从句主语 husbands 相同,所以可省略。动词是
do,没有 be 动词或助动词,所以要加上 \emph{-ing} 再省略连接词,成为:
\begin{itemize}
\item Many husbands enjoy \unct{doing the cooking}{简化名词从句}.
\end{itemize}

\subsection{ 介词的宾语}

\begin{enumerate}
\item \unct{He}{S} \unct{got}{V} \unct{used}{C} \unct{to}{介词} \unct{something}{O}.
\item He worked late into the night.
\end{enumerate}

整个例 2 就是例 1 中 something 的内容。要把例 2 放入 something的位置,还不能
直接用名词从句 that he worked late into the night的形式,因前面是介词 to,
不能直接放名词从句作宾语,所以例 2一定要先行简化。做法还是将相同的主语省略,
动词加上 \emph{-ing},成为:

\begin{itemize}
\item He got used to \unct{working late into the night}{简化名词从句}.

  他习惯了熬夜工作。
\end{itemize}

\subsection{补语位置}

\begin{itemize}
\item \unct{His favorite pastime}{S} \unct{is}{V} \unct{that he goes fishing on
    weekends}{C名词从句}.

  他最喜欢的消遣就是周末钓鱼。
\end{itemize}
省略名词从句的 he,动词加 \emph{-ing} 而成为:
\begin{itemize}
\item   His favorite pastime is going fishing on weekends .

他习惯了熬夜工作。)
\end{itemize}

\subsection{主语不能省略时}

有时省略名词从句的主语会造成句意的改变,这时要设法用其他方式来处理。以下几种
方式较为常见:

\subsection{ 改成 S+V+O+C 的句型}

但要如此修改,名词从句必须是处于宾语位置,而且主要从句的动词适用于S+V+O+C 的
句型。例如:
\begin{itemize}
\item \unct{I}{S} \unct{imagined}{V} \unct{that a beautiful girl was singing to
    me}{O名词从句}.

  我想象有个美女在对我唱歌。
\end{itemize}
以上的名词从句中,主语是 a beautiful girl,和主要从句的主语 I
不同。如果径行简化,省略主语与 be 动词,会变成:
\begin{itemize}
\item   I imagined singing to myself.

  我想象在对自己唱歌。
\end{itemize}
这个句子的意思就完全不一样了。所以,要完整保留原意,名词从句的主语 a
beautiful girl 不能省略,只能把 be 动词省略。在上例中恰好可以这样处理:
\begin{itemize}
\item \unct{I}{S} \unct{imagined}{S} \unct{a beautiful girl}{O} \unct{singing to me}{C}.
\end{itemize}
名词从句的主语 a beantiflil girl 放到宾语位置,原来的主语补语 singing to me放
在宾语补语的位置,就可顺利解决问题。原来的复句也简化为 S+V+O+C的句型。

\subsection{用所有格来处理}

\begin{itemize}
\item \unct{That he calls my girlfriend every day}{S 名词从句} \unct{is}{V}
  \unct{too much for me}{C}.

  他每天打电话给我女朋友真让我受不了。
\end{itemize}
若径行简化名词从句的主语 he,会成为:
\begin{itemize}
\item \unct{Calling my girlfriend}{S} every day \unct{is}{V} \unct{too much for
    me}{C}.

  每天打电话给我女朋友真让我受不了。
\end{itemize}
这句的意思变成是自己不爱打电话。所以,要保留原意,名词从句的主语 he不能省略。
但 calling my girlfriend every day取代了名词从句成为主要从句的主语,已经没有
位置可安插原来的主语he。这时可把原主语 he 改成所有格,就能放在 calling
\ldots{} 之前,成为:
\begin{itemize}
\item \unct{His calling my girlfriend every day}{S 简化名词从句} \unct{is}{V}
  \unct{too much for me}{C}.
\end{itemize}

名词从句简化为 Ving的形态,而主语不能省略时,大多可用所有格来处理主语的部分。

\subsection{加介词来处理}

这只适合一种特殊的句型。例如:
\begin{itemize}
\item \unct{I}{S} \unct{am}{V} \unct{worried}{C} \unct{that my son lies all the
    time}{名词从句}.

  我很发愁我儿子老说谎。
\end{itemize}
在简化前,首先要了解这个名词从句扮演的角色。在 S+V+C的句型后面,本来并没有名
词存在的空间,所以上述的句型要这样诠释:
\begin{itemize}
\item \unct{I}{S} \unct{am}{V} \unct{worried}{C} about the fact \unct{that my
    son lies all the time}{同位语\quad 名词从句}.
\end{itemize}
这句的名词从句 that my son lies all the time 是 the fact的同位语(即形容词从
句 which is that \ldots{} 的补语,其中 which is经简化而省略)。这个名词从句简
化后即可置入与它重复的 the fact的位置。因主语 my son 与主要从句主语 I 不同,
故可用所有格来处理,成为:
\begin{enumerate}
\item \unct{I}{S} \unct{am}{V} \unct{worried}{C} \unct{about}{介词} \unct{my
    son's lying all the time}{O 简化名词从句}.
\end{enumerate}

也可以将主语 my son 置于 about 后面的宾语位置,lying all the time作宾语补语:
\begin{enumerate}[resume]
\item \unct{I}{S} \unct{am}{V} \unct{worried}{C} \unct{about}{介词} \unct{my son}{O} \unct{lying all the time}{C}.
\end{enumerate}

句 1 和句 2 这两种处理方式在语法上都正确。在意思上又以句 1更接近原意。因为在
原句中,说话的人所担心的是一件事情(that my son lies all the time),简化
为 my son's lying all the time仍是一件事情,比较接近。但改成句 2 时,担心的对
象变成了人(my son),事情(lying all the time)则降格成了修饰语,所以意思和
原句稍有出入。

\subsection{如何处理被动态}

被动态中若省略主语和 be 动词,剩下的补语 Ven是形容词类,无法取代原来的名词从
句,所以必须进一步修改。例如:
\begin{itemize}
\item \unct{That anyone is called a liar}{S 名词从句} \unct{is}{V} the greatest \unct{insult}{C}.

  任何人被叫作骗子都是最大的侮辱。
\end{itemize}
这个名词从句的主语 anyone 没有特定的对象,是空泛的字眼,可省略。再省略be 动词
和连接词 that,本来算是完成了简化,可是:
\begin{itemize}
\item \unnormal{Called a liar}{S} \unnormal{is}{V} the greatest \unnormal{insult}{C}. (误)
\end{itemize}

剩下的补语 called a liar是形容词类,不能取代原来的名词从句作主语。如果
将 called 改成calling,虽然变成了名词类,但是被动的意味消失了:called a
liar是“打电话给一个骗子”。所以,为了维持被动态,called a liar不能更动,只能
借用前面的 be 动词来作词类变化,成为 being called a liar。be 动词本身没有意义,
把它加上 \emph{-ing}纯粹只有词类变化的功能,并不改变句意,因而成为:
\begin{itemize}
\item \unct{Being called a liar}{S 简化名词从句} \unct{is}{V} the greatest
  \unct{insult}{C}.

  任何人被叫作骗子都是最大的侮辱。
\end{itemize}

再看一个例子:
\begin{enumerate}
\item I am looking forward to \unbf{something}.
\item \unbf{I am invited to the party}.
\end{enumerate}
例 2 就是例 1 中 something 的内容,可以简化后放人 something的位置。但是例 2是
被动态,如果直接省略主语与 be 动词,会成为:
\begin{itemize}
\item I am looking forward to \unnormal{invited}{形容词} to the party. (误)
\end{itemize}
过去分词补语 invited \ldots{} 是形容词,不能直接放在介词 to的后面。若直接
将 invited 的词类改变,就这个例子而言意思也维持不变:
\begin{itemize}
\item I am looking forward to the \unct{invitation}{名词} to the party.

  我盼望着受邀去参加舞会。
\end{itemize}

如果按照前面的做法,加上 being 来改变 invited 的词类当然也可以:
\begin{itemize}
\item I am looking forward to \unct{being invited to the party}{简化名词从句}.
\end{itemize}

名词从句简化成 Ving 的形式,如果是被动态时,以 being Ven的形式就可以表示,并
仍以名词的形式保留下来。

\subsection{动词是单纯的 be动词}

若名词从句中是 be 动词,后面接一般的名词或形容词作补语,则须加
上 \emph{-ing}成为 being:

\begin{itemize}
\item \unct{That one is a teacher}{S 名词从句} \unct{requires}{V} a lot of \unct{patience}{O}.

  做老师的人就得很有耐心。
\end{itemize}
名词从句中是单纯的 be 动词,后面接 a teacher 作补语。简化时改成
being \ldots{} 才能保持“做”老师的味道:
\begin{itemize}
\item \unct{Being a teacher}{简化名词从句} requires a lot of patience.
\end{itemize}
若省略 be 动词,成为:
\begin{itemize}
\item   A teacher requires a lot of patience.
\end{itemize}
意思会稍有不同。又如:
\begin{itemize}
\item \unct{That he was busy}{S 名词从句} \unct{is}{V} no \unct{excuse}{C} for the negligence.

  “他很忙”并不能构成疏忽的借口。
\end{itemize}
这个名词从句是单纯的 be 动词后接形容词 busy 作补语。简化时也不能径行省略
be 动词,否则会剩下形容词 busy,无法充当主语。正确的做法仍是改成 \emph{-ing}:
\begin{itemize}
\item \unct{Being busy}{简化名词从句} is no excuse for the negligence.
\end{itemize}

\section{简化后剩下的补语是 to V形态时}

名词从句简化,若其中有情态助动词,含有不确定语气,就会成为不定词(to
V)。如:
\begin{itemize}
\item \unct{The children}{V} \unct{expect}{V} \unct{that they can get presents
    for Christmas}{O}.

  孩子们期望圣诞节能得到礼物。
\end{itemize}
这个名词从句中有助动词,表示不确定语气(还不一定拿得到)。简化时可以先把助动
词改写为be+to(所有的情态助动词都可如此改写以便简化),成为:
\begin{itemize}
\item The children expect that \unct{they are to get presents for Christmas}{名
    词从句}.
\end{itemize}
如此一来,名词从句中有了 be 动词,就可以把 that they are这三个没有内容的部分
简化,成为不定词的形态:
\begin{itemize}
\item   The children expect \unct{to get presents for Christmas}{O 简化名词从句}.
\end{itemize}

不定词即“不一定是什么词类”,可当名词、形容词、副词,所以不必顾虑词类是否符
合的问题。唯一要注意的是,\textbf{不定词不适合放在介词后面,这时要改为Ving 的形式}。
再看一个例子:
\begin{itemize}
\item \unct{I}{S} \unct{find}{V} \unct{it}{O} \unct{strange}{C} \unct{that man
    should fear ghosts}{名词从句}.

  我觉得人竟然怕鬼是很奇怪的事。
\end{itemize}
上面的名词从句是当作 find 的宾语使用。不过这个宾语从句后面还有宾语补
语strange,照写的话会产生断句的问题,所以用 it这个虚字(expletive)暂代一下宾
语位置,而把真正的宾语从句移到补语后面。

这个名词从句的主语是man,可以指任何人,所以是空泛的字眼,可以省略。助动
词 should就可简化为不定词,成为:
\begin{itemize}
\item I find it strange \unct{to fear ghosts}{简化名词从句}.
\end{itemize}

\subsection{主语不适合省略时}

名词从句的主语如果和主要从句不重复,又不是空泛的字眼,省略时往往会改变句意。
这时就要想办法改变这个主语,将它保留。在有些句型中可以把主语放入宾语位置,变
成S+V+O+C 的句型,例如:
\begin{itemize}
\item \unct{I}{S} \unct{want}{V} \unct{that you should go}{O 名词从句}.

  我希望你去。
\end{itemize}
名词从句的主语是you,有特定的对象,又和主要从句不重复,因而不适合省略。此时先
将 should改写为 be+to,成为:
\begin{itemize}
\item \unct{I}{S} \unct{want}{V} \unct{that you are to go}{O}.

  我希望你去。
\end{itemize}
然后省去 be 动词,主语的 you 放入宾语位置,主语补语 to go就成了宾语补语,成
为:
\begin{itemize}
\item \unct{I}{S} \unct{want}{V} \unct{you}{O} \unct{to go}{O}.
\end{itemize}

在大部分的句型中,不定词原来的主语可放在介词后的宾语位置以保留下来,例如:
\begin{itemize}
\item \unct{That the Clippers should beat the Lakers}{S 名词从句} \unct{was}{V} quite a marvelous \unct{feat}{C}.

  快船队竟然击败湖人队,真是相当了不起的。
\end{itemize}
名词从句的主语 the Clippers 不能省,又没有别处可安插,就可加介词for,简化
为:
\begin{itemize}
\item \unct{For the Clippers to beat the Lakers}{简化名词从句} was quite a marvelous feat.
\end{itemize}

\subsection{代表疑问句的名词从句简化}

名词从句有两种。一种是由直述句外加连接词 that而形成。这种名词从句简化时,无意
义的 that要省略。另一种是由疑问句改造,通常以疑问词来充当连接词。例如:
\begin{enumerate}
\item What should I do?
\item \unct{I}{S} \unct{don't know}{V} \unct{the question}{O}.
\end{enumerate}
例 1 就是例 2 中 the question 的内容,可直接用疑问词 what当连接词来取代,成
为:
\begin{itemize}
\item \unct{I}{S} \unct{don't know}{V} \unct{what I should do}{O 名词从句}.

  我不知如何是好。
\end{itemize}
这个名词从句省去主语 I,助动词改为不定词,成为:
\begin{itemize}
\item \unct{I}{S} \unct{don't know}{V} \unct{what to do}{O 简化名词从句}.
\end{itemize}
唯一不同之处在于:疑问句 what 是有意义的字,应该保留。语法书说 where to
V、how to V、when to V等是名词短语,其实这些都是由疑问词引导的名词从句简化而
成。

如果是 Yes/No question,没有疑问词,要制造名词从句时就得添加whether,例如:
\begin{enumerate}
\item Should I vote for Mary?
\item I can't decide the question.
\end{enumerate}
例 1 就是例 2 中的 the question。不过例 1是疑问句,又没有疑问词,要置入例 2
中,先要加上 whether,成为:
\begin{itemize}
\item \unct{I}{S} \unct{can't decide}{V} \unct{whether I should vote for Mary
    (or not)}{O 名词从句} .

  我无法决定要不要投票给玛丽。
\end{itemize}
whether 是由连接词 either \ldots{} or 变造而成。在这个名词从句中,主语 I与主要
从句主语相同,可以省略。助动词改写成不定词 to V 之后,即简化成:
\begin{itemize}
\item I can't decide \unct{whether to vote for Mary}{简化名词从句}.
\end{itemize}

\section{to V 与 Ving的比较}

不定词与动名词都可以当成名词类使用,两者之间有时不易区分。可是从简化从句的角
度来看,就很容易区分清楚。请看以下的例子:
\begin{itemize}
\item \unct{He}{S} \unct{forgot}{V} \unct{that he should see his dentist that
    day}{O 名词从句}.

  他忘了他那天应该去看牙医的。
\end{itemize}
这个名词从句中的动词 Should see是“应该看”,属于不确定语气,表示“该去但还没
去”。这种语气和不定词完全相同,而且助动词简化就成为不定词,所以可写成:
\begin{itemize}
\item \unct{He}{S} \unct{forgot}{V} \unct{to see his dentist that day}{O 名词从
    句}.
\end{itemize}
相反的,如果原本的句子是这样:
\begin{itemize}
\item \unct{He}{S} \unct{forgot}{V} \unct{that he saw the man before}{O 名词从句}.

  他忘了以前见过这个人。
\end{itemize}
这是真的见过,是确定的语气,所以没有助动词,只是单纯的动词saw。这个名词从句简
化时,因为没有助动词,也没有 be 动词,就只能加 \emph{-ing},成为:
\begin{itemize}
\item He forgot \unct{seeing the man before}{简化名词从句}.
\end{itemize}

另外再看看下面的例子:
\begin{itemize}
\item \unct{I}{S} \unct{love}{V} \unct{driving on the freeway}{O 简化名词从句}.

  我喜欢在高速公路上开车。
\end{itemize}
这句并没有“想去”开或“将去”开的意思,只是把“在高速公路上开车”当做一件事,
故没有不确定语气。名词从句可还原为that I drive on the freeway 或 that I am
driving on the freeway,都可简化成 driving on the freeway。下面这个例句则又不
同:
\begin{enumerate}
\item \unct{I}{S} \unct{would love}{V} \unct{to drive to work in my own car}{O
    简化名词从句}.

  我很想能够开自己的车去上班。
\end{enumerate}
这个句子有强烈的“希望能够”的暗示,但目前还不行。这就有不确定语气,牵涉到助
动词can。名词从句可还原成下句中的形状:
\begin{enumerate}[resume]
\item \unct{I}{S} \unct{would love}{V} \unct{that I can drive to work in my own
    car}{O 名词从句}.
\end{enumerate}

如果判断出名词从句中有不确定语气,或者能看出原来应有助动词,那么就会简化为不
定词的形状(如例1)。请看以下这个例子:
\begin{itemize}
\item \unct{I}{S} \unct{avoid}{V} \unct{being late to any appointment}{O 简化名
    词从句}.

  任何约会我都避免迟到。
\end{itemize}
说这句话的人只是把迟到当成一件事来谈,并没有“将要迟到”或“能够迟到”等语气,
所以没有助动词。将名词从句还原即成:
\begin{itemize}
\item \unct{I}{S} \unct{avoid}{V} \unct{that I am late to any appointment}{O 名
    词从句}.
\end{itemize}

这个名词从句简化时自然不会有不定词。下面的例子又不同:
\begin{itemize}
\item \unct{I}{S} \unct{hope}{V} \unct{to get to the concert on time}{O 简化名词
    从句}.

  我希望能赶上这场音乐会。
\end{itemize}
赶不赶得上并不确定,但是有浓厚的“希望能够”的语气,就会牵涉到助动词 can
了:
\begin{itemize}
\item I hope that I can get to the concert on time.

\item \unct{I}{S} \unct{hope}{V} \unct{that I can get to the concert on time}{O
    名词从句}.
\end{itemize}

若名词从句中有助动词,自然会简化为不定词。语法书论及 to V 和 Ving出现于动词后
面的宾语位置的选择时,会列出几份动词表,要求读者背哪些动词后面该用哪一个,以
及意思是否相同。这种死背方式不值得推荐。了解简化从句之后,读者便可发现这个区
分是顺理成章,不必死背。

\section{结语}

本章到目前为止已讨论过形容词从句与名词从句的简化,下一章将探讨副词从句的简化,
就可将所有“从属从句简化”介绍完毕。若读者能透彻了解这几章的内容,对读、写都
会有极大的帮助,TIME的复杂句型也不再会难倒你了。

\section{Test}

\subsection{练习一}

\paragraph{将下列各句中的名词从句(即画底线部分)改写为简化从句:}

\begin{enumerate}
\item \ul{That he sends flowers to his girlfriend every day} is the only way he
can think of to gain her favor.

\item \ul{That the legislator was involved in the fraud} is rather obvious.

\item The student denied \ul{that he had cheated in the exam}.

\item The researcher is certain \ul{that he has found a solution}.

\item The residents were not aware \ul{that they were being exposed to
radiation}.

\item I consider \ul{that this is a most unfortunate incident}.

\item \ul{That John comes to school late every day} cannot go on much longer.

\item \ul{That he was named the new CEO} came as a surprise to everybody.

\item I would like \ul{that you can look after the kids for me this evening}.

\item It is a privilege \ul{that one can live in these monumental times}.
\end{enumerate}

\subsection{练习二}

\paragraph{请选出最适当的答案填入空格内,以使句子完整。}

\begin{enumerate}
\item Don't worry; I'll show you \ttu.
\begin{tasks}(2)
  \task that you should do
  \task what to do
  \task what to do it
  \task that to do
\end{tasks}

\item Ministers are used to \ttu with respect.
\begin{tasks}(2)
  \task treated
  \task treating
  \task being treated
  \task treat
\end{tasks}

\item \ttu is one thing I cannot stand.
\begin{tasks}(2)
  \task Being lied
  \task Being lied to
  \task To being lied
  \task To be lied
\end{tasks}

\item The boy is worried \ttu.
\begin{tasks}(2)
  \task that will fail in the exam
  \task about failing in the exam
  \task failing in the exam
  \task about being failed in the exam
\end{tasks}

\item You mustn't forget \ttu before you leave for London.
\begin{tasks}(2)
  \task to give me a call
  \task giving me a call
  \task give me a call
  \task given me a call
\end{tasks}

\item They intend \ttu this coming Christmas.
\begin{tasks}(2)
  \task to get married
  \task getting married
  \task get married
  \task got married
\end{tasks}

\item To say you don't remember is \ttu you didn't pay any attention at the
  time.
\begin{tasks}(2)
  \task saying
  \task to say
  \task say
  \task said
\end{tasks}

\item The decision to emigrate does not necessarily mean \ttu in the country.
\begin{tasks}
  \task cutting off all ties
  \task that cuts off all ties
  \task that ties cut off
  \task cut off all ties
\end{tasks}

\item You can count on \ttu the election even before all the results are in.
\begin{tasks}(2)
  \task winning
  \task to win
  \task won
  \task that you will win
\end{tasks}

\item I never expected \ttu in this mess.
\begin{tasks}(2)
  \task involving
  \task involved
  \task to be involved
  \task involve
\end{tasks}

\end{enumerate}
\section{Answer}
\subsection{练习一答案}
\begin{enumerate}

\item \ul{Sending flowers to his girlfriend every day} is the only way he can think
  of to gain her favor.

\item \ul{The legislator's being involved in the fraud} is rather obvious. 或

  \ul{The legislator's involvement in the fraud} is rather obvious.

\item The student denied having cheated in the exam.

\item The researcher is certain about having found a solution.

\item The residents were not aware \ul{of being exposed to radiation}. 或

  The residents were not aware \ul{of their exposure to radiation}.

\item I consider \ul{this a most unfortunate incident}.

\item \ul{John's coming to school late every day} cannot go on much longer.

\item \ul{His being named the new CEO} came as a surprise to everybody.

\item I would like \ul{you to look after the kids for me this evening}.

\item It is a privilege \ul{to live in these monumental times}.

\end{enumerate}

\subsection{练习二答案}
\begin{enumerate}
\item (B) 原为名词从句 what you should do,简化为 B。

\item (C) 原为 They are treated with respect,简化为 C 以维持被动态,并以动名词形状置于介词 to 之后。
\item (B) lie(说谎)是不及物动词。“别人对我说谎”要这样表示:People lie to me. 改成被动态是: I am lied to (by people). 这个句子再简化为动名词就是 being lied to。

\item (B) 原为 about the possibility that he will fail in the exam,简化为 B。

\item (A) 原为名词从句 that you must give me a call,简化为 A。

\item (A) 原为名词从句 that they will get married,简化为 A。

\item (B) 选择不定词 to say 以求和前面的 to say 对称。两个 to say 都可视为名词从句 that you should say 的简化。

\item (A) 原为名词从句 that one cuts off all ties…,简化为 A。

\item (A) 原来是像 D 中的句子,可是从句不能放在介词 on 的后面,所以简化成 A。
  因为在介词后面,不能用不定词,所以助动词 will 可以忽略掉。

\item (C) 原为名词从句 that I would be involved…,简化为 C。
\end{enumerate}

\chapter{副词从句简化之一}

继前两章探讨形容词从句简化、名词从句简化之后,本章探讨的是比较复杂的副词从句
简化。在此重复一下重要的观念:所有从属从句简化的原则都一样,即为求精简,把从
属从句的主语与be动词省略,只留下补语。省略主语是为了避免重复,但如果省略会造
成句意模糊,主语就得另行处理;省略be 动词是因为它本身没有任何意义。

传统语法将副词从句的简化称为分词构句、独立短语等。这种标示方式不但不够周延,
也不够深入,造成许多学习者的困扰。若运用简化从句的观念就不会有这些问题。从简
化从句的角度来看,副词从句的简化可分成几种情况,本章先研究简化为Ving 补语的情
形。

\section{简化为 Ving补语}

若副词从句是一般语法书所谓的进行式(be+Ving),那么省略主语和 be动词后就只
剩 Ving 补语。反之,若没有 be动词可省略,也没有情态助动词可供改写,就得先改成
进行式,再省略 be动词,仍然可得到 Ving 的结果。例如:

\begin{itemize}
\item \unct{While he was lying on the couch}{副词从句}, \unct{the boy}{S} \unct{fell}{V} \unct{asleep}{C}.

  小男孩在沙发上躺着,就睡着了。
\end{itemize}
上例中副词从句的主语 he 就是主要从句的主语 the boy,这个重复就有可以省略的空
间。同时副词从句中有现成的 be 动词,是Linking Verb(连缀动词),本身没有意义,
因此,省去主语与 be动词,不会改变原句的意思:
\begin{enumerate}
\item \unct{While lying on the couch}{简化副词从句}, the boy fell asleep.
\end{enumerate}

\subsection{连接词是否保留}

副词从句因为已经简化,不再有主语、动词,所以上例中它的连接词 while也没有存在
的必要。不过,副词从句的连接词除了语法功能之外,还有词义的功
能:while和 before 不同,也和 if、although等不一样,虽然简化了,副词从句的连
接词有时还是要保留,至于保留与否则完全取决于修辞上是否清楚。简化是为了让句子
更简洁,可是绝不可伤害清楚性。在句子够清楚的前提下,副词从句的连接词可以一并
省去,上例即成为:
\begin{enumerate}[resume]
\item \unct{Lying on the couch}{简化副词从句}, the boy fell asleep.
\end{enumerate}

一般来说,while(包括 when等)是表示“当……之时”的连接词,because(包
括 as、since等)是“因为”的连接词,省略后通常不妨碍句子的清楚性。但还是要一
个一个句子去判断,看看省略之后读者是否可能会误解。

\subsection{所谓“分词构句”}

以例 2 而言,省去 while之后,句子仍然清楚,不过传统语法学家解释起来就大费周章。
他们只看到 lying on the couch 是现在分词短语,属于形容词类,但显然不是用来修
饰名词类的the boy(它不是用来特别指出哪一个男孩),而是修饰动词类的fell(用来
说明是何时、在何种状态下睡着)。以形容词修饰动词,这不是犯了词类错误吗?面对
这个矛盾,语法学家于是创造了一个名称:分词构句——lying on the couch 这个分词
短语本身就构成一个从句,一个修饰动词 fell的副词从句。

了解简化从句的来龙去脉后,就会了解“分词构句”一词实在是多此一举。lying
on the couch 本来就是副词从句 while he was lying on the couch
的简化,无需用任何特别名称来表示。当然,若把连接词 while 保留(如例
1),可以更明确表示这是副词从句。在这个例子中,是否要保留 while
属于个人的选择:若比较注重句子的清楚性就保留它,若比较注重简洁性就省略它。不论有无
while,都不影响一个事实:lying on the couch 是简化的副词从句。

\subsection{没有 be动词与助动词时}

如果原来的副词从句没有 be动词,也没有情态助动词(can、must、may),只有普通动
词,那么就会成为Ving 的形式,例如:
\begin{itemize}
\item \unct{Because we have nothing to do here}{副词从句}, \unct{we}{S}
  \unct{might as well go}{V} home.

  在这儿也没事做,我们还不如回家算了。
\end{itemize}
首先请观察副词从句中的 to do here,其实这是简化的形容词从句(形容词从句的简化
已经在前面章节介绍过),原来是that we can do here,修饰先行词 nothing。然后再
看看副词从句的动词have,这是普通动词,没有 be动词可省略,也没有情态助动词可供
改写。这个动词若不处理掉,句子将无法简化。所以必须加上be 动词,原来的动
词 have 就得变成 having: Because we are having nothing to do here, we might
as well go home.。请注意:这种修改不是为了要改成进行式(这个句子并不适合采用
进行式),而是为了做\textbf{词类变化}:把having nothing to do here 移入补语部
分,we are 便得以省略,成为:
\begin{itemize}
\item \unct{Having nothing to do here}{简化副词从句}, we might as well go home.
\end{itemize}

\subsection{应该省略的连接词}

\textbf{在做这种简化动作时,表示原因的连接词 because、since等等通常要省略},若保留
下来会显得相当刺眼。因为这种句型本身就强烈暗示因果关系,再加上because 会十分
累赘。

\subsection{应该保留的连接词}

反之,如果连接词省略会造成句意不清,就得保留,例如:
\begin{itemize}
\item \unct{Although we have nothing to do here}{副词从句}, \unct{we}{S} \unct{can't leave}{V} early.

  虽然这儿没事,我们还是不能提早离开。
\end{itemize}
副词从句的主语 we 与主要从句的主语相同,可以省略。动词 have是普通动词,可以改
成 having 保留下来,成为:
\begin{itemize}
\item \unct{Although having nothing to do here}{简化副词从句}, we can't leave early.
\end{itemize}
本来没事应该可以离开,但是却相反。这种“相反”的逻辑关系要靠连接词although 来
表示,所以 although不宜省略,不然会让读者搞不清楚:是因为没事才不能早走吗?

语法上 although这个连接词已无必要,只是为了表达逻辑关系而保留。如果省略它,用
别的方式来表示逻辑关系也未尝不可,例如:
\begin{itemize}
\item Having nothing to do here, we \unbf{still} can't leave early.
\end{itemize}
在主要从句中加个副词 still 就可取代 although来表达“相反”的逻辑,although 省
略也不会造成语意不清。再看下例:
\begin{itemize}
\item \unct{He}{S} \unct{raised}{V} \unct{his hand}{O}, \unct{as if he was
    trying to hit her}{副词从句}.

  他举起手来,好像要打她。
\end{itemize}
副词从句的 he was 省略之后,就简化为:
\begin{enumerate}
\item He raised his hand, \unct{as if trying to hit her}{简化副词从句}.
\end{enumerate}

例 1 的连接词 as if 不宜省略,不然会产生误解:
\begin{enumerate}[resume]
\item He raised his hand, \unct{trying to hit her}{简化副词从句}.
\end{enumerate}
例 2 中省略连接词 as if,意思就成为:他举起手来,“因为”要打她。读者看不到连
接词,往往会联想最常见的because,因而就产生误解。这时就不应省略连接词。

\subsection{being 的运用}

副词从句的 be 动词一般在简化时要省略,但有些状况下要以 being
的方式留下来,以下举几个例子说明:
\begin{itemize}
\item \unct{As I am a student}{副词从句}, \unct{I}{S} \unct{can't afford}{V}
  \unct{to get married}{O}.

因为现在我还是学生,所以结不起婚。
\end{itemize}

这个句子有几种简化方式。如果把副词从句中的 I am省略,剩下的补语是名词类的 a
student。假如连接词 as 再省略,只剩下 a student就省略得太过头了,读者无从判断
这是个简化的副词从句(因形状差太多),反而可能误会a student 是主语,或者是同
位语。为了避免误会,一个办法是保留连接词:
\begin{itemize}
\item \unct{As a student}{简化副词从句}, I can't afford to get married.
\end{itemize}
只要有连接词,读者可以清楚看出是简化从句,a student 是省略 I am以后留下的补语,
整个句意就很清楚。另一个办法是省略连接词as,借用无意义的 be 动词改成 being:
\begin{itemize}
\item \unct{Being a student}{简化副词从句}, I can't afford to get married.
\end{itemize}
being a student 因为有 being,所以 a student很明显是补语,意思是“身为学
生”或“是学生”。谁是?主语当然是和主要从句的主语I 相同:我是,这样句意也就
清楚了。

\subsection{兼作介词的连接词:before、after、since}

还有一种情况需要使用 being,情形稍微复杂一些,请看下面的例子:
\begin{itemize}
\item \unct{Before he was in school}{副词从句}, \unct{he}{S} \unct{used to be}{V} \unct{a naughty child}{O}.

  上学读书以前,他原本是个小顽童。
\end{itemize}
副词从句中有现成的 he was 可省略。如果省略,连接词 before也一并拿掉,就成为:
\begin{itemize}
\item In school, he used to be a naughty child.
\end{itemize}
这个句子本身没错,只不过和原句意思不同,成为:他从前在学校里很调皮。会产生句
意的出入,主要是因为表示时间关系的连接词before 被省略了。若把 before 保留呢?
\begin{itemize}
\item Before in school, he used to be a naughty child. (误)
\end{itemize}
保留 before 问题就更大了。因为 before这个字除了当连接词以外,也可以当介词
(例如 before 1977、before the war等等)。简化从句中如果留下before,因为已经
省去主语、动词,读者会判断这个 before是介词,不是连接词。那么 before 后面就
只能接名词类的东西。before in school这个组合因而成为一项语法错误。这是词类的
错误,修改方法是进行词类变化。若把in school 改成名词类,例如去掉 in,就可以放
在 before 之后,成为 before school。如此一来,语法问题是解决了,但是意思稍嫌
不清楚。因为 before school看起来不像“开始上学读书以前”,反而像“早上开始上
课前”。另一个改法就是借用无字面意义的be 动词来作词类变化:
\begin{itemize}
\item \unct{Before being in school}{简化副词从句}, he used to be a naughty child.
\end{itemize}
一旦有 be 动词存在,后面就可以接补语 in school。而 be 动词本身釆用being(动名
词)的形状,放在介词 before的后面也符合词类的要求,这样才算解决了问题。

副词从句的连接词中,before、after、since是身兼连接词与介词的双重词类。简化
时要注意:它会被视为介词,故后面只能接名词类,必要时得加上being 来作词类变
化。

\subsection{时态的问题}

简化副词从句还得注意时态问题,例如:
\begin{itemize}
\item \unct{After he wrote the letter}{副词从句}, he put it to mail.

  他写好了信,就拿去邮寄。
\end{itemize}
这两个从句中的动词 wrote 与 put 都是过去简单式,两者的先后顺序是靠连接
词after 来区分。在副词从句简化时,有以下两个选择:
\begin{enumerate}
\item \unct{After writing the letter}{简化副词从句}, he put it to mail.
\end{enumerate}

简化的步骤仍是省去相同的主语 he,把普通动词改为 Ving。如果像例 1选择把连接
词 after留下来,就可以清楚分出先后顺序,是正确的简化从句。附带一提的
是, after在从句简化后即成为介词,后面要接名词。writing the letter是动名词
短语,可以符合词类要求。然而若把连接词 after一并省略就会出现问题:
\begin{itemize}
\item \ul{Writing the letter}, he put it to mail. (误)
\end{itemize}
因为 after 省略了,读者看到的印象会是:When he was writing the letter, he
put it to mail.(他正在写信的时候,拿去邮寄。)这就不合理了。所以如果要省
略 after,在时态上要做如下的处理:
\begin{enumerate}[resume]
\item \unct{Having written the letter}{简化副词从句}, he put it to mail.
\end{enumerate}
这是用完成式与简单式的对比来交代写信在先,邮寄在后。句子还原后就能看得更清楚:
\begin{itemize}
\item When he had written the letter, he put it to mail.
\end{itemize}
\textbf{若连接词是不能表达先后功能的 when,就得靠动词时态来表达。}had written(过去完
成式)在先,put(过去简单式)在后。以这句来说,副词从句的动词had
written 没有 be 动词,也没有情态助动词(had是时态助动词),简化方法就只有
加 \emph{-ing} 成为 having written。连接词 when属于可省略之列。例 2 即是简化结果,也
是正确的简化从句做法。

\subsection{Dangling Modifier的错误}

\textbf{副词从句的简化}有一个相当严格的要求:\textbf{主语只有在与主要从句相同时才可省略}。
如果忽略这一点就径行省略,会产生语法、修辞的错误。这项错误一不小心就会发生,
修辞学中甚至有一个特别的名称来称呼它:\textbf{Dangling Modifiers(悬荡修饰语)}。请
看下例:
\begin{itemize}
\item \unct{When the child was already sleeping soundly in bed}{副词从句}, \unct{her mother}{S} \unct{came}{V} to kiss her goodnight.

  小孩已经在床上睡得很熟了,这时她妈妈来亲她一下道晚安。
\end{itemize}
副词从句的主语是小孩(the child),主要从句的主语却是她妈妈(her mother)。如
果忽略这一点而径行简化,省去主语与 be动词,就会得出这个结果:
\begin{itemize}
\item \unnormal{Already sleeping soundly in bed}{Dangling Modifier}, her mother
  came to kiss her goodnight. (误)
\end{itemize}
看到 already sleeping soundly in bed这个简化从句时,知道有个人在床上熟睡,可
是主语省略了,不知是谁在睡,这时候读者只能假定就是主要从句的主语her mother,
这个句子就因而发生了沟通的错误。简化副词从句属于副词类,是一个修饰语,可是却
找不到依归,有如悬荡在半空中,所以这是个被称为Dangling Modifier的错误。碰到这
种问题,有两种常用的修改方式,\textbf{其一是从主要从句下手:改变主要从句的结构,让
  它的主语与副词从句的主语相同。}上例可修改如下:
\begin{itemize}
\item \unct{Already sleeping soundly in bed}{简化副词从句}, \unct{the child}{S}
  \unct{did not know}{V} it when her mother came to kiss her goodnight.

  小孩在床上熟睡着,并不知道妈妈来亲她道晚安。
\end{itemize}

主语相同时,简化副词从句就可尘埃落定,找到修饰的对象。另一种改法是从副词从句下手:保留不同的主语。

\subsection{所谓“独立短语”}

副词从句简化时,若主语与主要从句不同就不能省略。这时可以选择\textbf{保留主语,只省
略be动词和连接词。在主语后面保留现在分词或过去分词的补语。}上面的例子可以修改
如下:
\begin{itemize}
\item \unct{The child already sleeping soundly in bed}{简化副词从句}, \unct{her
    mother}{S} \unct{came}{V} to kiss her goodnight.
\end{itemize}
传统语法称这种保留主语的简化副词从句为\textbf{“独立短语}”。那是把 already
sleeping soundly in bed \textbf{视为形容词短语看待,修饰前面的名词} the
child。可是名词 the child就无法成功纳入主要从句来诠释。传统语法分析不够深入,
因此碰到困难就取个名称来搪塞,“独立短语”的名称就是这样来的——无法纳入主要
从句中,就叫它“独立”好了!

从简化从句的角度来看就能完整地了解。简化时以不妨碍清楚性为原则。一般的副词从
句要省去主语,是因为和主要从句主语重复,省略不会影响语意。可是主语不同时,一
旦省略就会造成语意不清。这时的选择就是不省略,把主语保留下来,如此而已。

\subsection{保留主语时的注意事项}

简化副词从句时,如果主语不同而需保留,有两点必须注意:第一,连接词要省略。简
化从句一般是省略主语、be动词与连接词(视情形决定是否省略)。如果主语要保留,
连接词又留下,就只是省掉一个be 动词,那么并没有达到简化的效果。

\begin{itemize}
\item   When the child already sleeping soundly in bed, her mother came to
  kiss her goodnight. (误)
\end{itemize}
这个句子看起来不像简化从句,反而像写错了,漏掉一个 be 动词。

简化副词从句若保留主语,第二件注意事项是:后面必须配合分词补语(现在分词或过
去分词)。因为只有如此,才可明显看出这是省略be 动词的简化从句。The child
sleeping soundly 清楚说明 the child是主语,sleeping soundly 是补语,省略 be动
词与连接词,形成简化的副词从句。传统语法把“独立短语”视为“分词构句”的变化,
就是因为保留主语和使用分词补语有必然的关联性。

\section{结语}

副词从句的简化有很多变化,大约可以分成五种不同的情况来探讨。本章先谈了一种情
形:Ving补语。其他情形留待下一章继续探讨。下面附上一篇练习,复习从属从句的简
化。有些题目是复习前两章关于形容词从句与名词从句简化的观念,有些题目则要等到
副词从句全部讲完才能完全清楚。读者不妨先做做看。遇到不会做的题目先别着急,等
到简化从句讲完时再来回顾,就不会有问题了。

\section{Test}

\subsection{练习一}

\paragraph{将下列各句中的副词从句(即画底线部分)改写为简化从句:}

\begin{enumerate}
\item \ul{While he was watching TV}, the boy heard a strange noise coming from
the kitchen.

\item \ul{Because she lives with her parents}, the girl can't stay out very
late.

\item \ul{If you have finished your work}, you can help me with mine.

\item \ul{As he is a law-enforcement officer}, he cannot drink on duty.


\item The actor has been in a state of excitement \ul{ever since he was
nominated for the Oscar}.

\item \ul{After he addressed the congregation}, the minister left in a hurry.

\item \ul{As it was rather warm}, we decided to go for a swim.

\item \ul{When the students have all left}, the teacher started looking over
their examination sheets.

\item I know all about corn farming \ul{because I grew up in a Southern farm}.

\item \ul{As the door remained shut}, the servant could not hear what was going
on inside.
\end{enumerate}

\subsection{练习二}

\paragraph{请选出最适当的答案填入空格内,以使句子完整。}

\begin{enumerate}
\item \ttu on the sofa, we began to watch television.
\begin{tasks}(2)
  \task Sat
  \task Seat
  \task Seated
  \task Set
\end{tasks}

\item Returning to the room, \ttu.
\begin{tasks}
  \task the book was lost
  \task I found the book missing
  \task missing was book
  \task the book was missing
\end{tasks}

\item The average age of the Lishan apples \ttu today is about fifty years.
\begin{tasks}(2)
  \task grow
  \task grown
  \task growing
  \task to grow
\end{tasks}

\item Underground money lenders make most of their income from interest \ttu on loans.
\begin{tasks}(2)
  \task earn
  \task earned
  \task to earn
  \task was earned
\end{tasks}

\item \ttu the driveway, the house appeared to be much smaller than it had
  seemed to us as children many years ago.
\begin{tasks}(2)
  \task Standing in
  \task Seen from
  \task Crossing
  \task Driving down
\end{tasks}

\item After finishing my degree, \ttu.
\begin{tasks}
  \task my education will be employed by the university
  \task employment will be given to me by the university
  \task the university will employ me
  \task I will be employed by the university
\end{tasks}

\item The man \ttu the paper is my father.
\begin{tasks}(2)
  \task reads
  \task reading
  \task is reading
  \task read
\end{tasks}

\item \ttu, he washed the cup and put it away.
\begin{tasks}
  \task Drinking the coffee
  \task Having drunk the coffee
  \task Having drank the coffee
  \task After drank the coffee
\end{tasks}

\item \ttu to the south of China, not far away from the coast of Mainland, Hainan Island has long played an important role in China's tourism.
\begin{tasks}(2)
  \task Its location
  \task Locating
  \task Is located
  \task Located
\end{tasks}

\item John Williams wrote his first novel \ttu.
\begin{tasks}
  \task while he worked a porter at a hotel in Paris
  \task while working as a porter at a hotel in Paris
  \task while worked as a porter at a hotel in Paris
  \task while he was worked as a porter a hotel in Paris
\end{tasks}

\end{enumerate}

\section{Answer}

\subsection{练习一答案}
\begin{enumerate}
\item \ul{(While) waiching TV}, the boy heard a strange noise coming from the
  kitchen.

\item \ul{Living with her parents}, the girl can't stay out very late.

\item \ul{If having finished your work},you can help me with mine.

\item \ul{Being a law-enforcement officer}, he cannot drink on duty.

\item The actor has been in a state of excitement \ul{ever since being
    nominated for the Oscar}.

\item \ul{After addressing the congregation}, the minister left in a hurry. 或

  \ul{Having addressed the congregation}, the minister left in a hurry.
\item \ul{It being rather warm}, we decided to go for a swim.

\item \ul{The students having all left}, the teacher started looking over their
  examination sheets.

\item I know all about corn farming, \ul{having grown up in a Southern farm}.

\item \ul{The door remaining shut}, the servant could not hear what was going on
  inside.

\end{enumerate}

\subsection{练习二答案}
\begin{enumerate}
\item(C) seat 是及物动词,本句是 we were seated on the sofa 的简化。

\item (B) Returning to the room 是简化从句,必须与主要从句同一主语。四个答案中只有 B 的主语是人,符合这个要求。

\item (C) 空格部分是 that are growing today 的简化,成为 growing today(今天在长
  的)。若是 B,应为 that are grown today(今天种下去的)这一句的简化,文法也
  通,但是如果今天才种下去,不可能“平均年龄 50 岁”,所以不行。

\item (B) 这是 that is earned on loans 这个形容词从句的简化。

\item (B) 四个答案都是副词从句简化,条件是要与主要从句同一主语(the house),所
  以只有 A 或 B。the driveway 是“车道”,房子不能站在它里面,所以排除掉 A,
  剩下 B(When it was seen from 的简化)。

\item (D) After finishing my degree 是 After I finish my degree 的简化,所以主要
  从句只能用 I 作主语,因而只有 D。

\item (B) 空格以下是 who is reading the paper 的简化。

\item  (B) A 看起来像是 when he was drinking the coffee 的简化,既然还在喝,不应洗杯子。所以选完成式的 B,表示“喝完之后”。

\item (D) 这是 Hainan Island is located to the south… 这一句的简化。

\item (B) 这是 while he was working as a porter… 的简化。A 中 while he worked
  a porter 当 worked 的宾语使用,是误把不及物的 work 当作及物动词使用。
\end{enumerate}

\chapter{副词从句简化之二}

简化从句是比较复杂的句型。因为它有精简、浓缩的特色,也是修辞效果相当好的句型。
其中又以副词从句的简化最为复杂。上一章探讨了副词从句简化为Ving 形式的变化,本
章继续探讨副词从句简化的其他变化。

\section{简化为 Ven}

从属从句简化的共同原则是省略主语与 be动词。\textbf{副词从句中如果原本是被动态
  (be+Ven),那么简化之后没有了 be动词,就会成为 Ven 的形态。}例如:

\begin{itemize}
\item \unct{After he was shot in the knee}{副词从句}, he couldn't fight.

  膝盖中枪后,他就不能作战了。
\end{itemize}
例句中副词从句的主语 he 与主要从句的主语相同,可以简化。省去主语与 be动词后,
不再需要连接词,成为:
\begin{itemize}
\item \unct{Shot in the knee}{简化副词从句}, he couldn't fight.
\end{itemize}

\subsection{是否保留连接词}

上例中连接词 after 可以不留,因为 shot是过去分词,本身就表示“已经中
枪”、“中枪之后”,已有完成式的暗示,因而不再需要after 一词。但下面的例子则
不同:
\begin{itemize}
\item \unct{Although he was shot in the knee}{副词从句}, he killed three more enemy soldiers.

  虽然膝盖中枪,他仍多杀了三名敌军。
\end{itemize}
句中连接词 although带有“相反”的暗示,省去后意思会有出入,应该予以保留:
\begin{itemize}
\item \unct{Although shot in the knee}{简化副词从句}, he killed three more enemy soldiers.
\end{itemize}
或者,如果省略 although的话,也必须用其他方式来表示句中的“相反”暗示,例如:
\begin{itemize}
\item Shot in the knee, he \unbf{still} killed three more enemy soldiers.
\end{itemize}

\subsection{三个特殊的连接词}

另外,连接词如果要留下来,要注意一点:before、after、since这三个连接词也可以
当介词用。如果其中任何一个出现在简化从句中,由于没有了主语、动词,这个连接
词就得当介词看待,亦即:后面要接名词。例如:
\begin{itemize}
\item \unct{Before it was redecorated}{副词从句}, the house was in bad shape.

  这栋房子重新装潢之前状况很糟。
\end{itemize}
副词从句简化之后,连接词 before 不能省略,否则意思会不同,成为:
\begin{itemize}
\item   Redecorated, the house was in bad shape.
\end{itemize}
因为过去分词 redecorated有完成的暗示,上面这句的意思是“重新装潢后,这栋房子
状况很糟。”若要维持原意,则连接词before 不能省略。但是,before是可以当介词
使用的连接词,留下来又会有问题:
\begin{itemize}
\item Before redecorated, the house was in bad shape. (误)
\end{itemize}
上句的错误在于 before 此时是介词,后面却只有形容词类的redecorated,造成语法
错误。修改的办法是改变 redecorated的词性。若要保留它的被动态,就不能作词尾的
词类变化,只能在前面加 being来作词类变化:
\begin{itemize}
\item \unct{Before being redecorated}{简化副词从句}, the house was in bad shape.
\end{itemize}

be 动词是没有内容的字眼。在此加上 being一词,纯粹是因应词类变化的需求:用动名
词词尾的 \emph{-ing} 来变成名词,以符合before 介词的要求。另外,以这个例子而言,
忽略 redecorated的被动态,改成名词 redecoration,意思仍不失清楚:
\begin{itemize}
\item Before redecoration, the house was in bad shape.
\end{itemize}

除了 before 以外,after 和 since这两个连接词如需保留,也都要注意词类的问题。

\subsection{如何应用 having been}

许多学习者对 having been 颇觉困扰。在此用一个例子来说明它的用法:
\begin{enumerate}
\item \unct{Because they had been warned}{副词从句}, they proceeded carefully.

  因为已经得到警告,他们就很小心地进行。
\end{enumerate}
简化这个句子里的副词从句时,主语 they 当然可以先省掉。动词 had been warned 有
两种处理方式。 be 动词固然没有内容,可以省略,但是 had been 是be 动词的完成式,
有“已经……”的意味。如果要保留下来,就得先把had been 改成分词类的 having
been,成为:
\begin{enumerate}[resume]
\item \unct{Having been warned}{简化副词从句}, they proceeded carefully.
\end{enumerate}

另外,如果忽略例 1 中 had been 的完成式内容,把整个 be
动词的完成式视同一般的 be 动词,随主语一起省略,就可以把例 1 简化为:
\begin{enumerate}[resume]
\item \unct{Warned}{简化副词从句}, they proceeded carefully.
\end{enumerate}
这个句子中,warned一字是过去分词,本身就具有完成的暗示(表示“已经”受到警
告),所以把 had been 省略并不影响句意。

如果把例 2 和例 3 两句比较一下,当可发现:having been后面如果跟的是过去分词,
那么即使把 having been省略,在语法上同样正确(因为例 2 的 having \ldots{} 和
例 3 的 warned同属分词,词类相同),而且在意思上也相同。因为例 2 的 having
been是表达“已经” 的意思,而例 3 里的 warned同样表达了“已经”的意思。所
以,having been后面如果跟的是过去分词,就可省略,不会有任何影响。

\subsection{主语不同时}

副词从句简化为Ven,如果主语和主要从句的主语不同,就得把主语留下来,不得省略。
例如:

\begin{itemize}
\item \unct{When the coffin had been interred}{副词从句}, the minister said a few comforting
  words. (棺材入土后,牧师说了几句安慰的话。)
\end{itemize}
副词从句的主语是棺材,和主要从句的主语牧师不同,不能省略,不然会出现下面的结果:
\begin{itemize}
\item (Having been) interred, the minister said a few comforting words. (误)
\end{itemize}
这个意思是“入土之后,牧师说了几句安慰的话。”也就是牧师入土了,在地下说话!正确的做法是:主语不同时要把主语留下,动词加以简化,并省去连接词,成为:
\begin{itemize}
\item \unct{The coffin (having been) interred}{简化副词从句}, the minister said a few comforting
  words.
\end{itemize}

\subsection{简化为 to V}

如果原来的副词从句中有情态助动词(can、should、must之类),带有不确定语气,简
化之后就会成为不定词。例如:
\begin{itemize}
\item He studied hard \unct{in order that he could get a scholarship}{副词从句}.

  他用功读书,为的是要拿奖学金。
\end{itemize}
副词从句的动词 could get并不表示拿到了奖学金,只是想要拿,带有不确定语气。这
时就可简化为不定词。从前提过,所有的情态助动词都可改写为be+to 的形状,意思不
会有太大的变化。所以助动词的简化,去除了 be动词就剩下 to,成为不定词:
\begin{itemize}
\item He studied hard \unct{in order to get a scholarship}{简化副词从句}.
\end{itemize}

再看一个例子:
\begin{itemize}
\item I'll only be too glad \unct{if I can help}{副词从句}.

  如果帮得上忙,我非常乐意。
\end{itemize}
副词从句中的动词 can help有助动词在,仍是不确定语气:还没开始帮忙。简化后成
为:
\begin{itemize}
\item   I'll only be too glad \unct{to help}{简化副词从句}.
\end{itemize}

副词从句中凡有助动词存在,简化的结果都是一样:连接词省略,主语如果相同亦省略,
助动词拿掉be 动词之后变成 to,所以就剩下 to V 的结果。

\section{单纯的 be动词时}

如果副词从句的动词是单纯的 be动词,后面可能是一般的名词、形容词类的补语。要简
化时,首先得注意主语要和主要从句的主语相同,然后才可以把连接词留下来,省去主
语和be 动词,留下补语。例如:

\subsection{介词短语}

\begin{itemize}
\item \unct{When you are under attack}{副词从句}, you must take cover immediately.

  受到攻击时,要立刻寻找掩护。
\end{itemize}
这个副词从句的动词是 be 动词,补语是介词短语 under attack。简化后成为:
\begin{itemize}
\item \unct{When under attack}{简化副词从句}, you must take cover immediately.
\end{itemize}

\subsection{形容词}

\begin{itemize}
\item \unct{While it is small in size}{副词从句}, the company is very competitive.

  这家公司规模虽小,但很有竞争力。
\end{itemize}
副词从句中的补语是形容词 small,简化方式相同:
\begin{itemize}
\item \unct{While small in size}{简化副词从句}, the company is very competitive.
\end{itemize}

\subsection{名词}

\begin{itemize}
\item \unct{Although he was a doctor by training}{副词从句}, Asimov became a writer.

  虽然接受的是做医生的训练,但阿西莫夫后来成了作家。
\end{itemize}
副词从句中的补语是名词 a doctor,简化后成为:
\begin{itemize}
\item \unct{Although a doctor by training}{简化副词从句}, Asimov became a writer.
\end{itemize}
观察以上三种情形,可以作一归纳:\textbf{副词从句}的连接词不同于名词从句或形容词从句,
是\textbf{有意义的连接词,简化时常要留下来。}一旦留下连接词,那么它是由副词从句简化而
成这一点就十分明显。所以,拿掉主语与be动词后,不论什么词类的补语——名词、形
容词、介词短语——都可以留下来。不过有两点需要注意:\textbf{如果连接词
是before 与 after之类,简化后成为介词,后面只能接名词类。另外,表示原因的连
接词 because与 since,简化后通常不能原样留下来,要改成 because of,as a
result of之类的介词。}做法请看下面说明。

\section{改为介词短语}

副词从句的连接词有表达某种逻辑关系的意义。简化时有一种特别的做法,就是把连接
词改为意义近似的介词,整个从句简化为名词后,作为介词的宾语。

\begin{itemize}
\item \unct{When she arrived at the party}{副词从句}, she found all the people gone.

  她到达舞会场地时,发现人都走光了。
\end{itemize}
与连接词 when 近似的介词有 on 和 upon。上面的句子可以改写为:
\begin{itemize}
\item \unct{Upon arriving at the party}{介词短语}, she found all the people gone.
\end{itemize}

因为介词后面只有一个宾语的空间,所以句型要大幅精简,所有重复、空洞的字眼都
要删去,有意义的部分则尽量保留下来。通常可以把动词改成动名词(加 \emph{-ing}),
如上例的方式处理。不过也可以这样修改:
\begin{itemize}
\item \unct{Upon her arrival at the party}{介词短语}, she found all the people gone.
\end{itemize}
动词 arrive 直接改成名词arrival,符合词类要求而意思不变。下面的例子就有些不
同:
\begin{itemize}
\item \unct{When she completed the project}{副词从句}, she was promoted.

  她完成了这项计划,就被提升了。
\end{itemize}
同样的,副词从句可以改写为介词加动名词。
\begin{itemize}
\item \unct{Upon completing the project}{介词短语}, she was promoted.
\end{itemize}

可是动词 complete 如改成名词 completion,就会有问题:
\begin{itemize}
\item \ul{Upon completion the project}, she was promoted. (误)
\end{itemize}
错误在于 complete 的后面有宾语 the project。一旦变成名词的completion,原来的
宾语就无所归依,所以要再加介词 of 来处理:

\begin{itemize}
\item \unct{Upon completion of the project}{介词短语}, she was promoted.
\end{itemize}

再看一个例子:
\begin{itemize}
\item The construction work was delayed \unct{because it had been raining}{副词
    从句}.

  因为一直下雨,建筑工程就耽搁了。
\end{itemize}
上例中副词从句的连接词可以改为介词 because of,成为:
\begin{itemize}
\item The construction work was delayed \unct{because of rain}{介词短语}.
\end{itemize}
\textbf{副词从句中的虚主语 it,以及动词 had been 都可以省略},有意义的只有 rain一词要
留下来。再看这个例子:
\begin{itemize}
\item \unct{Althought he opposed it}{副词从句}, the plan was carried out.

  虽然他反对,这个计划还是施行了。
\end{itemize}
例句中连接词 although 和介词 despite 或 in spite of意思接近,可以改为:
\begin{itemize}
\item \unct{Despite his opposition}{介词短语}, the plan was carried out.
\end{itemize}
副词从句中的宾语 it,其内容与主要从句重复,是多余的字眼。Although
改为介词 Despite 后,只能接一个宾语,里面要放下 he opposed
这个部分的概念,于是将词类变化为 his opposition。再看下例:
\begin{itemize}
\item \unct{If there should be a fire}{副词从句}, the sprinkler will be started.

  万一失火,洒水器会开动。
\end{itemize}
例句中的连接词 if 和介词 in case of 近似。改写后,副词从句中的 there
should be 这几个没有内容的词都要省略,只要把有意义的 fire
一词放进去就好:
\begin{itemize}
\item \unct{In case of a fire}{介词短语}, the sprinkler will be started.
\end{itemize}

副词从句改写为介词短语,是大幅度的简化。许多连接词都找得到近似的介词。然
而,改过之后,只剩下一个名词的空间来装下整个从句的内容,所以要大量精简。装不
下时就不要这样简化,或者另辟蹊径。例如:
\begin{itemize}
\item \unct{Because}{副词连接词} \unct{the exam}{S} \unct{is}{V} \unct{only a
    week away}{C}, I have no time to waste.

  因为离考试只剩一个星期了,我不能再浪费时间。
\end{itemize}
这个副词从句的主语 the exam 和主要从句主语 I不同,不易简化,需改成介词短
语:
\begin{itemize}
\item \unct{With}{介词} \unct{the exam}{O} \unct{only a week away}{C}, I have
  no time to waste.
\end{itemize}
连接词 because 改成介词 with。原来的主语 the exam 作它的宾语。be动词省略后,
主语补语 only a week away 就成了宾语补语,完成了简化的工作。

\section{结语}

简化从句这个较庞大的概念,至此可告一段落。这个非常重要的观念,对于认识与写作
复杂的句型有极大的帮助。为了消化这个观念,本书拟在下一章采
用sentence-combining的方式,与读者共同将若干个单句组合成复合句,再进一步简化
到只剩一个完整的从句。这样做一方面可以复习语法句型观念,一方面也是英语写作的
最佳练习。读者亲自练习一下,应当会有更深一层的体验。

\section{Test}

\subsection{练习一}

\paragraph{将下列各句中的副词从句(即画底线部分)改写为简化从句:}

\begin{enumerate}
\item \ul{After he was told to report to his supervisor}, the clerk left in a
hurry.

\item \ul{Although he was ordered to leave}, the soldier did not move an inch.

\item The plan must be modified \ul{before it is put into effect}.

\item \ul{Because it had been bombed twice in the previous week}, the village
was a total wreck.

\item \ul{When all things are considered}, I cannot truly say that this was an
accident.

\item \ul{When the job was done}, the secretary went home.

\item He took on two extra jobs \ul{so that he could feed his family}.

\item \ul{If you are in doubt}, you should look up the word in the dictionary.

\item \ul{Because pork is so expensive}, I'm buying beef instead.

\item \ul{ When we consider his handicap}, he has done very well indeed.
\end{enumerate}

\subsection{练习二}

\paragraph{请选出最适当的答案填入空格内,以使句子完整。}

\begin{enumerate}
\item \ttu not a big star, the actor played in hundreds of films.
\begin{tasks}(2)
  \task Although
  \task He was
  \task Because
  \task Despite
\end{tasks}

\item Eisenhauer was president of Columbia University \ttu President of the USA.
\begin{tasks}(2)
  \task before he becomes
  \task before becoming
  \task before
  \task before became
\end{tasks}

\item Gold remains stable even \ttu to extremely high temperatures.
\begin{tasks}(2)
  \task when is heated
  \task it is heated
  \task when to heat
  \task when heated
\end{tasks}

\item \ttu, the stock market crashed.
\begin{tasks}(2)
  \task With investor confidence gone
  \task When investor confidence gone
  \task When investors lose confidence
  \task With investors lost con fidence
\end{tasks}

\item A monkey's brain is small \ttu with the human brain.
\begin{tasks}(2)
  \task when they are compared
  \task when compare
  \task compared
  \task to compare them
\end{tasks}

\item Picasso did many of his abstract paintings \ttu living in Paris.
\begin{tasks}(2)
  \task that he was
  \task during
  \task while
  \task and
\end{tasks}

\item \ttu at correct angles, diamonds reflect light brilliantly.
\begin{tasks}(2)
  \task When carved
  \task If it is carved
  \task Carving
  \task If carving
\end{tasks}

\item \ttu, the children gradually learned to be independent.
\begin{tasks}(2)
  \task Because their father gone
  \task Their father was gone
  \task Due to their father was gone
  \task With their father gone
\end{tasks}

\item She broke into tears \ttu the news.
\begin{tasks}(2)
  \task upon hearing
  \task because hearing
  \task when heard
  \task when she hears
\end{tasks}

\item \ttu the truth, I know nothing about it.
\begin{tasks}(2)
  \task To tell you
  \task Telling you
  \task I tell you
  \task I told you
\end{tasks}

\end{enumerate}

\section{Answer}

\subsection{练习一答案}
\begin{enumerate}

\item \ul{(Having been) told to report to his supervisor}, the clerk left in a
  hurry.

\item \ul{Although ordered to leave}, the soldier did not move an inch.

\item The plan must be modified \ul{before being put into effect}.

\item \ul{(Having been) bombed twice in the previous week}, the village was a
  total wreck.

\item \ul{All things considered}, I cannot truly say that this was an accident.
\item \ul{The job done}, the secretary went home.

\item He took on two extra jobs \ul{(so as) to feed his family}.

\item \ul{If in doubt}, you should look up the word in the dictionary.
\item \ul{With pork so expensive}, I'm buying beef instead. 或

  \ul{Pork being so expensive}, I'm buying beef instead.

\item \ul{Considering his handicap}, he has done very well indeed.

\end{enumerate}

\subsection{练习二答案}
\begin{enumerate}
\item A) 副词从句 Although he was not a big star 的简化。

\item(B) 副词从句 before he became President of the USA 的简化。

\item(D) 副词从句 even when it is heated… 的简化。

\item (A) 副词从句 Because investor confidence was gone 简化成介词短语。

\item (C) 副词从句 when it is compared… 的简化。

\item (C) 副词从句 while he was living… 的简化。

\item (A) 副词从句 When they are carved… 的简化。

\item (D) 副词从句 Because their father was gone 简化为介词短语。
\item (A) 副词从句 as soon as she heard the news 简化为介词短语。
\item (A) 副词从句 If I can tell you the truth 的简化。
\end{enumerate}

\chapter{简化从句练习}

简化从句,亦即一般短语书所谓的非限定从句(Non-finite Clauses),是高度精简的
句型,也是较具挑战性的句型,在 TIME中俯拾皆是。本书一连几章介绍这个比较庞大的
概念,现在已到了验收的时候。这一章就用sentence combining 的形态来练习如何精简
复杂的句子。

首先回顾一下简化从句的两大原则:

\textbf{一、对等从句中,相对应位置(主语与主语,动词与动词等)如果重复,择一弹性省略。}

\textbf{二、从属从句(名词从句、形容词从句与副词从句)中,省略主语与 be动词两部分,
  留下补语。不过主语若非重复或空洞的元素,就应设法保留,以免句意改变。}

这两项原则的共同目的都是为了增强句子的精简性:尽量删除两个从句间重复或空洞的
元素,但以不伤害清楚性为前提。现在就借一些例句的组合来练习如何写作高难度的句
型。

\paragraph{例一}

\begin{enumerate}
\item The patient had not responded to the standard treatment.

  病人对标准疗法没有反应。
\item This fact greatly puzzled the medical team.

  医疗小组对此深感不解。
\end{enumerate}

这两个简单句中,句 2 的主语 this fact 指的就是整个句 1叙述的那件事。两句经由
这个交叉建立了关系,可以考虑用关系从句(即形容词从句)连结起来。亦即把句2 的
交叉点 this fact 改写为关系词,附于句 1 上作关系从句,成为:
\begin{itemize}
\item The patient had not responded to the standard treatment, which greatly
  puzzled the medical team. (不够清楚)
\end{itemize}

如此组合这两句话,短语上看来可以,但修辞上有严重的缺点:关系词 which固然可以
代表逗点前的整句话(表示病人缺乏反应这一点令人困惑),但是它也可以代表逗号前
面的名词the standard treatment(表示标准治疗方式本身令人困惑)。如此一来,一
个句子有两种可能的解释,犯了模棱两可(ambiguous)的毛病,也就是没有把意义表达
清楚,不如尝试另一种组合方式。

既然整个句 1 是句 2 主语 this fact的内容,不妨把它改成名词从句(前面加上连接
词 that 即可),然后直接置于句2 中 this fact 的位置当主语使用,成为复句:
\begin{itemize}
\item \unbf{That the patient had not responded to the standard treatment} greatly
  puzzled the medical team.
\end{itemize}
这个句子中的名词从句(that引导的从句)可再进一步简化,一般做法是删除主语
与 be动词。但这个从句中主语是 the patient,在主要从句中并无重复,无法省略。动
词 had not responded其中也没有 be 动词可以省略,那么该怎么做?首先,动词简化
的通用原则是:

\textbf{一、有 be 动词即省略 be 动词;}

\textbf{二、有情态助动词(can、must、should 等)则改为不定词(to V);}

\textbf{三、除此之外的动词一律加上 \emph{-ing} 保留下来。}

以 had not responded 这个动词短语而言,符合第三种情形,所以改写为 not having
responded,以取代原先的名词从句。原来的主语 the patient改为所有格(the
patient's)置于前面,再删除无意义的连接词 that即完成了简化的动作,成为:
\begin{itemize}
\item \unbf{The patient's not having responded to the standard treatment}
  greatly puzzled the medical team.
\end{itemize}

另外,也可以直接进行词类变化,把动词改写为名词后,成为:
\begin{itemize}
\item \unbf{The patient's failure to respond to the standard treatment} greatly
  puzzled the medical team.
\end{itemize}

这种讲法读起来会比上一种讲法更自然一点。

\paragraph{例二}

\begin{enumerate}
\item The summer tourists are all gone.

  夏季的观光客都走光了。
\item The resort town has resumed its air of tranquillity.

  这个度假小镇又恢复了平静。
\end{enumerate}

这两句话之间没有重复的元素,但有逻辑关系存在:在观光客走了之后,或是因为观光
客都走了,小镇才得以恢复平静。这时可以用副词从句的方式,选择恰当的连接词
(after、because、now that 等)附在句 1 前面,再把句 1 与句 2 并列即可:
\begin{itemize}
\item \unbf{Now that the summer tourists are all gone}, the resort town has
  resumed its air of tranquillity.
\end{itemize}
Now that 引导的副词从句若要进一步简化,关键在主语、动词两个部分。主语the
summer tourists与主要从句并无重复,必须保留下来以免损害句意。动词部分有 be动
词(are),后面还有补语(gone)。这时若去掉 be动词,留下主语与补语,就破坏了
这个副词从句的结构,可以省略连接词 now that,成为:
\begin{itemize}
\item \unbf{(With) the summer tourists all gone}, the resort town has resumed its air of tranquillity.
\end{itemize}
如果最前面没有加上 with,而是以 the summer tourists all gone直接代表一个简化
的副词从句,这种讲法比较文诌诌,不够口语化。

较口语化的做法是,用介词 with 来取代连接词 now that 的意义,而把 the
tourists 放在 with 后面作它的宾语, all gone仍然作补语,即成为上句中多一
个 with 在前面的句型。

\paragraph{例三}

\begin{enumerate}
\item Confucius must have written on pieces of bamboo.

  孔子当年一定是在竹简上写字。
\item Confucius lived in the Eastern Zhou Dynasty.

  孔子是东周时代的人。
\item Paper was not available until the Eastern Han Dynasty.

  纸到东汉时期才有。
\end{enumerate}

这三句话中,句 1 和句 2有一个交叉:Confucius。经由这个交叉点建立关系,可用关
系从句的方式连结,将句2 的 Confucius 改写为关系词 who,成为:
\begin{itemize}
\item (1+2) Confucius, \unbf{who lived} in the Eastern Zhou Dynasty, must have
  written on pieces of bamboo.
\end{itemize}
这个关系从句(who lived in the Eastern Zhou Dynasty)可以进行简化,省略重复的
主语 who,再把普通动词 lived 改写为living,即成为简化形容词从句:
\begin{itemize}
\item Confucius, \unbf{living} in the Eastern Zhou Dynasty, must have written on
  pieces of bamboo.
\end{itemize}
东周时代的孔子为什么要用竹简写字?是因为句 3:纸到东汉时期才有。句 3的内容表
示原因,所以用副词从句的方式——外加连接词 because成为副词从句,与主要从句并
列,即得到:
\begin{itemize}
\item (+3) Confucius, living in the Eastern Zhou Dynasty, must have written on
  pieces of bamboo, \unbf{because paper was not available until} the Eastern Han Dynasty.
\end{itemize}

句中的副词从句(because之后的部分)如要进一步简化,又要观察主语与动词部分。主
语 paper没有重复,必须留下来。动词虽然是 be动词,可是\textbf{副词从句的简化中,一旦留
下主语,就得有个分词配合(传统语法称为分词构句)},所以使用be 动词来制造分
词 being,并省略连接词 because,即成为简化的副词从句:

\begin{itemize}
\item Confucius, living in the Eastern Zhou Dynasty, must have written on
  pieces of bamboo, \unbf{paper not being available} until the Eastern Han Dynasty.
\end{itemize}

\paragraph{例四}

\begin{enumerate}
\item The movable-type press was invented by Gutenberg.

  古登堡发明活版印刷。
\item The movable-type press was introduced to England in 1485.

  活版印刷在 1485 年引进英国。
\item This event marked the end of the Dark Ages there.

  这件事标示英国黑暗时期的结束。
\end{enumerate}

这个例子中的句 1 和句 2 也有一个交叉:the movable-type press,可以将它改写为
关系词 which,以关系从句方式连接:
\begin{itemize}
\item (1+2) The movable-type press, \unbf{which was} invented by Gutenberg, was
  introduced to England in 1485.
\end{itemize}
这个关系从句(which 引导的部分)可以直接简化,省略主语 which 和 be 动词
was,只保留补语 invented 这个部分,即成为简化的形容词从句:
\begin{itemize}
\item The movable-type press, \unbf{invented} by Gutenberg, was introduced to England
  in 1485.
\end{itemize}
句 3 中的主语 this event(这个事件)指的就是上面整句话的那个事件。这时候因为
上面的句子比较长,可以先加个同位语an event,再用它和句 3 主语 the event 的交
叉构成关系从句,成为:
\begin{itemize}
\item (+3) The movable-type press, invented by Gutenberg, was introduced to
  England in 1485, \unbf{an event which marked} the end of the Dark Ages there.
\end{itemize}
要进一步简化这个句子,可以把重复部分 an event 删除,再省略关系从句的主
语which,把动词 marked 改成分词 marking:
\begin{itemize}
\item   The movable-type press, invented by Gutenberg, was introduced to
  England in 1485, \unbf{marking} the end of the Dark Ages there.
\end{itemize}

\paragraph{例五}

\begin{enumerate}
\item Ben Kook was educated in an art college.

  本·库克曾在一所美术学院念书。
\item Ben Kook acts unusual at times.

  本·库克有时表现得与众不同。
\item Ben Kook deals with economic matters at these times.

  这时本·库克处理经济事务。
\end{enumerate}

句 1 和句 2之间有因果关系:因为在艺术学院读过书,所以才有与众不同的表现。那么
就在句1 前面加上连接词 because 成为副词从句,与句 2 的主要从句并列,成为:
\begin{itemize}
\item (1+2) \unbf{Because he was educated in an art college}. Ben Kook acts unusual at
  times.
\end{itemize}
这个句子中,简化 because 引导的副词从句,可以直接省略 he was,再把连接
词because 删去,只保留补语 educated 部分,成为:
\begin{itemize}
\item \unbf{Educated} in an art college, Ben Kook acts unusual at times.
\end{itemize}
这个句子要与句 3 连结,可以观察到句尾的 at times 就是句 3 结尾部分的 at
these times。以这个交叉改写为关系词 when,构成关系从句的形态:
\begin{itemize}
\item (+3) Educated in an art college, Ben Kook acts unusual \unbf{(at times)
    when he deals} with economic matters.
\end{itemize}
句中括弧部分的 at times 是副词类,属于次要元素,又与后面的 when重复,可以先行
省略。进一步的简化做法仍是一样:把主语 he 省略,动词 deals改成 dealing。不过,
由于原先的 at times 已经省略,所以与它重复的 when不宜省略。把 when 留下来,即
成为:

\begin{itemize}
\item Educated in an art college, Ben Kook acts unusual \unbf{when dealing} with
  economic matters.
\end{itemize}

\paragraph{例六}

\begin{enumerate}
\item   I'd like something.

  我希望一件事。
\item   You will meet some people.

  你去见见一些人。
\item   Then you can leave.

  然后你就可以走了。
\end{enumerate}

句 1 中的宾语 something 就是整个句 2 叙述的那件事,所以在句 2前面加上一个连接
词 that,成为名词从句,然后放入句 1 中 something的位置作为 like 的宾语:
\begin{itemize}
\item (1+2) I'd like \unbf{that you (will) meet} some people.
\end{itemize}
附带提一下,1+2 合并时,that 从句的语气成为祈使句的语气,所以助动词 will应省
略成原形动词,但简化时仍变成不定词。以下的例子若看到助动词上加个括弧都是同样
的原因。这里的名词从句要简化时,因主语you 与主要从句并无重复,所以要留下来,
把它放在 like后面的宾语位置。简化从句的做法是把助动词简化为不定词 to V,因为
情态助动词 must、should、will(would)、can(could)、 may(might)等都可以改写
成 be+to 的形式。省略 be 动词后就剩下to,所以上面这个从句中的 will meet 就改
成 to meet 当补语用,成为:
\begin{itemize}
\item I'd like \unbf{you to meet} some people.
\end{itemize}
再把句 3 加上去。句 3 是表示时间,可以用连接词 before 把它改成副词从句:
\begin{itemize}
\item (+3) I'd like you to meet some people \unbf{before you (can) leave}.
\end{itemize}
这个副词从句若进一步简化,得把 before 留下才能表达“在……之前”的意思。
但 before这个连接词也可当介词用,一旦后面的从句简化了,它就成为介词,只能
接名词形态。因此把重复的主语you 省略后,原来的动词 leave 要改成动名
词 leaving 的形态,成为:
\begin{itemize}
\item I'd like you to meet some people \unbf{before leaving}.
\end{itemize}

\paragraph{例七}

\begin{enumerate}
\item I have not practiced very much.

  我练习得不多。
\item I should have practiced very much.

  我应该多练习。
\item I am worried about something.

  我担心一件事。
\item I might forget something.

  我可能忘记什么事。
\item What should I say during the speech contest?

  在演讲比赛中我该说些什么?
\end{enumerate}

句 1 和句 2 可以用比较级 as \ldots{} as 的连接词合成复句:
\begin{itemize}
\item (1+2) I have not practiced \unbf{as much as I should} (have practiced).
\end{itemize}
因为“练习不够”,才会造成句 3 “我很担心”的结果。表示这种因果关系,可以使
用 because的副词从句来连接:
\begin{itemize}
\item (+3) \unbf{Because I have not practiced} as much as I should, I am
  worried about something.
\end{itemize}
Because 引导的副词从句,简化时可把重复的主语 I 省略。动词部分 have not
practiced 因为没有 be 动词,也没有情态助动词,就只能加上 \emph{-ing},成
为 not having practiced,再把连接词 Because 删去,成为:
\begin{itemize}
\item \unbf{Not having practiced} as much as I should, I am worried about something.
\end{itemize}
这个句子中,“担心的事情” something,就是句 4的内容“我可能会忘记什么事”。
因为 something 是放在介词 about的后面,要连成复句的话可以先改成 about the
possibility,再把句 4加上连接词 that,形成名词从句,作为 possibility 的同位语,
成为:
\begin{itemize}
\item (+4) Not having practiced as much as I should, \unbf{I am worried (about the
  possibility)} that I might forget something.
\end{itemize}
这个句子中的介词短语 about the possibility 意思和下文的 that从句重复,可以
省略。但是如果要简化其后的 that 从句,就得把介词 about留下来,简化的结果才
有地方安置。that 从句的简化,省去重复的主语 I之后,动词 might forget 的简化一
般是改成不定词 to forget。可是现在要放在介词 about 后面,不能用不定词的形态,
只能改成forgetting:
\begin{itemize}
\item Not having practiced as much as I should, I am worried \unbf{about forgetting}
  something.
\end{itemize}
现在,这个句子中“担心会忘记的”那件 something,就是句 5的问题:“演讲比赛该
说什么?”只要将这个疑问句改成非疑问句,就是一个名词从句,可直接取代上句中
的something,作为 forget 的宾语:
\begin{itemize}
\item (+5) Not having practiced as much as I should, I am worried about
  \unbf{forgetting what I should say} during the speech contest.
\end{itemize}
最后一步是简化 what 引导的名词从句。做法一样:省略主语 I,动词 should
say 改为不定词 to say:
\begin{itemize}
\item Not having practiced as much as I should, I am worried about forgetting
  \unbf{what to say} during the speech contest.
\end{itemize}

\paragraph{例八}

\begin{enumerate}
\item A. Fries was the leader of the College football team then.

  A.弗赖斯当时是学院足球队队长。
\item A. Fries is the director of a football club now.

  A.弗赖斯现在是一家足球俱乐部的主管。
\item A. Fries saw something.

  A.弗赖斯当时见到一件事。
\item The College football team lost in the important game.

  学院足球队在重要的球赛中失利。
\item A. Fries offered something.

  A.弗赖斯提议做一件事。
\item He would assume responsibility.

  弗赖斯愿意负责。
\item He would tender his resignation.

  弗赖斯将提出辞呈。
\end{enumerate}

这里一共有七个句子,要合并在一起,而且其中六个都得简化,只许留下一个完整的从
句。这可能是个不小的挑战,请读者仔细观察如何逐步完成整个动作。

首先,句 1 和句 2 分别叙述A.弗赖斯当时与现在的身份。这两句在内容与句型上对仗
工整,适合以对等从句方式表现,故加上对等连接词and 来连接:
\begin{itemize}
\item (1+2) A. Fries was the leader of the College football team then and
  \unbf{he is the director} of a football club now.
\end{itemize}
对等从句的简化方法是:两从句间相对应位置如有重复,则省略一个。因此把 and
右边那个从句重复的 he is 去掉,成为:
\begin{itemize}
\item (A) A. Fries was the leader of the College football team then \unbf{and
    the director} of a football club now.
\end{itemize}
这个描述弗赖斯身份的句子,我们称作句 A,先放着备用。下一步来组合 3 和 4两句。
句 3 中“弗赖斯见到”的 something 就是整个句 4的内容:“学院足球队比赛失利”。
所以把句 4 冠上连接词 that成为名词从句,置于句 3 中取代 something,作为 saw
的宾语:
\begin{itemize}
\item (3+4) A. Fries saw that \unbf{the College football team lost} in the important
  game.
\end{itemize}
that 引导的这个名词从句可以如此简化:主语 the Callege football team改为所有格
留下,动词 lost 直接改为名词的 lost,成为:
\begin{itemize}
\item (B) A. Fries saw \unbf{the College football team's loss} in the important
  game.
\end{itemize}
“弗赖斯眼见学院足球队失利。”这句话我们称作句 B,也先放着暂时不用。

接下来组合 5 和 6 两句。句 5 “弗赖斯提出”的 something,就是句 6的“他要负起
责任”。所以如法炮制把句 6 改成名词从句置入句 5 来取代something,成为:
\begin{itemize}
\item (5+6) A. Fries offered \unbf{that he (would) assume} responsibility.
\end{itemize}
这个句子可再将助动词简化为不定词 to V 的简化从句 he be to assume,而 be
动词可再省略成为:
\begin{itemize}
\item A. Fries offered \unbf{to assume} responsibility.
\end{itemize}

现在就用这个句子来把前面整理的结果堆砌上去。先把句 A 拿出来。句 A内容是描述弗
赖斯的职位,有补充形容A.弗赖斯身份的功能,所以拿它来做关系从句,将 A. Fries
改为关系词who,附于上句的主语 A. Fries 之后,成为:
\begin{itemize}
\item (+A) A. Fries, \unbf{who was the leader of the College football team then and
  the director of a football club now}, offered to assume responsibility.
\end{itemize}
句中这个 who 引导的关系从句可以简化,省略主语 who 和 be 动词was,留下名词类补
语(一般所谓的同位语),成为:
\begin{itemize}
\item A. Fries, \unbf{the leader of the College football team then and the
    director of a football club now}, offered to assume responsibility.
\end{itemize}
“当时的学院足球队队长,现今一家足球俱乐部的主管弗赖斯,表示要负责。”为什么?
因为句B:“他目睹学院足球队比赛失利。”现在把句 B拿出来用,它和上句的关系是因
果关系,所以加上连接词because,做成副词从句与上句并列:
\begin{itemize}
\item (+B) \unbf{Because he saw} the College football team's loss in the important
  game, A. Fries, the leader of the College football team then and the
  director of a football club now, offered to assume responsibility.
\end{itemize}
句子越来越长了,现在来简化一下。上句中 because 引导的副词从句,主语 he和主要
从句的 A. Fries 重复,可以省略。动词 saw 因无 be动词与助动词,可直接改
成 seeing,再把多余的 because 去掉,成为:
\begin{itemize}
\item \unbf{Seeing} the College football team's loss in the important game, A. Fries,
  the leader of the College football team then and the director of a
  football club now, offered to assume responsibility.
\end{itemize}

别忘了,一直未动用到句7:“弗赖斯打算提出辞呈。”从内容来看,它是说明上句
中“负责”(assume responsibility)的方式。也就是句 7 应拿来修饰上句中的原形
动词 assume一词。\textbf{“以……方式”的最佳表达是用 by 的介词短语},所以把
句7(He would tender his resignation.)直接放入 by 的后面,不过,by是介词,
后面只能接受名词短语,所以要将句 7简化为名词短语的形态。省略主语 he,动
词 would tender因为要放在介词后面,只能改成动名词 tendering,成为:
\begin{itemize}
\item (+7) Seeing the College football team's loss in the important game, A.
  Fries, the leader of the College football team then and the director of a
  football club now, offered to assume responsibility \unbf{by tendering his
    resignation}.

  眼见学院足球队在重大的比赛中失利,当时的学院足球队队长,也现在一家足球俱乐
  部的主管弗赖斯,表示要提出辞呈以示负责。
\end{itemize}
终于大功告成。读者经过这一番演练,当可了解上面这个句子实际上隐含多达七句话。
然而经过简化的过程,删掉了一切多余的元素,最后的结果并不显得太长或太复杂,这
就是简化从句的功效。

如开场白中所述,简化从句是高难度句型,颇富挑战性。读者若看到这里都能大致了解,
那么句型观念可说已相当完整,欠缺的只是大量的阅读功夫,那要靠日积月累的培养。
有清晰的句型观念,再加上大量的阅读,日后自然能写出一手好文章。

下面再附上一篇练习,请读者先自行尝试组合、简化其中的句子,再比对附在后面的参
考——只是参考,因为简化从句没有一定的做法,也没有标准答案。在告别句型之前,
还有一个问题要处理:倒装句。下一章我们就来研究这个也很重要的问题。

\section{Test}

\paragraph{将下列各题中的句子写在一起成为复句或合句,然后再简化到最精简的地步:}

\begin{enumerate}
\item Ben Book was educated in an art college. (because)

  Ben Book acts unusual.

  Ben Book deals with economic matters. (while)

\item I'd like something.

  You will meet some people. (that)


\item I'm not sure.

  What should I do?


\item He worked late into the night.

  He was trying to finish the report. (because)

\item The soldier was wounded in the war. (after)

  He was sent home.

\item He used to smoke a lot.

  He got married. (before)

\item I am afraid.

  The Democratic Party might win a majority. (that)

\item I have nothing better to do. (when)

  I enjoy something.

  I play poker. (that)

\item Mike won the contest. (when)

  Mike was awarded ten thousand dollars.

\item The motorcyclist was pulled over by the police car.

The motorcyclist did not wear a safety helmet. (who)

\item The mayor declined.

 The mayor was a very busy person. (who)

 The mayor was asked to give a speech at the opening ceremony. (when)

\item Tax rates are already very high. (although)

  Tax rates might be raised further to rein in inflation.

\item The resort town is crowded.

  There has been an influx of tourists for the holiday season. (because)

\item The student had failed in two tests. (though)

  The student was able to pass the course.

\item The president avoided the issue. (that)

  This was obvious to the audience.

\item Anyone could tell he was upset.

  He had the look on his face. (because)

\item Michael Crichton is in town.

 He is author of Jurassic Park. (who)

 He could promote his new novel. (so that)

\item I am a conservative. (although)

 I'd like to see something.

 The conservative party is chastised in the next election. (that)

\item The man found a fly in his soup. (when)

  The man called to the waiter.

\item It is a warm day. (because)

 We will go to the beach.

\end{enumerate}

\section{Answer}
\begin{enumerate}
\item Because he was educated in an art college. Ben Book acts unusual while he
  deals with economic matters. 简化为:

  Educated in an art college, Ben Book acts unusual while dealing with
  economic matters.

\item I'd like that you will meet some people. 简化为:

  I'd like you to meet some people.

\item I'm not sure what I should do. 简化为:

  I'm not sure what to do.

\item He worked late into the night because he was trying to finish the report.
  简化为:

  He worked late into the night trying to finish the report.

\item After the soldier was wounded in the war, he was sent home. 简化
  为:

  (After being) wounded in the war, the soldier was sent home.

\item He used to smoke a lot before he got married. 简化为:

  He used to smoke a lot before getting married.

\item I am afraid that the Democratic Party might win a majority. 简化为:

  I am afraid of the Democratic Party winning a majority.

\item When I have nothing better to do, I enjoy that I play poker. 简化为:

  When I have nothing better to do, I enjoy playing poker.
\item When Mike won the contest, he was awarded ten thousand dollars. 简化为:

  (Upon) winning the contest. Mike was awarded ten thousand dollars.

\item The motorcyclist who did not wear a safety helmet was pulled over by the
  police car. 简化为:

  The motorcyclist not wearing a safety helmet was pulled over by the police
  car.
\item The mayor, who was a very busy person, declined when he was asked to give
  a speech at the opening ceremony. 简化为:

  The mayor, a very busy person, declined when asked to give a speech at the
  opening ceremony.

\item The mayor, who was a very busy person, declined when he was asked to give
  a speech at the opening ceremony. 简化为:

  The mayor, a very busy person, declined when asked to give a speech at the
  opening ceremony.

\item Although tax rates are already very high, they might be raised further to
  rein in inflation. 简化为:

  Although already very high, tax rates might be raised further to rein in
  inflation.

\item The resort town is crowded because there has been an influx of tourists
  for the holiday season. 简化为:

  The resort town is crowded with an influx of tourists for the holiday
  season.

\item Though the student had failed in two tests, he was able to pass the
  course. 简化为:

  Though having failed in two tests, the student was able to pass the
  course.

\item That the president avoided the issue was obvious to the audience. 或

  It was obvious to the audience that the president avoided the issue. 简化
  为:

  The president's avoiding the issue was obvious to the audience. 或

  The president's avoidance of the issue was obvious to the audicnce.
\item Anyone could tell he was upset because he had the look on his face. 简化
  为:

  Anyone could tell he was upset, with the look on his face.

\item Michacl Crichton, who is author of Jurassic Park, is in town so that he
  could promote his new novel. 简化为:

  Michacl Crichton, author of Jurassic Park, is in town to promote his new
  novel.
\item Although I am a conservative, I'd like to see that the conservative party
  is chastised in the next election. 简化为:

  Although (being) a conservative, I'd like to see the conservative party
  chastised in the next election.
\item When the man found a fly in his soup, he called to the waiter. 简化
  为:

  Finding a fly in his soup, the man called to the waiter.
\item Because it is a warm day, we will go to the beach. 简化为:

  It being a warm day, we will go to the beach.
\end{enumerate}
\chapter{倒装句}

\textbf{倒装句是一种把动词(或助动词)移到主语前面的句型。}以这个定义来看,一般的疑
问句都可以算是倒装句。

撇开疑问句这种只具有语法功能的倒装句不谈,比较值得研究的是具有修辞功能的倒装
句。恰当地运用倒装句,可以强调语气、增强清楚性与简洁性,以及更流畅地衔接前后
的句子。以下分别就几种重要的倒装句来看看它倒装的条件,以及可达到的修辞效果。

\section{比较级的倒装}

在开始谈比较级的倒装前,有一些关于比较级的修辞问题应先弄清楚,请看这个例子:
\begin{enumerate}
\item Girls like cats more than boys. (不清楚)
\end{enumerate}
这个句子可能有两种意思:
\begin{enumerate}[resume]
\item Girls like cats more than boys do.

  女孩比男孩更喜欢猫。
\item Girls like cats more than they like boys.

  女孩比较喜欢猫,比较不喜欢男孩。
\end{enumerate}

比较级的句型通常会牵涉到两个从句互相比较。这两个从句间应有重复的部分才能比较。
一旦有重复,就有省略的空间。但是如果省略不当,就会伤害句子的清楚性。就像上面
的例1,可以作例 2 和例 3 两种不同的解释。修辞学上称这种句子为ambiguous(模棱
两可)。如果要表达例 2 的意思,那么句尾的 do就不能省略,否则读者有可能把它当
作例 3 来理解。

如果把例 2 修改一下,成为:
\begin{itemize}
\item Girls like cats more than boys, who as a rule are a cruel lot, \ul{do}. (不
  佳)
\end{itemize}
这个句子在 boys后面加上一个修饰它的关系从句。从刚才的分析中可了解到,句尾
的 do不能省略,否则读者无从判断 boys是主语还是宾语——是喜欢猫的人,还是被喜
欢的对象。

do 这个词既不能省略,把它留在句尾却又有修辞上的毛病。首先,do这个助动词和它的
主语 boys之间,因为关系从句的阻隔,距离太远,会伤害句子的清楚性。另外,助动
词 do所代表的是前面从句中的 like cats,但同样也因为距离太远而不够清楚。

要解决这个修辞上的问题,有个方法——倒装句。将 do 挪到主语 boys前面,成为:
\begin{itemize}
\item Girls like cats more than \unct{do boys}{倒装句}, who as a rule are a cruel lot.

  女孩比男孩更喜欢猫——男孩通常都很残酷。
\end{itemize}

如此一来,助动词 do 和主语 boys 放在一起了,而且 do 和它所代表的 like cats的
距离也减到最小,解决了所有的修辞问题。比较级需要用到倒装句的情形大抵都是这
样:

\textbf{一、从属从句中的助动词或 be 动词不宜省略。}

\textbf{二、主语后面有比较长的修饰语。}

\section{关系从句的倒装}

关系从句中的关系词,如果不是原来就在句首位置,就要向前移到句首让它发挥连接词
的功能。

例如:
\begin{enumerate}
\item The President is \unbf{a man}.

  总统是一个人。
\item A heavy responsibility, whether he likes it or not, falls on \unbf{him}.

  不论他喜不喜欢,他负有重大的责任。
\end{enumerate}
例 2 中的 him 就是例 1 的 a man,由这个交叉建立起关系,可以制造一个关系从句:
\begin{itemize}
\item The President is a man on \unnormal{whom}{关系词} \unnormal{a heavy
    responsibility}{关系从句主语}, whether he likes it or not, falls. (不
  佳)
\end{itemize}
介词短语 on whom因为内含关系词,要移到句首的位置。然而一经移动,就产生了修
辞上的问题。

首先,on whom 这个介词短语是当做副词使用,修饰动词falls。但是移到句首之后,
它与修饰的对象 falls之间隔了颇长的距离,这就会伤害修辞的清楚性。另外,关系从
句主语 a heavy responsibility 与它的动词 falls 之间也隔了一个插入的副词从
句whether \ldots{},主语动词间的距离过长又是一个不清楚性的来源。要解决这两个
问题还是得靠倒装句,把动词移到主语前面:
\begin{itemize}
\item The President is a man \unct{on whom falls a heavy responsibility}{倒装句},
  whether he likes it or not.

  总统负有重大责任,不论他喜不喜欢。
\end{itemize}
如此一来,关系词 whom 与先行词 a man 在一起,介词 on whom与它修饰的对
象 falls 在一起,而且动词 falls 又与它的主语 a heavy responsibility 在一起,
一举解决了所有问题。这就是倒装句的妙用。

要注意的是,\textbf{关系词必须先向句首移动,造成顺序的反常,才有倒装的可能。}如果关系
词没有移动就不能倒装。例如:
\begin{itemize}
\item The President is a man who bears a lot of responsibility.
\end{itemize}
这句话的意思和原来的句子差不多,不过它无法倒装。因为里面的关系从句原来是He
bears a lot of responsibility,主语 he 改成关系词who,由于原本就在句首,没有
移动位置,所以也就不能倒装。

\section{假设语句的倒装}

这种倒装比较单纯。在虚拟语气的副词从句中(往往是由 if 引导的),如果有be 动词
或助动词,就可以考虑倒装。做法是把连接词(例如 if)省略掉,把 be动词或助动词
移到主语前面来取代连接词的功能。例如:
\begin{itemize}
\item \unct{If I had been there}{副词从句}, I could have done something to help.

  如果当时我在场,就可以帮得上忙。
\end{itemize}
为了加强简洁性,可以把连接词 if 省略掉,用倒装句来取代,成为:
\begin{itemize}
\item \unct{Had I been there}{倒装句}, I could have done something to help.
\end{itemize}

但副词从句中若没有 be动词或助动词,就缺乏可倒装的工具,因而不能使用倒装。

\section{引用句的倒装}

\textbf{在直接引句(用到双引号者)与间接引句(没有用双引号者)中,都可以选择使用倒
  装句来凸显出句中的重点。}例如:
\begin{itemize}
\item \unct{The police}{S} \unct{said}{V}, \unct{“None was killed in the
    accident.”}{O 直接引句}

  警方说:“这桩车祸无人死亡。”
\end{itemize}
引用句往往出现在宾语位置,上面这个例子就是如此。不过,引用句的内文才是读者急
于知道的事情,至于是“谁说的”倒不那么关心。然而引用句的构造偏偏是“谁说
的”作为主语、动词,出现在前面,宾语从句拖在后面。选择倒装句就可以解决这个问
题:
\begin{itemize}
\item \unct{“None was killed in the accident.”}{O} \unct{said}{V} \unct{the police}{S}.
\end{itemize}
把读者最关心的引用句内文移到句首,可以达到强调语气的效果。因为宾语从句挪到句
首,与它关系密切的动词said也可以移到主语前面,成为倒装句。\textbf{不过在直接引句的情
况下,主语、动词也可以选择不必倒装},像上面这个例子,句尾部分可以维持the
police said(S+V)的顺序不必倒过来。接下来看间接引句:
\begin{itemize}
\item \unct{The WHO}{S} \unct{warns}{V}, \unct{that cholera is coming back}{O
    间接引句}.

  世界卫生组织警告:霍乱已死灰复燃。
\end{itemize}
这句话有一个间接引句,除了选择把整个宾语从句移到句首之外,也可以选择只把引用句的主语移到句首来加强语气,主要从句倒装,成为:
\begin{itemize}
\item Cholera, \unct{warns}{V} \unct{the WHO}{S}, is coming back.
\end{itemize}

不论直接引句还是间接引句,选择倒装的修辞原因都是为了凸显引用句的内容,把它摆
在句首最显著的地位。

\section{类似 there is/are的倒装}

这种倒装句是把地方副词挪到句首,句型和 there is/are的句型很接近,修辞功能在于
强调语气,以及衔接上下文。例如:
\begin{itemize}
\item \unct{There}{地方副词} \unct{goes}{V} \unct{the train}{S}!

  你看,火车开走了!
\end{itemize}
这个句子以倒装句处理,可以强调动词 goes,表示“正在开走”。再如:
\begin{itemize}
\item \unct{Here}{地方副词} \unct{is}{V} \unct{your ticket}{S} for the opera!

  你的歌剧票,拿去吧!
\end{itemize}
除了 here,there 之外,其他的地方副词也可以倒装,例如:
\begin{itemize}
\item \unct{In Loch Ness}{地方副词} \unct{dwells}{V} \unct{a mysterious monster}{S}.

  尼斯湖里住着一头神秘的水怪。
\end{itemize}
这个倒装句可以同时加强句首地方副词与句尾主语两个部分的语气。

有时候可以使用这种倒装句来加强上下文的衔接。例如:
\begin{itemize}
\item To the west of Taiwan lies Southern China.

\item To the east spreads the expanse of the Pacific.

\item 台湾西方是华南,东方是浩瀚的太平洋。
\end{itemize}
为了以空间顺序(spatial order)来组织上下文,这两个句子都用地方副词(To the
west \ldots,To the east \ldots)开头,也都动用倒装句来达到这个目的。

\section{否定副词开头的倒装}

如果把表示否定意味的副词(not、never, hardly)挪到句首来强调语气,就得使用倒
装句。例如:
\begin{itemize}
\item We \unbf{don't} have such luck \unbf{every day}.

  我们不是每天都能有这种运气。
\end{itemize}
如果为了强调“不是每天”,而把 not every day挪到句首,就要用倒装句。因为 not
和 every day 都是修饰动词的,而且 not是用来作否定句的副词,和助动词 do 不能分
开。一旦移到句首,助动词 do也要往前移来配合否定句的需要,就成为倒装句:
\begin{itemize}
\item \unbf{Not every day do} we have such luck.
\end{itemize}

再看一个例子:
\begin{itemize}
\item I will \unbf{not} stop waiting for you \unbf{until you are married}.

  除非你结婚,否则我会一直等你。
\end{itemize}
同样的,如果把 not until you are married移到句首来强调语气,就得把助动
词 will 倒装到主语前面来配合否定句的要求:
\begin{itemize}
\item \unbf{Not until you are married will} I stop waiting for you.
\end{itemize}

另外有一些副词,像 hardly,barely 等等,虽然不是一般否定句用的
not,不过功能与用法都类似,移到句首时也要倒装。例如:
\begin{itemize}
\item I had \unbf{hardly} sat down to work when the phone rang.

  我刚坐下来要做事,电话就响了。
\end{itemize}
把 hardly 移到句首也是为了加强语气,这时就要倒装:
\begin{itemize}
\item \unbf{Hardly had} I sat down to work when the phone rang.
\end{itemize}

不过,下面这个句子就不要倒装:
\begin{itemize}
\item \unbf{Hardly anyone} knew him.

  几乎没有人认识他。
\end{itemize}
这是因为 hardly 虽然在句首,不过它是用来修饰主语anyone,句首是它正常的位置,
没有经过调动,因而也不需要倒装。

同样的情形也见于 only 一字的变化。请看这个例子:
\begin{itemize}
\item \unbf{Only I} saw him yesterday.

  昨天只有我见到他。
\end{itemize}
Only 原本就是修饰主语I,放在它前面是正常位置,不需倒装。下面这个句子则不同:
\begin{itemize}
\item I saw him \unbf{only yesterday}.

  我见到他,不过是昨天的亊。
\end{itemize}
如果把 only yesterday调到句首来强调“不过是昨天而已”,意思是“不是更早以前的
事”,也有否定的意味,所以可以视同表示否定的副词移到句首的变化,需要倒装:
\begin{itemize}
\item \unbf{Only yesterday did I} see him.
\end{itemize}

再比较一下这两个句子:
\begin{enumerate}
\item \unbf{Gradually} they became close friends.
\item \unbf{Only gradually did they} become close friends.
\end{enumerate}
例 1 中的副词 gradually放在句首,是语法上许可的位置,而且没有否定意味,不必倒
装。可是例 2 中的only gradually 就带有强烈的否定意味,表示 not at
once 或是 not very fast,这时就得动用倒装句型了。

not only 和 but also 配合时,如果选择倒装,变化比较复杂。请看这个例子:
\begin{itemize}
\item He \unbf{not only} passed the exam \unbf{but also} scored at the top.

  他不但及格了,还考了第一。
\end{itemize}
句中的 but 是对等连接词。形成 not only \ldots{} but also
的相关词组(correlative)时,严格要求连接的对称。上例中的 passed the
exam 和 scored at the top 都是动词短语,符合对称的要求。

如果要把 not only 移到句首来强调语气,因为 not only是有否定功能的副词,所以要
用倒装句型。先直接倒装成为:
\begin{itemize}
\item \ul{Not only} did he pass the exam \ul{but also} scored at the top. (误)
\end{itemize}

前半句用倒装句是对的,错在对等连接词 but 的左右不对等。 左边 he passed
the exam 是从句,而右边的 scored at the top 却是动词短语。

修正的方法是把右边的动词短语也改成能对称的从句:
\begin{itemize}
\item   Not only did he pass the exam but also he scored at the top. (不佳)
\end{itemize}
这样改过来,but 的左右都是从句,满足了语法的要求,不过还是有缺憾。因
为also 和 only 一样都是属于 focusing adverbs,是一种有强调功能的副词。许多学
习者把 but also连在一起来背,不知它有时也该拆开。在 but 右边的 also 不应用来
强调he,而应用来强调 scored at the top(而且还考第一),这样才能呼应左边 not
only did he pass \ldots(不仅考及格)的语气。所以最佳的作法是
把 also 移到scored 的前面:
\begin{itemize}
\item Not only did he pass the exam \unbf{but he also} scored at the top.
\end{itemize}

这样才算满足了所有的语法修辞要求。

\section{结语}

以上的整理涵盖了英语中重要的倒装句型。另有一些简单的倒装句,例如:
\begin{itemize}
\item Mary is pretty. So \unct{is}{V} \unct{her sister}{S}.

  玛丽很美,她妹妹也很美。
\end{itemize}

以及不常用的倒装句,像某些祈使句的句型:
\begin{itemize}
\item Long \unct{live}{V} \unct{the King}{S}!

  国王万岁!
\end{itemize}
这些也是倒装句,可是不需要深入探讨。

看完了倒装句,整个英语句型问题至此总算尘埃落定。恭喜本书读者,至此你们已经建
立了相当完整的句型观念,对英语句型有了深入的理解。

\section{Test}

\paragraph{请选出最适当的答案填入空格内,以使句子完整。}

\begin{enumerate}

\item The students were warned that on no account \ttu to cheat.
\begin{tasks}(2)
  \task they were
  \task were they
  \task they should
  \task they can
\end{tasks}

\item \ttu make up for lost time.
\begin{tasks}
  \task Only by working hard we can
  \task By only working hard we can
  \task Only by working hard can we
  \task By only working hard can we
\end{tasks}

\item Rarely \ttu such nonsense.
\begin{tasks}(2)
  \task I have heard
  \task have I heard
  \task I do hear
  \task don't I hear
\end{tasks}

\item \ttu perched a large black bird.
\begin{tasks}(2)
  \task Often
  \task Suddenly
  \task On the wire
  \task It
\end{tasks}

\item Only just now \ttu to him about the things to heed while riding a
  motorcycle.
\begin{tasks}(2)
  \task I talked
  \task was I talking
  \task talked I
  \task I was talked
\end{tasks}

\item John was as confused about the rules \ttu.
\begin{tasks}
  \task as were the other contestants
  \task as the other contestants had
  \task than were the other contestants
  \task than the other contestants had
\end{tasks}

\item An IBM PC 286 is as powerful \ttu on NASA's Voyager II.
\begin{tasks}
  \task than the mainframe computer is
  \task than is the mainframe computer
  \task as the mainframe computer is powerful
  \task as is the mainframe computer
\end{tasks}

\item The New Testament is a book \ttu the life and teachings of Jesus.
\begin{tasks}(2)
  \task which can be found
  \task in which can be found
  \task which can find
  \task in which can find
\end{tasks}

\item Not until the doctor was sure everything was all right \ttu the emergency room.
\begin{tasks}(2)
  \task he left
  \task left he
  \task did he leave
  \task he did leave
\end{tasks}

\item \ttu, man could die out.
\begin{tasks}
  \task World War III should ever break out
  \task If should World War III ever break out
  \task If World War III should have broken out
  \task Should World War III ever break out
\end{tasks}

\item The results, \ttu, the leading journal of science, indicate that the experimental procedure is flawed.
\begin{tasks}(2)
  \task says Nature
  \task Nature says
  \task which says Nature
  \task which Nature says
\end{tasks}

\item Across the street from the station \ttu.
\begin{tasks}
  \task stood an old drugstore
  \task it stood an old drugstore
  \task where an old drugstore stood
  \task which stood an old drugstore
\end{tasks}

\item I tried to call some friends but \ttu.
\begin{tasks}(2)
  \task none could I reach
  \task could I reach none
  \task I could none reach
  \task I none could reach
\end{tasks}

\item \ttu trouble you again.
\begin{tasks}(2)
  \task Never will I
  \task Not I will ever
  \task Will not ever I
  \task Never I will
\end{tasks}

\item Not until you paint your first oil color \ttu the difference between
  theory and practice.
\begin{tasks}(2)
  \task you find out
  \task and find out
  \task finding out
  \task do you find out
\end{tasks}

\item \ttu a baby deer is born, it struggles to stand on its own feet.
\begin{tasks}(2)
  \task No sooner
  \task As soon as
  \task So soon as
  \task Not sooner that
\end{tasks}

\item \ttu the invention of the movable print, books were mostly copied by hand
  and cost far more than ordinary people could afford.
\begin{tasks}(2)
  \task After
  \task Until
  \task Not until
  \task Because of
\end{tasks}

\item \ttu did I find out that he was dead.
\begin{tasks}(2)
  \task A moment ago
  \task Only a moment ago
  \task An only moment ago
  \task For a moment
\end{tasks}

\item Henry James is \ttu is his philosopher brother William.
\begin{tasks}(2)
  \task famous and also
  \task as famous as
  \task famous so
  \task equally famous
\end{tasks}

\item \ttu does the recluse venture out of his hermitage.
\begin{tasks}(2)
  \task Seldom
  \task Often
  \task Occasionally
  \task Sometimes
\end{tasks}

\end{enumerate}

\section{Answer}
\begin{enumerate}
\item (B) on no account 是否定副词短语,移至 that 从句句首即需倒装。

\item (C) only by working hard 因有 only 修饰,在句首要倒装。

\item (B) rarely 有否定功能,置于句首要倒装。

\item (C) 地方副词置于句首,类似 there is/are 的句型,方可倒装,故选 C。

\item (B) 因有 only just now 在句首,要倒装。

\item (A) 前有 as confused,后面要有 as(A 或 B)。因为前面是 John was confosed,
  有 be 动词,后面不能用 had 来代表,应用 be 动词,故选 A,这是比较级的倒装。

\item (D) 上文 as 要求用 as 作连接,C 错在 powerfol 不应重复。

\item (B) 原句是 The life and teachings of Jesus can be found in the book,改成关系从句再倒装,即是 B。


\item (C) not until 移到句首要用倒装句型。
\item (D) 原句是 If World War \Ronum{3} should ever break out… 省略 If 后倒装即
  是 D。

\item (A) 原句是间接引句,Nature says the results… 改成倒装句成为 A 会比不倒装
  的 B 好,因为空格后的 the leading journal of science 是 Nature 的同位语,两
  者应该在一起。

\item (A) 地方副词 across the street from the station 移到句首而成倒装句,类
  似 there is/are 的句型。

\item (A) 是 I could reach none 的倒装。

\item (A) 是 I will never trouble you again 的倒装句。

\item (D) not until 移到句首要倒装。

\item (B) 答案 A 要用倒装句,C 和 D 都不是正确的连接词,只有 B 能引导后面那个没有倒装的从句。

\item (B) “活版印刷发明前,书原来都是用手抄,一般人根本买不起。”从句意来看,只
  有 until 符合。

\item (B) 下文是倒装句,所以选择要求倒装的 only。

\item (B) 比较级后面倒装了。

\item (A) 下文是倒装句,所以选择要求倒装的 seldom。
\end{enumerate}

%%% Local Variables:
%%% mode: latex
%%% TeX-master: "main"
%%% End:
