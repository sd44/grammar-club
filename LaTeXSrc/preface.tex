\chapter{序——我学英语的经验}

我是进了初中才开始接触到英语。初一的班导师是英语老师,他叫我们要\textbf{勤查字典},我
也就乖乖地\textbf{查了三年字典,背了些单词},也生吞活剥地记了些语法规则。在那种年纪,
记忆力好,理解力差,也不会想去把语法弄懂,背下来就算了!

\section*{启蒙的老师及语法书}

高中读的是新竹中学,高一的班导师也是英语老师,他要我们开始使用\textbf{英英字典},于
是我就查了三年英英字典。这在高中阶段我觉得是不错的训练,可以避免在两种完全不
同语系的单词间硬套,同时也可以训练阅读,以及培养用英语思考的习惯。

\textbf{一本好的语法书},对学英语的人有多么大的帮助!汤老师那套语法书就是这样的好书。
只不过完全用英语写成,要自习不大容易。此后我再也没有看过一本够好的语法书。

大学我念的是师大英语系,大一班导师当然也是英语老师。他要我们\textbf{丢开字典、大量阅
  读,要有一个晚上看完一本小说的能力。}

\section*{大学教育}

我不管老师上课时怎么跳,晚上\textbf{一定把书本逐字逐句看完}。当然有看不懂的,也有不
认识的字。不过我是以“看完”为目的,不懂也就算了。除了课本,另有一些重要的典
籍与作品,像圣经、希腊罗马神话与经典小说等等,我都到图书馆去借来看。好在这些
东西都是经过时间试炼的名作,不必勉强自己,看下去自然会欲罢不能。看这些东西,
感官刺激虽不如看电影,可是它比电影多一层\textbf{想象的空间,韵味无穷},是电影无法企
及的。就这样,我轻松愉快地念完了大学…… “不求甚解”式读书方法。

\section*{一个晚上看一本小说}

大概一个晚上要看一本。这一段时间的\textbf{密集阅读}对我的英语能力有“很大”的帮助。
小说是最优美丰富的文字,戏剧用的是口语(不过19世纪的口语和现在颇不相同),诗
歌是最浓缩的语言,散文比较平易近人,文学批评则是非常学术化的文体。这些东西看
了一堆下来,大概各式各样的英语都可以应付了。

研究所毕业后,在元培医专、新竹科学园区实验中学教了一阵子的书,又回到台北,进
入淡江英语系任教,并就近去读淡江的美国研究所博士班。在美研所所学,与其说对英
语有什么帮助,不如说是\textbf{进一步了解了美国的社会、政治、文化背景}。话说回来,要真
正懂一门语言,不了解那个国家和人民的话是办不到的,这一方面也就是我读美研所的
收获。

\section*{享受阅读乐趣的第一步——不求甚解}

教学的经验给我的帮助也很大。在从前教托福、GRE、GMAT这些留学英语测验时,开始接
触到词源分析,了解到英语单词的构成,也体会到\textbf{词源分析}是学习单词效果宏大的工具。

同时,为了教学所需,我以在师大学的教材教法为基础,再去阅读新的 ESL/EFL教学理
论,发现我误打误撞的那套“\textbf{不求甚解}”式阅读,竟然就是五种教学法之中最适合国内
学习者需求的\textbf{“阅读法”(the Reading Approach)}。这种方法不需要外在有英语环境,
只要找来适合自己程度的英语文章,由浅入深阅读下去,常见的英语单词与常用的语法
句型自然会大量出现,从上下文中就可以学会新的单词与用法,不需借助词典。

阅读法的作法,是将英文阅读分成四个步骤:精读(Intensive Reading)、广读(
Extensive Reading)、略读(Skimming)、扫描(Scanning):
\begin{description}
\item[精读] 借助文法翻译和句型分析,把一篇文章从单字到句型、甚至是文章的组织结构
  与时代文化背景,各方面全部要弄清楚。

  \textbf{大部分人阅读英文一直都停留在精读的阶段},所以阅读速度缓慢,阅读的份量因而也相当
  有限。结果就是一直有个无法突破的瓶颈,听、说、读、写各方面一直无法达到真正「流
  利」的程度。

\item[广读] 跟随个人兴趣,持续进行大量、快速、不求甚解的阅读。广读所需的工具,一
  是\textbf{字源分析}、一是\textbf{句型分析}。

\item[略读] 快速翻阅一本书或一篇文章、抓重点。如果你为了写论文在找资料,在图书馆
  中找到五十本书、或者在网路上找到五百篇文章,想要快速了解一下哪几本书、哪几篇文
  章对你写论文有帮助,值得借阅或印出,那么你就得略读这些书籍或文章、快速掌握它的
  主题。

\item[扫描] 这个工作很像搜寻引擎的搜寻功能:以最短时间在一本书或一篇文章当中搜寻
  出你要找的内容在何处。\textbf{考试时(如阅读测验)扫描能力会很有帮助。}

\end{description}

略读与扫描都是后话,最重要的工作还是广读。读到一个程度,累积了足够的 input,就会
有 output出来——可以拿起笔来\textbf{写}了。

然后还要通过\textbf{文法句型的训练},建立起从单句到复合句到减化子句的架构,写出来
的句子才能够正确又富于变化。

\section*{TIME 的挑战}

我学英语的经验,还有一个挑战要提——TIME 。从前我只是偶尔看一下 TIME,1980 年
代我开始在补习班讲授 TIME。这时候不是看看就算了,而是要\textbf{完全弄懂}才能去教。

\section*{懒人英语学习法}

英语只是个工具,但是这个工具的学习可以说是永无止境。现代英语教学法中的the
General Approach主张学习者应认清自己的学习风格,该怎么学应因人而异,
要\textbf{选择最适合自己学习风格的方法}。我学习英语的经验也许不是每个人都适合,
不过我觉得,好逸恶劳是大部分人的通病。如果你曾痛下决心把英语学好,却半途而废,
不能持之以恒,那么我这套懒人的方法可能也适合你。只要找你\textbf{爱看的书}来,
不必查字典,不求甚解,知道大概在说什么,能维持你阅读的兴趣就好了。或者
找\textbf{简单点的东西}来看,或者找《TIME 中文解读版》这类有\textbf{深度、优
  美}的文字来看,利用\textbf{翻译、注解}等等来了解文章在说什么就够了。这样你
自然能持续阅读下去,在不知不觉中吸收有意义的input 。只是茶余饭后看看闲书,没
有丝毫勉强,假以时日你的英语就会进步。


%%% Local Variables:
%%% mode: LaTeX
%%% TeX-master: "main"
%%% End:
