\section{形容词和副词}

\subsection{定语和表语}

能做定语或表语是形容词的主要特点。

\begin{description}
\item[定语 (Attributive)] 置于限定词(包括零冠词)和名词或代词之间,修饰名词或
  代词。
  \begin{itemize}
  \item The \unbf{red} car
  \item A very \unbf{interesting} book
  \end{itemize}

\item[表语(PREDICATIVE)] 主语或宾语的补语,用来说明主语性质、状态或身份的成分。
  \begin{itemize}
  \item This car is \unbf{red}.

  \item He thought the painting \unbf{ugly}.
  \end{itemize}
\end{description}

\subsection{以 -ly 结尾的形容词}

一些形容词以 -ly 结尾,但一般不作副词。如costly, cowardly, deadly, friendly,
likely, lively, lonely, lovely, silly, ugly, unlikely.
\begin{itemize}
\item She gave me a friendly smile.

\item Her singing was lovely.
\end{itemize}

friendly, lovely没有副词形式。
\begin{itemize}
\item She smiled in a \unbf{friendly} way. (not \sout{She smiled friendly.})
\item He gave a \unbf{silly} laugh. (not \sout{He laughed silly.})
\end{itemize}

early, hourly, nightly, daily, weekly, monthly, quarterly, yearly and leisurely 既是副词也是形容词。

\subsection{形容词和副词的同音同形异义词}

\begin{itemize}
\item a \unbf{fast} car drive \unbf{fast}
\item They are \unbf{close} friends, and they live \unbf{close} by.

 close取“接近,紧密”的意思,作形容词或副词时,发音是 \doulos{/kləʊs/ /kloʊs/}.
\item She had \unbf{short} hair. She cut her hair \unbf{short}.

\item Did you have to wait a \unbf{long} time? (to wait \unbf{long})

\item a \unbf{slow} car drive \unbf{slow/slowly}
\end{itemize}

有时一个副词可能有两种形式(如late和lately),一种与其形容词形式一样,一种
以ly结尾。这两种形式往往意思不同或用法不同。

一些诅咒语可以既作形容词又作副词,如bloody。
\begin{itemize}
\item 'You \unbf{bloody} fool. You didn't look where you were going.''I \unbf{bloody} did.'

  “你这个笨蛋,你没有看你往哪走吗?”“我他妈看了。”
\end{itemize}

dead作副词时,意思是“的确,完全,非常”,例如dead ahead, dead certain, dead
drunk, dead right, dead slow, dead straight, dead sure, dead tired.

deadly则是形容词,意思是“致命的”。表达这个意思的副词是fatally.

easy 在非正式词组里作副词:
\begin{itemize}
\item Go \unbf{easy}! (= Not too fast!)
\item \unbf{Easy} come, \unbf{easy} go.
\item Take it \unbf{easy}! (= Relax!)
\item \unbf{Easier} said than done.
\end{itemize}

fair也可用在某些词组的动词后面,用作副词。
\begin{itemize}
\item to play \unbf{fair}, to fight \unbf{fair}
\item to hit/win something \unbf{fair} and square

  fairly一般修饰形容词和副词,表示并不怎样高,凑合的意思。

\item `How was the film?' '\unbf{Faily} good.' 还好,还凑合。

\item I speak English \unbf{fairly} well -- enough for everyday purposes.

  我中文还好——足够应付日常需要。
\end{itemize}


fine在非正式文本中用作副词,等同于well。
\begin{itemize}
\item That suits me \unbf{fine}.

\item You're doing \unbf{fine}.
\end{itemize}
副词 finely 通常用作表示细微的调整。
\begin{itemize}
\item a \unbf{finely} tuned engine

\item \unbf{finely} chopped onions (= cut up very small)
\end{itemize}

free 在动词后面用作副词时意思是“免费”; freely 意思是“无限制地”:
\begin{itemize}
\item You can eat \unbf{free} in my restaurant whenever you like.

\item You can speak \unbf{freely} – I won't tell anyone what you say.
\end{itemize}

hard副词“用力地努力地”;hardly意思是“几乎不”:
\begin{itemize}
\item Hit it \unbf{hard}.
\item I trained really \unbf{hard} for the marathon.
\item I've \unbf{hardly} got any clean clothes left.

  我几乎没有什么干净衣服了。
\item Anna works \unbf{hard}, but her elder sister \unbf{hardly} works.

  安娜努力工作,但她的姐姐几乎不工作。
\end{itemize}

high意思“高度高”; highly意思“非常,高级的,赞赏的”:
\begin{itemize}
\item He can jump really \unbf{high}.
\item Throw it as \unbf{high} as you can.
\item It's \unbf{highly} amusing. 这非常有趣。
\item I can \unbf{highly} recommend it. 我极力推荐它。
\end{itemize}

just有很多个意思:
\begin{description}
\item[exactly] 正好;恰好
  \begin{itemize}
  \item This jacket is \unbf{just} my size.

    这件夹克正合我的尺码。

  \item You're \unbf{just} in time.

    你来得正是时候。

  \item It's \unbf{just} what I wanted!

    这正是我想要的!
  \end{itemize}

\item[at this moment] 此时,刚刚
  \begin{itemize}
  \item I've \unbf{just} heard the news.

    我刚听到这个消息。

  \item When you arrived he had only \unbf{just} left.

    你到时他刚走。
  \item She has \unbf{just} been telling us about her trip to Rome.

    她刚才一直在给我们讲她的罗马之行。
  \end{itemize}

\item[only, scarcely] 只,简直不
  \begin{itemize}
  \item Complete set of garden tools for \unbf{just} £15.99!
  \item I \unbf{just} want somebody to love me – that's all.
  \end{itemize}

\item[emphasiser] 强调其他词句
  \begin{itemize}
  \item You're \unbf{just} beautiful.

  \item I \unbf{just} love your dress.
  \end{itemize}
\end{description}
此外还有形容词just,其副词为justly,意思是“公平地,正义地”:
\begin{itemize}
\item He was \unbf{justly} punished for his crimes.
\end{itemize}

形容词和副词late意思相近,“迟到”;副词lately“最近地”:
\begin{itemize}
\item I hate arriving \unbf{late}.

\item I haven't been to the theatre much \unbf{lately}.
\end{itemize}

形容词和副词low意思几乎一致:
\begin{itemize}
\item a \unbf{low} bridge \qquad bend \unbf{low} 深弯腰
\end{itemize}

most是much的最高级,也可以用来构成形容词和副词的最高级。
\begin{itemize}
\item Which part of the concert did you like \unbf{most}?
\item This is the \unbf{most} extraordinary day of my life.
\end{itemize}
在正式文体中,可以作
“非常地”用。
\begin{itemize}
\item You're a \unbf{most} unusual person.
\end{itemize}
Mostly 意思是“主要 (mainly),大都 (most often)或在大多数场合 (in \unbf{most} cases)”
\begin{itemize}
\item My friends are \unbf{most}ly non-smokers.
\end{itemize}


right与状语连用时,意思是“正好,精确地”:
\begin{itemize}
\item She arrived \unbf{right} \unbf{after breakfast}.
\item The snowball hit me \unbf{right} \unbf{on the nose}.
\item Turn the gas \unbf{right} \unbf{down}.
\end{itemize}
Right 和rightly都有“正确地”意思。但副词right只能用在动词后面,并且是非正式
的用法:
\begin{itemize}
\item I rightly assumed that Henry was not coming.
\item You guessed right.
\item It serves you right. ( … rightly is not possible.)
\end{itemize}

straight 副词和形容词形式一样.
\begin{itemize}
\item A \unbf{straight} road goes \unbf{straight} from one place to another.
\end{itemize}

sure 在非正式用法中“当然地” :
\begin{itemize}
\item `Can I borrow your tennis racket?' `\unbf{Sure}.'
\end{itemize}

surely (not) 表示“意见,惊讶于”:
\begin{itemize}
\item Surely house prices will stop rising soon!
\item Surely you're not going out in that old coat?
\end{itemize}

\item wide The normal adverb is wide; widely suggests distance or separation. Compare:
\item The door was wide open.
\item She's travelled widely.
\item They have widely differing opinions.
\item Note also the expression wide awake (the opposite of fast asleep).
\item wrong Wrong can be used informally instead of wrongly after a verb. Compare:
\item I wrongly believed that you wanted to help me.
\item You guessed wrong.

3	comparatives and superlatives
Informal uses of adjective forms as adverbs are especially common with comparatives and superlatives.
Can you drive a bit slower?
Let's see who can do it quickest.












%%% Local Variables:
%%% mode: latex
%%% TeX-master: "main"
%%% End:
