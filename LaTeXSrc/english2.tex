\chapter{从句的句法和语义功能}

\section{从句的功能类别}

根据从句的潜在功能,可将其分为四种主要类别:
\begin{description}
\item[名词从句 NOMINAL CLAUSES] \index{概念!名词性从句 nominal clauses} 功能
  类似于\textbf{名词短语},可充当主语、宾语、补语、同位语和介词补语。

\item[状语从句 ADVERBIAL CLAUSES] \index{概念!状语从句 adverbial clauses} 功能方面更像是副词短语;但是在表达的明确性上,常常更像
  是介词短语。

\item[关系从句 RELATIVE CLAUSES] \index{概念!关系从句 relative clauses} 功能
  上与修饰性形容词相同,\textbf{修饰名词短语};\textbf{位置上和后置修饰的介词
    相同}。
  \begin{itemize}
  \item a man \unbf{who is lonely} \~{} \unbf{a lonely} man
  \item tourists \unbf{who come from Italy} \~{} tourists \unbf{from Italy}
  \end{itemize}

\item[比较从句 COMPARATIVE CLAUSES] \index{概念!比较从句 comparative
    clauses} 在修饰功能上与形容词和副词相似;语义上比较从句和他们的关联词等同
  于程度副词。
  \begin{itemize}
  \item She has \unbf{more} patience \unbf{than you have}.
  \item He's not \unbf{as} clever a man \unbf{as I thought}.
  \item I love you \unbf{more} deeply \unbf{than I can say}.
  \end{itemize}
\end{description}


\section{状语从句}

状语从句的语义分析比较复杂,因为\textbf{同一个从属连词所引导的从句意思可能不
  同},而且这样的情形为数不少。例如, since从句可以是时间从句,也可以是原因分
句。\textbf{另外,有些从句把两层意思结合起来}。

作时间状语的 ing 从句由以下从属连词引导: once, till, until, when, whenever,
while 和 whilst, as soon/long as:
\begin{itemize}
\item \unbf{Once having made a promise}, you should keep it.
\item The dog stayed at the entrance \unbf{until told to come in}.
\item Complete your work \unbf{as soon as possible}.
\end{itemize}

带有\textbf{until- 从句的主句}必须是\textbf{持续性}的,时间持续到until-从句的时间
为止。因为事件未发生的状态是持续性的,所以\textbf{否定从句总是持续性的},即使相应
的肯定从句并非如此。例如:
\begin{itemize}
\item I \unbf{didn't} start my meal \unbf{until} Adam arrived. [正确]
\item \sout{I \unbf{started} my meal \unbf{until} Adam arrived.} [错误]
\end{itemize}


地点从句主要由where和wherever引导,\textbf{where是具体的, wherever是非具体的}。
\begin{itemize}
\item \unbf{Where} the fire had been, we saw nothing but blackened ruins.

\item They went \unbf{wherever} they could find work. [to any place where]
\end{itemize}


\todo[inline]{本笔记对状语较少描述,以后可从 The Gramma Book等简明书籍中摘录。}

\section{句子性关系从句}

\textbf{修饰名词的关系从句}的先行词是\textbf{名词短语},而\textbf{句子性关系从句}的
先行词可以是:
\begin{description}
\item[主句的谓语或谓体] They say he \unbf{plays truant}, \unbf{which he doesn't}.

  \unbf{walks for an hour each morning}, \unbf{which would bore me}.

\item[主句或整个句子] Things then improved, \unbf{which surprises me}.

  Colin married my sister and I married his brother, \unbf{which makes Colin and me in-law}.

\item[之前多个句子] --- \unbf{which is how the kangaroo came to have a pouch}.

  所以袋鼠才有了育儿袋。

\end{description}

句子性关系从句与名词短语中的非限制性后置修饰从句相似,因为它们也用语调或标点
符号将其本身和先行词分隔开来。暂不详述。
\begin{itemize}
\item The plane may be several hours late, in which case there's no point in our waiting.

\item They were under water for several hours, from which experience they
  emerged unharmed.
\end{itemize}

\improve[inline]{区别需参见cref郎文语法17.11}

\section{非限定状语从句和无动词状语从句}

\begin{description}
\item[独立从句] 具有一个明显的\textbf{主语}但\textbf{不用从属连词}引导
  的\textbf{非限定性从句和无动词从句}。之所以成为独立从句是因为它们在句法上显
  然并\textbf{不与主句绑定在一起}。独立从句可以是 -ing, -ed 或无动词从句。
\end{description}

\begin{itemize}
\item \textbf{No further discussion a rising}, the meeting was brought to a close.
\item \textbf{Lunch finished}, the guests retired to the lounge.
\item \textbf{Christmas then only days away}, the family was pent up with excitement.
\end{itemize}

非限定从句和无动词从句中\textbf{没有主语时},用以识别主语的依附规则
是,\textbf{认为母句的主语就是其主语}:
\begin{itemize}
\item The oranges, \unbf{when (they are) ripe}, are picked and sorted
  mechanically.

\item \unbf{Driving home after work}, I accidentally went through a red
  light. [While I was driving home after work]

\item \unbf{To climb the rock face}, we had to take various precautions. [So that
  we could climb]

\end{itemize}

某些情况下,依附规则是不适用的,或者说至少是不严格的:
\begin{description}[style=nextline]
\item[从句是一个主语外接状语, 这时隐含的主语是说话者 I]

  \unbf{Putting it mildly}, you have caused us some inconvenience.

\item[隐含的主语是整个主句]

  I'll help you \unbf{if necessary}. [\ldots{} \textbf{if it is necessary}]

\item[隐含的主语是一个不定代词或支撑词 it]

  \unbf{When dining in the restaurant}, a jacket and tie are required. [\textbf{When one
  dines}]

  \unbf{Being Christmas}, the government offices were closed. [\textbf{Since it was}]
\end{description}

\textbf{没有从属连词引导的状语从句和无动词从句被称为增补从句},根据上下文,我
们可以用其表示时间、条件、原因、让步或状况关系。对读者或听者来说,这种伴随关系
的实质是从语境中推断的。
\begin{itemize}
\item \unct{Reaching}{When we Reached} the river, we pitched camp for the night.
\item Julia, \unct{being}{since she was} a nun, spent much of her time in
  prayer and meditation.
\item The sentence is ambiguous, (\unbf{if / when it is}) taken out of context.

\item We spoke \unbf{face to face}.
\end{itemize}

\section{比较从句}

在比较结构中,主句中的陈述与从句中的陈述进行比较。两句中共同的部分在从属分
句中可省略。
\begin{itemize}
\item Jane is \unbf{as} \unct{healthy}{比较成分} \unbf{as} \unct{her sister}{比较基础} (is).
\item Jane is \unct{healthier}{比较成分} \unbf{than} \unct{her sister}{比较基础} (is).
\end{itemize}

\subsection{比较成分的从句功能}

\textbf{比较成分可以是比较结果中除动词以外的任何一个成分}:
\begin{description}
\item[主语] \unbf{Most people} use this brand than (use) any other shampoo.

\item[直接宾语] She knows \unbf{more history} than most people (know).

\item[间接宾语] That toy has given \unbf{more children} happiness than any other (toy) (has).

\item[主语补语] Simo is \unbf{more relaxed} than he used to be.

\item[宾语补语] She thinks her children \unbf{more taller} than (they were) last year.
\item[状语] You've been working \unbf{much harder} than I (have).

\item[介词补足语] She's applied for \unbf{more jobs} than Joyce (has (applied for)).
\end{description}

由 more \ldots{} than, less \ldots{} than 和 as \ldots{} as 引导
的\textbf{不一定是比较从句,后面可接续一个明显的比较标准或状态}。
\begin{itemize}
\item I weigh more than \unbf{200 pounds}.

\item It goes faster than \unbf{100 miles per hour}.

\end{itemize}

另一种不接比较从句的类型:
\begin{itemize}
\item I was more angry than \unbf{frightened}. [frightened:
  \doulos{/ˈfraɪtnd/} 受惊的,害怕的。]
\item I was angry more than \unbf{frightened}.

\item \sout{I was angrier than frightened}.
\end{itemize}
上述最后一句错误。因为angrier为屈折形式的比较级,frightened(害怕的)是过去分
词作形容词用,两者不对等。

more of a \ldots{} 和less of a \ldots{} 与可分等级的名词中心语连用:
\begin{itemize}
\item He's more of a fool than I thought (he was).

\item It was less of a success than I imagined (it would be).
\end{itemize}

当对比涉及\textbf{同一阶上的两个点}且一点高于另一点时,则than之后的部分\textbf{不可以
扩展成从句}。Than的功能是非从句比较中的\unbf{介词}:
\begin{itemize}
\item It's hotter \unbf{than} just warm. (或 It's hotter than 90°C.)
\item She's wiser \unbf{than} merely clever.
\item They fought harder \unbf{than} that.
\item I was \unbf{more than} happy to hear that.
\end{itemize}

\subsection{比较从句中的省略}

由于两个从句在结构和内容上通常非常相似,因此\textbf{省略在比较从句中是常规而不是例外}。
以下是省略和代词、替代谓语和替代谓体的例子:

James and Susan often go to plays but
\begin{enumerate}
\item James enjoys the theater more than Susan enjoys the theater.
\item James enjoys the theater more than Susan enjoys it.
\item James enjoys the theater more than Susan does.
\item James enjoys the theater more than Susan.
\item James enjoys the theater more.

  因为前半句已经说明了对象两人,所以这里可以直接省略整个比较从句。
\end{enumerate}
\textbf{宾语一般不可省略,除非主要动词也省略,如第3、4句,此时功能词可留可不留。}
\begin{itemize}
\item James enjoys the theater more than Susan \sout{enjoys}.

  误!比较从句中宾语省略,主要动词未省略。
\end{itemize}
但是,如果\textbf{宾语本来就是比较成分},那么\textbf{可以省略宾语,而不省略主要动词}:
\begin{itemize}
\item James knows more about the theater is more than Susan \unbf{knows}.
\end{itemize}


如作最大限度的省略,有可能造成歧义:
\begin{itemize}
\item He loves his dog more than his children.
\end{itemize}
上例的意思可能是他比他的孩子更爱狗(his children作从句主语),也可能是他爱狗
超过爱孩子(his children作从句宾语)。因此最好根据实际情况补充说明:
\begin{itemize}
\item He loves his dog more than his children \unbf{does} his dog.

  他比他的孩子更爱狗。
\item He loves his dog more than he loves his children.

  他爱狗超过爱孩子
\end{itemize}


\subsection{部分对比 (partial contrast)}

对比可能\textbf{只影响时态}或\textbf{加上了情态助动词}而已。在这种情况下,一
般是省略比较从句情态助动词之后的部分:

\begin{itemize}
\item I hear it more clearly than I \unbf{did}. [than I used to hear it]

\item I get up later than I \unbf{should}. [than I should get up]
\end{itemize}

如果只是\textbf{时态}上的对比,在比较从句中可能只用一个状语来表示:
\begin{itemize}
\item She'll enjoy it more than (she enjoyed it或 she did)last year.
\end{itemize}
这就为下列例句中从句全部省略提供了基础:
\begin{itemize}
\item You are slimmer (than you were).
\item You're looking better (than you were (looking)).
\end{itemize}

对一个隐含或实际表达的从句存在\textbf{逆向呼应}的省略:
\begin{itemize}
\item I caught the bus from the town: but Harry came home \unbf{even later}. [later than I came home]
\end{itemize}


话语之外\textbf{实境已包含被比较信息}的省略:
\begin{itemize}
\item You should have come home earlier. [earlier than you did]
\end{itemize}

部分对比可能是主句或对比从句中的\textbf{上位从句}:
\begin{itemize}
\item \unbf{She thinks} she's fatter than she (really) is.
\item He's a greater painter than \unbf{people suppose} (he is).
\item She enjoyed it more than \unbf{I expected} (her to (enjoy it)).
\end{itemize}

\subsection{等量比较 as \ldots{} as}

as \ldots{} as 结构在语法上与 more \ldots{} than 结构相似,只是 \textbf{as 不
  能像 more 那样作限定词、代词和下加伏语};这些功能由 as many (具数) 和 as
much(不具数)来弥补。因此我们可以在必要时用 as many 和 as much 替代 more:

as (much/many) 可作:
\begin{description}
\item[限定词] Isabelle has \unbf{as many} books as her brother (has).

\item[名词短语中心词] \unbf{As many} of his friends are in New York as (are) here.

\item[作下加状语] I agree with you \unbf{as much} as ((I agree) with) Robert.

\item[形容词中心语的修饰语] The article was \unbf{as objective} as I expected (it
  would be).

\item[前置修饰形容词的修饰语] It was \unbf{as lively} a discussion as we thought it would be.

  [形容词短语也可后置 It was a discussion \unbf{as lively} as \ldots{}]

\item[副词的修饰语] I am \unbf{as severely} handicapped as you (are).

  [副词也可后置 I am handicapped as severely as \ldots{}]
\end{description}

as ADJ a NOUN as \ldots{} 也是个常见的句式。
\begin{itemize}
\item  I did have a good time, but not \unbf{as good a time} as I should have had.
\end{itemize}

当母句是否定句时,可用 so \ldots{} as 取代 as \ldots{} as, \textbf{从句全部或大部
省略时尤其如此}:
\begin{itemize}
\item He's not \unbf{as naughty} as he was.
\item He's not \unbf{so naughty} as he was.
\item He's not \unbf{so naughty} (now).
\end{itemize}

\subsection{enough 和 too}

表达足量或超越比较的结构主要由enough或too + to不定式表示。
\begin{description}
\item[足量比较] They're rich \unbf{enough to} own a car.

  The book is simple \unbf{enough to} understand.

\item[超越比较] They're not \unbf{too poor to} own a car.

  他们还没有穷到买不起一辆车。

  The book is not \unbf{too difficult to} understand.

  这本书不是太难理解。
\end{description}

\textbf{too 有否定意义},表示\textbf{太、过于 \ldots{} 以致不能},比较:
\begin{itemize}
\item She's \unbf{old enough} to do some work.
\item She's \unbf{too old} to do any work.
\end{itemize}

如果语境许可,动词不定式从句可省略。sufficient(ly) 和 excessive(ly) 分别
是 enough 和 too 的较为正式的同义词,正式用法。
\begin{itemize}
\item The book is \unbf{sufficiently} simple to understand.
\item The book is not \unbf{excessively} difficult to understand.
\end{itemize}

\subsection{so \ldots{} that 和 such \ldots{} that}

So 是副词,前置修饰一个形容词或副词;such 是前限定词,与中后限定词一起修饰名
词中心语。

当that从句是\textbf{否定}时,so/such结构和too+to 不定式结构之间有一种对应关
系:
\begin{itemize}
\item It's \unbf{so} good a movie \unbf{that} we mustn't miss it.

  It's \unbf{too} good a movie \unbf{to} miss.

\item It was \unbf{such} a pleasant day \unbf{that} I didn't want to go to school.

  It was \unbf{too} pleasant a day \unbf{to} go to school.
\end{itemize}

当that从句是肯定的,so/such结构和enough+to 不定式结构之间有一种对应关系:
\begin{itemize}
\item It flies \unbf{so} fast \unbf{that} it can beat the speed record.

  It flies fast \unbf{enough to} beat the speed record.

\item I had \unbf{such} a bad headache \unbf{that} I needed two aspirins.

  I had a bad \unbf{enough} headache \unbf{to} need two aspirins.
\end{itemize}


当 \textbf{so} 单独与\textbf{动词}连用时,表示程度高;\textbf{such} 接续
的\textbf{名词}短语没有形容词前置修饰时,同样表示程度高:
\begin{itemize}
\item I \unbf{so} (much) enjoyed it \unbf{that} I'm determined to go
  again.
\item There was \unbf{such} a (large) crowd \unbf{that} we couldn't see a thing.
\end{itemize}

正式的结构so/such \ldots{} as+to不定式,有时替代so/such \ldots{} that从句:
\begin{itemize}
\item  We went early \unbf{so as to} get good seats.

\item I'm not \unbf{so} stupid \unbf{as to} believe that.

\item Would you be \unbf{so} kind \unbf{as to} lock the door when you leave?
\end{itemize}












%%% Local Variables:
%%% mode: LaTeX
%%% TeX-master: "main"
%%% End:
