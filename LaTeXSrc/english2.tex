\section{形容词和副词}

\subsection{定语和表语}

能做定语或表语是形容词的主要特点。

\begin{description}
\item[定语 (Attributive)] 置于限定词(包括零冠词)和名词或代词之间,修饰名词或
  代词。
  \begin{itemize}
  \item The \unbf{red} car
  \item A very \unbf{interesting} book
  \end{itemize}

\item[表语(PREDICATIVE)] 主语或宾语的补语,用来说明主语性质、状态或身份的成分。
  \begin{itemize}
  \item This car is \unbf{red}.

  \item He thought the painting \unbf{ugly}.
  \end{itemize}
\end{description}

\subsection{以 -ly 结尾的形容词}

一些形容词以 -ly 结尾,但一般不作副词。如costly, cowardly, deadly, friendly,
likely, lively, lonely, lovely, silly, ugly, unlikely.
\begin{itemize}
\item She gave me a friendly smile.

\item Her singing was lovely.
\end{itemize}

friendly, lovely没有副词形式。
\begin{itemize}
\item She smiled in a \unbf{friendly} way. (not \sout{She smiled friendly.})
\item He gave a \unbf{silly} laugh. (not \sout{He laughed silly.})
\end{itemize}

early, hourly, nightly, daily, weekly, monthly, quarterly, yearly and leisurely 既是副词也是形容词。

\subsection{形容词和副词的同音同形异义词}

\begin{itemize}
\item a \unbf{fast} car drive \unbf{fast}
\item They are \unbf{close} friends, and they live \unbf{close} by.

 close取“接近,紧密”的意思,作形容词或副词时,发音是 \doulos{/kləʊs/ /kloʊs/}.
\item She had \unbf{short} hair. She cut her hair \unbf{short}.

\item Did you have to wait a \unbf{long} time? (to wait \unbf{long})

\item a \unbf{slow} car drive \unbf{slow/slowly}
\end{itemize}

有时一个副词可能有两种形式(如late和lately),一种与其形容词形式一样,一种
以ly结尾。这两种形式往往意思不同或用法不同。

一些诅咒语可以既作形容词又作副词,如bloody。
\begin{itemize}
\item 'You \unbf{bloody} fool. You didn't look where you were going.''I \unbf{bloody} did.'

  “你这个笨蛋,你没有看你往哪走吗?”“我他妈看了。”
\end{itemize}

dead作副词时,意思是“的确,完全,非常”,例如dead ahead, dead certain, dead
drunk, dead right, dead slow, dead straight, dead sure, dead tired.

deadly则是形容词,意思是“致命的”。表达这个意思的副词是fatally.

easy 在非正式词组里作副词:
\begin{itemize}
\item Go \unbf{easy}! (= Not too fast!)
\item \unbf{Easy} come, \unbf{easy} go.
\item Take it \unbf{easy}! (= Relax!)
\item \unbf{Easier} said than done.
\end{itemize}

fair也可用在某些词组的动词后面,用作副词。
\begin{itemize}
\item to play \unbf{fair}, to fight \unbf{fair}
\item to hit/win something \unbf{fair} and square

  fairly一般修饰形容词和副词,表示并不怎样高,凑合的意思。

\item `How was the film?' '\unbf{Faily} good.' 还好,还凑合。

\item I speak English \unbf{fairly} well -- enough for everyday purposes.

  我中文还好——足够应付日常需要。
\end{itemize}


fine在非正式文本中用作副词,等同于well。
\begin{itemize}
\item That suits me \unbf{fine}.

\item You're doing \unbf{fine}.
\end{itemize}
副词 finely 通常用作表示细微的调整。
\begin{itemize}
\item a \unbf{finely} tuned engine

\item \unbf{finely} chopped onions (= cut up very small)
\end{itemize}

free 在动词后面用作副词时意思是“免费”; freely 意思是“无限制地”:
\begin{itemize}
\item You can eat \unbf{free} in my restaurant whenever you like.

\item You can speak \unbf{freely} – I won't tell anyone what you say.
\end{itemize}

hard副词“用力地努力地”;hardly意思是“几乎不”:
\begin{itemize}
\item Hit it \unbf{hard}.
\item I trained really \unbf{hard} for the marathon.
\item I've \unbf{hardly} got any clean clothes left.

  我几乎没有什么干净衣服了。
\item Anna works \unbf{hard}, but her elder sister \unbf{hardly} works.

  安娜努力工作,但她的姐姐几乎不工作。
\end{itemize}

high意思“高度高”; highly意思“非常,高级的,赞赏的”:
\begin{itemize}
\item He can jump really \unbf{high}.
\item Throw it as \unbf{high} as you can.
\item It's \unbf{highly} amusing. 这非常有趣。
\item I can \unbf{highly} recommend it. 我极力推荐它。
\end{itemize}

just有很多个意思:
\begin{description}
\item[exactly] 正好;恰好
  \begin{itemize}
  \item This jacket is \unbf{just} my size.

    这件夹克正合我的尺码。

  \item You're \unbf{just} in time.

    你来得正是时候。

  \item It's \unbf{just} what I wanted!

    这正是我想要的!
  \end{itemize}

\item[at this moment] 此时,刚刚
  \begin{itemize}
  \item I've \unbf{just} heard the news.

    我刚听到这个消息。

  \item When you arrived he had only \unbf{just} left.

    你到时他刚走。
  \item She has \unbf{just} been telling us about her trip to Rome.

    她刚才一直在给我们讲她的罗马之行。
  \end{itemize}

\item[only, scarcely] 只,简直不
  \begin{itemize}
  \item Complete set of garden tools for \unbf{just} £15.99!
  \item I \unbf{just} want somebody to love me – that's all.
  \end{itemize}

\item[emphasiser] 强调其他词句
  \begin{itemize}
  \item You're \unbf{just} beautiful.

  \item I \unbf{just} love your dress.
  \end{itemize}
\end{description}
此外还有形容词just,其副词为justly,意思是“公平地,正义地”:
\begin{itemize}
\item He was \unbf{justly} punished for his crimes.
\end{itemize}

形容词和副词late意思相近,“迟到”;副词lately“最近地”:
\begin{itemize}
\item I hate arriving \unbf{late}.

\item I haven't been to the theatre much \unbf{lately}.
\end{itemize}

形容词和副词low意思几乎一致:
\begin{itemize}
\item a \unbf{low} bridge \qquad bend \unbf{low} 深弯腰
\end{itemize}

most是much的最高级,也可以用来构成形容词和副词的最高级。
\begin{itemize}
\item Which part of the concert did you like \unbf{most}?
\item This is the \unbf{most} extraordinary day of my life.
\end{itemize}
在正式文体中,可以作
“非常地”用。
\begin{itemize}
\item You're a \unbf{most} unusual person.
\end{itemize}
Mostly 意思是“主要 (mainly),大都 (most often)或在大多数场合 (in \unbf{most} cases)”
\begin{itemize}
\item My friends are \unbf{most}ly non-smokers.
\end{itemize}


right与状语连用时,意思是“正好,精确地”:
\begin{itemize}
\item She arrived \unbf{right} \unbf{after breakfast}.
\item The snowball hit me \unbf{right} \unbf{on the nose}.
\item Turn the gas \unbf{right} \unbf{down}.
\end{itemize}
Right 和rightly都有“正确地”意思。但副词right只能用在动词后面,并且是非正式
的用法:
\begin{itemize}
\item I rightly assumed that Henry was not coming.
\item You guessed right.
\item It serves you right. ( … rightly is not possible.)
\end{itemize}

straight 副词和形容词形式一样.
\begin{itemize}
\item A \unbf{straight} road goes \unbf{straight} from one place to another.
\end{itemize}

sure 在非正式用法中“当然地” :
\begin{itemize}
\item `Can I borrow your tennis racket?' `\unbf{Sure}.'
\end{itemize}

surely (not) 表示“意见,惊讶于”:
\begin{itemize}
\item Surely house prices will stop rising soon!

  房价肯定会停止上涨!
\item Surely you're not going out in that old coat?

  你该不会穿着那件旧外套出去吧。
\end{itemize}

wide作副词用时表示“宽地”, widely 表示距离很远,或差别很大:
\begin{itemize}
\item The door was wide open.

  当时那扇门大开着。
\item She's travelled widely.

  它曾到处游历。
\item They have widely differing opinions.

  他们的意见大相径庭。
\end{itemize}


\textbf{有些形容词的比较级和最高级形式用作副词}的现象,在规范英语中也很常见。试比较:
\begin{itemize}
\item Speak \unbf{clearer}! [\unbf{more clearly}]
\item This newsreader speaks \unbf{clearest} of all. [\unbf{most clearly}]
\item It's \unbf{easier} said than done. [\unbf{more easily}]
\item Ami The car ran went (the) \unbf{slower and slower}.
\end{itemize}

\subsection{以 a- 开头的形容词和副词}

某些以a- 开头的词给语言学家带来了难题。有些语法学家把它们划为形容词,有些语法
学家把它们划为副词。这些以 a- 开头的词起表语作用,但只有几个能随意用作定语.

只有比较少的副词可以且只能在be后面作表语,如地点副词 aboard, upstairs和时间副
词,如 now, tonight 。而形容词却还能和其他系动词连用。试比较系动词 be 和 seem
的不同句型:

$\text{The patient} \left\{
  \begin{aligned}
    &\text{was asleep/hungry/abroad/there.}\\
    &\text{seemed asleep/hungry.(\textbf{形容词,不能是副词}abroad/there等)}
  \end{aligned}
\right.$

a- 形容词不能在非be动词后面。因此我们可以用seem to be支持a- 形容词或副词:
\begin{itemize}
\item They seemed to be abroad/there/around/afraid.
\end{itemize}

\subsection{形容词后置}
后置

形容词有时可以后置 (postpositive), 也就是说,它们能紧跟在所修饰的名词或代词后
面。因此,形容词可以有三种不同的位置:
\begin{description}
\item[表语位置] This information is \unbf{useful}.
\item[定语位置] \unbf{useful} information
\item[后置] something \unbf{useful}
\end{description}

以 -body, -one, -thing, -where 结尾的复合不定代词和复合不定副词(
见\cref{tab:someany})只能用后置形容词来修饰,通常我们可以将其后置形容词(短
语)看成是一个缩简的关系分句:
\begin{itemize}
\item \unbf{something} (that is) \unbf{useful}
\item \unbf{Anyone} (who is) \unbf{intelligent} can do it.
\item I want to try on \unbf{something} (that is) \unbf{larger}.
\item We're not going \unbf{anywhere} \unbf{very exciting}.
\end{itemize}


带有补足语的形容词通常不能放在定语位置上,而是要求放在名词的后面,可以把这种后
置结构看成是缩简的关系分句(去掉了重复主语代词连接词和be动词的关系分句):
\begin{itemize}
\item I know an actor (who is) suitable for the part.
\item They have a house (which is) larger than yours.
\item The boys (who were) easiest to teach were in my class.
\end{itemize}

还有其他形容词后置情况,本书暂不论述。

\subsection{可作名词短语中心语的形容词}
\begin{itemize}
\item 凡是\textbf{能前置修饰人称名词的形容词} (the young people),可以作名词短语中心
  词 (the young). 这些中心词具有\textbf{复数}和\textbf{类指}的含义,指各种不同类别、种类或类型的
  人。
  \begin{itemize}
  \item \unbf{The young (people) in spirit} enjoy life.

  \item This is a system in which \unbf{the rich (people)} are cared for and
    \unbf{the poor (people)} are left to suffer.

    这是一个富人得到照顾,穷人受苦的制度。suffer \doulos{/ˈsʌfə(r)/} 受苦、受难。
  \end{itemize}
\item 一些表示民族的形容词可以作名词短语的中心词:
  \begin{itemize}
  \item You \unbf{French} and we \unbf{British} ought to be allies.
  \end{itemize}
\item 有些形容词可以用作含有\textbf{抽象意义}的名词短语的中心词,特别是一些\textbf{最高级形
    容词}。我们有时可以将其之后的\unbf{thing}省略。因抽象,后面动词用\unbf{单
    数}。
  \begin{itemize}
  \item The \unbf{latest (thing/news)} is that he is going to run for re-election.

  \item The \unbf{best (thing)} is yet to come.

  \item in \unbf{common (thing)}
  \end{itemize}

\end{itemize}

\subsection{副词的词态分类}

由千副词包罗万象,类别繁多,所以副词是传统词类中最模糊不清、最令人困惑的词类。
的确,不如干脆说,副词不像其他词类那样,可以有确切的定义。因此,有的语法学家把副
词中的某些类型的词全部。

从形态上来说,我们可以把副词分为三种类型。其中两种是封闭类(简单副词和复合副
词);另一种是开放类(派生副词):
\begin{description}
\item[简单副词] 如 just, only, well。许多简单副词表示位置和方向,如 back, down,
  near, out, under。

\item[复合副词] 如 somehow, somewhere, therefore;和文体上极为正式
  的whereupon, hereby, herewith, whereto.

\item[派生副词] 大多数派生副词有 -ly 后缀。

  新副词就是通过形容词(和分词形容词)加上 -ly 后缀产生出来的,如oldly,
  slowly等。

  其他一些不常见的派生后缀有(见\cref{tab:mainsuffix}):
  \begin{description}
  \item[-wise] clockwise
  \item[-ward(s)] northwards, towards
  \item[-style] cowboy-style
  \item[-fashion] schoolboy-fashion
  \end{description}

\end{description}

\subsubsection{形容词构成开放式 -ly 副词的规则}
\begin{itemize}
\item 以辅音字母 + -le 结尾的形容词,e 变y,如simple \~{} simply \qquad comfortable
  \~{} comfortably 。 例外有whole \~{} wholly。

\item 以辅音字母 +y 结尾的形容词,常常把 y 改成 i 再加 -ly。如happy \~{} happily。

  但有些词可以有两种拼法: dry \~{} drily/dryly \qquad sly \~{} slily/slyly。

  而另一些词构成副词时,要保留原来的 -y: spry \~{} spryly \qquad wry \~{} wryly

  请注意下列元音字母 + -e形容词的副词,如 coy \~{} coyly 但是gay \~{} gaily
  \qquad due \~{} duly \qquad true~truly.

\item ic 和 -ical 结尾的形容词可以构成以 -ically 结尾的相应副词:
  economic(al) \~{} economically \qquad tragic(al) \~{} tragically
  只有public \~{} publicly是例外。

\item 以 -ed结尾的分词可以构成以 -edly结尾的副词,但应读为\doulos{/ɪdli/}

  marked \doulos{/mɑːkt/} \qquad \doulos{/ˈmɑːkɪdli/}

  learned \doulos{/ˈlɜːnɪd/} \~{} learnedly \doulos{/ˈlɜːnɪdli/}
\end{itemize}

\subsection{附加副词和连词}
\subsubsection{联加副词和连词}

so 和 yet等可作联加副词, 在作连接词用和句法特点这两个方面都和并列连词and,
but相似。但其次序固定。

\begin{itemize}
\item We paid him a large amount of money. \unbf{So} he kept quiet about what he saw.

  以上两个分句次序如果颠倒,意思就不对了。原因指向了前文。从属连词because引导
  的分句没有这种问题。
\end{itemize}

联加副词前可以加并列连词,如前文可改为:
\begin{itemize}
\item We paid him a large amount of money, \unbf{and so} he kept quiet about what he saw.
\end{itemize}

\subsubsection{附加副词和连词}

when [时间] , where [在什么地方或去什么地方], how [方式] , why [理由]等从属连
词可以看成是溶合了连词和附加状语代用式 (pro-adjunct) 的特点。

\textbf{where和when引导状语分句}:
\begin{itemize}
\item He saw them
    $\left\{
      \begin{aligned}
        &\text{when}\\
        &\text{at the time(s) at which}
      \end{aligned}
      \right\} $ they were in Rome.

\item I'll go
    $\left\{
      \begin{aligned}
        &\text{where}\\
        &\text{to the place(s) to which}
      \end{aligned}
      \right\} $ they go.

\item We'll go
    $\left\{
      \begin{aligned}
        &\text{where}\\
        &\text{to the place(s) at which}
      \end{aligned}
      \right\} $ it is comfortable.

      at强调的是地点,较为静态;to强调的是方向和目的,较为动态。
\end{itemize}

where, when和用的较少的why还可以\textbf{引导关系分句(形容词分句)}:
\begin{itemize}
\item the place
    $\left\{
      \begin{aligned}
        &\text{where}\\
        &\text{at which}
      \end{aligned}
      \right\} $ he is staying.

\item the time
    $\left\{
      \begin{aligned}
        &\text{when}\\
        &\text{at which}
      \end{aligned}
      \right\} $ she was there.
\end{itemize}

where, when, why和how都可以\textbf{引导名词性分句}:
\begin{itemize}
\item I know
    $\left\{
      \begin{aligned}
        &\text{where}\\
        &\text{at which place}
      \end{aligned}
      \right\} $ he is staying.

\item I wonder
    $\left\{
      \begin{aligned}
        &\text{when}\\
        &\text{at which time}
      \end{aligned}
      \right\} $ she was here.

\item I realize
    $\left\{
      \begin{aligned}
        &\text{why}\\
        &\text{the reason for which}
      \end{aligned}
      \right\} $ he did it.

\item That was
    $\left\{
      \begin{aligned}
        &\text{how}\\
        &\text{the way in which}
      \end{aligned}
      \right\} $ they treated her.
\end{itemize}

这四个以 wh- 开头的词也可用作疑问代用式,其中,where = at what place, when =
at what time, why = for what reason, how = in what way。最为常用不再给出例
句。

\subsection{副词小品词up, down, back, away等}

down, in, up 等有时不是介词,而是副词。如以下句子中,左边介词(后接宾语),右
边为副词小品词(无宾语)。
\begin{taskitem}(2)
* I ran \unbf{down} the road.
* Please sit \unbf{down}.
* Something's climbing \unbf{up} my leg.
* She's not \unbf{up} yet.
* He's \unbf{in} his office.
* You can go \unbf{in}.
\end{taskitem}

这种短小的副词通常被叫作“副词小品词”,包括:about, above, across, ahead,
along, (a)round, aside, away, back, before, behind, below, by, down,
forward, home, in, near, off, on, out, over, past, through, under, up.

许多词既可用作副词小品词,也可用作介词,但也有例外: back, away (只能作副词小
品词); from, during (只能是介词).

副词小品词往往与动词连用,构成双词动词,有时会有全新意思(如break down, put
off, work out, give up),通常被叫做“\textbf{短语动词}”

副词小品词和形容词一样,往往用作be动词的补语。
\begin{itemize}
\item Why are all the lights \unbf{on}?
\item The match will be \unbf{over} by 4.30.
\item Hello! You're \unbf{back}!
\item I'm \unbf{off} – see you later!
\end{itemize}

\subsection{形容词和副词的比较级}

\subsubsection{可分等级的形容词和副词类型}

可分等级的形容词和副词可以有三种类型的比较,即:
\begin{description}
\item[向较高程度的比较] 通过屈折变化 -er 和 -est;迂回法 more和most的比较级、最高
  级表示(见\cref{tab:comparison})。

\item[相同程度的比较] 通过as/so \ldots{} as表示。
  \begin{itemize}
  \item Anna is \unbf{as tall as} Bill.
  \item Anna is not \unbf{as/so tall as} Bill.
  \end{itemize}

\item[向较低程度的比较] 通过little的比较级less 和最高级 least表示。
  \begin{itemize}
  \item This problem is \unbf{less difficult} than the previous one.

    这个问题比上题难度低。
  \item This is the \unbf{least difficult} problem of all.

    这是所有问题中最简单(不困难)的。
  \end{itemize}
\end{description}

\begin{table}[htbp!]
  \centering \small
  \begin{talltblr}[ caption = {形容词和副词的比较级},
    label = {tab:comparison},
    ]{
      width=\linewidth, colspec={llll},
      rowsep=2pt, colsep=4pt,
      row{1} = {font=\bfseries},
    }
    \toprule
    & 原级 & 比较级 & 最高级 \\\midrule
    \textbf{屈折变化形式} \\
    形容词 & high & higher & highest \\
    副词 & soon & sooner & soonest \\ \hline
    \textbf{迂回法形式} \\
    形容词 & complex & more complex & most compex \\
    副词 & comfortably & more comfortably & most comfortably \\
    \bottomrule
  \end{talltblr}%
\end{table}

\subsubsection{不规则的比较级、最高级形式}

不规则的比较级、最高级形式较少(见\cref{tab:composison})。

\begin{table}[htbp]
  \centering \small
  \begin{talltblr}[ caption = {不规则的比较级和最高级},
    label = {tab:composison},
    ]{
      width=\linewidth, colspec={lll},
      rowsep=2pt, colsep=4pt,
      row{1} = {c, font=\bfseries},
    }
    \toprule
    原型 & 比较级 & 最高级 \\  \midrule
    bad/sick/evil & worse & worst \\
    far(通用,进一步)& further & furthest  \\
    far (时空距离远) & farther & farthest \\
    good/well & better & best \\
    in & inner & innermost \\
    little & less & least \\
    many/much/a lot & more & most \\
    old & older/elder & oldest/eldest \\
    out & outer & outermost \\
    \bottomrule
  \end{talltblr}%
\end{table}

其中,farther/farthest 和 further /furthest 这两组词既是形容词又是副词。
farther/farthest主要只用来表达物理时空距离较远、最远。further /furthest则囊
括上述含义,并且还可以有“进一步,较多,最近” (more, additional, later) 的意思。

\begin{itemize}
\item I have to travel \unbf{further/farther} to work now.

  现在我得走更远的路去上班。

\item Let's consider this point \unbf{further}.

  让我们更深入地考虑这一点。

\item The school will be closed until \unbf{further} notice.

  学校将关闭,直至进一步的通知。
\end{itemize}

elder其实不是真正的比较级形式,因为在它后面不能跟 than,而要用规则屈折变
化older。elder只能指人,并多用在家庭成员出生顺序,如elder brother/sister表示
哥哥、姐姐。

\subsubsection{规则的比较级屈折变化}

\begin{itemize}
\item 以单个元音字母+单个辅音字母结尾的形容词,先双拼辅音字母,再加 -er和-est 。

  big \~{} bigger \~{} biggest \qquad sad \~{} sadder \~{} saddest

\item 以辅音字母 + y 结尾的形容词,先把 y 改为 i,再加 -er和-est。

  angry \~{} angrier \~{} angriest \qquad early \~{} earlier \~{} earliest

\item 词尾以哑音 -e 结尾,去掉 e,再加 -er和-est。

  pure \~{} purer \~{} purest \qquad brave \~{} braver \~{} bravest

  -ee结尾的去掉末尾的e,再加 -er和-est。

  free \~{} freer \~{} freest

\end{itemize}

\subsubsection{屈折法比较和迂回法比较之间的选择}
\begin{itemize}
\item 一般来讲,单音节形容词通常用屈折变化。

  例外是real, right, wrong 和介词 like只用迂回形式来构成比较级和最高级。

\item 大部分双音节形容词既可以用屈折变化,也可用迂回法。

  对于以 -ing, -ed, -ful, -less等类复合词的双音节形容词来说,只能用迂回法。

\item 三个及以上音节的形容词,只能用迂回法。

  带否定前缀 un- 的形容词两者都可用:
  \begin{itemize}
  \item unhappy \~{} unhappier / more unhappy \~{} unhappiest / most unhappy

  \item untidy \~{} untidier / more untidy \~{} untidiest/ most untidy
  \end{itemize}
\item 以 -ly 结尾的开放式副词可能因音节数量问题,不能用屈折变化,只能用迂回法。
\end{itemize}

\subsubsection{比较级和最高级中冠词的用法}

\begin{itemize}
\item 比较级出现在than结构中,一般不用加the。


\item 最高级 + of (all) 结构中,最高级前要加the。

\item 最高级作定语,修饰名词中心语,最高级前要加the或其他定指限定词。
  \begin{itemize}
  \item Anna is \unbf{the/their youngest} child.

  \item Della is the/our most efficient publisher. \quad efficient
    \doulos{/ɪˈfɪʃnt/} 效率高的;有功效的
  \end{itemize}

  如果形容词不是起定语作用,the 就可有可无:
  \begin{itemize}
  \item Anna is (the) youngest (of all).
  \end{itemize}
\end{itemize}

重复和并列比较级表示程度逐渐增强,不加冠词。
\begin{itemize}
\item She is getting better and better.

\item They are becoming more and more difficult.
\end{itemize}


\subsubsection{比较级的前置修饰语}

形容词和副词的原级可为强化语(如 very, quite, so 等)所前置修饰。
\begin{itemize}
\item The job was \unbf{very easy}.
\end{itemize}

形容词和副词的比较级,不论是屈折变化形式,还是迂回形式,都可由增强
语(如 much, far 或 very much) 前置修饰:
\begin{itemize}
\item The job was \unbf{(very) much/far easier(more difficult)} than I thought.
\item She writes \unbf{(very) much/far better} than she used to.
\end{itemize}

下面是常常与比较级连用的其他强化语(和强化名词短语):
\begin{itemize}
\item somewhat/rather easier than \ldots{}
\item a lot/great/good/ easier than \ldots{}
\end{itemize}

\section{状语的语义和语法}

\subsection{状语按语义分类}

状语按语义可分为以下几类(也是状语的作用)
\begin{enumerate}
\item \textbf{空间}

  \begin{description}
  \item[位置] The dog was asleep \unbf{on the grass}.
  \item[方向] They walked \unbf{down the hill}.
  \item[目标] She hurried \unbf{to the station}.
  \item[来源] This book cannot be taken \unbf{from the library}.
  \item[距离] We mustn't go \unbf{very much further}.
  \end{description}
\item \textbf{时间}

  \begin{description}
  \item[固定时间位置] She was born \unbf{in 1980}.
  \item[前跨延续] I shall be in Chicago \unbf{until Thursday}.

    以“现在时间”为基点,向前跨越。
  \item[后跨延续] We have been at the airport \unbf{since yesterday}.

    以“现在时间”为基点,向前跨越。或者说从过去某时间点到现在。

  \item[时间频度] They \unbf{very seldom} went to see their parents.
  \item [一个时间和另一个时间的关系] She must \unbf{still} be in her office.
  \end{description}

\item \textbf{方式过程}

  \begin{description}
  \item[方式] The minister explained his policy \unbf{very clearly}.
  \item[手段] \unbf{By her insight}, she grasped the patient's real problem.
  \item[工具] I have difficulty eating \unbf{with chopsticks}.
  \item[施事] Penicillin was discovered \unbf{by Sir Alexander Fleming}.
  \end{description}

\item \textbf{方面}, 用状语增加具体真实价值。
  \begin{itemize}
  \item She helped him \unbf{with his research}.

    她帮助他做研究。
  \item He's busy writing.
  \end{itemize}
\item \textbf{原因}
  \begin{description}
  \item[原因] She died \unbf{of cancer}.
  \item[理由] He bought the book \unbf{through an interest in China}.
  \item[目的] He bought the book \unbf{to study English}.
  \item[结果] He always studies hard, \unbf{so he has good grades}.
  \item[条件] \unbf{If he always studies hard}, he will have good grades.
  \item[让步] \unbf{Even though he studied hard}, he didn't have good grades.
  \end{description}

\item \textbf{情态},可以使用状语来改变句子的真实性(如增强或减弱)。
  \begin{description}
  \item[强调] She \unbf{certainly} helped him with his research.

  \item[近似] They are \unbf{probably} going to the zoo.

  \item[限制] I shall be in Chicago \unbf{only} until Thursday.
  \end{description}

\item \textbf{程度},程度状语在改变句子的真实性上与情态状语类似,但是,程度状语添加了一
  个特殊的语义成分,可分等级性。
  \begin{description}
  \item[增强语义] He \unbf{badly} needed consolation.

    他急需安慰。badly在这里是非常,很,严重的意思。

  \item[减弱语义] She helped him \unbf{a little} with his research.
  \end{description}
\end{enumerate}

\subsection{可构成状语的词类}

状语成分可以由很多词类来实现:
\begin{description}
\item[封闭类副词为中心词的副词短语] \unbf{(Just) then}, the telephone rang.
\item[以开放类副词为中心词的副词短语] You should have opened it \unbf{(a bit more) carefully}.
\item[名词短语] They had travelled \unbf{a very long way}.
\item[介词短语] Tom hurried \unbf{across the field}.
\item[无动词分句] \unbf{When in doubt} the answer is ``no''. doubt, 疑问。
\item[非限定性分句] \unct{She}{S} \unct{realized}{V}, \unct{lying there}{A}, \unct{what she must do}{O}.
\item[限定性分句] We sent for you \unbf{because you were absent yesterday}.

  我们叫你来是因为你昨天缺席了。
\end{description}

\subsection{状语的位置}

与其他句子成分相比,状语成分可以比较自由地被置于句内各个不同的位置上(简单了
解即可):
\begin{description}
\item[I] \unbf{by then} the book should have been returned to the library.
\item[iM] The book \unbf{by then} should have been returned to the library.
\item[M] The book should \unbf{by then} have been returned to the library.
\item[mM] The book should have \unbf{by then} been returned to the library.
\item[eM] The book should have been \unbf{by then} returned to the library.
\item[iE] The book should have been returned \unbf{by then} to the library.
\item[E] The book should have been returned to the library \unbf{by then}.
\end{description}

如上文中的符号所示,状语可位于句中三个主要位置:句首位置I(NITIAL),句中位
置M(EDIAL),句末位置E(ND),但是,句中位置又分有三个变体(句中首位iM,句中中
位mM和句中末位eM)以及句末位置下分的句末首位(iE)。\textbf{句中位置就是紧接在功能词
  或系词后面的位置。}

若不存在功能词,那么M的位置就简单的处于S和V 之间;若S被省略,M的位置则位于V的
前面。

状语位置的选择由语义和语法因素来决定,但是同时也由信息处理的要求和末端
重 (end weight)原则来决定。如果没有特殊因素需要考虑,状语应被置于E(句末位
置),事实上,状语多数被置于这个位置。

\subsubsection{各类状语位置}

连接状语 Connecting adverbials 和评论状语 comment adverbials多表示本句与其他
句子的关系,或者评价本句,所以通常放在句首:
\begin{itemize}
\item \unbf{However}, not everybody agreed.

\item \unbf{Fortunately}, nobody was hurt.
\end{itemize}

Adverbials of indefinite frequency, certainty and completeness

不定频度状语(always, often等),确定性状语(probably, definitely等) 和完整性状
语 (completely, almost等) 通常放在句中。
\begin{itemize}
\item My boss \unbf{often} travels to America
\item I've \unbf{definitely} decided to change my job.
\item There is \unbf{clearly} something wrong.
\item The builder said he had \unbf{almost} finished, but it wasn't true.
\item \unbf{Sometimes} I'd like to live alone somewhere else alone.
\end{itemize}

焦点状语Focusing adverbials (also, just, even等)可以放在句中或其他位置,依具
体状语而定。
\begin{itemize}
\item He's \unbf{even} been to Antarctica.

\item We are \unbf{only} going for two days.


\item \unbf{Once} you could do a thing like that.

  只有你才会做出那样的事。
\end{itemize}

Adverbials of manner (how), place (where) and time (when) most often go in end position.

方式、地点和时间状语通常放在句中:
\begin{itemize}
\item She brushed her hair \unbf{slowly}.
\item The children are playing \unbf{upstairs}.
\item I phoned Alex \unbf{this morning}.
\end{itemize}

时间状语也可以放在句首。
\begin{itemize}
\item \unbf{Tomorrow} I've got a meeting in Cardiff.
\end{itemize}

强调状语Emphasising adverbials (terribly, really等) 通常与其所强调的词放在一
起。
\begin{itemize}
\item I'm \unbf{terribly} sorry about last night.
\end{itemize}

程度状语 Degree adverbials (more, very much, most, a lot, so等) 据其功能位置可变

如有多条状语短句,通常按照方式、地点、时间的顺序排列。
\begin{itemize}
\item Put the butter \unbf{in the fridge} \unbf{at once}. (not … at once in the fridge.)
\item Let's go \unbf{to bed} \unbf{early}. (not … early to bed.)
\item I worked \unbf{hard} \unbf{yesterday}.
\item She sang beautifully \unbf{in the town} \unbf{hall last night}.
\end{itemize}


\section{介词和介词短语}

\subsection{介词补语}

介词连接句子中的两个语言单位,并且说明它们之间的关系。

介词补语通常都是名词
性短语(含名词短语、代词、-ing分句、wh-名词性分句)。

尽管that分句和不定式分句可以起名词作用,但是它们不能作介词补语。

\begin{itemize}
\item I was suprised $ \left\{
    \begin{aligned}
     &\text{\unbf{at} her angry response.}\\
     &\text{\unbf{at} hearing her objecti on.}\\
     &\text{\unbf{at} what she said.}\\
     &\text{to hear her objection.}\\
     &\text{that she responded so angrily. }
    \end{aligned}
  \right. $

\end{itemize}

\subsection{介词后置}

尽管介词一般在它本身补语的前面,但是,在一些情况下,介词必须后置:
\begin{itemize}
\item 带介词动词的被动语态结构,其中主语相当于主动语态里的介词补语:
  \begin{itemize}
  \item \unbf{The car} has been paid \unbf{for}.
  \end{itemize}

\item 介词补语主位化的不定式分句 或 -ing分句:
  \begin{itemize}
  \item \unbf{That man} is unpleasant to work \unbf{with}.

  \item \unbf{His advice} is not worth listening \unbf{to}.
  \end{itemize}

  因上,当疑问词或引导词是介词补语时,介词往往出现在句尾,尤其是非正式用法中。
  \begin{itemize}
  \item \unbf{What} are you looking \unbf{for}?
  \item \unbf{Who} is she talking \unbf{about}?
  \item \unbf{About whom} is she talking? 太正式,日常不大用。
  \item \unbf{What} kind of films are you interested \unbf{in}?
  \item Tell me \unbf{what} you're worried \unbf{about}.
  \end{itemize}

  可是,当名词与疑问词连用一体时,介词不后置。
  \begin{itemize}
  \item \unbf{With} what money? (不能说 \sout{What money with.})
  \end{itemize}

\item 一些简单介词(如through)和多数的复杂介词(如because of, in addition to) 不可
  以被后置。

\end{itemize}

\subsection{简单介词和复杂介词}

最普通的介词是一些单音节词项,如 at, for, in, on, to, with,除非被后
置,它们通常要非重读且元音弱化。

有一些多音节介词,它们中有的向来就是由单音节介词组成的复合词(例如: inside,
with in),有的源于分词(例如: during, concering, granted),有的由其他语言引
入(例如: despite , except)

介词数量的增加主要是由于介词与其他词组成了“复合介词”。复合介词主要有两大类:
\begin{itemize}
\item 在简单介词前带有分词、形容词、副词、或连词,如:owing to, devoid of, away
  from, because of。
\item 在简单介词后带有一个名词或另一个简单介词,如:in charge of, by means of,
  at variance with, in addition to, as a result of。
\end{itemize}

\subsection{表示时间和空间的介词}

较为简单,略。

\subsection{表示原因和目的的介词}

表示\textbf{原因、理由和动机}的介词短语有because of, on account of, for, out of:
\begin{itemize}
\item He lost his job \unbf{because of} his laziness.
\item She was fined \unbf{for} dangerous driving .
\item They died \unbf{from} exposure.
\end{itemize}

表示\textbf{目的、目标和对象},这里最普通的介词是 for:
\begin{itemize}
\item We had better set out \unbf{for} home.

  我们最好动身回家。

\item She is applying \unbf{for} a better job.

  她正在申请一份更好的工作。
\end{itemize}
当补语是生物名词时,for通常带有“\textbf{预定接受者}” 的意味,to 被用于表示“\textbf{实
  际接受者}”:
\begin{itemize}
\item He built a play-pen \unbf{for} the little girl.

\item She address the letter \unbf{to} Jim.
\end{itemize}

还可以使用 from 或 out of表示\unbf{来源}(与目的相反):
\begin{itemize}
\item I don't like to borrow \unbf{from} friends.

\item She let him \unbf{out of} the house.
\end{itemize}

当 as 是“作为某个角色”的意思时,后面出现的短语表名了原因。
\begin{itemize}
\item As a doctor, I ought to help you.
\end{itemize}

\subsection{由手段到刺激因素}


\section{简单句}

句子的构成:
\begin{description}
\item[简单句] 由单一的独立分句构成句子。
\item[多重句] 由一个或者多个分句作为直接成分。
  \begin{description}
  \item [联合句] 由两个或两个以上并列分句(COORDINATE clauses)构成的句子。
  \item [复合句] 由从属分句 (SUBORDINATE clause) 来担任句子中一个或一个以上成分
    (例如直接宾语或状语)的句子。
  \end{description}
\end{description}

\begin{itemize}
\item You can borrow the car \unct{that belongs to my sister}{C}.
\item You can borrow \unct{the car that belongs to my sister}{C}.
\end{itemize}

和本书不同“简单句“这一术语在其他语法书中经常用来指
一个不包含另一分句的独立分句,而不管所包含的分句是不是句
子的直接成分。有些语法书,把非限定结构(这种结构含有一个非
限定动词作为动词成分)看作是短语而不是分句。我们则把这种结
构看作是分句,因为可以把它分解为分句成分(参见 14.5). 非哏定
分句本身就是从属性质的,因此不能成为典型的简单句形式(另见














%%% Local Variables:
%%% mode: latex
%%% TeX-master: "main"
%%% End:
