\section{形容词和副词}

\subsection{定语和表语}

能做定语或表语是形容词的主要特点。

\begin{description}
\item[定语 (Attributive)] 置于限定词(包括零冠词)和名词或代词之间,修饰名词或
  代词。
  \begin{itemize}
  \item The \unbf{red} car
  \item A very \unbf{interesting} book
  \end{itemize}

\item[表语(PREDICATIVE)] 主语或宾语的补语,用来说明主语性质、状态或身份的成分。
  \begin{itemize}
  \item This car is \unbf{red}.

  \item He thought the painting \unbf{ugly}.
  \end{itemize}
\end{description}

\subsection{以 -ly 结尾的形容词}

一些形容词以 -ly 结尾,但一般不作副词。如costly, cowardly, deadly, friendly,
likely, lively, lonely, lovely, silly, ugly, unlikely.
\begin{itemize}
\item She gave me a friendly smile.

\item Her singing was lovely.
\end{itemize}

friendly, lovely没有副词形式。
\begin{itemize}
\item She smiled in a \unbf{friendly} way. (not \sout{She smiled friendly.})
\item He gave a \unbf{silly} laugh. (not \sout{He laughed silly.})
\end{itemize}

early, hourly, nightly, daily, weekly, monthly, quarterly, yearly and leisurely 既是副词也是形容词。

\subsection{形容词和副词的同音同形异义词}

\begin{itemize}
\item a \unbf{fast} car drive \unbf{fast}
\item They are \unbf{close} friends, and they live \unbf{close} by.

 close取“接近,紧密”的意思,作形容词或副词时,发音是 \doulos{/kləʊs/ /kloʊs/}.
\item She had \unbf{short} hair. She cut her hair \unbf{short}.

\item Did you have to wait a \unbf{long} time? (to wait \unbf{long})

\item a \unbf{slow} car drive \unbf{slow/slowly}
\end{itemize}

有时一个副词可能有两种形式(如late和lately),一种与其形容词形式一样,一种
以ly结尾。这两种形式往往意思不同或用法不同。

一些诅咒语可以既作形容词又作副词,如bloody。
\begin{itemize}
\item 'You \unbf{bloody} fool. You didn't look where you were going.''I \unbf{bloody} did.'

  “你这个笨蛋,你没有看你往哪走吗?”“我他妈看了。”
\end{itemize}

dead作副词时,意思是“的确,完全,非常”,例如dead ahead, dead certain, dead
drunk, dead right, dead slow, dead straight, dead sure, dead tired.

deadly则是形容词,意思是“致命的”。表达这个意思的副词是fatally.

easy 在非正式词组里作副词:
\begin{itemize}
\item Go \unbf{easy}! (= Not too fast!)
\item \unbf{Easy} come, \unbf{easy} go.
\item Take it \unbf{easy}! (= Relax!)
\item \unbf{Easier} said than done.
\end{itemize}

fair也可用在某些词组的动词后面,用作副词。
\begin{itemize}
\item to play \unbf{fair}, to fight \unbf{fair}
\item to hit/win something \unbf{fair} and square

  fairly一般修饰形容词和副词,表示并不怎样高,凑合的意思。

\item `How was the film?' '\unbf{Faily} good.' 还好,还凑合。

\item I speak English \unbf{fairly} well -- enough for everyday purposes.

  我中文还好——足够应付日常需要。
\end{itemize}


fine在非正式文本中用作副词,等同于well。
\begin{itemize}
\item That suits me \unbf{fine}.

\item You're doing \unbf{fine}.
\end{itemize}
副词 finely 通常用作表示细微的调整。
\begin{itemize}
\item a \unbf{finely} tuned engine

\item \unbf{finely} chopped onions (= cut up very small)
\end{itemize}

free 在动词后面用作副词时意思是“免费”; freely 意思是“无限制地”:
\begin{itemize}
\item You can eat \unbf{free} in my restaurant whenever you like.

\item You can speak \unbf{freely} – I won't tell anyone what you say.
\end{itemize}

hard副词“用力地努力地”;hardly意思是“几乎不”:
\begin{itemize}
\item Hit it \unbf{hard}.
\item I trained really \unbf{hard} for the marathon.
\item I've \unbf{hardly} got any clean clothes left.

  我几乎没有什么干净衣服了。
\item Anna works \unbf{hard}, but her elder sister \unbf{hardly} works.

  安娜努力工作,但她的姐姐几乎不工作。
\end{itemize}

high意思“高度高”; highly意思“非常,高级的,赞赏的”:
\begin{itemize}
\item He can jump really \unbf{high}.
\item Throw it as \unbf{high} as you can.
\item It's \unbf{highly} amusing. 这非常有趣。
\item I can \unbf{highly} recommend it. 我极力推荐它。
\end{itemize}

just有很多个意思:
\begin{description}
\item[exactly] 正好;恰好
  \begin{itemize}
  \item This jacket is \unbf{just} my size.

    这件夹克正合我的尺码。

  \item You're \unbf{just} in time.

    你来得正是时候。

  \item It's \unbf{just} what I wanted!

    这正是我想要的!
  \end{itemize}

\item[at this moment] 此时,刚刚
  \begin{itemize}
  \item I've \unbf{just} heard the news.

    我刚听到这个消息。

  \item When you arrived he had only \unbf{just} left.

    你到时他刚走。
  \item She has \unbf{just} been telling us about her trip to Rome.

    她刚才一直在给我们讲她的罗马之行。
  \end{itemize}

\item[only, scarcely] 只,简直不
  \begin{itemize}
  \item Complete set of garden tools for \unbf{just} £15.99!
  \item I \unbf{just} want somebody to love me – that's all.
  \end{itemize}

\item[emphasiser] 强调其他词句
  \begin{itemize}
  \item You're \unbf{just} beautiful.

  \item I \unbf{just} love your dress.
  \end{itemize}
\end{description}
此外还有形容词just,其副词为justly,意思是“公平地,正义地”:
\begin{itemize}
\item He was \unbf{justly} punished for his crimes.
\end{itemize}

形容词和副词late意思相近,“迟到”;副词lately“最近地”:
\begin{itemize}
\item I hate arriving \unbf{late}.

\item I haven't been to the theatre much \unbf{lately}.
\end{itemize}

形容词和副词low意思几乎一致:
\begin{itemize}
\item a \unbf{low} bridge \qquad bend \unbf{low} 深弯腰
\end{itemize}

most是much的最高级,也可以用来构成形容词和副词的最高级。
\begin{itemize}
\item Which part of the concert did you like \unbf{most}?
\item This is the \unbf{most} extraordinary day of my life.
\end{itemize}
在正式文体中,可以作
“非常地”用。
\begin{itemize}
\item You're a \unbf{most} unusual person.
\end{itemize}
Mostly 意思是“主要 (mainly),大都 (most often)或在大多数场合 (in \unbf{most} cases)”
\begin{itemize}
\item My friends are \unbf{most}ly non-smokers.
\end{itemize}


right与状语连用,意思是
She arrived right after breakfast.
The snowball hit me right on the nose.
Turn the gas right down.
Right and rightly can both be used to mean 'correctly'. Right is only used after verbs, and is usually informal. Compare:
I rightly assumed that Henry was not coming.
You guessed right.
It serves you right. ( … rightly is not possible.)
sharp Sharp can be used as an adverb to mean 'punctually'.
Can you be there at six o'clock sharp?
It also has a musical sense (to sing sharp means 'to sing on a note that is too high'), and is used in the expressions turn sharp left and turn sharp right (meaning 'with a big change of direction').
In other senses the adverb is sharply.
She looked at him sharply.
I thought you spoke to her rather sharply.
short Short is used as an adverb in the expressions stop short (= 'stop suddenly') and cut short (= 'interrupt'). Shortly means 'soon'; it can also describe an impatient way of speaking.
slow Slow is used as an adverb in road signs (e.g. slow – dangerous bend), and informally after go and some other verbs. Examples: go slow, drive slow.
sound Sound is used as an adverb in the expression sound asleep. In other cases, soundly is used (e.g. She's sleeping soundly).
straight The adverb and the adjective are the same. A straight road goes straight from one place to another.
sure Sure is often used to mean 'certainly' in an informal style, especially in American English.
'Can I borrow your tennis racket?''Sure.'
Surely (not) is used to express opinions or surprise (600 for details).
Surely house prices will stop rising soon!
Surely you're not going out in that old coat?
tight After a verb, tight can be used instead of tightly, especially in an informal style. Typical expressions: hold tight, packed tight (compare tightly packed).
well Well is an adverb corresponding to the adjective good (a good singer sings well). Well is also an adjective meaning 'in good health' (the opposite of ill). For details, 622.
wide The normal adverb is wide; widely suggests distance or separation. Compare:
The door was wide open.
She's travelled widely.
They have widely differing opinions.
Note also the expression wide awake (the opposite of fast asleep).
wrong Wrong can be used informally instead of wrongly after a verb. Compare:
I wrongly believed that you wanted to help me.
You guessed wrong.
3	comparatives and superlatives
Informal uses of adjective forms as adverbs are especially common with comparatives and superlatives.
Can you drive a bit slower?
Let's see who can do it quickest.
4	American English
In informal American English, many other adjective forms can also be used as adverbs of manner.
He looked at me real strange.
Think positive.












%%% Local Variables:
%%% mode: latex
%%% TeX-master: "main"
%%% End:
