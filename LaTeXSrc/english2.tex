\section{形容词和副词}

\subsection{定语和表语}
\label{sub:attrpred}

能做定语或表语是形容词的主要特点。

\begin{description}
\item[定语 (Attributive)] \index{概念!前置修饰@前置修
    饰 premodification}\index{概念!定语@定语 attributive}置于限定词(包括零冠
  词)和名词代词之间,修饰名词或代词的成分,前置修饰作用(另可
  见\cref{sub:nounimal})。
  \begin{itemize}
    形容词作定语:如“a small table”(一张小桌子),其中“small”是前置修饰语。
    代词作定语:如“this book”(这本书),其中“this”是前置修饰语。
    数词作定语:如“three boys”(三个男孩),其中“three”是前置修饰语。
    名词作定语:如“car factory”(汽车厂),其中“car”作为名词也可以充当前置修饰语。
    名词所有格作定语:如“Peter’s car”(彼得的车),其中“Peter’s”是前置修饰语。
  \end{itemize}

\item[表语(PREDICATIVE)] \index{概念!表语 predicative} 起主语或宾语补足语的
  作用,用来说明主语或宾语其性质、状态、身份的成分。
  \begin{itemize}
  \item This car is \unbf{red}.

  \item He thought the painting \unbf{ugly}.
  \end{itemize}
\end{description}

\subsection{以 -ly 结尾的形容词}

一些形容词以 -ly 结尾,但一般不作副词。如costly, cowardly, deadly, friendly,
likely, lively, lonely, lovely, silly, ugly, unlikely.
\begin{itemize}
\item She gave me a friendly smile.

\item Her singing was lovely.
\end{itemize}

friendly, lovely没有副词形式。
\begin{itemize}
\item She smiled in a \unbf{friendly} way. (not \sout{She smiled friendly.})
\item He gave a \unbf{silly} laugh. (not \sout{He laughed silly.})
\end{itemize}

early, hourly, nightly, daily, weekly, monthly, quarterly, yearly and leisurely 既是副词也是形容词。

\subsection{形容词和副词的同音同形异义词}

\begin{itemize}
\item a \unbf{fast} car drive \unbf{fast}
\item They are \unbf{close} friends, and they live \unbf{close} by.

 close取“接近,紧密”的意思,作形容词或副词时,发音是 \doulos{/kləʊs/ /kloʊs/}.
\item She had \unbf{short} hair. She cut her hair \unbf{short}.

\item Did you have to wait a \unbf{long} time? (to wait \unbf{long})

\item a \unbf{slow} car drive \unbf{slow/slowly}
\end{itemize}

有时一个副词可能有两种形式(如late和lately),一种与其形容词形式一样,一种
以ly结尾。这两种形式往往意思不同或用法不同。

一些诅咒语可以既作形容词又作副词,如bloody。
\begin{itemize}
\item 'You \unbf{bloody} fool. You didn't look where you were going.''I \unbf{bloody} did.'

  “你这个笨蛋,你没有看你往哪走吗?”“我他妈看了。”
\end{itemize}

dead作副词时,意思是“的确,完全,非常”,例如dead ahead, dead certain, dead
drunk, dead right, dead slow, dead straight, dead sure, dead tired.

deadly则是形容词,意思是“致命的”。表达这个意思的副词是fatally.

easy 在非正式词组里作副词:
\begin{itemize}
\item Go \unbf{easy}! (= Not too fast!)
\item \unbf{Easy} come, \unbf{easy} go.
\item Take it \unbf{easy}! (= Relax!)
\item \unbf{Easier} said than done.
\end{itemize}

fair也可用在某些词组的动词后面,用作副词。
\begin{itemize}
\item to play \unbf{fair}, to fight \unbf{fair}
\item to hit/win something \unbf{fair} and square

  fairly一般修饰形容词和副词,表示并不怎样高,凑合的意思。

\item `How was the film?' '\unbf{Fairly} good.' 还好,还凑合。

\item I speak English \unbf{fairly} well -- enough for everyday purposes.

  我中文还好——足够应付日常需要。
\end{itemize}


fine在非正式文本中用作副词,等同于well。
\begin{itemize}
\item That suits me \unbf{fine}.

\item You're doing \unbf{fine}.
\end{itemize}
副词 finely 通常用作表示细微的调整。
\begin{itemize}
\item a \unbf{finely} tuned engine

\item \unbf{finely} chopped onions (= cut up very small)
\end{itemize}

free 在动词后面用作副词时意思是“免费”; freely 意思是“无限制地”:
\begin{itemize}
\item You can eat \unbf{free} in my restaurant whenever you like.

\item You can speak \unbf{freely} – I won't tell anyone what you say.
\end{itemize}

hard副词“用力地努力地”;hardly意思是“几乎不”:
\begin{itemize}
\item Hit it \unbf{hard}.
\item I trained really \unbf{hard} for the marathon.
\item I've \unbf{hardly} got any clean clothes left.

  我几乎没有什么干净衣服了。
\item Anna works \unbf{hard}, but her elder sister \unbf{hardly} works.

  安娜努力工作,但她的姐姐几乎不工作。
\end{itemize}

high意思“高度高”; highly意思“非常,高级的,赞赏的”:
\begin{itemize}
\item He can jump really \unbf{high}.
\item Throw it as \unbf{high} as you can.
\item It's \unbf{highly} amusing. 这非常有趣。
\item I can \unbf{highly} recommend it. 我极力推荐它。
\end{itemize}

just有很多个意思:
\begin{description}
\item[exactly] 正好;恰好
  \begin{itemize}
  \item This jacket is \unbf{just} my size.

    这件夹克正合我的尺码。

  \item You're \unbf{just} in time.

    你来得正是时候。

  \item It's \unbf{just} what I wanted!

    这正是我想要的!
  \end{itemize}

\item[at this moment] 此时,刚刚
  \begin{itemize}
  \item I've \unbf{just} heard the news.

    我刚听到这个消息。

  \item When you arrived he had only \unbf{just} left.

    你到时他刚走。
  \item She has \unbf{just} been telling us about her trip to Rome.

    她刚才一直在给我们讲她的罗马之行。
  \end{itemize}

\item[only, scarcely] 只,简直不
  \begin{itemize}
  \item Complete set of garden tools for \unbf{just} £15.99!
  \item I \unbf{just} want somebody to love me – that's all.
  \end{itemize}

\item[emphasiser] 强调其他词句
  \begin{itemize}
  \item You're \unbf{just} beautiful.

  \item I \unbf{just} love your dress.
  \end{itemize}
\end{description}
此外还有形容词just,其副词为justly,意思是“公平地,正义地”:
\begin{itemize}
\item He was \unbf{justly} punished for his crimes.
\end{itemize}

形容词和副词late意思相近,“迟到”;副词lately“最近地”:
\begin{itemize}
\item I hate arriving \unbf{late}.

\item I haven't been to the theater much \unbf{lately}.
\end{itemize}

形容词和副词low意思几乎一致:
\begin{itemize}
\item a \unbf{low} bridge \qquad bend \unbf{low} 深弯腰
\end{itemize}

most是much的最高级,也可以用来构成形容词和副词的最高级。
\begin{itemize}
\item Which part of the concert did you like \unbf{most}?
\item This is the \unbf{most} extraordinary day of my life.
\end{itemize}
在正式文体中,可以作
“非常地”用。
\begin{itemize}
\item You're a \unbf{most} unusual person.
\end{itemize}
Mostly 意思是“主要 (mainly),大都 (most often)或在大多数场合 (in \unbf{most} cases)”
\begin{itemize}
\item My friends are \unbf{most}ly non-smokers.
\end{itemize}


right与状语连用时,意思是“正好,精确地”:
\begin{itemize}
\item She arrived \unbf{right} \unbf{after breakfast}.
\item The snowball hit me \unbf{right} \unbf{on the nose}.
\item Turn the gas \unbf{right} \unbf{down}.
\end{itemize}
right 和rightly都有“正确地”意思。但副词right只能用在动词后面,并且是非正式
的用法:
\begin{itemize}
\item I rightly assumed that Henry was not coming.
\item You guessed right.
\item It serves you right. ( … rightly is not possible.)
\end{itemize}

straight 副词和形容词形式一样.
\begin{itemize}
\item A \unbf{straight} road goes \unbf{straight} from one place to another.
\end{itemize}

sure 在非正式用法中“当然地” :
\begin{itemize}
\item `Can I borrow your tennis racket?' `\unbf{Sure}.'
\end{itemize}

surely (not) 表示“意见,惊讶于”:
\begin{itemize}
\item Surely house prices will stop rising soon!

  房价肯定会停止上涨!
\item Surely you're not going out in that old coat?

  你该不会穿着那件旧外套出去吧。
\end{itemize}

wide作副词用时表示“宽地”, widely 表示距离很远,或差别很大:
\begin{itemize}
\item The door was wide open.

  当时那扇门大开着。
\item She's traveled widely.

  它曾到处游历。
\item They have widely differing opinions.

  他们的意见大相径庭。
\end{itemize}


\textbf{有些形容词的比较级和最高级形式用作副词}的现象,在规范英语中也很常见。试比较:
\begin{itemize}
\item Speak \unbf{clearer}! [\unbf{more clearly}]
\item This newsreader speaks \unbf{clearest} of all. [\unbf{most clearly}]
\item It's \unbf{easier} said than done. [\unbf{more easily}]
\item Ami ran \unbf{(the) slowest}.
\item The car ran went (the) \unbf{slower and slower}.
\end{itemize}

以下表示时间的 -ly词作形容词、副词皆可: monthly, daily, hourly, nightly,
quarterly, weekly, yearly。

\subsection{以 a- 开头的形容词和副词}

某些以a- 开头的词给语言学家带来了难题。有些语法学家把它们划为形容词,有些语法
学家把它们划为副词。这些以 a- 开头的词起表语作用,但只有几个能随意用作定语.

只有比较少的副词可以且只能在be后面作表语,如地点副词 aboard, upstairs和时间副
词,如 now, tonight 。而形容词却还能和其他系动词连用。试比较系动词 be 和 seem
的不同句型:

$\text{The patient} \left\{
  \begin{aligned}
    &\text{was asleep/hungry/abroad/there.}\\
    &\text{seemed asleep/hungry.(\textbf{形容词,不能是副词}abroad/there等)}
  \end{aligned}
\right.$

a- 形容词不能在非be动词后面。因此我们可以用seem to be支持a- 形容词或副词:
\begin{itemize}
\item They seemed to be abroad/there/around/afraid.
\end{itemize}

\subsection{形容词后置}
后置

形容词有时可以后置 (postpositive), 也就是说,它们能紧跟在所修饰的名词或代词后
面。因此,形容词可以有三种不同的位置:
\begin{description}
\item[表语位置] This information is \unbf{useful}.
\item[定语位置] \unbf{useful} information
\item[后置] something \unbf{useful}
\end{description}

以 -body, -one, -thing, -where 结尾的复合不定代词和复合不定副词(
见\cref{tab:someany})只能用后置形容词来修饰,通常我们可以将其后置形容词(短
语)看成是一个缩简的关系分句:
\begin{itemize}
\item \unbf{something} (that is) \unbf{useful}
\item \unbf{Anyone} (who is) \unbf{intelligent} can do it.
\item I want to try on \unbf{something} (that is) \unbf{larger}.
\item We're not going \unbf{anywhere} \unbf{very exciting}.
\end{itemize}


带有补足语的形容词通常不能放在定语位置上,而是要求放在名词的后面,可以把这种后
置结构看成是缩简的关系分句(去掉了重复主语代词连接词和be动词的关系分句):
\begin{itemize}
\item I know an actor (who is) suitable for the part.
\item They have a house (which is) larger than yours.
\item The boys (who were) easiest to teach were in my class.
\end{itemize}

还有其他形容词后置情况,本书暂不论述。

\subsection{可作名词短语中心语的形容词}
\begin{itemize}
\item 凡是\textbf{能前置修饰人称名词的形容词} (the young people),可以作名词短语中心
  词 (the young). 这些中心词具有\textbf{复数}和\textbf{类指}的含义,指各种不同类别、种类或类型的
  人。
  \begin{itemize}
  \item \unbf{The young (people) in spirit} enjoy life.

  \item This is a system in which \unbf{the rich (people)} are cared for and
    \unbf{the poor (people)} are left to suffer.

    这是一个富人得到照顾,穷人受苦的制度。suffer \doulos{/ˈsʌfə(r)/} 受苦、受难。
  \end{itemize}
\item 一些表示民族的形容词可以作名词短语的中心词:
  \begin{itemize}
  \item You \unbf{French} and we \unbf{British} ought to be allies.
  \end{itemize}
\item 有些形容词可以用作含有\textbf{抽象意义}的名词短语的中心词,特别是一些\textbf{最高级形
    容词}。我们有时可以将其之后的\unbf{thing}省略。因抽象,后面动词用\unbf{单
    数}。
  \begin{itemize}
  \item The \unbf{latest (thing/news)} is that he is going to run for re-election.

  \item The \unbf{best (thing)} is yet to come.

  \item in \unbf{common (thing)}
  \end{itemize}

\end{itemize}

\subsection{副词的词态分类}

由千副词包罗万象,类别繁多,所以副词是传统词类中最模糊不清、最令人困惑的词类。
的确,不如干脆说,副词不像其他词类那样,可以有确切的定义。因此,有的语法学家把副
词中的某些类型的词全部。

从形态上来说,我们可以把副词分为三种类型。其中两种是封闭类(简单副词和复合副
词);另一种是开放类(派生副词):
\begin{description}
\item[简单副词] 如 just, only, well。许多简单副词表示位置和方向,如 back, down,
  near, out, under。

\item[复合副词] 如 somehow, somewhere, therefore;和文体上极为正式
  的whereupon, hereby, herewith, whereto.

\item[派生副词] 大多数派生副词有 -ly 后缀。

  新副词就是通过形容词(和分词形容词)加上 -ly 后缀产生出来的,如oldly,
  slowly等。

  其他一些不常见的派生后缀有(见\cref{tab:mainsuffix}):
  \begin{description}
  \item[-wise] clockwise
  \item[-ward(s)] northwards, towards
  \item[-style] cowboy-style
  \item[-fashion] schoolboy-fashion
  \end{description}

\end{description}

\subsubsection{形容词构成开放式 -ly 副词的规则}
\begin{itemize}
\item 以辅音字母 + -le 结尾的形容词,e 变y,如simple \~{} simply \qquad comfortable
  \~{} comfortably 。 例外有whole \~{} wholly。

\item 以辅音字母 +y 结尾的形容词,常常把 y 改成 i 再加 -ly。如happy \~{} happily。

  但有些词可以有两种拼法: dry \~{} drily/dryly \qquad sly \~{} slily/slyly。

  而另一些词构成副词时,要保留原来的 -y: spry \~{} spryly \qquad wry \~{} wryly

  请注意下列元音字母 + -e形容词的副词,如 coy \~{} coyly 但是gay \~{} gaily
  \qquad due \~{} duly \qquad true~truly.

\item ic 和 -ical 结尾的形容词可以构成以 -ically 结尾的相应副词:
  economic(al) \~{} economically \qquad tragic(al) \~{} tragically
  只有public \~{} publicly是例外。

\item 以 -ed结尾的分词可以构成以 -edly结尾的副词,但应读为\doulos{/ɪdli/}

  marked \doulos{/mɑːkt/} \qquad \doulos{/ˈmɑːkɪdli/}

  learned \doulos{/ˈlɜːnɪd/} \~{} learnedly \doulos{/ˈlɜːnɪdli/}
\end{itemize}

\subsection{附加副词和连词}
\subsubsection{联加副词和连词}

so 和 yet等可作联加副词, 在作连接词用和句法特点这两个方面都和并列连词and,
but相似。但其次序固定。

\begin{itemize}
\item We paid him a large amount of money. \unbf{So} he kept quiet about what he saw.

  以上两个分句次序如果颠倒,意思就不对了。原因指向了前文。从属连词because引导
  的分句没有这种问题。
\end{itemize}

联加副词前可以加并列连词,如前文可改为:
\begin{itemize}
\item We paid him a large amount of money, \unbf{and so} he kept quiet about what he saw.
\end{itemize}

\subsubsection{附加副词和连词}

when [时间] , where [在什么地方或去什么地方], how [方式] , why [理由]等从属连
词可以看成是溶合了连词和附加状语代用式 (pro-adjunct) 的特点。

\textbf{where和when引导状语分句}:
\begin{itemize}
\item He saw them
    $\left\{
      \begin{aligned}
        &\text{when}\\
        &\text{at the time(s) at which}
      \end{aligned}
      \right\} $ they were in Rome.

\item I'll go
    $\left\{
      \begin{aligned}
        &\text{where}\\
        &\text{to the place(s) to which}
      \end{aligned}
      \right\} $ they go.

\item We'll go
    $\left\{
      \begin{aligned}
        &\text{where}\\
        &\text{to the place(s) at which}
      \end{aligned}
      \right\} $ it is comfortable.

      at强调的是地点,较为静态;to强调的是方向和目的,较为动态。
\end{itemize}

where, when和用的较少的why还可以\textbf{引导关系分句(形容词分句)}:
\begin{itemize}
\item the place
    $\left\{
      \begin{aligned}
        &\text{where}\\
        &\text{at which}
      \end{aligned}
      \right\} $ he is staying.

\item the time
    $\left\{
      \begin{aligned}
        &\text{when}\\
        &\text{at which}
      \end{aligned}
      \right\} $ she was there.
\end{itemize}

where, when, why和how都可以\textbf{引导名词性分句}:
\begin{itemize}
\item I know
    $\left\{
      \begin{aligned}
        &\text{where}\\
        &\text{at which place}
      \end{aligned}
      \right\} $ he is staying.

\item I wonder
    $\left\{
      \begin{aligned}
        &\text{when}\\
        &\text{at which time}
      \end{aligned}
      \right\} $ she was here.

\item I realize
    $\left\{
      \begin{aligned}
        &\text{why}\\
        &\text{the reason for which}
      \end{aligned}
      \right\} $ he did it.

\item That was
    $\left\{
      \begin{aligned}
        &\text{how}\\
        &\text{the way in which}
      \end{aligned}
      \right\} $ they treated her.
\end{itemize}

这四个以 wh- 开头的词也可用作疑问代用式,其中,where = at what place, when =
at what time, why = for what reason, how = in what way。最为常用不再给出例
句。

\subsection{副词小品词up, down, back, away等}

down, in, up 等有时不是介词,而是副词。如以下句子中,左边介词(后接宾语),右
边为副词小品词(无宾语)。
\begin{taskitem}(2)
* I ran \unbf{down} the road.
* Please sit \unbf{down}.
* Something's climbing \unbf{up} my leg.
* She's not \unbf{up} yet.
* He's \unbf{in} his office.
* You can go \unbf{in}.
\end{taskitem}

这种短小的副词通常被叫作“副词小品词”,包括:about, above, across, ahead,
along, (a)round, aside, away, back, before, behind, below, by, down,
forward, home, in, near, off, on, out, over, past, through, under, up.

许多词既可用作副词小品词,也可用作介词,但也有例外: back, away (只能作副词小
品词); from, during (只能是介词).

副词小品词往往与动词连用,构成双词动词,有时会有全新意思(如break down, put
off, work out, give up),通常被叫做“\textbf{短语动词}”

副词小品词和形容词一样,往往用作be动词的补语。
\begin{itemize}
\item Why are all the lights \unbf{on}?
\item The match will be \unbf{over} by 4.30.
\item Hello! You're \unbf{back}!
\item I'm \unbf{off} – see you later!
\end{itemize}

\subsection{形容词和副词的比较级}

\subsubsection{可分等级的形容词和副词类型}

可分等级的形容词和副词可以有三种类型的比较,即:
\begin{description}
\item[向较高程度的比较] 通过屈折变化 -er 和 -est;迂回法 more和most的比较级、最高
  级表示(见\cref{tab:comparison})。

\item[相同程度的比较] 通过as/so \ldots{} as表示。
  \begin{itemize}
  \item Anna is \unbf{as tall as} Bill.
  \item Anna is not \unbf{as/so tall as} Bill.
  \end{itemize}

\item[向较低程度的比较] 通过little的比较级less 和最高级 least表示。
  \begin{itemize}
  \item This problem is \unbf{less difficult} than the previous one.

    这个问题比上题难度低。
  \item This is the \unbf{least difficult} problem of all.

    这是所有问题中最简单(不困难)的。
  \end{itemize}
\end{description}

\begin{table}[htbp!]
  \centering \small
  \begin{talltblr}[ caption = {形容词和副词的比较级},
    label = {tab:comparison},
    ]{
      width=\linewidth, colspec={llll},
      rowsep=2pt, colsep=4pt,
      row{1} = {font=\bfseries},
    }
    \toprule
    & 原级 & 比较级 & 最高级 \\\midrule
    \textbf{屈折变化形式} \\
    形容词 & high & higher & highest \\
    副词 & soon & sooner & soonest \\ \hline
    \textbf{迂回法形式} \\
    形容词 & complex & more complex & most complex \\
    副词 & comfortably & more comfortably & most comfortably \\
    \bottomrule
  \end{talltblr}%
\end{table}

\subsubsection{不规则的比较级、最高级形式}

不规则的比较级、最高级形式较少(见\cref{tab:composison})。

\begin{table}[htbp]
  \centering \small
  \begin{talltblr}[ caption = {不规则的比较级和最高级},
    label = {tab:composison},
    ]{
      width=\linewidth, colspec={lll},
      rowsep=2pt, colsep=4pt,
      row{1} = {c, font=\bfseries},
    }
    \toprule
    原形 & 比较级 & 最高级 \\  \midrule
    bad/sick/evil & worse & worst \\
    far(通用,进一步)& further & furthest  \\
    far (时空距离远) & farther & farthest \\
    good/well & better & best \\
    in & inner & innermost \\
    little & less & least \\
    many/much/a lot & more & most \\
    old & older/elder & oldest/eldest \\
    out & outer & outermost \\
    \bottomrule
  \end{talltblr}%
\end{table}

其中,farther/farthest 和 further /furthest 这两组词既是形容词又是副词。
farther/farthest主要只用来表达物理时空距离较远、最远。further /furthest则囊
括上述含义,并且还可以有“进一步,较多,最近” (more, additional, later) 的意思。

\begin{itemize}
\item I have to travel \unbf{further/farther} to work now.

  现在我得走更远的路去上班。

\item Let's consider this point \unbf{further}.

  让我们更深入地考虑这一点。

\item The school will be closed until \unbf{further} notice.

  学校将关闭,直至进一步的通知。
\end{itemize}

elder其实不是真正的比较级形式,因为在它后面不能跟 than,而要用规则屈折变
化older。elder只能指人,并多用在家庭成员出生顺序,如elder brother/sister表示
哥哥、姐姐。

\subsubsection{规则的比较级屈折变化}

\begin{itemize}
\item 以单个元音字母+单个辅音字母结尾的形容词,先双拼辅音字母,再加 -er和-est 。

  big \~{} bigger \~{} biggest \qquad sad \~{} sadder \~{} saddest

\item 以辅音字母 + y 结尾的形容词,先把 y 改为 i,再加 -er和-est。

  angry \~{} angrier \~{} angriest \qquad early \~{} earlier \~{} earliest

\item 词尾以哑音 -e 结尾,去掉 e,再加 -er和-est。

  pure \~{} purer \~{} purest \qquad brave \~{} braver \~{} bravest

  -ee结尾的去掉末尾的e,再加 -er和-est。

  free \~{} freer \~{} freest

\end{itemize}

\subsubsection{屈折法比较和迂回法比较之间的选择}
\begin{itemize}
\item 一般来讲,单音节形容词通常用屈折变化。

  例外是real, right, wrong 和介词 like只用迂回形式来构成比较级和最高级。

\item 大部分双音节形容词既可以用屈折变化,也可用迂回法。

  对于以 -ing, -ed, -ful, -less等类复合词的双音节形容词来说,只能用迂回法。

\item 三个及以上音节的形容词,只能用迂回法。

  带否定前缀 un- 的形容词两者都可用:
  \begin{itemize}
  \item unhappy \~{} unhappier / more unhappy \~{} unhappiest / most unhappy

  \item untidy \~{} untidier / more untidy \~{} untidiest/ most untidy
  \end{itemize}
\item 以 -ly 结尾的开放式副词可能因音节数量问题,不能用屈折变化,只能用迂回法。
\end{itemize}

\subsubsection{比较级和最高级中冠词的用法}

\begin{itemize}
\item 比较级出现在than结构中,一般不用加the。


\item 最高级 + of (all) 结构中,最高级前要加the。

\item 最高级作定语,修饰名词中心语,最高级前要加the或其他定指限定词。
  \begin{itemize}
  \item Anna is \unbf{the/their youngest} child.

  \item Della is the/our most efficient publisher. \quad efficient
    \doulos{/ɪˈfɪʃnt/} 效率高的;有功效的
  \end{itemize}

  如果形容词不是起定语作用,the 就可有可无:
  \begin{itemize}
  \item Anna is (the) youngest (of all).
  \end{itemize}
\end{itemize}

重复和并列比较级表示程度逐渐增强,不加冠词。
\begin{itemize}
\item She is getting better and better.

\item They are becoming more and more difficult.
\end{itemize}


\subsubsection{比较级的前置修饰语}

形容词和副词的原级可为强化语(如 very, quite, so 等)所前置修饰。
\begin{itemize}
\item The job was \unbf{very easy}.
\end{itemize}

形容词和副词的比较级,不论是屈折变化形式,还是迂回形式,都可由增强
语(如 much, far 或 very much) 前置修饰:
\begin{itemize}
\item The job was \unbf{(very) much/far easier(more difficult)} than I thought.
\item She writes \unbf{(very) much/far better} than she used to.
\end{itemize}

下面是常常与比较级连用的其他强化语(和强化名词短语):
\begin{itemize}
\item somewhat/rather easier than \ldots{}
\item a lot/great/good/ easier than \ldots{}
\end{itemize}

\section{状语的语义和语法}

\subsection{状语按语义分类}

状语按语义可分为以下几类(也是状语的作用)
\begin{enumerate}
\item \textbf{空间}

  \begin{description}
  \item[位置] The dog was asleep \unbf{on the grass}.
  \item[方向] They walked \unbf{down the hill}.
  \item[目标] She hurried \unbf{to the station}.
  \item[来源] This book cannot be taken \unbf{from the library}.
  \item[距离] We mustn't go \unbf{very much further}.
  \end{description}
\item \textbf{时间}

  \begin{description}
  \item[固定时间位置] She was born \unbf{in 1980}.
  \item[前跨延续] I shall be in Chicago \unbf{until Thursday}.

    以“现在时间”为基点,向前跨越。
  \item[后跨延续] We have been at the airport \unbf{since yesterday}.

    以“现在时间”为基点,向前跨越。或者说从过去某时间点到现在。

  \item[时间频度] They \unbf{very seldom} went to see their parents.
  \item [一个时间和另一个时间的关系] She must \unbf{still} be in her office.
  \end{description}

\item \textbf{方式过程}

  \begin{description}
  \item[方式] The minister explained his policy \unbf{very clearly}.
  \item[手段] \unbf{By her insight}, she grasped the patient's real problem.
  \item[工具] I have difficulty eating \unbf{with chopsticks}.
  \item[施事] Penicillin was discovered \unbf{by Sir Alexander Fleming}.
  \end{description}

\item \textbf{方面}, 用状语增加具体真实价值。
  \begin{itemize}
  \item She helped him \unbf{with his research}.

    她帮助他做研究。
  \item He's busy writing.
  \end{itemize}
\item \textbf{原因}
  \begin{description}
  \item[原因] She died \unbf{of cancer}.
  \item[理由] He bought the book \unbf{through an interest in China}.
  \item[目的] He bought the book \unbf{to study English}.
  \item[结果] He always studies hard, \unbf{so he has good grades}.
  \item[条件] \unbf{If he always studies hard}, he will have good grades.
  \item[让步] \unbf{Even though he studied hard}, he didn't have good grades.
  \end{description}

\item \textbf{情态},可以使用状语来改变句子的真实性(如增强或减弱)。
  \begin{description}
  \item[强调] She \unbf{certainly} helped him with his research.

  \item[近似] They are \unbf{probably} going to the zoo.

  \item[限制] I shall be in Chicago \unbf{only} until Thursday.
  \end{description}

\item \textbf{程度},程度状语在改变句子的真实性上与情态状语类似,但是,程度状语添加了一
  个特殊的语义成分,可分等级性。
  \begin{description}
  \item[增强语义] He \unbf{badly} needed consolation.

    他急需安慰。badly在这里是非常,很,严重的意思。

  \item[减弱语义] She helped him \unbf{a little} with his research.
  \end{description}
\end{enumerate}

\subsection{可构成状语的词类}

状语成分可以由很多词类来实现:
\begin{description}
\item[封闭类副词为中心词的副词短语] \unbf{(Just) then}, the telephone rang.
\item[以开放类副词为中心词的副词短语] You should have opened it \unbf{(a bit more) carefully}.
\item[名词短语] They had traveled \unbf{a very long way}.
\item[介词短语] Tom hurried \unbf{across the field}.
\item[无动词分句] \unbf{When in doubt} the answer is ``no''. doubt, 疑问。
\item[非限定性分句] \unct{She}{S} \unct{realized}{V}, \unct{lying there}{A}, \unct{what she must do}{O}.
\item[限定性分句] We sent for you \unbf{because you were absent yesterday}.

  我们叫你来是因为你昨天缺席了。
\end{description}

\subsection{状语的位置}

与其他句子成分相比,状语成分可以比较自由地被置于句内各个不同的位置上(简单了
解即可):
\begin{description}
\item[I] \unbf{by then} the book should have been returned to the library.
\item[iM] The book \unbf{by then} should have been returned to the library.
\item[M] The book should \unbf{by then} have been returned to the library.
\item[mM] The book should have \unbf{by then} been returned to the library.
\item[eM] The book should have been \unbf{by then} returned to the library.
\item[iE] The book should have been returned \unbf{by then} to the library.
\item[E] The book should have been returned to the library \unbf{by then}.
\end{description}

如上文中的符号所示,状语可位于句中三个主要位置:句首位置I(NITIAL),句中位
置M(EDIAL),句末位置E(ND),但是,句中位置又分有三个变体(句中首位iM,句中中
位mM和句中末位eM)以及句末位置下分的句末首位(iE)。\textbf{句中位置就是紧接在功能词
  或系词后面的位置。}

若不存在功能词,那么M的位置就简单的处于S和V 之间;若S被省略,M的位置则位于V的
前面。

状语位置的选择由语义和语法因素来决定,但是同时也由信息处理的要求和末端
重 (end weight)原则来决定。如果没有特殊因素需要考虑,状语应被置于E(句末位
置),事实上,状语多数被置于这个位置。

\subsubsection{各类状语位置}

连接状语 Connecting adverbials 和评论状语 comment adverbials多表示本句与其他
句子的关系,或者评价本句,所以通常放在句首:
\begin{itemize}
\item \unbf{However}, not everybody agreed.

\item \unbf{Fortunately}, nobody was hurt.
\end{itemize}

Adverbials of indefinite frequency, certainty and completeness

不定频度状语(always, often等),确定性状语(probably, definitely等) 和完整性状
语 (completely, almost等) 通常放在句中。
\begin{itemize}
\item My boss \unbf{often} travels to America
\item I've \unbf{definitely} decided to change my job.
\item There is \unbf{clearly} something wrong.
\item The builder said he had \unbf{almost} finished, but it wasn't true.
\item \unbf{Sometimes} I'd like to live alone somewhere else alone.
\end{itemize}

焦点状语Focusing adverbials (also, just, even等)可以放在句中或其他位置,依具
体状语而定。
\begin{itemize}
\item He's \unbf{even} been to Antarctica.

\item We are \unbf{only} going for two days.


\item \unbf{Once} you could do a thing like that.

  只有你才会做出那样的事。
\end{itemize}

Adverbials of manner (how), place (where) and time (when) most often go in end position.

方式、地点和时间状语通常放在句中:
\begin{itemize}
\item She brushed her hair \unbf{slowly}.
\item The children are playing \unbf{upstairs}.
\item I phoned Alex \unbf{this morning}.
\end{itemize}

时间状语也可以放在句首。
\begin{itemize}
\item \unbf{Tomorrow} I've got a meeting in Cardiff.
\end{itemize}

强调状语Emphasizing adverbials (terribly, really等) 通常与其所强调的词放在一
起。
\begin{itemize}
\item I'm \unbf{terribly} sorry about last night.
\end{itemize}

程度状语 Degree adverbials (more, very much, most, a lot, so等) 据其功能位置可变

如有多条状语短句,通常按照方式、地点、时间的顺序排列。
\begin{itemize}
\item Put the butter \unbf{in the fridge} \unbf{at once}. (not … at once in the fridge.)
\item Let's go \unbf{to bed} \unbf{early}. (not … early to bed.)
\item I worked \unbf{hard} \unbf{yesterday}.
\item She sang beautifully \unbf{in the town} \unbf{hall last night}.
\end{itemize}


\section{介词和介词短语}

\subsection{介词补语}

介词连接句子中的两个语言单位,并且说明它们之间的关系。

介词补语通常都是名词
性短语(含名词短语、代词、-ing分句、wh-名词性分句)。

\textbf{尽管that分句和不定式分句可以起名词作用,但是它们不能作介词补语。}

\begin{itemize}
\item I was surprised $ \left\{
    \begin{aligned}
     &\text{\unbf{at} her angry response.}\\
     &\text{\unbf{at} hearing her objecti on.}\\
     &\text{\unbf{at} what she said.}\\
     &\text{to hear her objection.}\\
     &\text{that she responded so angrily. }
    \end{aligned}
  \right. $

\end{itemize}

\subsection{介词后置}

尽管介词一般在它本身补语的前面,但是,在一些情况下,介词必须后置:
\begin{itemize}
\item 带介词动词的被动语态结构,其中主语相当于主动语态里的介词补语:
  \begin{itemize}
  \item \unbf{The car} has been paid \unbf{for}.
  \end{itemize}

\item 介词补语主位化的不定式分句 或 -ing分句:
  \begin{itemize}
  \item \unbf{That man} is unpleasant to work \unbf{with}.

  \item \unbf{His advice} is not worth listening \unbf{to}.
  \end{itemize}

  因上,当疑问词或引导词是介词补语时,介词往往出现在句尾,尤其是非正式用法中。
  \begin{itemize}
  \item \unbf{What} are you looking \unbf{for}?
  \item \unbf{Who} is she talking \unbf{about}?
  \item \unbf{About whom} is she talking? 太正式,日常不大用。
  \item \unbf{What} kind of films are you interested \unbf{in}?
  \item Tell me \unbf{what} you're worried \unbf{about}.
  \end{itemize}

  可是,当名词与疑问词连用一体时,介词不后置。
  \begin{itemize}
  \item \unbf{With} what money? (不能说 \sout{What money with.})
  \end{itemize}

\item 一些简单介词(如through)和多数的复杂介词(如because of, in addition to) 不可
  以被后置。

\end{itemize}

\subsection{简单介词和复杂介词}

最普通的介词是一些单音节词项,如 at, for, in, on, to, with,除非被后
置,它们通常要非重读且元音弱化。

有一些多音节介词,它们中有的向来就是由单音节介词组成的复合词(例如: inside,
with in),有的源于分词(例如: during, concerning, granted),有的由其他语言引
入(例如: despite , except)

介词数量的增加主要是由于介词与其他词组成了“复合介词”。复合介词主要有两大类:
\begin{itemize}
\item 在简单介词前带有分词、形容词、副词、或连词,如:owing to, devoid of, away
  from, because of。
\item 在简单介词后带有一个名词或另一个简单介词,如:in charge of, by means of,
  at variance with, in addition to, as a result of。
\end{itemize}

\subsection{表示时间和空间的介词}

较为简单,略。

\subsection{表示原因和目的的介词}

表示\textbf{原因、理由和动机}的介词短语有because of, on account of, for, out of:
\begin{itemize}
\item He lost his job \unbf{because of} his laziness.
\item She was fined \unbf{for} dangerous driving .
\item They died \unbf{from} exposure.
\end{itemize}

表示\textbf{目的、目标和对象},这里最普通的介词是 for:
\begin{itemize}
\item We had better set out \unbf{for} home.

  我们最好动身回家。

\item She is applying \unbf{for} a better job.

  她正在申请一份更好的工作。
\end{itemize}
当补语是生物名词时,for通常带有“\textbf{预定接受者}” 的意味,to 被用于表示“\textbf{实
  际接受者}”:
\begin{itemize}
\item He built a play-pen \unbf{for} the little girl.

\item She address the letter \unbf{to} Jim.
\end{itemize}

还可以使用 from 或 out of表示\unbf{来源}(与目的相反):
\begin{itemize}
\item I don't like to borrow \unbf{from} friends.

\item She let him \unbf{out of} the house.
\end{itemize}

当 as 是“作为某个角色”的意思时,后面出现的短语表明了原因。
\begin{itemize}
\item \unbf{As} a doctor, I ought to help you.
\end{itemize}

\subsection{表示由手段到刺激因素的介词}

介词还可以表示\textbf{手段}、\textbf{工具}用来回答 ``How?'' 问句,其中 by 可以表达使用的手
段, with 可以表达使用的工具,例如:
\begin{itemize}
\item I go to work \unbf{by} car.
\item The thief entered \unbf{by} the back door.
\item She won the match \unbf{with} her speed.
\item He managed to open the car \unbf{without} a key.
\end{itemize}

与手段和工具相反的是\textbf{施事者},施事者为\textbf{生物名词},可以引发某事。它可以用
介词 by 来表达。
\begin{itemize}
\item This picture was painted \unbf{by} Degas.

\item I was bitten \unbf{by} a neighbour's dog.
\end{itemize}

刺激和反应主要是用 at, with, about, in, of 和 to 来表达:
\begin{itemize}
\item I'm surprised \unbf{at/with} her attitude.

\item They were all angry \unbf{at/with} Tom for making such a stupid mistake.

\end{itemize}

\subsection{表示。。。的介词}

略。


\section{简单句}

\subsection{主语--谓语一致}

在英语中,最重要的一致关系类型是主语和谓语动词之间第三人称数的一致。单
数主语需要用单数动词,复数主语需要复数动词。

注意:在名词短语表示主语时,\textbf{由名词短语的中心词来决定名词短语的单复数}:
\begin{itemize}
\item The change in husbands' attitudes is most obvious in their families.
\item The changes in husbands' attitude are most obvious in their families.
\end{itemize}

分句、介词短语和副词作为主语一般算作单数:
\begin{itemize}
\item \unbf{Smoking cigarettes} \unbf{is} dangerous to your health .

\item \unbf{In the evenings} \unbf{is} best for me.
\end{itemize}

名字、标题、引文即使是复数名词短语,也算作单数。

\textbf{就近原则} (proximity) 是指\textbf{动词与紧靠在他前面的名词短语相一致},而不
是与主语的名词短语中心词相一致。如 either \ldots{} or \ldots{},
neither \ldots{} nor \ldots{}.
\begin{itemize}
\item Either your brakes \unbf{or your eyesight} \unbf{is} at fault.
\item Neither you, nor I, \unbf{nor anyone else} \unbf{knows} the answer.
\end{itemize}

在中国教学中,there be \ldots{} 也采用就近原则,但这其实是过时或者不准确
的,注意随机应变吧。
\begin{itemize}
\item There \unbf{is/are} an apple, two pears and some oranges on the table.
\end{itemize}

\section{句子类型和话语功能}

简单句可以分为四种主要句法类型,并与话语功能息息相关。
\begin{description}
\item[陈述句] 句子有主语,并且主语位于动词前:
  \begin{itemize}
  \item \unbf{Richard} \unbf{gave} Tom a watch for his birthday.
  \end{itemize}

  另外,一些句子因情景中已暗含主语,省略了主语
  \begin{itemize}
  \item (I'm) Sorry I couldn't be there.

  \item (It's) Good to see you.
  \item (I'll) See you later.
  \end{itemize}

  另外 there/here be 陈述句中,主语在谓语 be 动词之后。

\item [疑问句] 一般分为如下两种:
  \begin{description}
  \item [一般 (yes-no) 疑问句] 助动词位于主语之前:
    \begin{itemize}
    \item \unbf{Did} \unbf{Richard} \unbf{gave} Tom a watch for his birthday?
    \end{itemize}

  \item [特殊 (Wh-) 疑问句] 特殊疑问词位于句首,并且通常功能词位于主语前:
    \begin{itemize}
    \item \unbf{What} \unbf{did} \unbf{Richard} give Tom for his birthday?
    \end{itemize}
  \end{description}

\item[祈使句] 一般没有明显的主语,并且使用动词原形:
  \begin{itemize}
  \item \unbf{Give} Tom a watch for his birthday.
  \end{itemize}


\item[感叹句] 句子由 what 或 how 引导,后接(陈述)主谓语序。
  \begin{itemize}
  \item \unbf{What} a fine watch \unbf{he received for his birthday}!

  \item \unbf{How} quickly \unbf{you eat}!
  \end{itemize}
\end{description}

\subsection{Wh- 疑问句}

Wh- 疑问句是由简单的疑问词协助构成的 (或 wh- 词),如: who/ whom/whose,
what, which, when, where, how, why。

\textbf{不同于 yes-no 疑问句的是,Wh-疑问句一般是降调。}

\textbf{带有 wh- 词的介词性补语}:
\begin{description}
\item[正式体]介词 + wh- 位于句首
  \begin{itemize}
  \item \unbf{To whom} should I write?
  \item \unbf{In which} city did you grow up?
  \item \unbf{At what} time does the train depart?
  \end{itemize}
\item[非正式体] wh- 位于句首
  \begin{itemize}
  \item \unbf{Whom} should I write \unbf{to}?
  \item \unbf{Which} city did you grow up \unbf{in}?
  \item \unbf{What} time does the train depart \unbf{(at)}?
  \end{itemize}

\item[工具、原因和目的] 就表示工具,原因,和目的的附加成分发问,wh- 位于句
  首
  \begin{itemize}
  \item \unbf{What} shall I mend it with?
  \item \unbf{What} are you fighting for?
  \end{itemize}
\end{description}

\subsection{回响疑问句}

回响疑问句顾名思义就是重复部分或者是全部已讲过的话,以期得到确认。
\begin{itemize}
\item A: The Browns are emigrating. \qquad B: \textsc{\'Emigrating?}
\item A: He's a doctor. \qquad B: \textsc{Wh\'at} is he?
\item A: I'll pay for it. \qquad B: You'll \textsc{wh\'at}?
\item A: Have you ever been to BJ? \qquad B: Have I ever been \textsc{wh\'ere}?
\end{itemize}

\subsection{简单句和多重句}

句子的构成:
\begin{description}
\item[简单句] 由单一的独立分句构成句子。
\item[多重句] 由一个或者多个分句作为直接成分。
  \begin{description}
  \item [联合句] 由两个或两个以上并列分句(COORDINATE clauses)构成的句子。
  \item [复合句] 由从属分句 (SUBORDINATE clause) 来担任句子中一个或一个以上成分
    (例如直接宾语或状语)的句子。
  \end{description}
\end{description}

\begin{itemize}
\item You can borrow the car \unct{that belongs to my sister}{C}.
\item You can borrow \unct{the car that belongs to my sister}{C}.
\end{itemize}

和本书不同,“简单句”这一术语在其他语法书中经常用来指一个不包含另一分
句的独立分句,而不管所包含的分句是不是句子的直接成分。有些语法书,把非
限定结构(这种结构含有一个非限定动词作为动词成分)看作是短语而不是分句。
我们则把这种结构看作是分句,因为可以把它分解为分句成分。非限定分句本身
就是从属性质的,因此不能成为典型的简单句形式

\section{替代形式和省略}

\subsection{替代形式}

\subsubsection{the same}

\begin{itemize}
\item A: Can I have \unbf{a cup of tea}, please?

  B: Give me \unbf{the same}, please.

\item Yesterday I felt \unbf{sad} and today I feel \unbf{the same}.

\item The Denison house is \unbf{small but very comfortable}, and ours is
  just \unbf{the same}.

\end{itemize}

\subsubsection{one, ones, some}

有两种替代形式的 one :一种复数形式是 some;另一种复数形式是ones。两种
都是\textbf{非重音}(因此与数字 one 区分),而且都替代\textbf{可数名词}。

some 也可以替代\textbf{不可数名词}。
\begin{itemize}
\item Have you any \unbf{knives}? I need a sharp \unbf{one}.
\item I like those \unbf{shoes}, but let's buy these \unbf{ones}.

\item Shall I pass \unbf{the butter}? Or have you got \unbf{some} already?
\end{itemize}


\subsubsection{so}
\begin{itemize}
\item You asked me to leave, and 'so I  \textbf{\textsc{d\`id}} .
\item You asked me to leave, and  I \textbf{\textsc{d\`id so}}.
\item A: It's starting to snow. B: 'So it \textbf{\textsc{\`is}} !
\end{itemize}

\subsection{省略}

\subsubsection{省略的位置分类}

\begin{description}
\item[句首省略] (I) hope he's there.
\item[句中省略] Jill owns a Volvo and Fred (owns) a BMW.
\item[句尾省略] I know that we haven't studied hard yet, but we will (study hard).
\end{description}

\subsection{省略的还原类型}

\subsubsection{情景省略}

典型的情景省略是句首省略,尤其是采取\textbf{省略主语、功能词或同时省略}这两个成
分的形式。

陈述句中的省略:
\begin{itemize}
\item (I) Told you so.
\item (I'm) Sorry I couldn't be there.
\item (It's) Good to see you.
\item (I'll) See you later.
\end{itemize}

疑问句中的省略:
\begin{itemize}
\item (Are you) In trouble?
\item (Is there) Anybody in?
\item (Do you) Want some?
\item (Have you) Got any money?
\item (Does) Anybody need a lift?
\end{itemize}

\subsubsection{结构省略}
\subsubsection{篇章省略}


\section{并列}

\section{复合句}

\subsection{从属和上位分句}

\textbf{复合句}像简单句,因为它\textbf{只包含一个主要分句};但是和简单句不一样的是,它\textbf{有一个
或一个以上的从属分句作为它的句子成分。}

\begin{itemize}
\item Although I admire her reasoning, I reject her conclusion.
\end{itemize}

\subsection{非限定性分句省略方法}


由于\textbf{非限定性动词分句}(谓词为分词、不定式的句子,
见 \cref{subsec:iffinite})没有时态标记和情态助动词,又常常没有主语和从
属连词;也可以\textbf{根据句子的语境来还原时态、语态、人称和数},因此是一种很
有价值的\textbf{压缩句子}的方法。

\begin{itemize}
\item \unbf{When} (she was) \unbf{questioned}, she denied being a member of the group.
\item (Since/Because/As they were) \unbf{Considered works of art}, they were
  admitted into the country without customs duties.

  它们被视为艺术品,被准许免关税进入该国。
\end{itemize}

在上下文中找不到与名词性成分指称的联系时,那么主语可能是\textbf{不确定主语}或是
\textbf{说话者}。
\begin{itemize}
\item \unbf{To be an administrator} is to have the worst job in the
  world. [(For) a person]
\item It's hard work \unbf{to be a student}. [不确定主语,根据上下文而定,
  如: (for) anyone]
\item It's hard work \unbf{to be honest}. [不确定主语,根据上下文而定]
\end{itemize}

助动词 have 有时用在 to 不定式中 (to have happened) 或 -ing 分词
中(having happened),前者更能表示未来时间或不确定性。

\subsection{无动词分句省略方法}

SVC, SVA 两种句型中,其中V为系动词且无实际意义,因此可以\textbf{省去系动词V},成
为\textbf{无动词分句}。

\begin{itemize}
\item Seventy-three people have drowned in the area, many of them
  (\unbf{are}) children.

  drown 意思是溺水,被动主动形式皆可。

\item  Mary sat in the front seat, her hands (\unbf{were}) in her lap.
\end{itemize}

如果可以根据上下文还原主语,也可\textbf{省去主语}。
\begin{itemize}
\item Whether (\unbf{he is}) right or wrong, he always comes off worst in
  argument.

\item One should avoid taking a trip abroad in August where (\unbf{it is}) possible.

\item We can meet again tomorrow, if (\unbf{it is}) necessary.

\end{itemize}

\subsection{时间性 since- 分句的完成时}

当整个结构指持续到现在(可能也包括现在)的一段时间时,\textbf{时间性since- 分句}一般使用
\textbf{一般过去时或现在完成时},\textbf{主句}则一般使用\textbf{现在完成时}。

\begin{itemize}
\item I \unbf{have lost} ten pounds since I started swimming.
\item Since \unbf{leaving} home, Larry \unbf{has written} to his parents just once.

  上句中leaving其实是she left的省略转化(见\cref{subsec:toing})。

\item Max \unbf{has been tense} since he\textbf{'s been taking} drugs.
\item I\textbf{'v had} a dog ever since I\textbf{'ve owned} a house.
\end{itemize}

英语是世俗流变的,而非学院派的,不要教条。其实非正式文体中,越来越多since- 对
应主句使用过去时。

如果整个时间段都在过去,那么主句和分句都使用\textbf{过去完成时或一般过去时}:
\begin{itemize}
\item Since he knew(had known) her, she was(had been) a journalist.
\end{itemize}

\subsection{其他时间分句的完成时}

如同since- ,当 after- 分句或 when- 分句指两个\textbf{过去时间的顺序}时,时间分句中的
动词可能使用\textbf{过去完成时},尽管\textbf{一般过去时}更常见。
\begin{itemize}
\item We ate our meal after/when we (had) returned from the game.
\end{itemize}

如果时间分句和条件分句指的是\textbf{将来时间的顺序}时,那么句中使用\textbf{现在完成
  时}是很普遍的:
\begin{itemize}
\item When they\unbf{'ve scored} their next goal, we\textbf{'ll go} home.

\item After they \unbf{have left}, we \unbf{can smoke}.
\end{itemize}


\subsection{直接引语和间接引语}

\begin{description}
\item[直接引语] direct speech, 援引他人的话或文字。

\item[间接引语] reported speech, 以第三者的身份转述他人的话或文字。
\end{description}

直接引语转为间接引语时,产生的动词变化关系被称作\textbf{时态呼应}。另外一定注意\textbf{人称变
  化}。

如果转述的时间在原话之后,那么一般需要改变动词形式,这种变化被称作\textbf{时态后
  移}(见\cref{tab:speech})。

\begin{table}[htbp!]
  \centering
  \begin{talltblr}[ caption = {直接引语到间接引语的时态后移},
    label = {tab:speech},
    ]{
      width=\linewidth, colspec={cc},
      rowsep=2pt, colsep=4pt,
      row{1} = {c, font=\bfseries},
    }
    \toprule
    直接引语 & 间接引语 \\ \midrule
    一般现在 & 一般过去 \\
    一般过去 & 一般过去或过去完成 \\
    现在/过去完成 & 过去完成 \\
    \bottomrule
  \end{talltblr}%
\end{table}

时态后移的例子:
\begin{itemize}
\item Paul said\unbf{, ``I'm felling} ill.''

  Paul said \unbf{(that) he was felling} ill.

\item Anna said\unbf{, ``I've lost} my phone.'' [I've = I have]

  Anna said \unbf{(that) she'd lost} her phone.' [she'd = she had]

\item Lucy said\unbf{, ``I can speak} English.''

  Lucy said \unbf{that she could speak} English.
\end{itemize}

\textbf{如果原来的时间指示关系在转述时依然有效,那么时态后移就不是必需的。}例如:
\begin{itemize}
\item Their teacher had told them that the earth \unbf{moves} around the sun.
\item Sam told me last night that he \unbf{is} now an American citizen.
\item I didn't know that our meeting \unbf{is} next Tuesday.
\end{itemize}

\subsection{间接陈述句、疑问句、感叹句和祈使句}

所有的语句类型都可变为间接引语,转变连接词见\cref{tab:reportedcon}。

\begin{table}[htbp!]
  \centering \small
  \begin{talltblr}[ caption = {间接句型及其连接词},
    label = {tab:reportedcon},
    note{a} = {间接祈使句不带主语}
    ]{
      width=\linewidth, colspec={cc},
      rowsep=2pt, colsep=4pt,
      row{1} = {c, font=\bfseries},
    }
    句子类型 & 从属连接词 \\ \midrule
    间接陈述句 & that \\
    间接疑问句 & wh- 或 if \\
    间接感叹句 & wh-  \\
    间接祈使句 & that或 to V. \\
    \bottomrule
  \end{talltblr}%
\end{table}

\begin{itemize}
\item ``Are you ready yet?'' asked Joan. [ yes-no 疑问句]

  Joan asked (me) \unbf{whether I was ready yet}.

\item ``When will the plane leave?'' I wondered. [ Wh- 疑问句]

 I asked her \unbf{when the plane would leave}.
\item ``Are you tired or not?'' I asked her.  [选择疑问句]

  I asked her \unbf{whether or not she was tired}.

\item ``What a brave boy you are!'' Margaret told him. [感叹句]

  Margaret told him \unbf{what a brave boy he was}.

\item ``Clean your teeth at once,'' Leo said to his son. [祈使句]

  Leo told to his son \unbf{to clean the teeth} at once.

  Leo \unbf{insisted that his son clean the teeth} at once.

  Leo \unbf{insisted on his son cleaning the teeth} at once.
\end{itemize}

\section{从属分句的句法和语义功能}

\subsection{从属分句的功能类别}

根据从属分句的潜在功能,可将其分为四种主要类别:
\begin{description}
\item[名词性分句] 功能类似于\textbf{名词短语},可充当主语、宾语、补语、同位语
  和介词补语。由于间接宾语一般指人,因此名词性分句是唯一可以充当间接宾语的分
  句类型。

\item[状语分句] 功能方面更像是副词短语;但是在表达的明确性上,常常更像
  是介词短语。

\item[关系分句] 功能上与修饰性形容词相同,修饰名词短语;位置上和后置修
  饰的介词相同。
  \begin{itemize}
  \item a man \unbf{who is lonely} \~{} \unbf{a lonely} man
  \item tourists \unbf{who come from Italy} \~{} tourists \unbf{from Italy}
  \end{itemize}

\item[比较分句] 在修饰功能上与形容词和副词相似;语义上比较分句和他们的
  关联词等同于程度副词。
  \begin{itemize}
  \item She has \unbf{more} patience \unbf{than you have}.
  \item He's not \unbf{as} clever a man \unbf{as I thought}.
  \item I love you \unbf{more} deeply \unbf{than I can say}.
  \end{itemize}
\end{description}

\subsection{名词性分句}

\subsubsection{that- 分句}
\label{subsub:thatclause}

名词性 that- 分句所承担的功能有:
\begin{description}
\item[主语] \unbf{That he passed the exam} surprised everyone.

  \unbf{That you don't know Chinese} is a pity.

  \textbf{直接作为主语的 that- 分句中,that 不可省略},否则句子结构混乱、
  易产生歧义,。

\item[宾语] I believe \unbf{that she will come}.
\item[主语补语] The truth is \unbf{that they are moving to another city}.
\item[形容词补语] I'm glad \unbf{that you are so friendly}.

\item[同位语] The belief \unbf{that no one is infallible} is well-founded.

  没有人是一贯正确的这一信念是有充分根据的。
\end{description}

that 的省略在简短、不复杂的分句中尤为常见。但在一些容易产生歧义的情况
下,that 不可省略:除上文所述作为主语的that以外,还有以下情况:
\begin{itemize}
\item  They told us once again \unbf{that the situation was serious}.

  他们再次告诉我们,形势严峻。

\item They told us \unbf{that once again the situation was serious}.

  他们告诉我们,形势再次严峻。

  以上两句中的that是为说明\textbf{状语once again的归属}。

\item \unbf{I realize that I'm in charge} and \unbf{that everybody accepts my
  leadership}.

  我意识到我是领导人,每人都接受我的领导。

  两个that说明前后两个句子为\textbf{并列关系},方便断句。

\item I realize \unbf{that I'm in charge and everybody accepts my leadership}.

  我意识到,我是领导人并且每人都接受我的领导。

\item \unbf{That she ever said such a thing} I simply don't believe.

\end{itemize}

\subsubsection{wh- 疑问分句}


\textbf{从属wh- 名词性关系分句具有名词性 that- 分句的所有功能而且还可充当介词
  补语}。

\begin{description}
\item[主语] \unbf{How the book will sell} depends on the readers.

\item[直接宾语] I can't imagine \unbf{what they want with your address}.

\item[主语补语] The problem is \unbf{who will water my plants when I am away}.

\item[同位语] Your question, \unbf{why did the leak occur}, remains unanswered.

  也可以不用同位语:

  Your question \unbf{of why} did the leak occur remains unanswered.

\item[形容词补语] I'm not sure \unbf{which she prefers}.

\item[介词补语] They did not consult us on \unbf{how to do this work}.
\end{description}

\subsubsection{yes-no 和选择疑问句}

Yes-no 分句由从属连词 whether 或 if 引导:
\begin{itemize}
\item Do you know \unbf{whether/if the banks are open}?
\end{itemize}

选择疑问分句由关联词由关联词 whether \ldots{} or或 if \ldots{} or构成,\textbf{如果第
二个部分是一个完整的分句,那么从属连词要重复}:
\begin{itemize}
\item They didn't say \unbf{whether} it will \textsc{r\'ain} or be \textsc{s\`un}ny.
\item He didn't tell us \textsc{whether} to wait for him \unbf{or}
  \textsc{(whether)} to go on without him.

  \textbf{if不能引导to不定式},只能用whether。另外,如果省略第二个分句中的to,
  则连接词不重复。

  He didn't tell us \unbf{whether} to wait for him \unbf{or} go on without him.

\item I can't find out \unbf{if} the flight has been de\textsc{l\'ayed}
  \unbf{or} \textsc{c\`an}celled.

\item I can't find out \unbf{whether/if} the flight has been de\textsc{l\'ayed} \unbf{or}
  \unbf{whether/if} has been \textsc{c\'an}celled.

\end{itemize}

\subsubsection{wh- 名词性关系分句}

wh- 名词性关系分句和wh- 疑问分句相似,因为它们也由一个wh- 成分引导。在很多方面,它们
更像是名词短语。
\begin{description}
\item[主语] \unct{Whoever did that}{The person who did that} should admit if frankly.

  \unct{Whatever book you see}{The books that you see} is yours to take.
\item[主语补语] The autumn is \unbf{when the leaves fall}.
\item[直接宾语] I took \unct{what books she gave me}{the books that she gave me}.
\item[间接宾语] He gave \unbf{whoever asked for it} a copy of his latest article.
\item[宾语补语] You can call me \unbf{whatever you like}.
\item[介词补语] You should vote for \unbf{whichever candidate you think the best}.
\end{description}

\textbf{如同名词短语,名词性关系分句做形容词补语时要求有介词:}
\begin{itemize}
\item He's aware of \unbf{what I write}.
\end{itemize}

wh-成分可以表达一个\textbf{具体的}(不能使用后缀 -ever) 或是\textbf{非具体的}
(一般使用后缀 \textbf{-ever}) 意思。

具体的:
\begin{itemize}
\item I took \unbf{what was on the kitchen table}.
\item May is \unbf{when she takes her last examination}.
\end{itemize}


非具体的:
\begin{itemize}
\item \unbf{Whoever breaks this law} deserves to be locked up.

  违反这项法律的人应被监禁。
\item I'll send \unbf{whatever is necessary}.

  我会发送任何必要的东西
\end{itemize}

在一个 to- 不定式分句中出现\textbf{主语}时,通常需要主语前面有\textbf{for}:
\begin{itemize}
\item \unbf{For us to take part in the discussion} would be a conflict of interest.
\end{itemize}
当分句充当直接宾语时,一般没有 for:
\begin{itemize}
\item He likes everyone \unbf{to relax.}
\end{itemize}

\subsubsection{to 不定式分句}

to 不定式分句有如下功能:
\begin{description}
\item[主语] \unbf{To be neutral} in this conflict is impossible.
\item[直接宾语] He likes \unbf{to relax}.
\item[主语补语] The best excuse is \unbf{to say that you have an examination
    tomorrow}.

\item[同位语] Your dreamer, \unbf{to become a farmer,} requires the
  energy and perseverance.

\item[形容词补语] I'm very glad \unbf{to meet her}.
\end{description}

\subsubsection{-ing 分句}

-ing 分句有如下功能:
\begin{description}
\item[主语] \unbf{Watching television} makes them always happy.
  直
\item[直接宾语] He enjoys \unbf{playing football}.

\item[形容词补语] They are busy \unbf{preparing a barbecue}.
\end{description}

如果-ing分句有主语,那么 \textbf{-ing 的主语可以是属格或宾格(代词有宾格)或
其他名词短语通格}:
\begin{itemize}
\item I object to \unbf{his/Jim's receiving} an invitation. [object to doing
  sth,可在 -ing 前加属格]
\item I object to \unbf{him/Jim receiving} an invitation. [object to sb
  doing sth, sb是宾格]

  我反对他/Jim接受邀请。

\item \unbf{My forgetting her name} was embarrassing.


  等同于:

  I was embarrassed \unbf{because/that I forgot her name}.

  \unbf{I, who forgot her name,} was embarrassed. [名词性 -ing 分句可用来指一
  个\textbf{事实或行动},这里的 -ing 其实是 连接词+过去式的简化:去掉连接词,
  过去式变现在分词]

\item \unbf{Your driving a car to New York in your condition} disturbs me greatly.

  你这种情况下开车去纽约让我很不安。

\end{itemize}

\subsubsection{不带to的不定式分句}

假拟分裂句或者说“cleft sentence”,是一种特殊句式,它通过使用特定的结构来强
调句子中的某个成分,通常是为了强调动作的执行者、动作本身或是动作发生的时间、
地点等。

名词性不带to的不定式常见于在假拟分裂句中作主语或主语补语,需指向后文,未完,暂略。
\begin{description}
\item[主语] \unbf{Turn off the tap} was all I did.
\item[主语补语] What that plan does is \unbf{(to) ensure a fair pension for all}.
  [to可省可不省]
\end{description}

\improve[inline]{暂略,需参考cref之后假拟分裂句概念 郎文语法15.11.}


\subsection{状语分句}

状语分句的语义分析比较复杂,因为\textbf{同一个从属连词所引导的分句意思可能不
  同},而且这样的情形为数不少。例如, since分句可以是时间分句,也可以是原因分
句。\textbf{另外,有些分句把两层意思结合起来}。

作时间状语的 ing 分句由以下从属连词引导: once, till, until, when, whenever,
while 和 whilst连接词可以引导\textbf{作时间状语的ing分句};\textbf{-ed分句和无动词分句}(as soon
as也可)。
\begin{itemize}
\item \unbf{Once having made a promise}, you should keep it.
\item The dog stayed at the entrance \unbf{until told to come in}.
\item Complete your work \unbf{as soon as possible}.
\end{itemize}

带有\textbf{until- 分句的主句}必须是\textbf{持续性}的,时间持续到until-分句的时间
为止。因为事件未发生的状态是持续性的,所以\textbf{否定分句总是持续性的},即使相应
的肯定分句并非如此。例如:
\begin{itemize}
\item I \unbf{didn't} start my meal \unbf{until} Adam arrived. [正确]
\item \sout{I \unbf{started} my meal \unbf{until} Adam arrived.} [错误]
\end{itemize}


地点分句主要由where和wherever引导,\textbf{where是具体的, wherever是非具体的}。
\begin{itemize}
\item \unbf{Where} the fire had been, we saw nothing but blackened ruins.

\item They went \unbf{wherever} they could find work. [to any place where]
\end{itemize}

\subsection{句子性关系分句}

\textbf{修饰名词的关系分句}的先行词是\textbf{名词短语},而\textbf{句子性关系分句}的
先行词可以是:
\begin{description}
\item[主句的谓语或谓体] They say he \unbf{plays truant}, \unbf{which he doesn't}.

  \unbf{walks for an hour each morning}, \unbf{which would bore me}.

\item[主句或整个句子] Things then improved, \unbf{which surprises me}.

  Colin married my sister and I married his brother, \unbf{which makes Colin and me in-law}.

\item[之前多个句子] --- \unbf{which is how the kangaroo came to have a pouch}.

  所以袋鼠才有了育儿袋。

\end{description}

句子性关系分句与名词短语中的非限制性后置修饰分句相似,因为它们也用
语调或标点符号将其本身和先行词分隔开来。暂不详述。
\begin{itemize}
\item The plane may be several hours late, in which case there's no point in our waiting.

\item They were under water for several hours, from which experience they
  emerged unharmed.
\end{itemize}

\improve[inline]{区别需参见cref郎文语法17.11}

\section{非限定分句和无动词分句的主语}

\textbf{独立分句}是指具有一个明显的主语但\textbf{不用从属连词}引导
的\textbf{非限定性分句和无动词分句}。之所以成为独立分句是因为它们在句法上显然
并\textbf{不与主句绑定在一起}。独立分句可以是 -ing, -ed 或无动词分句:
\begin{itemize}
\item \textbf{No further discussion a rising}, the meeting was brought to a close.
\item \textbf{Lunch finished}, the guests retired to the lounge.
\item \textbf{Christmas then only days away}, the family was pent up with excitement.
\end{itemize}

非限定分句和无动词分句中\textbf{没有主语时},用以识别主语的依附规则是,\textbf{认为领
句的主语就是其主语}:
\begin{itemize}
\item The oranges, \unbf{when (they are) ripe}, are picked and sorted
  mechanically.

\item \unbf{Driving home after work}, I accidentally went through a red
  light. [While I was driving home after work]

\item \unbf{To climb the rock face}, we had to take various precautions. [So that
  we could climb]

\end{itemize}

某些情况下,依附规则是不适用的,或者说至少是不严格的:
\begin{description}[style=nextline]
\item[分句是一个主语外接状语, 这时隐含的主语是说话者 I]

  \unbf{Putting it mildly}, you have caused us some inconvenience.

\item[隐含的主语是整个主句]

  I'll help you \unbf{if necessary}. [\ldots{} if it is necessary]

\item[隐含的主语是一个不定代词或支撑词 it]

  \unbf{When dining in the restaurant}, a jacket and tie are required. [When one
  dines]

  \unbf{Being Christmas}, the government offices were closed. [Since it was]
\end{description}


\textbf{没有从属连词引导的状语分句和无动词分句被称为增补分句},根据上下文,我
们可以用其表示时间、条件、原因、让步或状况关系。对读者或听者来说,这种伴随关系
的实质是从语境中推断的。

\begin{itemize}
\item \unct{Reaching}{When we Reached} the river, we pitched camp for the night.
\item Julia, \unct{being}{since she was} a nun, spent much of her time in
  prayer and meditation.
\item The sentence is ambiguous, (if / when it is) taken out of context.

\item We spoke \unbf{face to face}.
\end{itemize}

\subsection{比较分句}

在比较结构中,主句中的陈述与从属分句中的陈述进行比较。两句中共同的部分在从属分
句中可省略。
\begin{itemize}
\item Jane is \unbf{as} \unct{healthy}{比较成分} \unbf{as} \unct{her sister}{比较基础} (is).
\item Jane is \unct{healthier}{比较成分} \unbf{than} \unct{her sister}{比较基础} (is).
\end{itemize}

\subsubsection{比较成分的分句功能}

\textbf{比较成分可以是比较结果中除动词以外的任何一个成分}:
\begin{description}
\item[主语] \unbf{Most people} use this brand than (use) any other shampoo.

\item[直接宾语] She knows \unbf{more history} than most people (know).

\item[间接宾语] That toy has given \unbf{more children} happiness than any other (toy) (has).

\item[主语补语] Simo is \unbf{more relaxed} than he used to be.

\item[宾语补语] She thinks her children \unbf{more taller} than (they were) last year.
\item[状语] You've been working \unbf{much harder} than I (have).

\item[介词补语] She's applied for \unbf{more jobs} than Joyce (has (applied for)).
\end{description}

由 more \ldots{} than, less \ldots{} than 和 as \ldots{} as 引导
的\textbf{不一定是比较分句,后面可接续一个明显的比较标准或状态}。
\begin{itemize}
\item I weigh more than \unbf{200 pounds}.

\item It goes faster than \unbf{100 miles per hour}.

\end{itemize}

另一种不接比较分句的类型:
\begin{itemize}
\item I was more angry than \unbf{frightened}.
\item I was angry more  than \unbf{frightened}.

\item \sout{I was angrier than frightened}.
\end{itemize}
上述最后一句错误。因为angrier为屈折形式的比较级,frightened(害怕的)是过去分
词作形容词用,两者不对等。

more of a \ldots{} 和less of a \ldots{} 与可分等级的名词中心语连用:
\begin{itemize}
\item He's more of a fool than I thought (he was).

\item It was less of a success than I imagined (it would be).
\end{itemize}

当对比涉及\textbf{同一阶上的两个点}且一点高于另一点时,则than之后的部分\textbf{不可以
扩展成分句}。Than的功能是非分句比较中的\unbf{介词}:
\begin{itemize}
\item It's hotter \unbf{than} just warm. (或 It's hotter than 90°C.)
\item She's wiser \unbf{than} merely clever.
\item They fought harder \unbf{than} that.
\item I was \unbf{more than} happy to hear that.
\end{itemize}

\subsubsection{比较分句中的省略}

由于两个分句在结构和内容上通常非常相似,因此\textbf{省略在比较分句中是常规而不是例外}。
以下是省略和代词、替代谓语和替代谓体的例子:

James and Susan often go to plays but
\begin{enumerate}
\item James enjoys the theater more than Susan enjoys the theater.
\item James enjoys the theater more than Susan enjoys it.
\item James enjoys the theater more than Susan does.
\item James enjoys the theater more than Susan.
\item James enjoys the theater more.

  因为前半句已经说明了对象两人,所以这里可以直接省略整个比较分句。
\end{enumerate}
\textbf{宾语一般不可省略,除非主要动词也省略,如第3、4句,功能词可留可不留。}
\begin{itemize}
\item James enjoys the theater more than Susan \sout{enjoys}.

  误!比较分句中宾语省略,主要动词未省略。
\end{itemize}
但是,如果\textbf{宾语本来就是比较成分},那么\textbf{可以省略宾语,而不省略主要动词}:
\begin{itemize}
\item James knows more about the theater is more than Susan \unbf{knows}.
\end{itemize}


如作最大限度的省略,有可能造成歧义:
\begin{itemize}
\item He loves his dog more than his children.
\end{itemize}
上例的意思可能是他比他的孩子更爱狗(his children作分句主语),也可能是他爱狗
超过爱孩子(his children作分句宾语)。因此最好根据实际情况补充说明:
\begin{itemize}
\item He loves his dog more than his children \unbf{does} his dog.

  他比他的孩子更爱狗。
\item He loves his dog more than he loves his children.

  他爱狗超过爱孩子
\end{itemize}


\subsubsection{部分对比 (partial contrast)}

对比可能\textbf{只影响时态}或\textbf{加上了情态助动词}而已。在这种情况下,一
般是省略比较分句情态助动词之后的部分:

\begin{itemize}
\item I hear it more clearly than I \unbf{did}. [than I used to hear it]

\item I get up later than I \unbf{should}. [than I should get up]
\end{itemize}

如果只是时态上的对比,在比较分句中可能只用一个状语来表示:
\begin{itemize}
\item She'll enjoy it more than (she enjoyed it或 she did)last year.
\end{itemize}
这就为下列例句中从属分句全部省略提供了基础:
\begin{itemize}
\item You are slimmer (than you were).
\item You're looking better (than you were (looking)).
\end{itemize}

对一个隐含或实际表达的分句存在\textbf{逆向呼应}的省略:
\begin{itemize}
\item I caught the bus from the town: but Harry came home \unbf{even later}. [later than I came home]
\end{itemize}


话语之外情境已包含被比较信息的省略:
\begin{itemize}
\item You should have come home earlier. [earlier than you did]
\end{itemize}

部分对比可能是主句或对比分句中的\textbf{上位分句}:
\begin{itemize}
\item \unbf{She thinks} she's fatter than she (really) is.
\item He's a greater painter than \unbf{people suppose} (he is).
\item She enjoyed it more than \unbf{I expected} (her to (enjoy it)).
\end{itemize}

\subsubsection{enough 和 too}

表达足量或超越比较的结构主要由enough,和too + to不定式表示。
\begin{description}
\item[足量比较] They're rich enough to own a car.

  The book is simple enough to understand.

\item[超越比较] They're not too poor to own a car.

  他们还没有穷到买不起一辆车。

  The book is not too difficult to understand.

  这本书不是太难理解。
\end{description}

\textbf{too 有否定意义},表示\textbf{太、过于 \ldots{} 以致不能},比较:
\begin{itemize}
\item She's \unbf{old enough} to do some work.
\item She's \unbf{too old} to do any work.
\end{itemize}
不定式分句可带有明显的主语:
\begin{itemize}
\item It moves too quickly \unbf{for most people to see (it)}.
\item He was old enough \unbf{for us to talk to him seriously}.
\end{itemize}
当不定式分句中没有主语时,其主语就是\unbf{领句中的主语}或一个\unbf{不定主语}:
\begin{itemize}
\item She writes quickly enough (for her) to finish the paper on time.

\item He was old enough (for others) to talk to him seriously.
\end{itemize}
但也有可能是模棱两可的:
\begin{itemize}
\item She was too young to date. [to date others 或 for others to date her]
\end{itemize}

\subsubsection{so \ldots{} that 和 such \ldots{} that}

So 是副词,前置修饰一个形容词或副词,such 是前限定词,与中后限定词一起修饰名
词中心语。

当that分句是\textbf{否定}时,so/such结构和too+to 不定式结构之间有一种对应关
系:
\begin{itemize}
\item It's \unbf{so} good a movie \unbf{that} we mustn't miss it.

  It's \unbf{too} good a movie \unbf{to} miss.

\item It was \unbf{such} a pleasant day \unbf{that} I didn't want to go to school.

  It was \unbf{too} pleasant a day \unbf{to} go to school.
\end{itemize}

当that分句是肯定的,so/such结构和enough+to 不定式结构之间有一种对应关系:
\begin{itemize}
\item It flies \unbf{so} fast \unbf{that} it can beat the speed record.

  It flies fast \unbf{enough to} beat the speed record.

\item I had \unbf{such} a bad headache \unbf{that} I needed two aspirins.

  I had a bad \unbf{enough} headache \unbf{to} need two aspirins.
\end{itemize}


当 \textbf{so} 单独与\textbf{动词}连用时,表示程度高;\textbf{such} 接续
的\textbf{名词}短语没有形容词前置修饰时,同样表示程度高:
\begin{itemize}
\item I \unbf{so} (much) enjoyed it \unbf{that} I'm determined to go
  again.
\item There was \unbf{such} a (large) crowd \unbf{that} we couldn't see a thing.
\end{itemize}

正式的结构so/such \ldots{} as+to不定式,有时替代so/such \ldots{} that分句:
\begin{itemize}
\item  We went early \unbf{so as to} get good seats.

\item I'm not \unbf{so} stupid \unbf{as to} believe that.

\item Would you be \unbf{so} kind \unbf{as to} lock the door when you leave?
\end{itemize}















%%% Local Variables:
%%% mode: LaTeX
%%% TeX-master: "main"
%%% End:
