\chapter{TIPS}

本章收录一些还未划入以上章节的tips.


\begin{itemize}

\item 可用介词短语替换的关系副词
  \begin{description}
  \item[where] prep + which + [place]
  \item[when] prep + which + [time]
  \item[why] prep + which + [reason]
  \item[how] prep + which + [manner/way]
  \end{description}
\item 英语中有三种“\textbf{横线}”:
  \begin{description}
  \item[hyphen -] 连字符,用来组成复合词,如mother-in-law ;或者连接因换行被
    分隔的单个单词。

  \item[en dash --] 表示数字范围,如 pages 13--67,the years 1929--1934。

  \item[em dash ---] 破折号,与中文破折号相似。如 three kinds of croissants---plain, almond, and chocolate
  \end{description}

\end{itemize}



\section{形容词性关系从句}

\begin{itemize}
\item The book \unbf{which} you ordered last month has arrived.

  You ordered the book last month. It has arrived.

  你上月订购的这本书到货了。
\end{itemize}

注: 名词性关系从句见\cref{subsubsec:whnoun}。

\subsection{限制性和非限制性关系从句}




\section{关系从句简化的结果}

关系从句简化之后往往剩下分词。在此先初步介绍,完整的简化从句概念留待复句探
讨完毕之后再详加整理。

\subsection{Ven}

\begin{itemize}
\item  Toys \unbf{made in Taiwan} are much better now.

  现在台湾制造的玩具好多了。
\end{itemize}
这个过去分词短语 made in Taiwan 就是关系从句的简化。原句是:
\begin{itemize}
\item  Toys \unbf{which are made in Taiwan} are much better now.
\end{itemize}
这个关系从句中主语 which 与 Toys 重复,动词是空的 be动词。去掉这两个部分后
剩下的分词 made in Taiwan还是形容词类,因而可以简化。

\subsection{Ving}

\begin{itemize}
\item  Children \unbf{living in orphanages} make a lot of friends.

  在孤儿院生活的小朋友可以交很多朋友。
\end{itemize}
同样的,分词部分是 who are living in orphanages这个关系从句的简化,原因也相
同,省掉主语 who 和 be动词,剩下形容词短语 living \ldots{} 来取代关系从句。

\subsection{being Ven}

\begin{itemize}
\item  The vase \unbf{being auctioned} now is a Ming china.

  正在拍卖的花瓶是明朝的瓷器。
\end{itemize}
这里的 being auctioned 是 which is being auctioned 的简化。其中 being表示“正
在”,auctioned 表示“被拍卖”,如果没有 being,只剩下过去分词auctioned,就
有“完成”的暗示,读者可能会以为“已经卖掉了”。加上 being是为了去除这种误会,
增加表达的清楚性。

\section{状语从句简化的结果}

状语从句如果简化为分词,传统语法就叫做“分词构句”。之所以取这个名称,是因为
在传统语法的观察中分词是形容词,而状语从句是副词类,简化之后词类不一致,所以
取一个名称来称呼它。其实这里的变化和上一节的变化差不多,只是原来的从句词类不
同。

\subsection{Ven}

\begin{itemize}
\item  \unbf{Wounded in war}, the soldier was sent home.

在战场上受了伤,士兵就被遣送回家了。
\end{itemize}
这个分词短语是 After/Because he was wounded in war这个状语从句的简化。简化的
原因仍然一样:主语 he 就是 the soldier,所以可以省略,动词是 be 动词was,故可
以省略。一旦主语动词没有了,语法上也不需要连接词了。所以 After或 Because 也就
可以省掉,而只剩下补语部分的分词短语 wounded in war。这就是分词构句。

\subsection{Ving}

\begin{itemize}
\item The pigeon, \unbf{after flying 200 miles}, was caught up in a net.

  这只鸽子在飞了 200英里之后被网子网住了。
\end{itemize}

本句中底线部分原来是状语从句 after it flew 200 miles。因为主语 it 就是the
pigeon,因而可以省略。再下来\textbf{没有 be动词可省,就要先改成进行式(after
  it was flying \ldots)使它有 be动词,才可成功地省掉动词,剩下补语部分。}去
掉主语、动词后,连接词 after也可以省掉。可是状语从句的连接词有意义,Before,
When 和 After的意义都不一样,所以可以选择留下来,就成了 after flying 200
miles。这也是分词构句。

\subsection{having Ven}

\begin{itemize}
\item \unbf{Having finished the day's work}, the secretary went home.

  做完一天的工作后,秘书回家去了。
\end{itemize}

加底线部分原本是状语从句 She had finished the day's work。简化的原因还是因为
主语相同。而由于\textbf{动词部分没有 be动词}就不能进一步简化下去,因此改成 having
finished \ldots{}的现在分词形态,等于前面省去 be 动词,而留下补语部分。

\section{结语}

这是动状词的最后一讲。若要深入探讨分词,就得接触到简化从句,而简化从句又得建
立在复句结构上。所以,进一步的探讨要等复句结构介绍完毕后才能进行。而在进入复
句之前,还有一些小细节要先处理。下一章我们就来谈谈形容词的用法。
