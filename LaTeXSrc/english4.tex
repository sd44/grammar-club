\chapter{TIPS}

本章收录一些还未划入以上章节的tips.


\begin{itemize}

\item 可用介词短语替换的关系副词
  \begin{description}
  \item[where] prep + which + [place]
  \item[when] prep + which + [time]
  \item[why] prep + which + [reason]
  \item[how] prep + which + [manner/way]
  \end{description}
\item 英语中有三种“\textbf{横线}”:
  \begin{description}
  \item[hyphen -] 连字符,用来组成复合词,如mother-in-law ;或者连接因换行被
    分隔的单个单词。

  \item[en dash --] 表示数字范围,如 pages 13--67,the years 1929--1934。

  \item[em dash ---] 破折号,与中文破折号相似。如 three kinds of croissants---plain, almond, and chocolate
  \end{description}

\end{itemize}
